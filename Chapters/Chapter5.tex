\chapter{Wobbling Motion in Nuclei}

The pioneering work of Bohr and Mottelson \cite{bohr1998nuclear} which was done more than 50 years ago lead to some interesting features regarding the collective phenomena in triaxial nuclei. Namely, they pointed out that a specific precessional motion of the nucleus's spin will take place when the rotational energy is sufficient. The angular momentum for triaxial nuclei is not aligned any of the principal axes of the ellipsoid, but it \emph{precesses} and \emph{wobbles} around one of these axes. They called this phenomenon \textbf{wobbling motion} (w.m.).
This combined motion comes from a consequence regarding the MOI. Indeed, the asymmetry of the three MOI makes the quantum mechanical nature of rotation to be possible around any of the three axes. As such, a \emph{main} rotation around the axis with the largest MOI will be the most energetically favorable, but the other two directions can \emph{quantum mechanically disturb} this main rotation, leading to this unique characteristic of triaxial nuclei.

The non-uniformity nature of w.m. was firstly studied for the `pure' rigid-rotators that correspond to the even-even nuclei. In this case, the w.m. can be treated as small amplitude oscillations of the total angular momentum $\mathbf{I}$ around the axis corresponding to the largest MOI.

\section{Wobbling Motion in Even-Even Nuclei}

The analytical expressions for wobbling excitations were firstly evaluated by Bohr and Mottelson using the so-called \emph{Harmonic Approximation} (HA). This can be described as a small-amplitude limit for the Triaxial Rigid Rotor Hamiltonian that was discussed in Chapter \ref{chapter4} (see Section \ref{trm-model}). In this limit, the projection of the total angular momentum onto the axis with largest MOI can be approximated $I_3\approx I$, meaning that the nucleus will do most of its rotation around this axis, with little-to-no `disturbance' from the other two principal of the triaxial rotor. 

For a quantitative description of this simple wobbler, one can consider the case when the $3$-axis has the largest MOI, and the following order holds true:
\begin{align}
    \mathcal{I}_3>\mathcal{I}_2>\mathcal{I}_1\ ,
\end{align}
or equivalently:
\begin{align}
    A_3<A_2<A_1\ .
\end{align}

The Hamiltonian can be written as:
\begin{align}
    \hat{H}_\text{rot}={\color{red}A_3I_3^2}+{\color{blue}\left(A_1I_1^2+A_2I_2^2\right)}
\end{align}

The wobbling excitations which cause oscillations with small amplitudes for $\mathbf{I}$ about the $3$-axis are assumed to have a harmonic-like behavior, meaning that the final energy spectrum of a simple wobbler (even-even wobbler) will have the typical $\hbar\omega(n+1/2)$ behavior. Thus, in the HA, the eigenvalues of the rotor Hamiltonian can be expressed in terms of a \emph{wobbling phonon number} ($n_w$) and a \emph{wobbling frequency}:
\begin{align}
    E_{I,n}={\color{red}A_3I(I+1)}+{\color{blue}\hbar\omega_w\left(n_w+\frac{1}{2}\right)}\ .
    \label{eq-wobbling-energy-evenA}
\end{align}

Notice that the two colored terms emphasize the main rotation around $3$-axis (indicated by {\color{red}red}) and the disturbed motion with small oscillations around the other two axes (indicated by {\color{blue}blue}). Consequently, the wobbling character of the system will be generated by the latter harmonic term. The wobbling phonon number $n_w$ is related to the `strength' of the tilting for $\mathbf{I}$, indicating the fact that an increasing number for $n_w$ will result in oscillations with larger amplitudes around the other two axes.

In order to show the expression for the wobbling frequency from Eq. \ref{eq-wobbling-energy-evenA}, some additional terms need to be defined, starting from the three inertia parameters $A_k$.