\chapter{Wobbling Motion in Nuclei}

The pioneering work of Bohr and Mottelson \cite{bohr1998nuclear} which was done more than 50 years ago lead to some interesting features regarding the collective phenomena in triaxial nuclei. Namely, they pointed out that a specific precessional motion of the nucleus's spin will take place when the rotational energy is sufficient. The angular momentum for triaxial nuclei is not aligned any of the principal axes of the ellipsoid, but it \emph{precesses} and \emph{wobbles} around one of these axes. They called this phenomenon \textbf{wobbling motion} (w.m.).
This combined motion comes from a consequence regarding the MOI. Indeed, the asymmetry of the three MOI makes the quantum mechanical nature of rotation to be possible around any of the three axes. As such, a \emph{main} rotation around the axis with the largest MOI will be the most energetically favorable, but the other two directions can \emph{quantum mechanically disturb} this main rotation, leading to this unique characteristic of triaxial nuclei.

The non-uniformity nature of w.m. was firstly studied for the `pure' rigid-rotators that correspond to the even-even nuclei. In this case, the w.m. can be treated as small amplitude oscillations of the total angular momentum $\mathbf{I}$ around the axis corresponding to the largest MOI.

\section{Wobbling Motion in Even-Even Nuclei}

The analytical expressions for wobbling excitations were firstly evaluated by Bohr and Mottelson using the so-called \emph{Harmonic Approximation} (HA). This can be described as a small-amplitude limit for the Triaxial Rigid Rotor Hamiltonian that was discussed in Chapter \ref{chapter4} (see Section \ref{trm-model}). In this limit, the projection of the total angular momentum onto the axis with largest MOI can be approximated $I_3\approx I$, meaning that the nucleus will do most of its rotation around this axis, with some `disturbance' from the other two principal of the triaxial rotor. 

For a quantitative description of this simple wobbler, one can consider the case when the $3$-axis has the largest MOI, and the following order holds true:
\begin{align}
    \mathcal{I}_3>\mathcal{I}_2>\mathcal{I}_1\ ,
\end{align}
or equivalently:
\begin{align}
    A_3<A_2<A_1\ .
\end{align}

The Hamiltonian can be written as:
\begin{align}
    \hat{H}_\text{rot}={\color{red}A_3I_3^2}+{\color{blue}\left(A_1I_1^2+A_2I_2^2\right)}
    \label{general-rotor-ham-evenA}
\end{align}

The wobbling excitations which cause oscillations with small amplitudes for $\mathbf{I}$ about the $3$-axis are assumed to have a harmonic-like behavior, meaning that the final energy spectrum of a simple wobbler (even-even nucleus) will have the typical $\hbar\omega(n+1/2)$ behavior. Since this oscillator motion can be explained as `vibrations' of the total angular momentum around a steady position, where each wobbling excitation consists of an additional vibrating phonon, one can express the Hamiltonian in terms of \emph{boson} creation and annihilation operators. As such, the quantum mechanical treatment implies \cite{bohr1998nuclear}:
\begin{align}
    b^\dagger=\frac{1}{\sqrt{2I}}I_+\ ,\ b=\frac{1}{\sqrt{2I}}I_-\ ,\ \left[b,b^\dagger\right]\approx 1\ .
\end{align}

This initial quantization allows one to write Eq. \ref{general-rotor-ham-evenA} as a rotational term and a wobbling-specific one:
\begin{align}
    \hat{H}_\text{rot}&={\color{red}A_3I_3^2}+{\color{blue}H_w}\ ,\\
    {\color{blue}H_w}&={\color{blue}t_1\left(n+\frac{1}{2}\right)+\frac{1}{2}t_2(b^\dagger b^\dagger + bb)}\ ,
\end{align}
where the \emph{number of boson excitations} is denoted by $n$. Each wobbling quanta will carry an angular momentum of one unit less with respect to the $3$-axis.

Thus, in the HA, the eigenvalues of the rotor Hamiltonian can be expressed in terms of a \emph{wobbling phonon number} ($n_w$) and a \emph{wobbling frequency}:
\begin{align}
    E_{I,n}={\color{red}A_3I(I+1)}+{\color{blue}\hbar\omega_w\left(n_w+\frac{1}{2}\right)}\ .
    \label{eq-wobbling-energy-evenA}
\end{align}

Notice that the two colored terms emphasize the main rotation around $3$-axis (indicated by {\color{red}red}) and the disturbed motion with small oscillations around the other two axes (indicated by {\color{blue}blue}). Consequently, the wobbling character of the system will be generated by the latter harmonic term. The wobbling phonon number $n_w$ is related to the `strength' of the tilting for $\mathbf{I}$, indicating the fact that an increasing number for $n_w$ will result in oscillations with larger amplitudes around the other two axes. The phonon number takes values $n_w=0,1,\dots$.

In order to show the expression for the wobbling frequency from Eq. \ref{eq-wobbling-energy-evenA}, some additional terms need to be defined, starting from the three inertia parameters $A_k$. If the following quantities are defined:
\begin{align}
    t_1&=I(A_2+A_1-2A_3)\ , \\
    t_2&=I(A_2-A_1)\ ,
\end{align}
then one can write the wobbling frequency as \cite{wen2015wobbling}:
\begin{align}
    \hbar\omega_w=\sqrt{t_1^2-t_2^2}=2I\sqrt{(A_1-A_3)(A_2-A_3)}
\end{align}

As a quantitative analysis of the wobbling frequency and the rotor energy, one can take three arbitrary values for the moments of inertia (and, implicitly, the inertia factors $A_k$) and see the behavior of both $E_{I,n}$ and $\hbar\omega_w$ with increasing angular momentum and wobbling phonon number. Keep in mind that depending on the value of the wobbling phonon number, different spin sequences will be allowed. More precisely, from the invariance of the rotor w.r.t. rotations by $\pi$ about the principal axes for even-even nuclei, the signature quantum number $\alpha$ can take the values 0 and 1. Each wobbling band will have an alternating signature, starting with $\alpha=0$ for $n_w=0$ then $\alpha=1$ for $n_w=1$ and so on: even spin sequences appear for even values of $n_w$ and odd spin sequences appear for odd values of $n_w$.

The rotor energy from Eq. \ref{eq-wobbling-energy-evenA} is graphically represented for an arbitrary set of moments of inertia as a function of the nuclear angular momentum $I$ in Fig. \ref{fig-even-even-wobbling-energies}. This pedagogical example contains rotational bands up to $n_w=5$ in the wobbling phonon number. From Fig. \ref{fig-even-even-wobbling-energies}, one can see the linear dependence on the total angular momentum and, moreover, the wobbling energy and frequency both are increasing with increasing spin.

\begin{figure}
    \centering
    \includegraphics[scale=0.7]{Chapters/Figures/wobbling-evenA.pdf}
    \includegraphics[scale=0.74]{Chapters/Figures/wobblingFreq-evenA.pdf}
    \caption{\textbf{Left:} The rotor energy for an even-even nucleus with three different moments of inertia, with the main rotation around the $3$-axis. Each wobbling band has alternating signature number $\alpha$ (starting with $\alpha=0$ for the ground state $n_w=0$ band). Notice the even/odd spin sequences for each band. \textbf{Right:}: The wobbling frequency plotted together with the linear terms $t_1$ and $t_2$ that are used to express $\hbar\omega_w$. Same set of MOI were used across both figures. The unit for $\mathcal{I}_i$ is $\hbar^2\ \text{MeV}^-1$.}
    \label{fig-even-even-wobbling-energies}
\end{figure}


Another instructive analysis would be the evolution of the components of $\mathbf{I}$ as a function of the polar and azimuthal angles $\theta,\varphi$. Indeed, expressing the three components as:
\begin{align}
    I_1&=I'\sin\theta\cos\varphi\ ,\\
    I_2&=I'\sin\theta\sin\varphi\ ,\\
    I_3&=I'\cos\theta\ ,
    \label{angular-momentum-polar-components}
\end{align}
where $I'=\sqrt{I(I+1)}$, one can make a graphical representation for these components. In Fig. \ref{figs-angular-momentum-components-polar}, the components $I_1$ and $I_2$ are represented in the $\theta,\varphi$ plane, for a fixed value $I=10\hbar$. Since the third component is independent of the azimuthal angle $\varphi$, it has been dismissed.

\begin{figure}
    \centering
    \includegraphics[scale=0.66]{Chapters/Figures/angular_components-TRM-1.pdf}
    \includegraphics[scale=0.66]{Chapters/Figures/angular_components-TRM-2.pdf}
    \caption{The geometrical representation of the first and second component of the total angular momentum $\mathbf{I}$, as function of the polar angles, according to Eq. \ref{angular-momentum-polar-components}.}
    \label{figs-angular-momentum-components-polar}
\end{figure}

The other relevant observables which can be calculated for simple wobbler within the HA are the two quadrupole moments $Q_{20,22}$ and the intraband + interband $B(E2)$ transition probabilities. The quadrupole components are expressed in terms of the intrinsic quadrupole moment $Q_0$ and the triaxiality parameter as \cite{shoji2006microscopic}:
\begin{align}
    Q_{20}=Q_0\cos\gamma\ ,\ Q_{22}=\frac{1}{\sqrt{2}}Q_0\sin\gamma\ .
\end{align}

These components can be furthermore used to determine the intraband $B(E2)$ transition probabilities \cite{wen2015wobbling}:
\begin{align}
    B(E2;(n,I)\to(n,I-2))=\frac{5}{16\pi}Q_{22}^2\ ,
\end{align}
and also the interband transitions:
\begin{align}
    B(E2;(n,I)\to(n-1,I-1))&=\frac{5}{16\pi}\frac{n}{I}\left(\sqrt{3}Q_{20}w_1+\sqrt{2}Q_{22}w_2\right)^2\ ,\\
    B(E2;(n,I)\to(n+1,I-1))&=\frac{5}{16\pi}\frac{n+1}{I}\left(\sqrt{3}Q_{20}w_2+\sqrt{2}Q_{22}w_1\right)^2\ ,
\end{align}
where the two coefficients $w_{1,2}$ are defined as \cite{bohr1998nuclear}:
\begin{align}
    w_1&=\left[\frac{1}{2}\left(\frac{t_1}{\sqrt{t_1^2-t_2^2}}+1\right)\right]^{1/2}\ ,\\
    w_2&=\left[\frac{1}{2}\left(\frac{t_1}{\sqrt{t_1^2-t_2^2}}+1\right)\right]^{1/2}\ .
\end{align}

