\chapter{Wobbling Motion in Nuclei}

The pioneering work of Bohr and Mottelson \cite{bohr1998nuclear} which was done more than 50 years ago lead to some interesting features regarding the collective phenomena in triaxial nuclei. Namely, they pointed out that a specific precessional motion of the nucleus's spin will take place when the rotational energy is sufficient. The angular momentum for triaxial nuclei is not aligned any of the principal axes of the ellipsoid, but it \emph{precesses} and \emph{wobbles} around one of these axes. They called this phenomenon \textbf{wobbling motion} (w.m.).
This combined motion comes from a consequence regarding the MOI. Indeed, the asymmetry of the three MOI makes the quantum mechanical nature of rotation to be possible around any of the three axes. As such, a \emph{main} rotation around the axis with the largest MOI will be the most energetically favorable, but the other two directions can \emph{quantum mechanically disturb} this main rotation, leading to this unique characteristic of triaxial nuclei.

The non-uniformity nature of w.m. was firstly studied for the `pure' rigid-rotators that correspond to the even-even nuclei. In this case, the w.m. can be treated as small amplitude oscillations of the total angular momentum $\mathbf{I}$ around the axis corresponding to the largest MOI.