\chapter{Triaxial Nuclei and Their Signatures}

In the following chapter, some theoretical background that is necessary for understanding triaxiality will be presented, with examples from literature and also some results obtained by this team. It is also instructive to realize why the nuclear community focuses their attention to the highly deformed nuclei, and moreover, nuclei which depart so much from the spherical shapes that they become \emph{triaxial}.

Presenting the theoretical formalism that is used to describe triaxial nuclei, and mention the \emph{fingerprints} of nuclear triaxiality is that last step before diving into the recently developed framework for odd-$A$ nuclei which the team created, and showing the results. 

\section{Non-axial nuclei}

The discussion regarding excitation energies of a rigid rotator from the previous chapter was focused on pure rotators or nuclei with axially symmetric shapes (i.e,, prolate or oblate). Remember that deformation is still required in order to define a collective spectra with rotational character. Moreover, the relevant quantities that are involved in the rotational motion for a deformed nucleus are the moments of inertia corresponding to the principal axes of the deformed ellipsoid: $\mathcal{I}_{1,2,3}$, and within previous calculations, two moments of inertia were supposed to be identical.

Even though calculations are performed with the rigid-like MOI or the irrotational-like, their dependence on the deformation parameters $\beta_2$ and $\gamma$ is present (recall expressions given in Eq. \ref{eq-irrotational-rigid-mois}). Taking a closer look at their evolution with $\gamma$, one can see that indeed, identical MOI can only occur at certain values (see Fig. \ref{fig-irrotational-rigid-mois}). As such, the nuclei can be regarded (when referring to their ground state) as such:
\begin{itemize}
    \item \textbf{Spherical:} all MOI are identical and no deformations are present
    \item \textbf{Axially-symmetric:} two identical MOI and only the $\beta_2$ parameter plays a role in the collective phenomena of these nuclei
    \item \textbf{Triaxial (Axially-asymmetric):} all three MOI are different (usually one of them is very large when compared to the other two), the quadrupole deformation parameter as well as the triaxiality parameter $\gamma$ are present
\end{itemize}

Now, across the chart of nuclides, most of the isotopes are either spherical either symmetric in their ground state \cite{budaca2018tilted}, but triaxial shapes might also occur as ground-state \cite{moller2006global}. The apparition of triaxial nuclei implies some kind of `stability' on the $\gamma$-parameter, since a dynamical character of the triaxiality will indicate some transitional states rather than nuclear stability. Indeed, a rigid (fixed) value for $\gamma$ is required around the minimal region of the potential energy surface such that triaxial stable nuclei can exist. Thus, one can distinguish between the $\gamma$-soft nuclei in which this parameter has a dynamical character and the $\gamma$-rigid ones, which could in fact exhibit stable triaxial deformation \cite{dracoulis2013isomers}.
Even more interesting are the structures which occur at very large values of quadrupole deformation $\beta_2$ and $\gamma$ in the vicinity of $\approx 30^\circ$ (where maximal triaxiality occurs). It will be shown that these last nuclei will lead to band structures that are called \emph{Triaxial Strongly Deformed} bands (TSD for short) \cite{odegaard2001evidence,jensen2002evidence}.

Concluding, the triaxial nuclei are a special class of nuclei in which there is an asymmetry between the moments of inertia (given by a $\gamma$ value within the corresponding interval), and moreover, the quadrupole deformation is high enough such that it stabilizes the entire system.

\subsection{Triaxial Rotor Model}

The most general Hamiltonian for a \emph{triaxial system} is given in terms of the components of the total angular momentum operator $\hat{I}$ and the moments of inertia for the deformed ellipsoid, similarly as it was the case for the \emph{rotational Hamiltonian} within the symmetry case:
\begin{align}
    \hat{H}=\frac{\hat{I}_1^2}{2\mathcal{I}_1}+\frac{\hat{I}_2^2}{2\mathcal{I}_2}+\frac{\hat{I}_3^2}{2\mathcal{I}_3}\ ,
\end{align}
where the indices correspond to each of the principal axes of the rotational ellipsoid (principal axes are the ones in which the components of the MOI tensor are diagonal). More often, the notations $A_{1,2,3}=\frac{\hbar^2}{2\mathcal{I}_{1,2,3}}$ are used in the Hamiltonian's expression, leading to $A_1\neq A_2\neq A_3$ for the triaxial nuclei. Shi et al. \cite{wen2015wobbling} show a very straightforward way of obtaining the eigenvalues for the \emph{triaxial rigid rotor}, following the quantum treatment made by Davydov and Filippov for the rigid rotor without symmetry axis \cite{davydov1958rotational}:
\begin{align}
    \hat{H}&=\hat{H}_\text{diag}+\hat{H}_\text{non-diag}\ ,\nonumber\\
    \hat{H}_\text{diag}&=\left[\frac{1}{2}\left(A_1+A_2\right)\left(\hat{I}^2-\hat{I}_3^2\right)+A_3\hat{I}_3^2\right]\ ,\nonumber\\
    \hat{H}_\text{non-diag}&=\frac{1}{4}\left(A_1-A_2\right)\left(\hat{I}_+^2+\hat{I}_-^2\right)\ .
\end{align}
This way of expressing the Hamiltonian is useful because there is a clear difference between a term which is diagonal and one that mixes states with different $\Delta K=\pm2$ quantum number. It is important to emphasize that this kind of Hamiltonian is still invariant to rotations with $\pi$ around the principal axes. This is useful because one can solve the eigenvalue problem with the basis $\ket{IMK}$, were the wave-function is described as:
\begin{align}
    \ket{IMK}=\sqrt{\frac{2I+1}{16\pi^2(1+\delta_{K0})}}\left[\ket{IMK}+(-)^I\ket{IM-K}\right]\ ,
\end{align}
where $\ket{IM\pm K}$ are the Wigner $D_{MK}^I$ - functions that determine the \emph{orientation} of the nucleus itself (their are functions of the three Euler angles), the $K$ quantum number is the projection of $I$ onto the 3-axis of the body-fixed frame (intrinsic frame of reference), and the $M$ number represents the projection of $I$ onto the $z$-axis of the laboratory frame. This wave-function is quite similar to the one defined in Eq. \ref{RAL-bands-wave-function}, when the decoupled states in the Rotation Aligned Bands were studied. Indeed, using this basis, the total Hamiltonian can be diagonalized \cite{wen2015wobbling} as such:
\begin{align}
    \hat{H}_{IK}=\frac{1}{2}(A_1+A_2)\left[I(I+1)-K^2\right]+A_3K^2\ ,
\end{align}
with the notation $H_{IK}\equiv\bra{IK}\hat{H}\ket{IK}$. The second, non-diagonal term will have the energy states given as:
\begin{align}
    \hat{H}_{IK\pm2}=\frac{1}{4}(A_1-A_2)\sqrt{(I\mp K)(I\pm K +1)(I \mp K -1)(I\pm K +2)}\ ,
\end{align}
where $\hat{H}_{IK\pm 2}\equiv \bra{IK}\hat{H}\ket{I\pm K}$. Finally, using these matrix elements, the energies can be obtained by solving the eigenvalue equation for given spins $I$. Such calculations were performed by the team for an even-even nucleus $^{158}$Er (see Fig. 8 from  \cite{raduta2017semiclassical}) and the results of the diagonalization procedure were in complete agreement with alternative descriptions for $\hat{H}$.

\subsection{Triaxial Particle + Rotor Model}

In this section, a general discussion about the Hamiltonian for a system which is composed on one single-particle (valence nucleon) in a high-$j$ shell and one triaxial core. Usually, such a treatment will be applied to odd-$A$ triaxial nuclei, and in fact the team's developed framework is founded on the triaxial PRM \cite{davydov1958rotational}. Davydov et al. developed this model for explaining the low-lying collective spectra of $2^+$ states within some transitional nuclei.

The Hamiltonian of this system will be composed of a term which corresponds to the even-even core and that of a single-particle which is moving in a quadrupole deformed mean-field (remember that quadrupole deformation are the only relevant effects when discussing triaxial structures).
\begin{align}
    \hat{H}=\hat{H}_\text{rot}+\hat{H}'_\text{sp}\ .
\end{align}

Here, an important remark must be done, regarding the notation of the terms. Usually, the single-particle Hamiltonian is composed of a term which gives its \emph{intrinsic} energy which comes from the $j$ shell in which the nucleon will orbit and a term that characterizes the effective interaction between the particle itself and the deformed mean-field generated by the core. Consequently, the Hamiltonian $\hat{H}'_\text{sp}$ should be written as: $$\hat{H}'_\text{sp}=\hat{h}_0^j+\hat{H}_\text{int}^\text{quad}$$, where $\hat{h}_0^j$ is the contribution of the total particle energy coming from the position of the $j$-shell and $\hat{H}_\text{int}^\text{quad}$ is the interaction of the valence nucleon with the even-even core (that is of quadrupole type). For example, Ring et al. \cite{ring2004nuclear} uses the SHO for describing $\hat{h}_0^j$. The interaction Hamiltonian is in fact a $\gamma$-deformed Nilsson potential, which was previously discussed (see Chapter), with it's general expression in this case written as:
\begin{align}
    \hat{H}_\text{int}^\text{quad}=\kappa\beta r^2\left[\cos\gamma Y_{2}^{0}+\frac{\sin\gamma}{\sqrt{2}}\left(Y_2^2+Y_2^{-2}\right)\right].
\end{align}
However, this generalized expression can be written for the single-particle $j$ shell in the following way:
\begin{align}
    \hat{H}_\text{int}^\text{quad}=\frac{V}{j(j+1)}\left[\cos\gamma(3j_z^2-\mathbf{j}^2)-\sqrt{3}\sin\gamma(j_x^2-j_y^2)\right]\ ,
\end{align}
where the entire interaction strength between the particle and the core is embedded within the value of a parameter (usually adjustable) called \emph{single-particle potential strength} $V$. This parameter is very important in the present research, since the obtained theoretical results regarding energy spectra was obtained through the determination of $V$ numerically.

% Within the literature (e.g., see similar work in Ref. \cite{peng2003description,wang2008description,tanabe2017stability}).