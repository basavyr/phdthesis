\chapter{Triaxial Nuclei and Their Signatures}

In the following chapter, some theoretical background that is necessary for understanding triaxiality will be presented, with examples from literature and also some results obtained by this team. It is also instructive to realize why the nuclear community focuses their attention to the highly deformed nuclei, and moreover, nuclei which depart so much from the spherical shapes that they become \emph{triaxial}.

Presenting the theoretical formalism that is used to describe triaxial nuclei, and mention the \emph{fingerprints} of nuclear triaxiality is that last step before diving into the recently developed framework for odd-$A$ nuclei which the team created, and showing the results. 

\section{Non-axial nuclei}

The discussion regarding excitation energies of a rigid rotator from the previous chapter was focused on pure rotators or nuclei with axially symmetric shapes (i.e,, prolate or oblate). Remember that deformation is still required in order to define a collective spectra with rotational character. Moreover, the relevant quantities that are involved in the rotational motion for a deformed nucleus are the moments of inertia corresponding to the principal axes of the deformed ellipsoid: $\mathcal{I}_{1,2,3}$, and within previous calculations, two moments of inertia were supposed to be identical.

Even though calculations are performed with the rigid-like MOI or the irrotational-like, their dependence on the deformation parameters $\beta_2$ and $\gamma$ is present (recall expressions given in Eq. \ref{eq-irrotational-rigid-mois}). Taking a closer look at their evolution with $\gamma$, one can see that indeed, identical MOI can only occur at certain values (see Fig. \ref{fig-irrotational-rigid-mois}). As such, the nuclei can be regarded (when referring to their ground state) as such:
\begin{itemize}
    \item \textbf{Spherical:} all MOI are identical and no deformations are present
    \item \textbf{Axially-symmetric:} two identical MOI and only the $\beta_2$ parameter plays a role in the collective phenomena of these nuclei
    \item \textbf{Triaxial (Axially-asymmetric):} all three MOI are different (usually one of them is very large when compared to the other two), the quadrupole deformation parameter as well as the triaxiality parameter $\gamma$ are present
\end{itemize}

Now, across the chart of nuclides, most of the isotopes are either spherical either symmetric in their ground state \cite{budaca2018tilted}.