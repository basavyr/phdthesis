\chapter{Triaxial Nuclei and Their Signatures}
\label{chapter4}
In the following chapter, some theoretical background that is necessary for understanding triaxiality will be presented, with examples from literature and also some results obtained by this team. It is also instructive to realize why the nuclear community focuses their attention to the highly deformed nuclei, and moreover, nuclei which depart so much from the spherical shapes that they become \emph{triaxial}.

Presenting the theoretical formalism that is used to describe triaxial nuclei, and mention the \emph{fingerprints} of nuclear triaxiality is that last step before diving into the recently developed framework for odd-$A$ nuclei which the team created, and showing the results. 

\section{Non-axial nuclei}

The discussion regarding excitation energies of a rigid rotator from the previous chapter was focused on pure rotators or nuclei with axially symmetric shapes (i.e,, prolate or oblate). Remember that deformation is still required in order to define a collective spectra with rotational character. Moreover, the relevant quantities that are involved in the rotational motion for a deformed nucleus are the moments of inertia corresponding to the principal axes of the deformed ellipsoid: $\mathcal{I}_{1,2,3}$, and within previous calculations, two moments of inertia were supposed to be identical.

Even though calculations are performed with the rigid-like MOI or the irrotational-like, their dependence on the deformation parameters $\beta_2$ and $\gamma$ is present (recall expressions given in Eq. \ref{eq-irrotational-rigid-mois}). Taking a closer look at their evolution with $\gamma$, one can see that indeed, identical MOI can only occur at certain values (see Fig. \ref{fig-irrotational-rigid-mois}). As such, the nuclei can be regarded (when referring to their ground state) as such:
\begin{itemize}
    \item \textbf{Spherical:} all MOI are identical and no deformations are present
    \item \textbf{Axially-symmetric:} two identical MOI and only the $\beta_2$ parameter plays a role in the collective phenomena of these nuclei
    \item \textbf{Triaxial (Axially-asymmetric):} all three MOI are different (usually one of them is very large when compared to the other two), the quadrupole deformation parameter as well as the triaxiality parameter $\gamma$ are present
\end{itemize}

Now, across the chart of nuclides, most of the isotopes are either spherical either symmetric in their ground state \cite{budaca2018tilted}, but triaxial shapes might also occur as ground-state \cite{moller2006global}. The apparition of triaxial nuclei implies some kind of `stability' on the $\gamma$-parameter, since a dynamical character of the triaxiality will indicate some transitional states rather than nuclear stability. Indeed, a rigid (fixed) value for $\gamma$ is required around the minimal region of the potential energy surface such that triaxial stable nuclei can exist. Thus, one can distinguish between the $\gamma$-soft nuclei in which this parameter has a dynamical character and the $\gamma$-rigid ones, which could in fact exhibit stable triaxial deformation \cite{dracoulis2013isomers}.
Even more interesting are the structures which occur at very large values of quadrupole deformation $\beta_2$ and $\gamma$ in the vicinity of $\approx 30^\circ$ (where maximal triaxiality occurs). It will be shown that these last nuclei will lead to band structures that are called \emph{Triaxial Strongly Deformed} bands (TSD for short) \cite{odegaard2001evidence,jensen2002evidence}.

Concluding, the triaxial nuclei are a special class of nuclei in which there is an asymmetry between the moments of inertia (given by a $\gamma$ value within the corresponding interval), and moreover, the quadrupole deformation is high enough such that it stabilizes the entire system.

\subsection{Triaxial Rotor Model}
\label{trm-model}

The most general Hamiltonian for a \emph{triaxial system} is given in terms of the components of the total angular momentum operator $\hat{I}$ and the moments of inertia for the deformed ellipsoid, similarly as it was the case for the \emph{rotational Hamiltonian} within the symmetry case:
\begin{align}
    \hat{H}=\frac{\hat{I}_1^2}{2\mathcal{I}_1}+\frac{\hat{I}_2^2}{2\mathcal{I}_2}+\frac{\hat{I}_3^2}{2\mathcal{I}_3}\ ,
    \label{general-rotor-hamiltonian}
\end{align}
where the indices correspond to each of the principal axes of the rotational ellipsoid (principal axes are the ones in which the components of the MOI tensor are diagonal). More often, the notations $A_{1,2,3}=\frac{\hbar^2}{2\mathcal{I}_{1,2,3}}$ are used in the Hamiltonian's expression, leading to $A_1\neq A_2\neq A_3$ for the triaxial nuclei. Shi et al. \cite{wen2015wobbling} show a very straightforward way of obtaining the eigenvalues for the \emph{triaxial rigid rotor}, following the quantum treatment made by Davydov and Filippov for the rigid rotor without symmetry axis \cite{davydov1958rotational}:
\begin{align}
    \hat{H}&=\hat{H}_\text{diag}+\hat{H}_\text{non-diag}\ ,\nonumber\\
    \hat{H}_\text{diag}&=\left[\frac{1}{2}\left(A_1+A_2\right)\left(\hat{I}^2-\hat{I}_3^2\right)+A_3\hat{I}_3^2\right]\ ,\nonumber\\
    \hat{H}_\text{non-diag}&=\frac{1}{4}\left(A_1-A_2\right)\left(\hat{I}_+^2+\hat{I}_-^2\right)\ .
\end{align}
This way of expressing the Hamiltonian is useful because there is a clear difference between a term which is diagonal and one that mixes states with different $\Delta K=\pm2$ quantum number. It is important to emphasize that this kind of Hamiltonian is still invariant to rotations with $\pi$ around the principal axes. This is useful because one can solve the eigenvalue problem with the basis $\ket{IMK}$, were the wave-function is described as:
\begin{align}
    \ket{IMK}=\sqrt{\frac{2I+1}{16\pi^2(1+\delta_{K0})}}\left[\ket{IMK}+(-)^I\ket{IM-K}\right]\ ,
\end{align}
where $\ket{IM\pm K}$ are the Wigner $D_{MK}^I$ - functions that determine the \emph{orientation} of the nucleus itself (their are functions of the three Euler angles), the $K$ quantum number is the projection of $I$ onto the 3-axis of the body-fixed frame (intrinsic frame of reference), and the $M$ number represents the projection of $I$ onto the $z$-axis of the laboratory frame. This wave-function is quite similar to the one defined in Eq. \ref{RAL-bands-wave-function}, when the decoupled states in the Rotation Aligned Bands were studied. Indeed, using this basis, the total Hamiltonian can be diagonalized \cite{wen2015wobbling} as such:
\begin{align}
    \hat{H}_{IK}=\frac{1}{2}(A_1+A_2)\left[I(I+1)-K^2\right]+A_3K^2\ ,
\end{align}
with the notation $H_{IK}\equiv\bra{IK}\hat{H}\ket{IK}$. The second, non-diagonal term will have the energy states given as:
\begin{align}
    \hat{H}_{IK\pm2}=\frac{1}{4}(A_1-A_2)\sqrt{(I\mp K)(I\pm K +1)(I \mp K -1)(I\pm K +2)}\ ,
\end{align}
where $\hat{H}_{IK\pm 2}\equiv \bra{IK}\hat{H}\ket{I\pm K}$. Finally, using these matrix elements, the energies can be obtained by solving the eigenvalue equation for given spins $I$. Such calculations were performed by the team for an even-even nucleus $^{158}$Er (see Fig. 8 from  \cite{raduta2017semiclassical}) and the results of the diagonalization procedure were in complete agreement with alternative descriptions for $\hat{H}$.

\subsection{Triaxial Particle + Rotor Model}

In this section, a general discussion about the Hamiltonian for a system which contains one single-particle and one triaxial core will be made. Usually, such a treatment is applied to odd-$A$ triaxial nuclei, and in fact, the current team's developed framework that will be described in the following chapters is founded on the triaxial PRM \cite{davydov1958rotational}. Davydov et al. developed this model for explaining the low-lying collective spectra of $2^+$ states within some transitional nuclei.

The Hamiltonian of this system is composed of a term which corresponds to the even-even core and another one for a single-particle which is moving in a quadrupole deformed mean-field (remember that quadrupole deformation are the only relevant effects when discussing triaxially-deformed structures).
\begin{align}
    \hat{H}=\hat{H}_\text{rot}+\hat{H}'_\text{sp}\ .
    \label{triaxial-prm-general-hamiltonian}
\end{align}
Here, an important observation must be made regarding the second term. Usually, the single-particle Hamiltonian $\hat{H}'_\text{sp}$ is composed of an \emph{intrinsic energy} coming from the $j$-shell in which the nucleon is orbiting (one can think of it as the Fermi energy level), and an \emph{effective energy} that characterizes the interaction between the particle and the deformed mean-field generated by the core (which is of quadrupole nature). Consequently, the Hamiltonian $\hat{H}'_\text{sp}$ should be written as: $$\hat{H}'_\text{sp}=\hat{h}_0^j+\hat{H}_\text{int}^\text{quad}\ ,$$where $\hat{h}_0^j$ is the former term and $\hat{H}_\text{int}^\text{quad}$ is the latter one.
%the contribution of the total particle energy coming from the position of the $j$-shell and $\hat{H}_\text{int}^\text{quad}$ is the interaction of the valence nucleon with the even-even core (that is of quadrupole type). 
For example, Ring et al. \cite{ring2004nuclear} uses the SHO for describing $\hat{h}_0^j$. The interaction Hamiltonian is in fact a $\gamma$-deformed Nilsson potential, which was previously discussed (see Section \ref{nilsson-model-section}), with its general expression:
\begin{align}
    \hat{H}_\text{int}^\text{quad}=\kappa\beta r^2\left[\cos\gamma Y_{2}^{0}+\frac{\sin\gamma}{\sqrt{2}}\left(Y_2^2+Y_2^{-2}\right)\right].
    \label{quadrupole-deformed-potential-beta}
\end{align}
Furthermore, this deformed potential gives the energy splittings for the nucleonic orbits (recall the Nilsson diagrams from Figs. \ref{nillson-diagram} - \ref{nillson-diagram-2}), and it can be expressed in terms of the atomic mass $A$ and the quadrupole deformation parameter $\beta_2$ \cite{peng2003description}:
\begin{align}
    \hat{H}_\text{int}^\text{quad}=\frac{206}{A^{1/3}}\beta_2\left[\cos\gamma Y_2^0+\frac{\sin\gamma}{\sqrt{2}}(Y_2^2+Y_2^{-2})\right]\ .
    \label{quadrupole-deformed-potential-V}
\end{align}
However, the generalized expressions from Eqs. \ref{quadrupole-deformed-potential-beta} - \ref{quadrupole-deformed-potential-V} can be re-written for a single-particle characterized by its total angular momentum $\mathbf{j}$ in the following way:
\begin{align}
    \hat{H}_\text{int}^\text{quad}=\frac{V}{j(j+1)}\left[\cos\gamma(3j_z^2-\mathbf{j}^2)-\sqrt{3}\sin\gamma(j_x^2-j_y^2)\right]\ ,
    \label{single-particle-nilsson-defored-potential}
\end{align}
where the entire interaction strength between the particle and the core is embedded within the value of a parameter (usually adjustable) called \emph{single-particle potential strength} $V$. This parameter is very important in the present research, since the theoretical results regarding energy spectra were obtained through the numerical determination of $V$ (as it will be shown in the following chapters). Obviously, the three components of the single-particle angular momentum are represented by $j_x$, $j_y$, and $j_z$.

Peng et al. \cite{peng2003description} gave a description for odd-odd nuclei within $A\approx 100$ and $A\approx 130$ mass regions using a Hamiltonian in which two valence nucleons were coupled to the triaxial core. Indeed, their Hamiltonian was constructed in the following way:
\begin{align}
    \hat{H}=\hat{H}_\text{intr}+\hat{H}_{coll}\ ,
\end{align}
where $\hat{H}_\text{coll}$ is the typical rotor Hamiltonian (such as the one defined in Eq. \ref{general-rotor-hamiltonian}) and $\hat{H}_\text{intr}$ represents the sum of a proton and neutron contribution:
\begin{align}
    \hat{H}_\text{intr}=h_p+h_n\ .
    \label{intrinsic-proton-neutron-hamiltonian}
\end{align}
$\hat{H}_\text{intr}$ can be viewed as an intrinsic part that describes the deformation of the core and the motion of the nucleons within the mean-field. A remarking feature of this approach is that the problem can be extrapolated to a case with multiple protons and neutrons, invoking some sort of `scalability' of their model. The single-particle energies for protons and neutrons from Eq. \ref{intrinsic-proton-neutron-hamiltonian} have a somewhat similar form as the one stated in Eq. \ref{single-particle-nilsson-defored-potential}, namely it is given as:
\begin{align}
    h_{p(n)}=\pm\frac{1}{2}C_\beta\left[\left(j_3^2-\frac{j(j+1)}{3}\right)\cos\gamma+\frac{1}{2\sqrt{3}}(j_+^2+j_-^2)\sin\gamma\right]\ ,
    \label{single-particle-energies-hpn}
\end{align}
with the alternating signs corresponding to the proton and neutron, respectively. Note the switch of axes labelling from $(x,y,z)$ to $(1,2,3)$, and the introduction of ladder operators $(j_+,j_-)$ that change the expression of the potential initially stated in Eq. \ref{single-particle-nilsson-defored-potential}. The \emph{coupling parameter} $C_\beta$ is a measure of energy (typically expressed in MeV) with its value depending linearly on $\beta_2$:
\begin{align}
    C_\beta=\frac{195}{j(j+1)}\frac{1}{A^{1/3}}\beta_2\ ,
\end{align}
% \subsubsection*{Numerical application - Deformed Potential}
thus one can see that the single-particle potential strength $V$ is in fact related to this coupling parameter.

It is worth giving some quantitative results concerning the quadrupole potential and the single-particle Hamiltonian described above. As such, a simple numerical application will be employed, obtaining graphical representations with the behavior of these quantities with respect to the deformation parameters. Firstly, the potential $\hat{H}_\text{int}^\text{quad}$ will be analyzed in the polar plane defined by the angles $(\theta,\varphi)$ (since the spherical harmonics are defined in terms of them). For nuclei near the $A\approx 160$ region, it is common to have values of $\beta_2\in\left[0.2,0.4\right]$ and $\gamma\approx 20^\circ$, so for testing purposes one can fix $\beta_2$ and $\gamma$ within those ranges. Therefore, the quadrupole potential $\hat{H}_\text{int}^\text{quad}$ can be numerically evaluated for known polar angles, and the results can be seen in Figs. \ref{figs-deformed-quadrupole-potential-1} - \ref{figs-deformed-quadrupole-potential-2}.
\begin{figure}
    \centering
    \includegraphics[scale=0.66]{Chapters/Figures/quadrupole-potentialV-1.pdf}
    \includegraphics[scale=0.66]{Chapters/Figures/quadrupole-potentialV-2.pdf}
    \caption{The quadrupole potential $\hat{H}_\text{int}^\text{quad}$ as defined in Eq. \ref{quadrupole-deformed-potential-V}, represented as a function of the angular coordinates $\theta$ and $\varphi$. Calculations were done with fixed parameters $\beta_2$, $\gamma$, and for $A=163$.}
    \label{figs-deformed-quadrupole-potential-1}
\end{figure}
\begin{figure}
    \centering
    \includegraphics[scale=0.66]{Chapters/Figures/quadrupole-potentialV-3.pdf}
    \includegraphics[scale=0.66]{Chapters/Figures/quadrupole-potentialV-4.pdf}
    \caption{The quadrupole potential $\hat{H}_\text{int}^\text{quad}$ as defined in Eq. \ref{quadrupole-deformed-potential-V}, represented as a function of the angular coordinates $\theta$ and $\varphi$. Calculations were done with fixed parameters $\beta_2$, $\gamma$, and for $A=163$.}
    \label{figs-deformed-quadrupole-potential-2}
\end{figure}

Concerning the numerical evaluation of $h$ given in Eq. \ref{single-particle-energies-hpn} (hereafter, the indices $p,n$ are dismissed), one can take the diagonal components of $h$ and apply the rules only for one proton. As a result, the \emph{mixing terms} $j_+$ and $j_-$ from $h$ will not contribute at all, since their action on protonic states $\ket{jk}$ will give zero. Only the $(j_3^2-j(j+1)/3)$ term will affect the diagonal components of $h$. For a proton state $\ket{jk}$, where $k=-j,\dots,j$, the diagonal element $\bra{jk}h\ket{jk}$ will be:
\begin{align}
    h_{jk}\equiv\bra{jk}h\ket{jk}=\frac{1}{2}C_\beta\cos\gamma\bra{jk}\left(j_3^2-\frac{j(j+1)}{3}\right)\ket{jk}\ ,
\end{align}
which can be simplified to:
\begin{align}
    h_{jk}=\frac{1}{2}C_\beta\cos\gamma\left(k^2-\frac{j(j+1)}{3}\right)\ .
    \label{hdiag-equation}
\end{align}
These matrix elements will be functions of deformation parameters as well as the projection $k$ of the particle's angular momentum onto the $3$-axis. Since $j_3$ is applied twice on $\ket{jk}$ state, the final result will be $k^2$, so only the positive values can be considered for the numerical application $k=1/2,\dots,j$. The qualitative evolution of $h_{jk}$ as function of $\beta_2$ and $\gamma$ is studied for a nucleus with $A=167$ in which the odd-particle is either a proton in the $h_{11/2}$-shell or in the $i_{13/2}$-shell. The behavior of $h_{jk}$ w.r.t. the quadrupole deformation parameter can be seen in Fig. \ref{hdiag-beta-evolution}, while Fig. \ref{hdiag-gamma-evolution} shows the behavior of $h_{jk}$ w.r.t. the triaxiality parameter $\gamma$.
\begin{figure}
    \centering
    \includegraphics[scale=0.7]{Chapters/Figures/singleParticle-hdiag-1.pdf}
    \includegraphics[scale=0.7]{Chapters/Figures/singleParticle-hdiag-2.pdf}
    \caption{The evolution with quadrupole deformation parameter $\beta_2$ for the diagonal matrix elements of $h_{jk}$ defined in Eq. \ref{hdiag-equation} (proton on a defined $j$-shell), at a fixed triaxiality parameter $\gamma$. See text for details.}
    \label{hdiag-beta-evolution}
\end{figure}
\begin{figure}
    \centering
    \includegraphics[scale=0.7]{Chapters/Figures/hdiag-gamma-1.pdf}
    \includegraphics[scale=0.7]{Chapters/Figures/hdiag-gamma-2.pdf}
    \caption{The evolution with triaxiality parameter $\gamma$ for the diagonal matrix elements of $h_{jk}$ defined in Eq. \ref{hdiag-equation} (proton on a defined $j$-shell), at a fixed deformation parameter $\beta_2$. See text for details.}
    \label{hdiag-gamma-evolution}
\end{figure}

Regarding this numerical application, one can observe the fact that since only the cosine function is present in the diagonal components of $h$, the single-particle energies are independent on the sign of $\gamma$ (since cosine is an even function). However, this is not the case when calculations are done for the non-diagonal elements, since states with different $k$ will add contributions to $h$ via the sine function. For the evolution of $h_{jk}$ wr.t. $\gamma$, one can see that the lines in Fig. \ref{hdiag-beta-evolution} (especially ones from the right inset) are almost similar, indicating that the diagonal components are not very sensitive to nuclear triaxiality. On the other hand, a difference in $\beta_2$ between the components will result in large gaps.

Going back to the general Hamiltonian given in Eq. \ref{triaxial-prm-general-hamiltonian} and putting together Eqs. \ref{quadrupole-deformed-potential-beta}-\ref{single-particle-nilsson-defored-potential}, it can be brought to a `final' form. With this, rotor part, the single-particle contributions, and the quadrupole deformation of the mean-field are clearly depicted \cite{ring2004nuclear}:
\begin{align}
    % \hat{H}=\sum_i\frac{R_i^2}{2\mathcal{I}_i}+h_0^j+\kappa\beta r^2\left[\cos\gamma Y_2^0+\frac{\sin\gamma}{\sqrt{2}}\left(Y_2^2+Y_2^{-2}\right)\right]\ ,\\ 
    \hat{H}=\sum_i\frac{R_i^2}{2\mathcal{I}_i}+h_0^j+\frac{V}{j(j+1)}\left[\cos\gamma(3j_z^2-\mathbf{j}^2)-\sqrt{3}\sin\gamma(j_x^2-j_y^2)\right]\ .
    \label{eq-triaxial-prm-full-hamiltonian}
\end{align}

Regarding the dynamical character of $\hat{H}$ defined in Eq. \ref{eq-triaxial-prm-full-hamiltonian}, the coupling of high-$j$ particle with a rotor will create physical effects which can be classified in the following way \cite{ring2004nuclear}:
\begin{itemize}
    \item rotational energy is always minimized by the rotation of the core around the axis with the largest MOI
    \item alignment of the odd particle will always prefer a maximal mass overlap with the core (this configuration minimizes the potential energy the most)
    \item the Coriolis effect will always tend to align the particle with the core, in terms of their a.m.
\end{itemize}

Concluding this section, by using the triaxial PRM one can describe the energies of a an odd-$A$ triaxial nucleus, and moreover, as it will be shown, quantities such as transition probabilities (electric quadrupole type) and quadrupole moments can be calculated. It is a starting ground for multiple theoretical descriptions which aim to verify, confirm, and even predict phenomena specific to highly deformed shapes.

\section{Stable Triaxial Deformation}
\label{potential-energy-surfaces-section}

As it was mentioned, across the chart of nuclides, stable isotopes do exist in an \emph{equilibrium} at which no symmetries are involved with regards to their shape. Quantitatively, the stability of a triaxial nucleus is related to the presence (existence) of minima within the \emph{Potential Energy Surface}. This energy surface characterizes the shape of the nuclear surface with respect to the deformation parameters $\beta$ and $\gamma$. One can speculate on the fact that for an ellipsoidal shape, the total energy at a fixed angular momentum, function of the deformation will be given as \cite{ring2004nuclear}:
\begin{align}
    E(\beta,\gamma,I)=E_\text{LDM}(\beta,\gamma,I)+E_\text{shell}(\beta,\gamma,I)-\bar{E}_\text{corr}(\beta,\gamma,I)\ ,
    \label{energy-surface-correction-terms}
\end{align}
where $E_\text{LDM}$ is the deformation energy for a rotating ellipsoid (characterized by the rigid-like MOI, since at high spin values this approximation holds true), $E_\text{shell}$ is the energy within the shell model (can even use the Nilsson's deformed variant), and $\bar{E}_\text{corr}$ is usually calculated as an \emph{averaged value} of the Nilsson-Strutinsky corrected potential \cite{brack1972funny} (which is beyond the scope of this work). Finding these energies can be done by working at constant $\omega$ or at constant I, when diagonalizing the deformed-single particle potential in the rotating frame \cite{ring2004nuclear}.

Solving thus Eq. \ref{energy-surface-correction-terms} results in having a \emph{qualitative} analysis for the behavior of the nucleonic matter at high spins. These `solutions' will consist of graphical representations in the $(\beta,\gamma)$-plane, where values with different magnitudes for the potential energy appear. The lower these values are, the `more stable' the deformation is. Typically, one can find regions of stability across the plane with well-defined values for $\beta$ and $\gamma$ at which nucleus is stable and deformed. On the other hand, if the energy variation is rather flat in the $\gamma$ degree of freedom, but centered around a finite $\beta$, then this nucleus is called \emph{$\gamma$-unstable}. These energy surfaces are a key indicator of stability with respect to the deformation parameters. Examples of energy surfaces for some nuclei are shown in Figs. \ref{pes-example-set-1} - \ref{pes-example-set-2}. It is remarkable the fact that small quantum fluctuations will exist around the minima.
In order to explain the theoretical implications of Figs. \ref{pes-example-set-1} - \ref{pes-example-set-2}, one needs to understand that based on a MHO, a so-called \emph{Ultimate Cranker} (UC) numerical implementation has been developed \cite{bengtsson1989method,bengtsson1990high} (a Nilsson deformed potential in which pairing interaction is taken into account). This implementation is able to give the values for $(\beta,\gamma)$ at which a nucleus might achieve stability, practically identifying local and global minima within the potential energy surface. In a future section, multiple isotopes with known stable triaxiality will be classified in terms of the deformation parameters, creating thus an `inventory' with these deformed nuclei.

\begin{figure}
    \centering
    \includegraphics[width=0.49\textwidth]{Chapters/Figures/174Hf-PES.pdf}
    \includegraphics[width=0.49\textwidth]{Chapters/Figures/even-A-PES.pdf}
    \caption{\textbf{Left:} Potential energy surface for $^{174}$Hf, evaluated at a constant rotational frequency. Positive parity and a signature $\alpha=0$ were considered. Four minima that could be attributed to known configurations are identified which are marked with red colored dots (normal deformation) and blue dots (strong deformation). This figure was adapted from Ref. \cite{djongolov2003extending}. \textbf{Right:} PES for some even-even nuclei. Note the rather shallow triaxial minima for all nuclei, and the $\gamma$-soft nature of all the nuclei except for $^{106}$Pd, whose minimum is achieved at a null triaxiality parameter. This figure was taken from Ref. \cite{nomura2021examining}.}
    \label{pes-example-set-1}
\end{figure}

\begin{figure}
    \centering
    \includegraphics[scale=0.17]{Chapters/Figures/135Pr-PES.pdf}
    \includegraphics[scale=0.38]{Chapters/Figures/163Lu-PES.pdf}
    \caption{\textbf{Top:} PES for the odd-$A$ nucleus $^{135}$Pr, with units of MeV. The figure is taken from the work of Frauendorf et al. \cite{frauendorf2014transverse}. \textbf{Bottom:} PES for the odd-$A$ isotope $^{163}$Lu evaluated at a fixed spin $I=53/2$ with positive parity and positive signature. The red dot corresponds to a global minimum for normal deformed structure, while the blue diamonds represent two additional minima which correspond to strong deformation. This figure is adapted from Ref. \cite{jensen2002wobbling}.}
    \label{pes-example-set-2}
\end{figure}

Similarly, PES calculations for other nuclei will also point out the stable deformations (with triaxial character), extracting the deformation parameters. For instance, within the $A\approx160$ mass region there are multiple nuclei in which TSD bands are identified. Best example are the results concerning Lu-Hf isotopes, where more than 20 such bands were experimentally confirmed \cite{gu2007theoretical}. It is interesting that fact that the nuclei with even $N$ (especially the Lu isotopes) have ground state TSD bands which emerge from a configuration based on the $\pi=i_{13/2}$ nucleon. In fact, Ødegård et al. \cite{odegaard2001evidence} point out the fact that nuclei with proton number $Z\approx 71$ and neutron number $N\approx 94$ have stable triaxial shapes with large quadrupole deformation around the values $(\beta_2,\gamma)=(0.38,\pm 20^\circ)$, while the lowest energy state corresponding to the odd nucleon is the $i_{13/2}$ proton \cite{schnack1995superdeformed}.

Given the degree of shell filling, only some specific triaxial shapes with fixed value of $\gamma$ are favored by the aligned particle (see Ref. \cite{hamamoto1983intrinsic} for more details). In the case of $^{163}$Lu, the aligned proton from $j=13/2$ shell favors a triaxiality of $\gamma=20^\circ$ (see Fig. \ref{pes-example-set-2}). Similarly, other nuclei will have different high-$j$ nucleon configurations which will drive the system to large triaxial deformations, by favoring a particular value of $\gamma$.

Now that the stability of asymmetric nuclear shapes has been visualized in terms of energy surfaces and deformation parameters (whose magnitudes are more or less given by the favoring of the single-particle configuration), it is instructive to understand how one can identify triaxial nuclei (both experimentally and theoretically). This will be done in the following section.

\section{Fingerprints of Triaxiality}

Within recent years, it has been shown that triaxiality plays an important role on features such as:
\begin{itemize}
    \item Calculating nucleon separation energies \cite{moller2006global}
    \item Protonic emission probabilities \cite{qi2019recent,budaca2022deformation}
    \item Determination of fission barrier height \cite{moller2009heavy,lu2012potential}
    \item Nuclear fragmentation \cite{palumbo1985splitting}
\end{itemize}

Unfortunately, triaxiality is still quite challenging to measure it directly \cite{hamamoto2016interplay,budaca2018tilted}. On the experimental side, a tremendous effort was made in order to construct setups that can study nuclei at very high spins (were collective phenomena previously discussed appear). Over the last 25 years, with the advance in technology of detectors \cite{henning2012stability}, some large facilities were built, with the sole purpose of studying high-spin nuclei with great degree of precision on the spectroscopic measurements that are performed. Facilities such as the EUROBALL \cite{simpson1997euroball} or GAMMASPHERE \cite{lee1990gammasphere} opened \emph{new frontiers} on the nuclear physics at high-spin; being able to measure excited spectra of nuclei having angular momentum within the range $60-80\hbar$. Regarding the reactions involved, most of the measurements of highly excited (and rapidly rotating) are obtained through \emph{Fusion-Evaporation Reactions} in which heavy-ion collisions take place. Since this work does not focus on the experimental setups and procedures, only some literature covering these reactions are mentioned: \cite{gu2007theoretical,henning2012stability,ayangeakaa2013exotic,matta2017exotic,das2018nuclear,lewis2019lifetime,sensharma2021wobbling}.

Even though stable triaxiality is an elusive phenomenon, \emph{two clear fingerprints} are known to pinpoint asymmetric nuclear shapes: \textbf{wobbling motion} and \textbf{chiral symmetry breaking}. There has been an extensive study of these two phenomena, and both indicate a clear correspondence between their physical significance and the lack of symmetry for the underlying nuclei. Since wobbling is the main focus of this work, a brief introduction for chiral symmetry breaking will be made below, while a separated chapter will be devoted to the theoretical and experimental evidence regarding wobbling motion. Moreover, all the theoretical results obtained within the team's framework will be presented.

\subsection{Chiral motion}
\label{chiral-section}

Firstly introduced by Frauendorf \cite{frauendorf1997tilted}, it concerns nuclei in which the nucleonic configurations lead to a system that lacks \emph{chiral symmetry}, meaning that the \emph{left-handed} version is not identical with the \emph{right-handed} one. The two systems can be transformed into each other via the chiral operator which combines a rotation with the time-reversal operator: $\chi_\text{chiral}=\mathcal{T}\mathcal{R}(\pi)$. The left/right handedness of the nuclear system comes from the coupling of three different angular momenta, typically a valence proton, a valence neutron, and a rotational core. As such, the chiral symmetry breaking is expected to appear in odd-odd nuclei.

In that pioneering work, Frauendorf showed that for a triaxial nucleus, the collective angular momentum $\mathbf{R}$ will align along the intermediate ($m$) axis of the deformed ellipsoid (note that the $m$-axis is also the axis with largest MOI). The energy is minimized when the triaxial system rotates about the largest MOI, and for irrotational-like flow this corresponds to the $m$-axis. Regarding the valence nucleons (particles or holes), the location of their corresponding Fermi level will dictate the alignment with the ellipsoid \cite{frauendorf1997tilted,starosta2001chiral}. For a valence proton with the energy located in the bottom part of the high-$j$ subshell, it will align its a.m. with the short $s$-axis of the ellipsoid. Furthermore, a valence neutron located in the upper part of the high-$j$ shell will align its a.m. with the long $l$-axis. Reasoning behind this type of alignment has been discussed already and it is related to the overlap between the quasi-particle wave-function and that of the triaxial core. For the (particles + core) coupling, a \emph{maximal overlap of their wave-functions will minimize the interaction energy}, hence the particle will align its a.m. with the short axis. On the other hand, for the (hole + core) coupling, \emph{the energy is minimized for a minimal overlap of the wave-functions}, meaning that the hole will tend to align its a.m. with the long axis.

When the chiral symmetry is broken in the intrinsic frame of reference (body-fixed frame), the `restoration' of the symmetry (through quantum tunnelling \cite{zhang2007chiral}) in the laboratory frame will manifest itself as the emergence of two almost degenerate doublet bands with $\Delta I=1\hbar$ \cite{frauendorf1997tilted}. This interesting behavior has been drawing a lot of attention lately, with most of the experimental identifications around the transitional nuclei $A\approx 130$. Great research towards a complete spectroscopy of some chiral nuclei has been done in Refs. \cite{starosta2001chiral,meng2008chiral,budaca2018tilted,budaca2018semiclassical,budaca2021chiral}. Examples of experimental chiral doublets are shown in Figs. \ref{chiral-bands-1} - \ref{chiral-bands-2} for some odd-odd nuclei.

\begin{figure}
    \centering
    \includegraphics[width=0.99\textwidth]{Chapters/Figures/Chiral_136Pm.pdf}
    \caption{The energy spectra for the odd-odd $^{136}$Pm isotope, in which the chiral doublet built on the configuration $\pi h_{11/2}\otimes\nu h_{11/2}$ appears. Experimental data is from Ref. \cite{mccutchan2018nuclear} and the level scheme is adapted from Ref. \cite{bhat1992evaluated}.}
    \label{chiral-bands-1}
\end{figure}

\begin{figure}
    \centering
    \includegraphics[scale=0.7]{Chapters/Figures/chiral_bands_132La.pdf}
    \caption{The energy spectra for the odd-odd $^{132}$La isotope, in which the chiral doublet built on the configuration $\pi h_{11/2}\otimes\nu h_{11/2}$ appears, with each double being properly colored in red and blue, respectively. Experimental data is taken from from Refs. \cite{grodner2004dsam,grodner2005lifetime}.}
    \label{chiral-bands-2}
\end{figure}

The geometrical interpretation of the left-handed and right-handed systems can be seen in Fig. \ref{chiral-geometry}, where the ellipsoid is `surrounded' by the two valence nucleons, each with its aligned a.m. w.r.t. principal axes. Regarding axes labelling, it is worth mentioning that for the principal axes of the triaxial ellipsoid, when a positive triaxiality parameter ($\gamma\in[0^\circ,60^\circ]$) is employed, the long axis will become the $3$-axis, the short axis will become the $1$-axis, and the intermediate axis will become the $2$-axis. Such an association for the principal axes is fully consistent with the one from Ref. \cite{frauendorf2001spontaneous} (i.e., see Table I therein).

\begin{figure}
    \centering
    \includegraphics[width=0.99\textwidth]{Chapters/Figures/chiral_handedness.pdf}
    \caption{Left- and right-handed chiral systems for a triaxial odd-odd nucleus, indicating the mutually perpendicular angular momentum vectors. The two valence nucleons with their orbits are colored with blue (proton) and magenta (neutron). These figures were inspired from \cite{starosta2001chiral}. Considering the discussion regarding alignments, the proton a.m. $\mathbf{j}_\pi$ is aligned with the $s$-axis, the neutron a.m. $\mathbf{j}_\nu$ with the $l$-axis, and the even-even core a.m. $\mathbf{R}$ with the $m$-axis of the ellipsoid. The alignments for the core and the particles are described in text.}
    \label{chiral-geometry}
\end{figure}

\subsection{Wobbling Motion}

The second fingerprint of triaxiality is the `core' topic of this work, and it will be discussed in detail in the next chapter, where some theoretical ideas will be presented together with the latest experimental results. A great amount of progress towards understanding nuclear phenomena at high spin has been made by the scientific community which devoted research to this particular topic.

Going back to the discussion about the triaxial shapes and their asymmetry between the MOI, this key feature leads to the possibility of defining a rotation about any of the three axis. Although the main rotational motion will be around the axis with the largest MOI (since this is energetically the cheapest \cite{bohr1954rotational}), the other two axes will contribute to a `total' (superimposed) rotation such that the angular momentum $\mathbf{I}$ of the system will precess around a steady position. The precessional motion has oscillator-like behavior, meaning that $\mathbf{I}$ will not only precess, but its projection around the principal axes will oscillate around a steady position.
The easiest way of understanding wobbling motion is a classical analog such as the \emph{spinning asymmetric top}, but for a nucleus the spinning motion is \emph{quantized} and a typical phonon number will be attributed to the system.

As it will be shown in the following chapter, treating the wobbling phenomenon in a `classical' will lead to very good results concerning the physical quantities of interest. Moreover, there are many attempts (albeit quantum mechanical or semi-classical approximations) at describing wobbling motion in which the parameters used to describe the Hamiltonian have clear analogies with quantities that describe classical systems, keeping thus a close contact with well-defined concepts regarding system dynamics.

Other classical analogies for the nuclear wobbling motion can be seen in celestial bodies such as planets. Indeed, Earth has a precession given by its rotation and a small amplitude polar motion. As a result, the Earth's `angular momentum' will oscillate with respect to one of the body's fixed axis.
