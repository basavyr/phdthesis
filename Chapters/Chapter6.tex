\chapter{A New Theoretical Formalism}

In this chapter, a formalism that describes the wobbling properties in odd-$A$ nuclei will be presented. The model was developed recently by the current team (i.e., Raduta and Poenaru) and applied to $^{135}$Pr \cite{raduta2020new}, and the `Lu family' with $^{161,165,167}$Lu \cite{raduta2020approach} and $^{163}$Lu \cite{raduta2020approach,raduta2020towards,poenaru2021parity,poenaru2021extensive1,poenaru2021extensive2}. This framework is an original contribution to the field of nuclear structure, with focus on the theoretical aspects of collective phenomena in nuclei.

\section{Previous Work - Foundation}
\label{foundation}

From a development standpoint, it is instructive to review the previous `stages' that lead to the current work, since the description of odd-$A$ nuclei achieved in here is strongly connected with former calculations done by the team. Consequently, in this section a brief overview of the wobbling motion in $^{163}$Lu done by Raduta et al. in 2017 \cite{raduta2017semiclassical} will be given. Indeed, by using a \emph{semi-classical} approach, the wobbling properties of this odd-mass isotope were accurately described.

The semi-classical approaches tend to be very useful when applied to problems having a \emph{quantal} Hamiltonian that contains quantities which behave as in the \emph{classical limit} when certain constraints or approximations are made. Moreover, these methods always keep the dynamics of the system in close contact with the classical features, which in principle are easier to interpret (i.e., a clear physical meaning). For this reason, the collective properties of $^{163}$Lu were characterized by such a method. Dealing with an odd nucleus it makes sense to adopt a Particle + Rotor Model in which the odd quasi-particle couples to an even-even core. From the total Hamiltonian $H=H_\text{rot}+H_\text{sp}$ (recall discussion on the PRM model \ref{triaxial-prm-general-hamiltonian} and also QTR Hamiltonian \ref{oddA-QTR-general-hamiltonian}), one obtained the energy spectrum for the four wobbling bands in this nucleus. Remarking the fact that there is another debate regarding the `true' nature of the fourth triaxial strongly-deformed band. For example, Jensen et al. \cite{jensen2004coexisting} suspect that this band is built from single-particle excitations, meaning that the states to not show wobbling mechanism. However, Tanabe et al. \cite{tanabe2008selection} is in favor of attributing $n_w=3$ for TSD4. More details on the interpretation of TSD4 will be discussed in the following sections.

After the initial Hamiltonian problem was established, classical equations of motions which fully describe the motion of $\mathcal{Q}$ and $\mathscr{C}$ (recall notations from Table \ref{notation-table-wobbling} that will still be used throughout the remaining chapters) are obtained through the Variational Principle (VP) by following a textbook procedure. The procedure of applying VP translates to the \emph{dequantization} procedure of an initial quantal Hamiltonian, thus making the change from a quantum space $S_\text{qt}$ to a classical space $S_\text{cls}$. This kind of transition $S_\text{qt}\to S_\text{cls}$ was also used previously for describing collective phenomena \cite{raduta2007semiclassical,chen2016two,budaca2018tilted}. For keeping a short overview of the foundational stage, the VP method applied to $H$ is skipped for now, but it will be detailed in a subsequent section. Furthermore, the classical equations of motions were solved with several restrictions, and the energy spectrum was analytically obtained (see Eq. 3.23 from the same reference). The excited bands TSD2, TSD3, and TSD4 were constructed via a phonon operator (denoted by $\Gamma_1^\dagger$ in Eq. 3.43 from \cite{raduta2017semiclassical}, which acts every state $I$). This approach makes it possible to have $n_w=3$ phonon excitations that are obtained by applying the phonon operator three times on the spin states belonging to the yrast band TSD1. In fact, since the states from TSD4 have negative parity, they are obtained by applying twice $\Gamma_1^\dagger(\pi=+1)$ and once $\Gamma_1^\dagger(\pi=-1)$, where the former phonon has positive parity and the latter has a negative parity. A schematic representation with the mechanism of action for the phonon operator from this formalism is shown in Fig. \ref{phonon-operator-schematic}.
\begin{figure}
    \centering
    \includegraphics[width=0.75\textwidth]{Chapters/Figures/w0_phonon_operator.pdf}
    \caption{The mechanism of action for the phonon operator introduced in Ref. \cite{raduta2017semiclassical} for creating excited states of a given angular momentum, from the ground state bands. The core's angular momentum is defined as $\mathbf{R}_\mathscr{C}$, while the quasi-particle a.m. is denoted by $\mathbf{j}_\mathcal{Q}$. The ground-state band (yrast) emerges from the coupling of the quasi-particle with the triaxial core, giving rise to a series of spins $I_i=R_i+j$. By acting with $\Gamma_1^\dagger$ $n$-times on any of these states, then an excited level is obtained in a $n_w=n$ wobbling band. Keep in mind that one phonon operator increases the spin of a state by one unit.}
    \label{phonon-operator-schematic}
\end{figure}


From the seminal work from 2017 of the team, several features are emphasized:
\begin{itemize}
    \item The four TSD bands are considered as zero-, one-, two-, and three-wobbling phonon bands for TSD1, TSD2, TSD3, and TSD4, respectively
    \item Each excited band is obtained by acting on the yrast (TSD1) band with one-, two-, and three-phonons, respectively (e.g., a state $I$ from TSD2 is obtained by acting with the wobbling-phonon operator on a state $I-1$ from TSD1 and so on), according to Fig. \ref{phonon-operator-schematic}
    \item Wobbling structure for the group TSD1-2-3 emerged from a proton $i_{13/2}$ ($\mathcal{Q}_p$) where all spin states have positive parity
    \item The band TSD4 has spin states with negative parity, and it is built on a proton from the $h_{9/2}$ orbital (also a $\mathcal{Q}_p$)
    \item In the expression of the rotor Hamiltonian, the rigid-like MOI were adopted (see Eq. \ref{eq-irrotational-rigid-mois}) which depend on $\gamma$ and $\mathcal{I}_0$
    \item Analytical expressions for the energies were expressed in terms of total spin and wobbling phonon numbers
    \item Experimental data was reproduced through a fitting procedure, with the free parameters $\mathcal{I}_0^{-1}$ (rotor part) and a \emph{scaling factor} $s=V\cdot \mathcal{I}_0$ (single-particle part)
    \item The model assumes similar MOI across all four bands
    \item Deformation parameters $\beta_2$ and $\gamma$ were a priori fixed (taken from literature)
\end{itemize}

A year later, the team also extended this method into analyzing the wobbling properties of two more isotopes: $^{165,167}$Lu \cite{raduta2018wobbling}, and the model showed again that it was an useful tool in providing a realistic description of the wobbling motion in odd-mass nuclei. Hereafter, when referring to the procedure realized by Raduta and Poenaru in \cite{raduta2017semiclassical} and \cite{raduta2018wobbling}, the term $\mathbf{W_0}$ will be used. Throughout the following chapters, comparisons between the newly developed and improved theory and $\mathbf{W_0}$ will be made when necessary, in order to understand several key differences.

\section{Re-interpretation of The Wobbling Motion}

The $\mathbf{W_0}$ formalism can be regarded as a cornerstone in the description of an odd-$A$ particle-triaxial-rotor system done by the team. In a more recent follow-up work (2020) done by Raduta and Poenaru \cite{raduta2020approach,raduta2020towards}, a new interpretation of the wobbling bands was made, relative not only to the odd-$A$ $^{163}$Lu, but to an entire set of wobblers. The way of obtaining yrast and excited states within the collective spectrum turned out to be a first within literature, especially for $^{163}$Lu and $^{135}$Pr, since these nuclei have been drawing a lot of attention lately. The model starts with the typical QTR Hamiltonian:
\begin{align}
    \hat{H}=\hat{H}_\text{rot}+\hat{H}_\text{sp}\ ,
    \label{total-ham-approach-w1}
\end{align}
where $\hat{H}_\text{sp}$ represents the quasi-particle that moves inside the quadrupole mean-field as described in Eqs. \ref{eq-nilsson-ham-spherical-harmonics} and \ref{single-particle-nilsson-defored-potential} (recall discussion on the $\gamma$-deformed Nilsson potential and also Sections \ref{trm-model} - \ref{tprm-model}):
\begin{align}
    \hat{H}_\text{sp}=\epsilon_j+\frac{V}{j(j+1)}\left[\cos\gamma\left(3j_3^3-\mathbf{j}_\mathcal{Q}^2\right)-\sqrt{3}\sin\gamma\left(j_1^2-j_2^2\right)\right]\ ,
    \label{single-particle-ham-approach-w1}
\end{align}
with $\epsilon_j$ is the intrinsic energy of the particle from the corresponding $j$-shell (as it was discussed in Section \ref{tprm-model}). The total angular momentum of the odd-particle + core system is $\mathbf{I}=\mathbf{R}_\mathscr{C}+\mathbf{j}_{\mathcal{Q}_p}$. The components of the total angular momentum are: $$\mathbf{I}=\{\hat{I}_1,\hat{I}_2,\hat{I}_3\}\ ,$$ while the a.m. components for the $\mathcal{Q}_p$ are given as $$\mathbf{j}_{\mathcal{Q}_p}=\{\hat{j}_1,\hat{j}_2,\hat{j}_3\}\ .$$ The axes labelling for the triaxial ellipsoid is $(1,2,3)$. Knowing this, one can express the rotor part as:
\begin{align}
    \hat{H}_\text{rot}=\sum_{k=1}^3A_k(\hat{I}_k-\hat{j}_k)^2\ ,
    \label{rotor-ham-approach-w1}
\end{align}
where the inertial factors $A_k$ are expressed in terms of the three MOI:$$A_k=\frac{1}{2\mathcal{I}_k}\ .$$
Note that the Hamiltonian from Eq. \ref{rotor-ham-approach-w1} is just as the one expressed in Eq. \ref{general-rotor-hamiltonian}, except that here, the components of $\mathbf{R}_\mathscr{C}$ are given in terms of $\mathbf{I}$ and $\mathbf{j}_{\mathcal{Q}_p}$.

Obviously, the next task would be to obtained the eigenvalues of the Hamiltonian, which would comprise the energies of the system. One can proceed with the diagonalization procedure of $\hat{H}$ using a set of states that manifest the invariance to rotation by $\pi$ around a particular axis (the $D_2$ symmetry), since the Hamiltonian exhibits such a property. However, it is more practical to describe the system only though a few variables that have a classical counterpart. By doing this, the system dynamics will be close to the classical analogy of a rotating body lacking axial symmetry. The semi-classical approach that best fits this need os the Time-Dependent Variational Principle (TDVP), to which one associates the Time-Dependent Variational Equation (TDVE). When applying this Variational Principle on an initial problem, the requirement is to have a variational state which is constructed carefully, such that it embeds the necessary degrees of freedom of the underlying physics. Furthermore, VP will provide the time dependence of the variables that comprise the variational state itself \cite{budaca2018tilted}. Additionally, for each variable that parametrizes the state (usually the variables are complex), the TDVE will yield its equation of motion, thus finding the `connection' between the initial quantal variable (belonging to $S_\text{qt})$ and the classical variable (belonging to $S_\text{cls}$).

\subsection{Variational Principle}

The discussion regarding the TDVP and TDVE from above can therefore be summarized in the following equation, which must be associated to the quantum Hamiltonian ($\hat{H}\subset S_\text{qt}$) defined in Eq. \ref{total-ham-approach-w1}:
\begin{align}
    \delta\int_0^t\bra{\Psi_{IjM}}\hat{H}-i\frac{\partial}{\partial t'}\ket{\Psi_{IjM}}dt'=0\ .
    \label{tdve-approach-w1}
\end{align}
Obviously, the variational state $\ket{\Psi_{IjM}}\equiv\Psi_\text{trial}$ (also known as the \emph{trial function}) must be chosen in such a way that it comprises the entire space $S_\text{qt}$ of the original quantal Hamiltonian. This can be achieved if $\Psi_\text{trial}$ is a \emph{coherent state} \cite{glauber1963coherent}, which due to its \emph{completeness} property will span all the basis vector states from $S_\text{qt}$. Keep in mind that for $S_\text{qt}$ the states belong to the Hilbert space of $\hat{H}$. For the present case, the trial function is defined as:
\begin{align}
    \Psi_\text{trial}\equiv\ket{\Psi_{IjM}}=\mathcal{N}e^{z\hat{I}_-}e^{s\hat{j}_-}\ket{IMI}\ket{jj}\ ,
    \label{trial-function-appeoach-w1}
\end{align}
where the ladder operators for the total and single-particle a.m. are represented by $\hat{I}_-$ and $\hat{j}_-$, respectively. The factor $\mathcal{N}$ is the normalization constant, which keeps the trial function normalized to unity. Its value is given by \cite{raduta2007semiclassical,raduta2020new}:
\begin{align}
    \mathcal{N}=\left(1+|z|^2\right)^{-2I}\left(1+|s|^2\right)^{-2j}\ .
\end{align}
The states that are involved in $\Psi_\text{trial}$ are as follow: $\ket{IMK}$ represent the Wigner-D functions (i.e., eigenstates of $\hat{I}^2$ and $\hat{I}_3$), while $\ket{j\Omega}$ are the wave-functions of the odd quasi-particle $\mathcal{Q}$ (the eigenstates of $\hat{j}^2$ and $\hat{j}_-$). Notice that for both cases, the trial function contains their extremal form, namely the states of maximum projections $(K=I,\Omega=j)$. For $\ket{IMK}$, one has:
\begin{align}
    \ket{IMK}&=\sqrt{\frac{2I+1}{8\pi^2}}\mathcal{D}_{MK}^I\ , \nonumber\\
    \ket{IM-K}&=\sqrt{\frac{2I+1}{8\pi^2}}\mathcal{D}_{M-K}^I\ .
    \label{IMK-wigner-functions}
\end{align}
The second equation denotes the \emph{time-reversed} states, which are obtained by changing the projection $K$ to $-K$. The single-particle states can be defined more generally as:
\begin{align}
    \ket{\chi}&=\sum_{j\Omega}c_{j\Omega}\ket{\chi_{j\Omega}}=\sum_{j\Omega}c_{j\Omega}\ket{j\Omega}\nonumber\\
    \ket{\bar{\chi}}=&\sum_{j\Omega}c_{j-\Omega}\ket{\bar{\chi}_{j\Omega}}=\sum_{j\Omega}c_{j-\Omega}\ket{j-\Omega}=\sum_{j\Omega}(-1)^{j-1/2}c_{j\Omega}\ket{j-\Omega}\ .
    \label{j-Omega-single-particle-states}
\end{align}
The wave-functions are expressed in terms of the projections $\Omega=-j,\dots,j$. The states $\ket{\bar{\chi}_{j\Omega}}$ are the time-reversed ones and they are degenerate with $\ket{\chi_{j\Omega}}$. Because of the $D_2$ symmetry for $\hat{H}$, the total wave-function can be formulated as the a sum over the components $K, \Omega$ and over $j$. Denoting the total wave-function of the odd-mass nucleus as $\ket{\Psi}_\text{odd}$ (not to be confused with the trial function $\Psi_\text{trial}$), it must comprise both the intrinsic ($\mathbf{I}=\mathbf{R}_\mathscr{C}+\mathbf{j}_\mathcal{Q}$) and the single-particle ($\mathbf{j}_\mathcal{Q}$) degrees of freedom, namely:
\begin{align}
    \ket{\Psi_\text{odd}}&=\ket{\Psi_\text{intr.}}\otimes\ket{\Psi_\mathcal{Q}}=\nonumber\\
    &=\sqrt{\frac{2I+1}{16\pi^2}}\sum_K C_K\sum_{j,\Omega}c_{j\Omega}\left[\mathcal{D}_{MK}^I\ket{\chi_{j\Omega}}+(-1)^{I-j}\mathcal{D}_{M-K}^I\ket{\bar{\chi}_{j\Omega}}\right]=\nonumber\\
    &=\sqrt{\frac{2I+1}{16\pi^2}}\sum_K C_K \left[\mathcal{D}_{MK}^I\ket{\chi}+(-1)^{I-1/2}\mathcal{D}_{M-K}^I\ket{\bar{\chi}}\right]\ ,
    \label{total-wavefunction-oddA-general}
\end{align}
where the entire summation must be evaluated under the restriction $(K-\Omega)=even$. Such a restraint gives the allowed set of values for both $K$ and $\Omega$ to be:
\begin{align}
    \dots,-\frac{11}{2},-\frac{7}{2},-\frac{3}{2},\frac{1}{2},\frac{5}{2},\frac{9}{2},\frac{13}{2},\dots
\end{align}

These states are from the quantal space $S_\text{qt}$, belonging to the Hamiltonian from Eq. \ref{total-ham-approach-w1}. Remarking the fact that for a known quasi-particle $\mathcal{Q}$ from a specific $j$-shell, one must keep $j$ constant within the summations.

\subsection{Classical Equations of Motion}
\label{equations-of-motion-section}

Returning to the trial function $\ket{\Psi_{IjM}}$ from Eq. \ref{trial-function-appeoach-w1}, defined as a coherent state with products between  $\ket{IMK}$ (Eq. \ref{IMK-wigner-functions}) and $\ket{j\Omega}$ (Eq. \ref{j-Omega-single-particle-states}), its expression must be further discussed in terms of $z$ and $s$. Indeed, these two variables represent complex functions of time, and they are associated with the motion of the core and the odd-particle, respectively. Their expressions are:
\begin{align}
    z=\rho e^{i\varphi}\ ,\ s=\sigma e^{i\psi}\ .
    \label{z-s-variables}
\end{align}
The next step would be to evaluate the averages of both $\hat{H}$ and the time derivative $\frac{\partial}{\partial t}$ on the variational state $\ket{\Psi_{IjM}}$ as:
\begin{align}
    \left\langle \hat{H} \right\rangle&=\bra{\Psi_{IjM}}\hat{H}\ket{\Psi_{IjM}}\nonumber\\
    \left\langle \frac{\partial}{\partial t} \right\rangle&=\bra{\Psi_{IjM}}\frac{\partial}{\partial t}\ket{\Psi_{IjM}}\ .
    \label{hamiltonian-average}
\end{align}
The expressions of these quantities were defined in Ref. \cite{raduta2017semiclassical} with respect to the variables $z$ and $s$. However, it would be more useful to change the form of $\rho$ and $\sigma$ in the following way:
\begin{align}
    \rho \to r&=\frac{2I}{1+\rho^2}\ ,\ 0\leq r\leq 2I\ ,\nonumber\\
    \sigma \to f&=\frac{2j}{1+\sigma^2}\ ,\ 0\leq f\leq 2j\ .
    \label{changed-rho-sigma-variables}
\end{align}
These two new variables keep the same correspondence, meaning that $r$ is related to the core and $f$ is related to the odd-particle. Moreover, by doing such a transformation, the TDVE (Eq. \ref{tdve-approach-w1}) will provide the equations of motion in a canonical form (unlike the initial $\rho$ and $\sigma$). In fact, this was the reason behind the change of variable. The set of equations of motion are:
\begin{align}
    \frac{\partial \mathcal{H}}{\partial r}=\dot{\varphi}\ ,\ \frac{\partial \mathcal{H}}{\partial \varphi}=-\dot{r}\ ,\nonumber\\
    \frac{\partial \mathcal{H}}{\partial f}=\dot{\psi}\ ,\ \frac{\partial \mathcal{H}}{\partial \psi}=-\dot{f}\ ,
    \label{eq-of-motion-approach-w1}
\end{align}
where $\mathcal{H}$ is just the average of $\hat{H}$ on the trial function, as defined in Eq. \ref{hamiltonian-average}. It now plays the role of \emph{classical energy function} (CEF) and moreover, it is a constant of motion, implying that the energy of the system must be conserved. Remarking the fact that the \emph{classical image} of the initial problem (i.e., the Hamiltonian of an odd-$A$ triaxial nucleus) is now properly reached, through the equations of motion (Eq. \ref{eq-of-motion-approach-w1}) in the canonical form. The explicit form of the equations of motion provided by VP are \cite{raduta2020approach}:
\begin{align}
    \dot{\varphi}=&\frac{2i-1}{I}(I-r)\left(A_1\cos^2\varphi+A_2\sin^2\varphi-A_3\right)-\nonumber\\
                 &-2\sqrt{\frac{f(2j-f)}{r(2I-r)}}(I-r)\left(A_1\cos\varphi\cos\psi+A_2\sin\varphi\sin\psi\right)+2A_3(j-f)\ ,\nonumber\\
    \dot{\psi}=&\frac{2j-1}{j}(j-f)\left(A_1\cos^2\psi+A_2\sin^2\psi-A_3\right)-\nonumber\\
               &-2\sqrt{\frac{r(2I-r)}{f(2j-f)}}(j-f)\left(A_1\cos\varphi\cos\psi+A_2\sin\varphi\sin\psi\right)+2A_3(I-r)-\nonumber\\
               &-V\frac{2j-1}{j^2(j+1)}(j-f)\sqrt{3}\left(\sqrt{3}\cos\gamma+\sin\gamma\cos2\psi\right)\ , \nonumber\\
    \label{eq-of-motion-explicit-coordinates}
\end{align}
for the canonical coordinates $(\varphi,\psi)$, and \cite{raduta2020approach}:
\begin{align}
    -\dot{r}=&\frac{2I-1}{2I}r(2I-r)(A_2-A_1)\sin2\varphi+\nonumber\\
             &+2\sqrt{rf(2I-r)(2j-f)}\left(A_1\sin\varphi\cos\psi-A_2\cos\varphi\sin\psi\right)\ ,\nonumber\\
    -\dot{f}=&\frac{2j-1}{2j}f(2j-1)(A_2-A_1)\sin2\psi+\nonumber+\\
             &+2\sqrt{rf(2I-r)(2j-f)}\left(A_1\cos\varphi\sin\psi-A_2\sin\varphi\cos\psi\right)+\nonumber\\
             &+V\frac{2j-1}{j^2(j+1)}f(2j-f)\sqrt{3}\sin\gamma\sin2\psi\ ,
    \label{eq-of-motion-explicit-momenta}
\end{align}
for the canonical momenta $(r,f)$, respectively.

Additionally, the \emph{classical coordinates} encompassed in $S_\text{cls}$ are the generalized momentum and generalized coordinates, which here are represented by $(r,f)$ and $(\varphi,\psi)$, respectively. Keep in mind that the two sets of equations of motion and canonical coordinates correspond to the core and the single-particle. Thus, the dequantization procedure was properly described, obtaining the classical dynamics of the system.

\subsection{Classical Energy Function (CEF)}

Regarding the structure of the CEF, its expression in terms of the canonical variables is given as:
\begin{align}
    \mathcal{H}&\equiv\bra{\Psi_{IjM}}\hat{H}\ket{\psi_{IjM}}=\nonumber\\
    &=\frac{I}{2}(A_1+A_2)+A_3I^2+\frac{2I-1}{2I}r(2I-r)\mathcal{A}_\varphi+\frac{j}{2}(A_1+A_2)+A_3j^2+\nonumber\\
    &+\frac{2j-1}{2j}f(2j-f)\mathcal{A}_\psi-2\sqrt{rf(2I-r)(2j-f)}\mathcal{A}_{\varphi\psi}+\nonumber\\
    &+A_3\left[r(2j-f)+f(2I-r)\right]-2A_3Ij+V\frac{2j-1}{j+1}\mathcal{A}_\gamma\ .
    \label{full-classical-energy-function}
\end{align}
Since the expression is quite lengthy, some \emph{canonical factors} were introduced in Eq. \ref{full-classical-energy-function}, namely $\mathcal{A}_\varphi$, $\mathcal{A}_\psi$, $\mathcal{A}_{\varphi\psi}$, and $\mathcal{A}_\gamma$. They depend on the canonical coordinates as follows:
\begin{align}
    \mathcal{A}_\varphi&=A_1\cos^2\varphi+A_2\sin^2\varphi-A_3\ ,\nonumber\\
    \mathcal{A}_\psi&=A_1\cos^2\psi+A_2\sin^2\psi-A_3\ ,\nonumber\\
    \mathcal{A}_{\varphi\psi}&=A_1\cos\varphi\cos\psi+A_2\sin\varphi\sin\psi\ ,\nonumber\\
    \mathcal{A}_\gamma&=\cos\gamma-\frac{f(2j-f)}{2j^2}\sqrt{3}\left(\sqrt{3}\cos\gamma+\sin\gamma\cos2\psi\right)\ .
    \label{classical-energy-function-A-factors}
\end{align}

\subsubsection{Canonical Factors - Qualitative Analysis}

It is worth analyzing the behavior of the canonical factors defined in Eq. \ref{classical-energy-function-A-factors}, since their evolution with respect to the generalized coordinates and MOI ordering will provide a better understanding regarding the behavior of the CES, as it will be shown in the following sections. Firstly, the factor $\mathcal{A}_\varphi$ is graphically represented for different orderings of the inertia parameter $A_k$. This is shown in Fig. \ref{fig-A-varphi-canonical}. Because both $\mathcal{A}_\varphi$ and $\mathcal{A}_\psi$ have similar expressions (only the coordinate is changed), the plots are equivalent.
\begin{figure}
    \centering
    \includegraphics[width=0.49\textwidth]{Chapters/Figures/A_varphi_1.pdf}
    \includegraphics[width=0.49\textwidth]{Chapters/Figures/A_varphi_2.pdf}
    \caption{The evolution of $\mathcal{A}_\varphi$ with respect to the generalized coordinate $\varphi$, at different MOI orderings. The values $1,3,7$ were chosen and they are interchanged between the three inertia factors. This figure is equivalent for $\mathcal{A}_\psi$.}
    \label{fig-A-varphi-canonical}
\end{figure}

For the mixed term $\mathcal{A}_{\varphi,psi}$ from Eq. \ref{classical-energy-function-A-factors} it is more suitable to create a contour plot, since it depends on both canonical coordinates $(\varphi,\psi)$. As such, representations with different MOI orderings have been depicted in Fig. \ref{fig-A-mixed-canonical}, by lettings both coordinates vary inside the interval $[0,\pi]$.
\begin{figure}
    \centering
    \includegraphics[width=0.99\textwidth]{Chapters/Figures/A_mixed.pdf}
    \caption{The mixed canonical factor $\mathcal{A}_{\varphi\psi}$ from Eq. \ref{classical-energy-function-A-factors}, as function of the coordinates $\varphi$ and $\psi$. Both variables vary within the interval $[0,\pi]$. All insets share a common labelling for the OX and OY axes.}
    \label{fig-A-mixed-canonical}
\end{figure}

Lastly, the factor $\mathcal{A}_gamma$ must also be represented as a contour plot, because it depends on the canonical coordinates of the single-particle, namely the set $(f,\psi)$. Its graphical representation for a few values of $\gamma$ shown in Fig. \ref{fig-A-gamma-canonical}.
\begin{figure}
    \centering
    \includegraphics[width=0.99\textwidth]{Chapters/Figures/A_gamma.pdf}
    \caption{The canonical factor $\mathcal{A}_\gamma$ from Eq. \ref{classical-energy-function-A-factors}, as a function of the coordinates $(f,\psi)$ corresponding to the single-particle $\mathcal{Q}$. The $j$-shell has been fixed to $j=13/2$. In terms of the two variables, $f$ varies within $[0,2j]$ while $\psi$ is inside the interval $[0,\pi]$. All insets share a common labelling for the OX and OY axes.}
    \label{fig-A-gamma-canonical}
\end{figure}

From the representations shown in Figs. \ref{fig-A-varphi-canonical} - \ref{fig-A-gamma-canonical}, one can interpret the results as possible conditions for stability of an odd-$A$ system with regards to the wobbling regime. More precisely, smaller values of $\mathcal{A}$ will correspond to a lower total energy. Keep in mind that all these canonical factors take part in the structure of the CEF, meaning that lower energies could imply a greater stability to some extent.

\subsubsection{Critical Region}

Given the general expression of the CEF from Eqs. \ref{full-classical-energy-function} - \ref{classical-energy-function-A-factors}, one can group the terms in a part that is independent on the coordinates, a part that depends only on the core's coordinates $(r,\varphi)$, a third part that depends on the single-particle's coordinates $(f,\psi)$, and finally a \emph{mixed} term, which contains both sets of classical coordinates. As a result, $\mathcal{H}$ can be summarized (coloring is just to emphasize the grouping of the terms):
\begin{align}
    \mathcal{H}(r,\varphi;f,\psi)={\color{blue}\mathcal{H}_\text{free}}+{\color{red}\mathcal{H}_\mathscr{C}(r,\varphi)+\mathcal{H}_\mathcal{Q}(f,\psi)+\mathcal{H}_\text{mixed}(r,\varphi;f,\psi)}\ .
    \label{classical-energy-function-terms}
\end{align}
From the critical conditions associated to the classical energy function, namely:
\begin{gather*}
    \left(\frac{\partial \mathcal{H}}{\partial r}\right)=0\ ,\ \left(\frac{\partial \mathcal{H}}{\partial \varphi}\right)=0\ ,\\
    \left(\frac{\partial \mathcal{H}}{\partial f}\right)=0\ ,\ \left(\frac{\partial \mathcal{H}}{\partial \psi}\right)=0\ ,
\end{gather*}
it is possible to obtain the points at which $\mathcal{H}$ is \emph{minimal} (provided by the sign of its corresponding Hessian). In order to meet such a criteria for $\mathcal{H}$, one needs to set an ordering of the three inertia factors. Choosing the largest MOI to be around the $3$-axis and $\mathcal{I}_3>\mathcal{I}_2>\mathcal{I}_1$ (or, equivalently $A_1<A_2<A_3$), the function achieves a minimum value at:
\begin{align}
    p_0=(I,0;j,0)
    \label{cef-minimum-point-p0}
\end{align}

\subsection{Wobbling Frequencies}

By performing a linearization procedure on the classical equations of motion (from Eq. \ref{eq-of-motion-approach-w1} or explicitly in Eqs. \ref{eq-of-motion-explicit-coordinates} - \ref{eq-of-motion-explicit-momenta}) around the minimum point of $\mathcal{H}$ (i.e., point $p_0$), an algebraic equation of fourth degree will show up, with a new variable which will be denoted with $\Omega$. Reasoning behind this labelling will become clear later on. For now, the equation for $\Omega$ is given generally as \cite{raduta2020towards}:
\begin{align}
    \Omega^4+B\Omega^2+C=0\ ,
    \label{omega-equation-linearized}
\end{align}
where the coefficient $B$ is \cite{raduta2020approach}:
\begin{multline}
    B=-\bigg\{\left[ (2I-1)(A_3-A_1)+2jA_1\right]\left[(2I-1)(A_2-A_1)+2jA_1\right]+8A_2A_3Ij+\\
    +\left[(2j-1)(A_3-A_1)+2IA_1+V\frac{2j-1}{j(j+1)}\sqrt{3}\left(\sqrt{3}\cos\gamma+\sin\gamma\right)\right]\cdot\\
    \left.\cdot\left[(2j-1)(A_2-A_1)+2IA_1+V\frac{2j-1}{j(j+1)}2\sqrt{3}\sin\gamma\right]\right\}\ ,
    \label{omega-B-term}
\end{multline}
and the coefficient $C$ is \cite{raduta2020approach}:
\begin{multline}
    C=\bigg\{\left[(2I-1)(A_3-A_1)+2jA_1\right]\bigg[(2j-1)(A_3-A_1)+2IA_1+\\
    +V\frac{2j-1}{j(j+1)}\sqrt{3}(\sqrt{3}\cos\gamma+\sin\gamma)\bigg]-4IjA_3^2\bigg\}\bigg\{\left[(2I-1)(A_2-A_1)+2jA_1\right]\cdot\\
    \cdot\left[(2j-1)(A_2-A_1)+2IA_1+V\frac{2j-1}{j(j+1)}2\sqrt{3}\sin\gamma\right]-4IjA_2^2\bigg\}\ .
    \label{omega-C-term}
\end{multline}
In a work done by Raduta et al \cite{raduta2017semiclassical}, a similar equation as the one from Eq. \ref{omega-equation-linearized} was obtained via a Random Phase Approximation (RPA) method applied to another (quantized) energy function. It is of crucial importance to extract from Eq. \ref{omega-equation-linearized} only those solutions that are real and positive, so one has to study the stability conditions for the equation. Deriving certain restriction on the parameters involved in $B$ and $C$ is therefore required. Obviously, the solutions to the general equation are:
\begin{align}
    \Omega_1 \to \left(\frac{-B-\sqrt{B^2-4 C}}{2}\right)^{1/2}\ ,&\ \Omega_2 \to \left(\frac{-B+\sqrt{B^2-4 C}}{2}\right)^{1/2}\ ,\nonumber\\
    \Omega_3 \to -\left(\frac{-B-\sqrt{B^2-4 C}}{2}\right)^{1/2}\ ,&\ \Omega_4 \to -\left(\frac{-B+\sqrt{B^2-4 C}}{2}\right)^{1/2}\ .
    \label{omega-1-2-3-4-solutions}
\end{align}
Since the positiveness condition mentioned above, it is clear that only $\Omega_1$ and $\Omega_2$ will be taken into consideration. The conditions where Eq. \ref{omega-equation-linearized} has vanishing solutions will be now analyzed in terms of $B$ and $C$.

\textit{Case i)} $B>0$ \textit{and} $C=0$

The situation $C=0$ imposes that the terms (or at least one) inside the curly brackets from Eq. \ref{omega-C-term} will equate to zero. In order to simplify the calculations, some notations should be introduced. Firstly, from the definition of $A_k=1/(2\mathcal{I}_k)$ one can exploit the fact that MOI (both in rigid representation as well as the irrotational ones) are typically expressed in terms of $\beta_2$, $\gamma$, and a common term (recall Eq. \ref{eq-irrotational-rigid-mois}):
\begin{align}
    \mathcal{I}_k=\mathcal{I}_0\cdot h(\beta_2,\gamma;k)\ ,
\end{align}
where $h$ is a trigonometric function defining moments of inertia either in the rigid picture or in the irrotational picture. Going back to the inertia factors, it results that:
\begin{align}
    A_k=\frac{1}{2\mathcal{I}_k}=\frac{1}{\mathcal{I}_0}\cdot h'(\beta_2,\gamma;k)\equiv\frac{1}{\mathcal{I}_0}\bar{A}_k\ .
    \label{A-bar-general}
\end{align}
Using thus Eq. \ref{A-bar-general}, each sub-term from $B$ and $C$ can be re-written with the following notations:
\begin{align}
    P_{31}&=(2I-1)(\bar{A}_3-\bar{A}_1)+2j\bar{A}_1\ ,\ P_{21}=(2I-1)(\bar{A}_2-\bar{A}_1)+2j\bar{A}_1\ , \nonumber\\
    Q_{31}&=(2j-1)(\bar{A}_3-\bar{A}_1)+2I\bar{A}_1\ ,\ Q_{21}=(2j-1)(\bar{A}_2-\bar{A}_1)+2I\bar{A}_1\ ,\nonumber\\
    G_1&=\frac{2j-1}{j(j+1)}\sqrt{3}\left(\sqrt{3}\cos\gamma+\sin\gamma\right)\ ,\ G_2=\frac{2j-1}{j(j+1)}2\sqrt{3}\sin\gamma\ .
    \label{P-Q-G1-G2-factors}
\end{align}
This transformation is very useful because the constant $1/\mathcal{I}_0$ can be factored out from both $B$ and $C$, leaving only one `independent variable' within equations, namely $S=\mathcal{I}_0V$, which will be considered a \emph{scaling factor}. Getting back to $C=0$, after some rearrangement one gets:
\begin{align}
    P_{31}G_1S+P_{31}Q_{31}-4Ij\bar{A}_3^2=&0\ ,\nonumber\\
    P_{21}G_2S+P_{21}Q_{21}-4Ij\bar{A}_2^2=&0\ .
    \label{S-parameter-equations-set1}
\end{align}
Indeed, the above formulas represent a set of linear equations in the newly introduced variable $S$, which are of the form $aS+b=0$. As a physical interpretation for $S$, it is remarkable the fact that it comprises both a rotational part of the odd-mass system (through $\mathcal{I}_0$) but also single-particle contribution (through the potential strength $V$) so it \emph{encodes} the effect of collective rotation and deformation. By a straightforward manipulation of Eq. \ref{S-parameter-equations-set1}, the two solutions which provide vanishing a $C$ term are:
\begin{align}
    S_{01}=\frac{4Ij\bar{A}_3^2-P_{31}Q_{31}}{P_{31}G_1}\ \text{and}\ S_{02}=\frac{4Ij\bar{A}_2^2-P_{21}Q_{21}}{P_{21}G_2}\ .
    \label{C-Term-zero-solutions}
\end{align}

\textit{Case ii)} $B=0$ \textit{and} $C=\text{arbitrary}$

When $B$ must vanish, a second-degree algebraic equation for the variable $S$ will emerge. Using the same notations that where introduced in the case $i)$ for the sub-terms within $C$, it results:
\begin{align}
    P_{31}P_{21}+8Ij\bar{A}_2\bar{A}_3+\left(Q_{31}+SG_1\right)\left(Q_{21}+SG_2\right)=0\ , \nonumber
\end{align}
or:
\begin{align}
    G_1G_2S^2+\left(Q_{21}G_1+Q_{31}G_2\right)S+P_{31}P_{21}+8Ij\bar{A}_2\bar{A}_3=0\ .
    \label{S-parameter-equations-set2}
\end{align}
The solutions from this second-degree equation are:
\begin{align}
    S_{11}&=-\frac{\sqrt{\left(G_1 Q_{21}+G_2 Q_{31}\right){}^2-4 G_1 G_2 \left(8 I j \bar{A}_2 \bar{A}_3+P_{21} P_{31}\right)}+G_1 Q_{21}+G_2 Q_{31}}{2 G_1 G_2}\ ,\nonumber\\
    S_{12}&=\frac{\sqrt{\left(G_1 Q_{21}+G_2 Q_{31}\right){}^2-4 G_1 G_2 \left(8 I j \bar{A}_2 \bar{A}_3+P_{21} P_{31}\right)}-G_1 Q_{21}-G_2 Q_{31}}{2 G_1 G_2}\ .
    \label{B-Term-zero-solutions}
\end{align}
Gathering the outcomes from Eqs. \ref{B-Term-zero-solutions} and \ref{C-Term-zero-solutions}, four solutions for the variable $S$ were obtained, i.e., $S=\{S_{01},S_{02},S_{11},S_{12}\}$.

Returning to the case $i)$ which provides vanishing $C$ term, it would be instructive to see for what values of $\mathcal{I}_0$ and $V$ the left-hand sides from Eq. \ref{S-parameter-equations-set1} equate to zero. Denoting the first left-hand side with $f_1=f_1(\mathcal{I}_0,V)$ and the second one with $f_2=f_2(\mathcal{I}_0,V)$, the contour lines that make $f_1$ and $f_2$ cancel out are represented in Fig. \ref{fig-vanishing-f1-f2}.
\begin{figure}
    \centering
    \includegraphics[scale=0.8]{Chapters/Figures/f1f2_solutions-edited.pdf}
    \caption{The contour lines that show for what values of $\mathcal{I}_0$ and $V$ the left-hand sides from Eq. \ref{S-parameter-equations-set1} are zero. The two sides are denoted here by $f_1$ and $f_2$, respectively. The calculations were done for fixed values of $I, j, \gamma, A_1, A_2, A_3$. Regarding the coloring, darker red corresponds to negative values of $f_1$, while lighter red means positive $f_1$. Similarly for the $f_2$ term with blue color. A 50/50 ratio for each term has been chosen relative to the width of the plot just for a clearer difference between the positive/negative zones of the two functions.}
    \label{fig-vanishing-f1-f2}
\end{figure}

Furthermore, the left-hand side of Eq. \ref{S-parameter-equations-set2} from case $ii)$ can also be zero for certain intervals of $\mathcal{I}_0$ and $V$. A restriction on $V$ to only have positive values is adopted throughout the formalism (to be consistent with the literature \cite{shou2009coupling,tanabe2017stability,poenaru2021parity}). This forces only a single contour (denoted with $F=0$) to appear above the OX axis. A representation showing this line is done in Fig. \ref{fig-vanishing-F}, evaluated at arbitrary $I, j, \gamma, A_1, A_2, A_3$. Keep in mind that the $V$ parameter is restricted to the interval $[0,5]$ as it was the case for Fig. \ref{fig-vanishing-f1-f2}.
\begin{figure}
    \centering
    \includegraphics[scale=0.8]{Chapters/Figures/F_solutions_B-term.pdf}
    \caption{Graphical representation with the contour $F=0$ as function of the parameters $\mathcal{I}_0$ and $V$, where $F$ represents the left-hand side of Eq. \ref{S-parameter-equations-set2}. Arbitrary values for $I, j, \gamma, A_1, A_2, A_3$ were chosen. Darker color signifies negative values for $F$, while opposite holds for the lighter shading.}
    \label{fig-vanishing-F}
\end{figure}

Taking a look at the regions depicted in Figs. \ref{fig-vanishing-f1-f2} - \ref{fig-vanishing-F}, one can assume that they represent a clear indicator regarding the stability of Eq. \ref{omega-equation-linearized}. Obviously, when values of $(\mathcal{I}_0,V)$ lie near the contours lines within the plots, the equation reaches instability and no real solutions emerge. About the units of measure for the two quantities that were used as plot variables, they are $\hbar^2\text{MeV}^{-1}$ and $\text{MeV}$, respectively. They were dismissed from the plots because the precise numerical values of those representations were not necessary.

Regarding the solutions given in Eq. \ref{omega-1-2-3-4-solutions}, the selected ones are $\Omega_1$ and $\Omega_2$. Considering their structure, it is important to retrieve all the conditions that grant positive square roots. These conditions are sketched in Table \ref{omega-positive-conditions}.
\begin{table}
    \centering
    \resizebox{0.69\textwidth}{!}{%
    \begin{tabular}{|c|c|}
    \hline
    Solution   & Positive square root \\ \hline
    $\Omega_1$ & $B<0\ \text{and}\ 0\leq C\leq \frac{B^2}{4}$                      \\ \hline
    $\Omega_2$ & $\left(B\leq 0\ \text{and}\ C\leq \frac{B^2}{4}\right)$ or $\left(B<0\ \text{and}\ C\leq 0\right)\ $                     \\ \hline
    \end{tabular}%
    }
    \caption{The conditions for which the square roots that appear in the two solutions $\Omega_{1,2}$ given in Eq. \ref{omega-1-2-3-4-solutions} are positive, such that real quantities can be obtained. The trivial solution $B=C=0$ has been dismissed.}
    \label{omega-positive-conditions}
\end{table}

Going back to the CEF in the form shown in Eq. \ref{classical-energy-function-terms}, the free term ${\color{blue}\mathcal{H}_\text{free}}$ is the one that does not depend on any canonical variable. The other three terms (depicted by red color) have a dependence on either the core's coordinates $(r,\varphi)$ or the particle's coordinates $(f,\psi)$. However, the remarking thing is that all these terms (including the mixed one) are comprised in the two solutions $\Omega_1$ and $\Omega_2$. In fact, one expects such a thing, since $\Omega_{1,2}$ arise from the linearization of the equations of motion around the minimum point $p_0=(r_0=I,\varphi_0=0;f=j,\psi=0)$. As it was shown in $\mathbf{W_0}$ and also Refs. \cite{raduta2020approach,raduta2020new}, the solutions $\Omega_{1,2}$ give the `final' terms (besides the minimal value $\mathcal{H}_\text{min}$) in the total energy spectrum for an odd-mass system. More precisely, the CEF from \ref{classical-energy-function-terms} combined with the two real and positive solutions of Eq. \ref{omega-equation-linearized} will result in the final spectrum \cite{poenaru2021extensive1}:
\begin{align}
    E_{I,n_1,n_2}=\epsilon_j+{\color{blue}\mathcal{H}_\text{min}^I}+{\color{red}\mathcal{F}^I_{n_{w_1}n_{w_2}}}\ ,
    \label{tsd-bands-compressed-spectrum}
\end{align}
where the \emph{phonon} term $\mathcal{F}_{n_{w_1}n_{w_2}}^I$ is the sum of the two solutions $\Omega_{1,2}$:
\begin{align}
    \mathcal{F}_{n_{w_1}n_{w_2}}^I=\hbar\Omega_1^I\left(n_{w_1}+\frac{1}{2}\right)+\hbar\Omega_2^I\left(n_{w_2}+\frac{1}{2}\right)\ ,
    \label{phononic-term-tsd-energies}
\end{align}
and it can be regarded as originating from $\mathcal{H}_\mathscr{C}+\mathcal{H}_\mathcal{Q}+\mathcal{H}_\text{mixed}$. This was obtained from the linearization procedure mentioned at the start of the section. The single-particle energy of the odd-particle belonging to a particular $j$-shell (as per $\hat{H}_\text{sp}$ from Eq. \ref{single-particle-ham-approach-w1}) is represented by $\epsilon_j$, which should be adopted as a constant/fixed value. Notice that $\mathcal{H}_\text{min}^I$ from Eq. \ref{tsd-bands-compressed-spectrum} is in fact the \emph{free term} that appears in Eq. \ref{classical-energy-function-terms} and it does not depend on the coordinates $(r, \varphi\ ;\ f, \psi)$. The coloring from Eq. \ref{tsd-bands-compressed-spectrum} is consistent with the one from Eq. \ref{classical-energy-function-terms} so that the analogy between each component can be viewed. Putting together Eq. \ref{tsd-bands-compressed-spectrum} and Eq. \ref{phononic-term-tsd-energies}, the general spectrum of an odd-mass triaxial nucleus will be:
\begin{align}
    E_{I,n_1,n_2}=\epsilon_j+\mathcal{H}_\text{min}^I+\hbar\Omega_1^I\left(n_{w_1}+\frac{1}{2}\right)+\hbar\Omega_2^I\left(n_{w_2}+\frac{1}{2}\right)\ .
    \label{tsd-bands-general-spectrum}
\end{align}
The conditions for which $\Omega_{1,2}$ exist and meet the criteria of positiveness were summarized in Table \ref{omega-positive-conditions}, and throughout the calculations the ordering for $\Omega_1$ and $\Omega_2$ will be chosen accordingly.

The physical interpretation of $\Omega_1$ and $\Omega_2$ from Eq. \ref{tsd-bands-general-spectrum} is a remarking feature of this formalism. Indeed, in the same manner as for the simple wobbling motion (recall the spectrum defined in Eq. \ref{eq-wobbling-energy-evenA}) or the odd-$A$ spectrum discussed in Chapter \ref{chapter-5} (recall Section \ref{chapter-5-odd-wobbling-theory}), here $\Omega$ also represents a wobbling frequency. The advantage of using the TDVE approach is that the Hamiltonian is properly `split' and two wobbling frequencies emerge: one for the core and one for the odd-particle. This semi-classical approach is very useful for visualizing the two interacting systems directly from the analytical expression (i.e., Eq. \ref{tsd-bands-general-spectrum}). The description for odd-mass systems depicted in Chapter \ref{chapter-5} resulted having a singular wobbling frequency for the QTR Hamiltonian (see Eq. \ref{wobbling-freq-oddA}) and also the interplay between the core and the particle were not clearly separated from its expression.

Concerning the two numbers from Eq. \ref{tsd-bands-general-spectrum}, i.e., $n_{w_1}=0,1,\dots$ and $n_{w_2}=0,1,\dots$, they are the wobbling phonon numbers, which are also separated in terms of the core and the single particle. These integers represent the number of excited quanta, and as it will be shown, the triaxial bands in the studied isotopes will be built with such excited quanta. The entire discussion for obtaining the energy spectrum for a triaxial system, starting from the dequantization of $\hat{H}$ and ending with the two harmonic frequencies for the core and the particle, can be summarized in a diagram where the steps are schematically shown together with a final physical interpretation of $\Omega_{1,2}$ and $n_{w_{1,2}}$. The resulting drawing is shown in Fig. \ref{TDVE-wobbling-complete-sketch}.
\begin{figure}
    \centering
    % \includegraphics[]{}
    \caption{\textbf{TO FINISH: }An illustration depicting the dequantization of the initial Hamiltonian for an odd-mass triaxial nucleus. From the classical canonical variables given by TDVE, a set of two wobbling frequencies are obtained, which are ascribed to the even-even core and the single-particle. Each frequency can be regarded as a precessional motion of either $\mathbf{R}_\mathscr{C}$ or $\mathbf{j}_\mathcal{Q}$.}
    \label{TDVE-wobbling-complete-sketch}
\end{figure}

\subsection{Alternative Approach}

In the following, a different description for the phonon term $\mathcal{F}_{n_{w_1}n_{w_2}}^I$ from Eq. \ref{phononic-term-tsd-energies} will be made and nevertheless the method will provide similar results regarding $\Omega_1$ and $\Omega_2$. This approach will help to understand the definition of the wave-functions of the triaxial bands that will be described later on. Starting again with the CEF from Eq. \ref{classical-energy-function-terms} and expanding it around $p_0$ up to second order, the obtained equation will be \cite{raduta2020approach}:
\begin{align}
    \mathcal{H}=&{\color{blue}\mathcal{H}_\text{min}}+\nonumber\\
                &+{\color{red}\bigg\{}\frac{1}{2I}\left[(2I-1)(A_3-A_1)+2jA_1\right]r'^{2}+\frac{I}{2}\left[(2I-1)(A_2-A_1)+2jA_1\right]\varphi'^2+\nonumber\\
                &+\frac{1}{2j}\left[(2j-1)(A_3-A_1)+2IA_1+V\frac{2j-1}{j(j+1)}\sqrt{3}\left(\sqrt{3}\cos\gamma+\sin\gamma\right)\right]f'^2+\nonumber\\
                &+\frac{j}{2}\left[(2j-1)(A_2-A_1)+2IA_1+V\frac{2j-1}{j(j+1)}2\sqrt{3}\sin\gamma\right]\psi'^2-\nonumber\\
                &-2A_3r'f'-2IjA_2\varphi'\psi'{\color{red}\bigg\}}\ .
    \label{classical-energy-function-deviations}
\end{align}
where the primed coordinates $r',\varphi',f',\psi'$ are the deviations from the minimum point $p_0=(r_0,\varphi_0;f_0,\psi_0)$, which are expressed as:
\begin{align}
    r'=(r-r_0)\ ,\ \varphi'=(\varphi-\varphi_0)\ ,\nonumber\\
    f'=(f-f_0)\ ,\ \psi'=(\psi-\psi_0)\ .
\end{align}
Note that the curly brackets from Eq. \ref{classical-energy-function-deviations} are only used to emphasize the three terms $\mathcal{H}_\mathscr{C}(r,\varphi)+\mathcal{H}_\mathcal{Q}(f,\psi)+\mathcal{H}_\text{mixed}(r,\varphi;f,\psi)$, so that one can clearly see which are the terms with separate canonical variables and which are the mixed ones. Again one illustrates the grouped factors as per Eq. \ref{classical-energy-function-terms} through a similar coloring. It is clear from this expression that $\mathcal{H}_\text{min}$ is none other than the free term $\mathcal{H}_\text{free}$.

\paragraph*{\textit{i) Non-coupling terms}}
Ignoring the coupling terms from Eq. \ref{classical-energy-function-deviations}, the CEF will consist in the sum of two independent harmonic oscillators with the frequencies:
\begin{align}
    \omega_1=&\left\{\left[(2I-1)(A_3-A_1)+2jA_1\right]\left[(2I-1)(A_2-A_1)+2jA_1\right]\right\}^{1/2}\ ,\nonumber\\
    \omega_2=&\left[(2j-1)(A_3-A_1)+2IA_1+V\frac{2j-1}{j(j+1)}\sqrt{3}\left(\sqrt{3}\cos\gamma+\sin\gamma\right)\right]^{1/2}\cdot\nonumber\\
    &\cdot\left[(2j-1)(A_2-A_1)+2IA_1+V\frac{2j-1}{j(j+1)}2\sqrt{3}\sin\gamma\right]^{1/2}\ .
    \label{small-omega-1-2}
\end{align}

In order to obtain real solutions for $\omega_1$, the following conditions for the three inertia factors must hold:
\begin{align}
    A_3>SA_1\ \text{and}\ \ A_2>SA_1\ ,\nonumber\\
    \text{while:}\ A_3>A_2\ \text{or}\ A_3<A_2\ ,
\end{align}
while for the second frequency, the following restrictions must hold:
\begin{align}
    A_3>&S'A_1-VT\ \text{and}\ A_2>S'A_1-VT'\ ,\nonumber
    % &\text{while:}\ A_3>A_2\ \text{or}\ A_3<A_2\ ,
\end{align}
with the terms $S$, $S'$, $T$, and $T'$ defined as:
\begin{align}
    S=\frac{2I-1-2j}{2I-1}\ &,\  S'=\frac{2j-2I-1}{2j-1}\ ,\nonumber\\
    T=\frac{1}{j(j+1)}\sqrt{3}\left(\sqrt{3}\cos\gamma+\sin\gamma\right)\ &,\ T'=\frac{1}{j(j+1)}2\sqrt{3}\sin\gamma\ .
\end{align}
The positiveness for $V$ will make sure that the conditions given for $\omega_2$ are always satisfied.

\paragraph*{\textit{ii) Coupling terms}}
In order to treat the terms that contain products $r'f'$ and $\varphi'\psi'$ from the expansion in Eq. \ref{classical-energy-function-deviations}, a \emph{quantization} should be employed on the canonical variables. Indeed, from the classical coordinates (obtained via the TDVE) of the phase space $S_\text{cls}$, one can also go back to a $S_\text{qt}$ space, where each coordinate will have a corresponding operator defined in that space. As such, the following transformations will be utilized \cite{raduta2020approach}:
\begin{align}
    \varphi\to&\hat{q}_1\ ,\ r\to\hat{p}_1\ ,\ \left[\hat{q}_1,\hat{p}_1\right]=\iu\ ,\nonumber\\
    \psi\to&\hat{q}_2\ ,\ f\to\hat{p}_2\ ,\ \left[\hat{q}_2,\hat{p}_2\right]=\iu\ .
    \label{canonical-coordinates-quantized}
\end{align}
The notation of the operators is consistent with the fact that $(\varphi,\psi)$ represent the canonical coordinates and $(r,f)$ depict the canonical momenta. Furthermore, each set of operators a corresponding creation and annihilation operator defined as:
\begin{align}
    a^\dagger=\frac{1}{\sqrt{2}k_1}\left(k_1^2\hat{q}_1+ \iu \hat{p}_1\right)\ ,\ a=\frac{1}{\sqrt{2}k_1}\left(k_1^2\hat{q}_1+ \iu \hat{p}_1\right)\ ,
    \label{creation-operators-a}
\end{align}
for the variables $(r,\varphi)$ of the core and:
\begin{align}
    b^\dagger=\frac{1}{\sqrt{2}k_2}\left(k_2^2\hat{q}_2+ \iu \hat{p}_2\right)\ ,\ b=\frac{1}{\sqrt{2}k_2}\left(k_2^2\hat{q}_2+ \iu \hat{p}_2\right)\ ,
    \label{creation-operators-b}
\end{align}
for the single-particle variables $(f,\psi)$. With the terms defined in Eqs. \ref{creation-operators-a} - \ref{creation-operators-b}, the operators $(\hat{q}_1,\hat{p}_1)$ and $(\hat{q}_2,\hat{p}_2)$ will acquire the following form:
\begin{align}
    \hat{q}_1=&\frac{1}{\sqrt{2}k_1}\left(a^\dagger+a\right)\ ,\ \hat{p}_1=\frac{\iu k_1}{\sqrt{2}}\left(a^\dagger-a\right)\ ,\nonumber\\
    \hat{q}_2=&\frac{1}{\sqrt{2}k_2}\left(b^\dagger+b\right)\ ,\ \hat{p}_2=\frac{\iu k_2}{\sqrt{2}}\left(b^\dagger-b\right)\ .
    \label{canonical-transformations-ab-qp}
\end{align}

The transformations that go from $(\hat{q}_1,\hat{p}_1)$ and $(\hat{q}_2,\hat{p}_2)$ to $(a^\dagger, a)$ and $(b^\dagger, b)$ are \emph{canonical}. The two constant factors $k_1$ and $k_2$ that appear in Eqs. \ref{creation-operators-a} - \ref{canonical-transformations-ab-qp} are called \emph{canonicity factors} (firstly introduced by the team in \cite{raduta2017semiclassical}), and they are fixed such that `dangerous' terms like $(a^\dagger)^2+a^2$ and $(b^\dagger)^2+b^2$ do not appear in the two frequencies. One should not confuse them with the canonical factors $\mathcal{A}$, which were defined in Eq. \ref{classical-energy-function-A-factors} as a way of expressing the CEF. The expressions for $k_1$ and $k_2$ are \cite{raduta2020approach}:
\begin{align}
    k_1=&\left[\frac{(2I-1)(A_2-A_1)+2jA_1}{(2I-1)(A_3-A_1)+2jA_1}\cdot I^2\right]^{1/4}\, \nonumber\\
    k_2=&\left[\frac{(2j-1)(A_2-A_1)+2IA_1+V\frac{2j-1}{j(j+1)}2\sqrt{3}\sin\gamma}{(2j-1)(A_3-A_1)+2IA_1+V\frac{2j-1}{j(j+1)}\sqrt{3}\left(\sqrt{3}\cos\gamma+\sin\gamma\right)}\cdot j^2\right]^{1/4}\ .
    \label{canonicity-factors}
\end{align}
It can be seen that the canonicity factors from Eq. \ref{canonicity-factors} exhibit an $f(x)\approx x^{1/2}$ behavior with respect to the total and single-particle angular momentum, respectively. Moreover, the triaxiality parameter only affects $k_2$, which is consistent with the particle + core system, i.e., the single-particle will drive the entire system to a large degree of deformation and it will stabilize the structure.

Introducing a set of additional notations (in a similar fashion as for the calculations done in Eqs. \ref{S-parameter-equations-set1} and \ref{S-parameter-equations-set2}):
\begin{align}
    \Delta_{31}=&(2I-1)(A_3-A_1)+2jA_1\ ,\ \Delta_{21}=(2I-1)(A_2-A_1)+2jA_1\ ,\nonumber\\
    \Sigma_{31}=&(2j-1)(A_3-A_1)+2IA_1\ ,\ \Sigma_{21}=(2j-1)(A_2-A_1)+2IA_1\ ,
    \label{Sigma-Delta-factors}
\end{align}
while the factors $G_1$ and $G_2$ from Eq. \ref{P-Q-G1-G2-factors} will be kept the same. Compared to the calculations from Eqs. \ref{S-parameter-equations-set1} - \ref{S-parameter-equations-set2}, no prior factorization for the inertia factors is made here. With the notations given in Eq. \ref{Sigma-Delta-factors}, the two oscillator frequencies and the canonicity factors achieve a more practical shape, namely:
\begin{align}
    \omega_1=&\left(\Delta_{31}\cdot\Delta_{21}\right)^{1/2}\ ,\nonumber\\
    \omega_2=&\left(\Sigma_{31}+VG_1\right)^{1/2}\cdot\left(\Sigma_{21}+VG_2\right)^{1/2}\ ,
\end{align}
for the wobbling frequencies, and:
\begin{align}
    k_1=&\left(\frac{\Delta_{21}}{\Delta_{31}}\cdot I^2\right)^{1/4}\, \nonumber\\
    k_2=&\left[\frac{\Sigma_{21}+VG_2}{\Sigma_{31}+VG_1}\cdot j^2\right]^{1/4}\ ,
\end{align}
for the two canonicity factors. Notice the single-particle strength appearing explicitly for $\omega_2$ and also $k_2$.

From the quantization made in Eq. \ref{canonical-coordinates-quantized}, with the operators $(a^\dagger,a;b^\dagger,b)$ defined in Eqs. \ref{creation-operators-a} - \ref{creation-operators-b}, and the factors defined in Eq. \ref{canonicity-factors}, this \emph{new quantal representation} of the classical energy $S_\text{cls}\supset\mathcal{H}\to\hat{H}\subset S_\text{qt}$ will achieve the following form:
\begin{align}
    \hat{H}={\color{blue}\mathcal{H}_\text{min}}+{\color{red}\bigg\{}\omega_1\left(a^\dagger a+\frac{1}{2}\right)+&\omega_2\left(b^\dagger b+\frac{1}{2}\right)+A_3k_1k_2\left(a^\dagger b^\dagger+ba-a^\dagger b-b^\dagger a\right)-\nonumber\\
                                                                                &-IjA_2\frac{1}{k_1k_2}\left(a^\dagger b^\dagger + ba + a^\dagger b + b^\dagger a \right){\color{red}\bigg\}}\ .
    \label{quantized-Hamiltonian-CEF}
\end{align}
Since $\mathcal{H}_\text{min}$ does not depend on the canonical variables, its structure will stay unchanged. The next two terms are the harmonic oscillators with the two frequencies of oscillation $\omega_1$ and $\omega_2$, while the last two contain mixed products of creation/annihilation operators from both the core and single-particle degrees of freedom. Indeed, this structure can be also regarded in a similar way as the grouped terms from Eq. \ref{classical-energy-function-terms} or Eq. \ref{classical-energy-function-deviations}, similarly depicted here by the blue and red colors. The Hamiltonian defined in Eq. \ref{quantized-Hamiltonian-CEF} will have the following commutation rules with the creation and annihilation operators (Eqs. \ref{creation-operators-a} - \ref{creation-operators-b}):
\begin{align}
    \left[\hat{H},a^\dagger\right]=&\omega_1a^\dagger+A_3k_1k_2(b-b^\dagger)-IjA_2\frac{1}{k_1k_2}(b+b^\dagger)\ ,\nonumber\\
    \left[\hat{H},a\right]=&-\omega_1a-A_3k_1k_2(b^\dagger-b)+IjA_2\frac{1}{k_1k_2}(b^\dagger+b)\ ,\nonumber\\
    \left[\hat{H},b^\dagger\right]=&\omega_2b^\dagger+A_3k_1k_2\left(a-a^\dagger\right)-IjA_2\frac{1}{k_1k_2}\left(a+a^\dagger\right)\ ,\nonumber\\
    \left[\hat{H},b\right]=&-\omega_2b-A_3k_1k_2\left(a^\dagger-a\right)+IjA_2\frac{1}{k_1k_2}\left(a^\dagger+a\right)\ .
    \label{H-quantal-linear-system-a-b-operators}
\end{align}

One can see that Eq. \ref{H-quantal-linear-system-a-b-operators} forms a linear system, which can be solved with the introduction of a special \emph{phonon operator} \cite{raduta2017semiclassical,raduta2018wobbling}:
\begin{align}
    \Gamma^\dagger=X_1a^\dagger-Y_1a+X_2b^\dagger-Y_2b\ .
\end{align}
The four \emph{phonon amplitudes} $(X_1,Y_1)$ and $(X_2,Y_2)$ are complex numbers defined in such a way that the following restrictions (commutation rules) hold true:
\begin{align}
    \left[\hat{H},\Gamma^\dagger\right]=\Omega\Gamma^\dagger\ ,\ \left[\Gamma,\Gamma^\dagger\right]=1\ ,
    \label{H-Gamma-phonon-operator-commutator}
\end{align}
and they verify the relation:
\begin{align}
    \left|X_1\right|^2-\left|Y_1\right|^2+\left|X_2\right|^2-\left|Y_2\right|^2=1\ .
\end{align}
The amplitudes were firstly calculated in Ref. \cite{raduta2007semiclassical} and their expressions as functions of $\omega_{1,2}$, $k_{1,2}$, and $\Omega$ can be seen in Appendix B therein. Furthermore, the factor $\Omega$ from Eq. \ref{H-Gamma-phonon-operator-commutator} turns out to be the wobbling frequency obtained in the previous section. Indeed, one finds that the system of equations given in Eq. \ref{H-quantal-linear-system-a-b-operators} verifies:
\begin{align}
    \Omega^4+B'\Omega^2+C'=0\ ,
\end{align}
where the two primed coefficients are \cite{raduta2020approach}:
\begin{align}
    B'=&-\left(\omega_1^2+\omega_2^2+8A_2A_3Ij\right)\ ,\nonumber\\
    C'=&(\omega_1\omega_2)^2-4\left[A_3^2(k_1k_2)^2+A_2^2\frac{(Ij)^2}{\left(k_1k_2\right)^2}\right]\left(\omega_1\omega_2\right)+16(A_2A_3Ij)^2\ .
    \label{B-C-prime-terms}
\end{align}
Manipulating the obtained terms from Eq. \ref{B-C-prime-terms}, it can be shown that they are actually equivalent with the factors $B$ and $C$ defined in Eq. \ref{omega-B-term} and Eq. \ref{omega-C-term}, respectively. Notice the fact that the first term does not contain any canonicity factor (i.e., neither $k_1$ nor $k_2$), however, $C'$ depends quadratically on the product $k_1k_2$. Moreover, the contribution from both oscillator frequencies attained when the coupling terms were ignored (case $i)$ with Eq. \ref{small-omega-1-2}) comes into play for both terms. Obviously, in reference to the core $\mathscr{C}$ and single-particle $\mathcal{Q}$ degrees of freedom, there is an interplay between them in $B'$ but also $C'$ (through the fact that $\omega_2$ contains the single-particle potential strength $V$ and the triaxiality parameter $\gamma$ parameters).

Prior to conclude this section, some graphical representations of the oscillator frequencies $(\omega_1,\omega_2)$ and the canonicity factors $(k_1,k_2)$ will be made, in order to get a qualitative interpretation of their behavior with respect to certain parameters. As such, in Fig. \ref{fig-omega1-omega2} the oscillator frequencies are shown as function of the total angular momentum. Notice that the two quantities vary linearly with spin (recall Eq. \ref{small-omega-1-2}), and they also intersect at a particular spin $I'$, depending on the magnitude of the three inertia factors $A_k$. Moreover, in Fig. \ref{fig-omega1-omega2} the region with $I\leq j$ is captured too (marked by the hatched gray area), although one should keep in mind that usually in isotopes the wobbling bands have band-heads which lie higher than that (especially for the excited $n_w>0$ bands). The plots were made only for a single ordering of the three inertia factors, given that reversing $A_2$ and $A_3$ will not produce significant modifications of the figures (i.e., $\omega_1$ stays the same and $\omega_2$ changes only very slightly).
\begin{figure}
    \centering
    \includegraphics[scale=0.8]{Chapters/Figures/omega-1-2-frequencies-1.pdf}
    \caption{The oscillator frequencies $\omega_1$ and $\omega_2$ from Eq \ref{small-omega-1-2}. Calculations are given for arbitrary values of the single particle potential $V$, triaxiality parameter $\gamma$ and single-particle $j$-shell. For this particular example, $j=13/2$. The inertia factor ordering is given as $A_3>A_2>A_1$, according to the conditions of existence for $\omega_1$ and $\omega_2$. The intersection point between $\omega_1$ and $\omega_2$ is marked by the red dot-dashed line around $I\approx 24\hbar$. See text for details on the two colored regions split by the black vertical line.}
    % backup paragraph
    %The region with $I\leq j$ is only for illustrative purpose, since in a realistic representation the total spin of the nucleus will always be larger than $j$
    \label{fig-omega1-omega2}
\end{figure}

In Fig. \ref{fig-omega2-V-Gamma} a different approach is also taken into consideration for $\omega_2$. Indeed, since its dependence is given by both parameters $\gamma$ and $V$, a contour plot within this space is provided, while the other relevant parameters are kept constant (i.e., total a.m. single-particle a.m., and inertia factors). From this representation one can notice the rather constant grow in magnitude for $\omega_2$ when the single-particle potential $V$ increases. Additionally, for a given $V$, the frequency does not vary too much much across the $\gamma$ range, particularly for $V<2$. The rather abrupt changes only happen in the low $\gamma$ and $V>2$ region. Throughout the contour-like representations that vary $V$ and $\gamma$, the range was kept to $[0,5]\ \text{MeV}$ for $V$ and $[0,60]\  ^\circ$ for $\gamma$, respectively.
\begin{figure}
    \centering
    \includegraphics[scale=0.8]{Chapters/Figures/omega-2-gamma-V.pdf}
    \caption{The oscillator frequency $\omega_2$ from Eq. \ref{small-omega-1-2} within the $(\gamma,V)$ plane. Arbitrary values were given for the three inertia factors ($A_3>A_2>A_1$), the total spin, and the single-particle angular momentum. The range for $V$ was set to $[0,5]\ \text{MeV}$ and $\gamma\in[0,60]$ degrees. The value of $\gamma=25^\circ$ is marked with the dashed vertical line just to guide the eye. Each contour line represents an energy difference of about $0.20$ MeV.}
    \label{fig-omega2-V-Gamma}
\end{figure}

Concerning the canonicity factors, they are represented separately in Fig. \ref{fig-k-1-2-factors}. The plots show the evolution of both quantities with respect to the total angular momentum and at different orderings for $A_k$. For the adopted numerical calculations, the values of $A_k$ remained unchanged and only $A_2$ and $A_3$ were reversed. Remarkable the fact that for $k_2$, reversing the factors $A_{2,3}$ will change the behavior with respect to spin from an increasing one ($A_3>A_2$) to a decreasing type ($A_2>A_3$). The same cannot be said about $k_1$, where both orderings give an increasing trend w.r.t. spin. Note that just for a pedagogical purpose, the plots contain the regions where $I\leq j$, similarly as per Fig. \ref{fig-omega1-omega2}.
\begin{figure}
    \centering
    \includegraphics[scale=0.7]{Chapters/Figures/k1_factor.pdf}
    \includegraphics[scale=0.7]{Chapters/Figures/k2_factor.pdf}
    \caption{The canonicity factors defined in Eq. \ref{canonicity-factors} are graphically represented as function of spin $I$ for arbitrary MOI, $V$ and $\gamma$. The single-particle angular momentum is set to $j=13/2$. For the calculations, the two $A_k$ orderings kept the same numerical values and only $A_2$ and $A_3$ were interchanged. The regions where $I\leq j$ (hatched area) and $I\geq j$ (green area) that are split by the vertical dashed line are pointed out as well.}
    \label{fig-k-1-2-factors}
\end{figure}

Since the structure of $k_2$ is dependent on the triaxiality parameter and the single-particle potential strength, a set of contour plots within this plane can also be created, in a similar fashion as the one obtained for $\omega_2$ from Fig. \ref{fig-omega2-V-Gamma}. For this case though, by reversing $A_2$ and $A_3$ will in fact produce significant changes on the values of $k_2$ (recall Fig. \ref{fig-k-1-2-factors} where the condition $A_2>A_3$ gave decreasing behavior for the canonicity factor). Consequently, two such contour plots are depicted in Fig. \ref{fig-k2-factor-contour}.
\begin{figure}
    \centering
    \includegraphics[width=0.49\textwidth]{Chapters/Figures/k2_CP.pdf}
    \includegraphics[width=0.5\textwidth]{Chapters/Figures/k2_reversed_CP.pdf}
    \caption{The canonicity factor $k_2$ from Eq. \ref{canonicity-factors} in the ($\gamma,V)$ plane, evaluated at fixed inertia factors and angular momenta. In each figure the terms $A_2$ and $A_3$ have been reversed but the other parameters remained unchanged. The value $\gamma=25^\circ$ is marked by the vertical dashed line just to guide the eye.}
    \label{fig-k2-factor-contour}
\end{figure}

This alternative description for obtaining a set of solutions that describe a harmonic motion of the core $\mathscr{C}$ and the odd single-particle $\mathcal{Q}$ comprised $a)$ the required steps in acquiring analytical results in regards to the energy spectra and $b)$ qualitative analyses for some of the quantities that emerged from calculations. Furthermore, by quantizing the classical variables obtained from TDVE, it is possible to attain a new classical energy function of a triaxial nucleus, which can be summarized in the following way: \emph{The CEF is composed of 1) a term that is independent of any canonical variable and 2) two harmonic vibrators with the phonon energies} $\left\{\Omega_1,\Omega_2\right\}$, \emph{portraying the precessional motion of the core and the odd particle}. The oscillators emerge from the solutions of a fourth degree algebraic equation, whose origin was deducted in a `quantized manner', i.e., the description shown above (starting with Eq. \ref{canonical-coordinates-quantized}). This is indeed a remarkable feature of the current model, since the dequantization procedure of the initial Hamiltonian and further \emph{re-quantization} of $\mathcal{H}$ can be done consistently without any loss of the underlying system dynamics and degrees of freedom. The final form of the energy spectrum will be defined analytically by Eq. \ref{tsd-bands-compressed-spectrum}, where the first term of the expansion of $\mathcal{H}$ around its minimum point is given by $\mathcal{H}_\text{min}$, and the two oscillators (from the expansion up to second order) are comprised in the phonon term $\mathcal{F}_{n_{w_1}n_{w_2}}^I$ (Eq. \ref{phononic-term-tsd-energies}).

Regarding the energy spectrum $E_{I,n_{w_1},n_{w_2}}$ obtained in Eq. \ref{tsd-bands-compressed-spectrum} and Eq. \ref{tsd-bands-general-spectrum}, when both wobbling phonon numbers are $0$, it simply reflects the \emph{zero-point-motion} for the system, i.e., the true ground state. Even when $n_{w_1}=n_{w_2}=0$, the core and particle will contribute to the total energy of the system, through the smallest `quantum fluctuations` characterized by $\Omega_1/2$ and $\Omega_2/2$, which is consistent with the harmonic description of the triaxial rigid rotators. This will be crucial when the real wobbling spectrum of the studied nuclei will be interpreted with the current formalism.

\section{A New Band Structure in Lu Isotopes}

Following up to the team's work with regards to the so-called $\mathbf{W_0}$ approach that was outlined in Section \ref{foundation}, two more research papers were devoted to the study of wobbling motion in Lu isotopes (see Refs. \cite{raduta2020approach,raduta2020towards}), but with a different perspective in reference to the band structure. In what follows, the new interpretation (hereafter referred to as $\mathbf{W_1}$) will be described, while also pointing out key distinctions between it and $\mathbf{W_0}$.

Recalling the mechanism that generates the wobbling states via $\mathbf{W_0}$, this was depicted in Fig. \ref{phonon-operator-schematic}, and in this scheme it was shown that from the ground states (calculated variationally according to Eq. \ref{tdve-approach-w1}), one can obtain energy levels for bands $n_w=1, 2, \dots$ by acting with a phonon operator once, twice, and so on, on a given spin state $I$. Moreover, a spin state $I$ from a band $n_w=n'$ emerged by means of the phonon operator which acts on $I-1$ from the band $n_w=n'-1$. The formalism $\mathbf{W_0}$ clearly shows the fact that the variational method can be applied to the ground-band and furthermore get excited bands through an additional term (i.e., the phonon operator), which can be successively applied on ground states. On the other hand, $\mathbf{W_1}$ changes this landscape by aiming at a more \emph{compact} method in attaining the wobbling spectrum. In order to better understand this procedure, a specific case-study for $^{163}$Lu will be employed in the following subsection and a generalization to other isotopes can be inherently adopted going further.

\subsection{Variational States}

The isotope $^{163}$Lu has been studied successfully by the team within the $\mathbf{W_0}$ and also through the current $\mathbf{W_1}$ model. In both approaches one considered the wobbling spectrum as being composed of four TSD bands. In $\mathbf{W_0}$ the bands were ground (TSD1) and three excited (TSD2-3-4), which was `in line' with other studies on this isotope \cite{schonwasser2003one,jensen2004coexisting,tanabe2008selection}. The general characteristics of the four triaxial bands and their particle + core structure were already described at the beginning of the chapter from the perspective of $\mathbf{W_0}$. Experimentally, the collective structure of $^{163}$Lu is summarized in Table \ref{lu-163-table-info}.
\begin{table}
    \centering
    \resizebox{\textwidth}{!}{%
    \begin{tabular}{|c|c|c|c|c|c|c|}
    \hline
    Band & Spins & $\pi$ & $\alpha$ & $\mathcal{Q}_p:\ \pi(l_j)$ & $\mathscr{C}:\mathbf{W_0}$ & $\mathscr{C}:\mathbf{W_1}$ \\ \hline
    TSD1 & $13/2,17/2 \dots 97/2$ & $+1$ & $+1/2$ & $\pi(i_{13/2})$ & $0^+,2^+,4^+,\dots$ & $0^+,2^+,4^+,\dots$ \\ \hline
    TSD2 & $27/2,31/2 \dots 91/2$ & $+1$ & $-1/2$ & $\pi(i_{13/2})$ & $\text{TSD1}+1\Gamma^\dagger$ & $1^+,3^+,5^+,\dots$ \\ \hline
    TSD3 & $33/2,37/2 \dots 85/2$ & $+1$ & $+1/2$ & $\pi(i_{13/2})$ & $\text{TSD1}+2\Gamma^\dagger$ & $\text{TSD2}+\Gamma^\dagger$ \\ \hline
    TSD4 & $47/2,51/2 \dots 83/2$ & $-1$ & $-1/2$ & $\pi(h_{9/2})$  & $\text{TSD1}+3\Gamma^\dagger$ & $0^+,2^+,4^+,\dots$ \\ \hline
    \end{tabular}%
    }
    \caption{The data concerning spins, parity, and signature assignments for $^{163}$Lu from experimental measurements \cite{reich2010nuclear}. The last two columns represent the key difference between formalisms $\mathbf{W_0}$ and $\mathbf{W_1}$, namely the particle + core coupling. Note that in this new approach, the band TSD3 is the one-phonon band built on top of TSD2.}
    %The last two columns represent the quasi-particle $\mathcal{Q}$ and the core $\mathscr{C}$ (according to the notations from Table \ref{notation-table-wobbling}) for each TSD band as per $\mathbf{W_0}$. Note that $\mathcal{Q}$ is different for TSD4 and the core angular momentum $\mathscr{C}$ is only defined for TSD1 because the excited bands are created through $\Gamma^\dagger$ phonon operator (see Section \ref{foundation}).}
    \label{lu-163-table-info}
\end{table}

Within the \emph{renormalization} of $\mathbf{W_1}$ for the nucleus $^{163}$Lu, one applies the variational principle (i.e., Eq. \ref{tdve-approach-w1}) for all states in TSD1 and also TSD2. Additionally, since the coupling scheme $\mathcal{Q}+\mathscr{C}$ is different for TSD4 (i.e., the proton with $j_2=9/2$ couples with the even-even core as opposed to the $j_1=13/2$ proton for the other bands) then the VP can be employed for TSD4 as well. The states from the band TSD3 emerge as excited wobbling states, which are formed by the action of a phonon operator on TSD2 (similar to the one defined in $\mathbf{W_0}$). \emph{Indeed, this successive application of TDVE in obtaining variational states is a remarking aspect of} $\mathbf{W_1}$.

All the states from TSD1 are produced by coupling the $j_1=13/2$ odd proton from $i$-shell with a triaxial even-even core having the spin sequence $\mathscr{C}_1=0,2,4,\dots$, forming a zero-phonon wobbling band. Both the core and the single-particle have positive parity. On the other hand, all states in TSD2 are built by the same odd-particle, but with a different \emph{core}: $\mathscr{C}_2=1,3,5\dots$, which also has positive parity. Notice the even/odd dissimilarity between the two angular momentum sequences of the cores $\mathscr{C}_1$ and $\mathscr{C}_2$. The wobbling bands TSD1 and TSD2 from $^{163}$Lu can be explained by means of a semiclassical model, through an even-odd staggering of states $(0^+,1^+)$, $(2^+,3^+)$ of collective nature. Moreover, the negative parity states belonging to the band TSD4 can be explored semi-classically via TDVE with the collective core $\mathscr{C}_1=0,2,4$ coupled to a negative parity proton $j_2=9/2$.

In terms of quasi-particle + triaxial core couplings, it is remarkable the fact that the `final picture' for $^{163}$Lu is regarded as a positive parity core of even spin states that generates the bands TSD1 and TSD4 (i.e., $\mathscr{C}_1$), a core with odd spin states of positive parity states forming TSD2 (i.e., $\mathscr{C}_2$), and finally the one-phonon states in TSD3, which are activated from the ground band TSD2. This is emphasized within the last two columns of Table \ref{lu-163-table-info}. One can encapsulate this current renormalization under the following set of \emph{equalities}:
\begin{align}
    \{TSD1\}&\equiv\left\{\mathscr{C}_1\left[0^+,2^+,4^+,\dots\right] \otimes \mathcal{Q}_1[j_1=13/2^+]\right\}\ ,\ \nonumber\\
    \{TSD2\}&\equiv\left\{\mathscr{C}_2\left[1^+,3^+,5^+,\dots\right] \otimes \mathcal{Q}_1[j_1=13/2^+]\right\}\ ,\ \nonumber\\
    \{TSD4\}&\equiv\left\{\mathscr{C}_1\left[0^+,2^+,4^+,\dots\right] \otimes \mathcal{Q}_2[j_2=9/2^-]\right\}\ ,
    \label{renormalized-bands-structure-TSD124}
\end{align}
for TSD1-2-4 and:
\begin{align}
    \{TSD3\left[\dots,I^+,(I+2)^+,\dots\right]\}\equiv\left\{TSD2\left[\dots,(I-1)^+,(I+1)^+,\dots\right]+\Gamma^\dagger\right\}\ ,
    \label{renormalized-bands-structure-TSD3}
\end{align}
for TSD3. The two odd protons from Table \ref{lu-163-phonon-numbers} (quasi-particles of particle character) are denoted with $\mathcal{Q}_1$ and $\mathcal{Q}_2$ in Eqs. \ref{renormalized-bands-structure-TSD124} - \ref{renormalized-bands-structure-TSD3} and these notations will be used hereafter. Thus, the two equations constitute the renormalization of $\mathbf{W_1}$, which is successfully applied not only to $^{163}$Lu but also other isotopes.% and their angular momenta will be labelled $j_1=13/2$ for $\mathcal{Q}_1$ and $j_2=9/2$ for $\mathcal{Q}_2$. The $p$-abbreviation from the quasi-particle notation (as per the rules in Table \ref{notation-table-wobbling}) will be dismissed from now on because all the discussed isotopes have odd the odd nucleon with \emph{particle} character. The labelling for both $\mathcal{Q}_1$ and $\mathcal{Q}_2$ is given in Table \ref{lu-163-phonon-numbers}.

With the Eqs. \ref{renormalized-bands-structure-TSD124} - \ref{renormalized-bands-structure-TSD3} as `recipes', one can obtain the wobbling energies for this isotope as per Eq. \ref{tsd-bands-general-spectrum} by adopting the spin values $I$ and the wobbling phonon numbers $n_{w_1}$ and $n_{w_2}$ of each band in particular. These values are given in Table \ref{lu-163-phonon-numbers}. Notice that for TSD1-2-4 the pair of phonon numbers are $(0,0)$, due to all of them being variational ground states. Moreover, the TSD3 states are activated by the phonon number $n_{w_1}$ (i.e., $n_{w_2}=0$), which is consistent with the theory from $\mathbf{W_0}$ where the excited wobbling states were generated only through the phonon operator for $\Omega_1$. As it will be seen, all four bands have the wobbling frequency $\Omega_2$ in the zero-point energy, that is $\frac{1}{2}\Omega_2^I$. In reference to TSD3, given that the band is obtained as one-phonon excitations built on top of TSD2, then by acting with the operator on a state $I$ from TSD2 it will increase the angular momentum by one unit (see Fig. \ref{phonon-operator-schematic} from Section \ref{foundation}). Consequently, for a state $I\in\{TSD3\}$, the phonon operator is indeed $\mathcal{F}_{10}^{I-1}$ and $I-1\in\{TSD2\}$.
\begin{table}
    \centering
    \resizebox{\textwidth}{!}{%
    \begin{tabular}{|c|c|c|c|c|c|c|}
    \hline
    Bands & $n_{w_1}$ & $n_{w_2}$ & $\mathcal{F}_{n_{w_1}n_{w_2}}^I$ & $I_0$    & $I_t$    & $\mathcal{Q}$    \\ \hline
    TSD1  & $0$       & $0$       & $\mathcal{F}_{00}^I=\frac{1}{2}\left(\Omega_1^I+\Omega_2^I\right)$ & $13/2^+$ & $97/2^+$ & $j_1^\pi=13/2^+\stackrel{not}{\equiv}\mathcal{Q}_1$ \\ \hline
    TSD2  & $0$       & $0$       & $\mathcal{F}_{00}^I=\frac{1}{2}\left(\Omega_1^I+\Omega_2^I\right)$ & $27/2^+$ & $91/2^+$ & $j_1^\pi=13/2^+\stackrel{not}{\equiv}\mathcal{Q}_1$ \\ \hline
    TSD3  & $1$       & $0$       & $\mathcal{F}_{10}^{I-1}=\frac{3}{2}\Omega_1^{I-1}+\frac{1}{2}\Omega_2^{I-1}$ & $33/2^+$ & $85/2^+$ & $j_1^\pi=13/2^+\stackrel{not}{\equiv}\mathcal{Q}_1$ \\ \hline
    TSD4  & $0$       & $0$       & $\mathcal{F}_{00}^I=\frac{1}{2}\left(\Omega_1^I+\Omega_2^I\right)$ & $47/2^-$ & $83/2^-$ & $j_2^\pi=9/2^-\stackrel{not}{\equiv}\mathcal{Q}_2$  \\ \hline
    \end{tabular}%
    }
    \caption{The wobbling phonon numbers of $^{163}$Lu that correspond to the phonon frequencies $\Omega_1$ and $\Omega_2$, respectively (see Eq. \ref{phononic-term-tsd-energies}). For completeness the first ($I_0$) and last (i.e., $I_t$ for \emph{terminus}) spin states of each band are given. The quasi-particles involved in the particle + rotor coupling are denoted according to Eqs. \ref{renormalized-bands-structure-TSD124} - \ref{renormalized-bands-structure-TSD3}. See text for the $\mathcal{F}_{10}^{I-1}$ term.}
    \label{lu-163-phonon-numbers}
\end{table}

Taking the information comprised in Table \ref{lu-163-phonon-numbers} in order to compute the phonon terms (Eq. \ref{phononic-term-tsd-energies}) and by using the general energy formula obtained in Eq. \ref{tsd-bands-general-spectrum}, one can determine the wobbling spectrum of $^{163}$Lu through the following set of equations:
\begin{align}
    E_{I,0,0}^\text{TSD1}=&\epsilon_{13/2}+\mathcal{H}_\text{min}^I+\mathcal{F}_{00}^I=\epsilon_1+\mathcal{H}_\text{min}^I+\frac{1}{2}\left(\Omega_1^I+\Omega_2^I\right)\ ,\nonumber\\
    E_{I,0,0}^\text{TSD2}=&\epsilon_{13/2}+\mathcal{H}_\text{min}^I+\mathcal{F}_{00}^I=\epsilon_1+\mathcal{H}_\text{min}^I+\frac{1}{2}\left(\Omega_1^I+\Omega_2^I\right)\ ,\nonumber\\
    E_{I,1,0}^\text{TSD3}=&\epsilon_{13/2}+\mathcal{H}_\text{min}^{I-1}+\mathcal{F}_{10}^{I-1}=\epsilon_1+\mathcal{H}_\text{min}^{I-1}+\frac{1}{2}\left(3\cdot\Omega_1^{I-1}+\Omega_2^{I-1}\right)\ ,\nonumber\\
    E_{I,0,0}^\text{TSD4}=&\epsilon_{9/2}+\mathcal{H}_\text{min}^{I}+\mathcal{F}_{00}^{I}=\epsilon_2+\mathcal{H}_\text{min}^{I}+\frac{1}{2}\left(\Omega_1^{I}+\Omega_2^{I}\right)\ ,
    \label{lu163-absolute-energies-tsd1234}
\end{align}
where $\epsilon_{13/2}$ and $\epsilon_{9/2}$ are the single-particle shell energies for $\mathcal{Q}_1$ and $\mathcal{Q}_2$, respectively. For each band, the spins are the ones from Table \ref{lu-163-table-info}. Keep in mind that the wobbling frequencies and the minimal energy both depend on the single-particle angular momentum $j$ as well, so it must be properly inserted for each band as per Table \ref{lu-163-phonon-numbers}. An alternative way of depicting the energy from TSD3 is through TSD2:
\begin{align}
    E_{I,1,0}^\text{TSD3}=E_{I-1,0,0}^\text{TSD2}+\Omega_1^{I-1}\ .\nonumber
\end{align}

In the numerical calculations the set of formulas obtained in Eq. \ref{lu163-absolute-energies-tsd1234} represent the absolute values. As such, they will be subtracted from the band-head energy of $^{163}$Lu, obtaining thus the spectrum of \emph{excitation energies} (recall Eq. \ref{excitation-energy-general-formula}). The band-head energy $E_{I_b,0,0}^\text{TSD1}$ is energy corresponding to the spin state $I_b=13/2$ from TSD1. Its expression is as follows:
\begin{align}
    E_{13/2,0,0}^\text{TSD1}=\epsilon_{13/2}+\mathcal{H}_\text{min}^{13/2}+\mathcal{F}_{00}^{13/2}=\epsilon_{13/2}+\mathcal{H}_\text{min}^{13/2}+\frac{1}{2}\left(\Omega_1^{13/2}+\Omega_2^{13/2}\right)\ , \nonumber
\end{align}

An important aspect of working with the excitation energies within the numerical computations is that the single-particle energies will practically cancel out. Only for TSD4 there will be the constant term $\epsilon_{9/2}-\epsilon_{13/2}=-0.344$ MeV \cite{raduta2020towards}, which is just the difference for the spherical shell model states of $\mathcal{Q}_1$ and $\mathcal{Q}_2$. Regarding the two wobbling frequencies in the case of $^{163}$Lu their expressions are:
\begin{align}
    \Omega_{1,2}=\left[\frac{1}{2}\left(-B\mp\sqrt{B^2-4C}\right)\right]^{1/2}\ ,
    \label{wobbling-frequencies-Omega-1-2}
\end{align}
such that the frequency ordering $\Omega_1<\Omega_2$ holds true. The minimal energy term $\mathcal{H}_\text{min}$ is given in terms of the three inertia factors, the single-particle potential strength $V$ and the triaxiality $\gamma$ as:
\begin{align}
    \mathcal{H}_\text{min}^I=\left(A_2+A_3\right)\frac{I+j}{2}+A_1(I-j)^2-V\frac{2j-1}{j+1}\sin\left(\gamma+\frac{\pi}{6}\right)\ .
    \label{minimal-energy-term-hmin}
\end{align}

\subsection{Fitting Parameters}

So far, a compact wobbling spectrum was obtained for $^{163}$Lu within the $\mathbf{W_1}$ formalism via Eq. \ref{lu163-absolute-energies-tsd1234}. With the expression of $\mathcal{H}_\text{min}^I$ from Eq. \ref{minimal-energy-term-hmin} and the wobbling frequencies evaluated through Eq. \ref{wobbling-frequencies-Omega-1-2} the analytic formulas for the TSD bands are acquired. In order to get numerical results for the excitation energies, a fitting procedure will be adopted with a set of free parameters which need to be employed first.

From the structure of the classical energy function (or even the initial quantal Hamiltonian) it can be seen that the rotational term is expressed in terms of the inertia factors $A_k$ and the single-particle term is explicitly specified through the single-particle strength $V$ and triaxiality $\gamma$. Therefore, it is best suited to pick the following quantities as fitting parameters throughout the numerical computations:
\begin{enumerate}
    \item moments of inertia: $\mathcal{I}_1$, $\mathcal{I}_2$, $\mathcal{I}_3$
    \item single-particle potential strength $V$
    \item triaxiality parameter $\gamma$
\end{enumerate}
whose values will be denoted as:
\begin{align}
    \mathcal{P}_\text{fit}=\left[\mathcal{I}_1,\mathcal{I}_2,\mathcal{I}_3,V,\gamma\right]\ .
    \label{fitting-parameters-p-fit}
\end{align}

The actual fitting technique consists of finding the set $\mathcal{P}_\text{fit}$ that best reproduces the experimental data concerning the wobbling spectrum. The numerical implementation aims at determining $\mathcal{P}_\text{fit}$ such that \cite{poenaru2021extensive1}:
\begin{align}
    \chi^2=\frac{1}{N_T}\sum_i\frac{\left(E_\text{exp}^{(i)}-E_\text{th}^{(i)}\right)^2}{E_\text{exp}^{(i)}}\ ,
    \label{chi-2-fitting-function}
\end{align}
is \emph{minimal}, where $N_T$ represents the total number of states for the isotope. In the case of $^{163}$Lu, the $\chi^2$-function will contain all states belonging to TSD1-2-3 and TSD4. As an observation, the band-head energy $E_{13/2,0,0}^\text{TSD1}$ will be excluded because its corresponding excitation energy is null. Obviously, the energy $E_\text{exp}^{(i)}$ is in fact the \emph{experimental excitation energy} and it is evaluated by subtracting from all energies the band-head level of $1.738\ \text{MeV}$. From a computational standpoint, the problem thereby consists of a minimization method that has to be tackled on the $\chi^2$-function.

This new approach is different than $\mathbf{W_0}$ in the sense that the moments of inertia are now considered as separate (independent) `free' values, whereas in \cite{raduta2017semiclassical} one of the parameters was the scaling factor $1/\mathcal{I}_0$ emerging from the adoption of the rigid-body MOI (see Eq. \ref{eq-irrotational-rigid-mois}). Moreover, in the present model the three moments of inertia have no angular momentum dependence, i.e., they are fixed across the entire spin range of each isotope. On the other side, unlike $\mathbf{W_0}$ where $\gamma$ and $\beta$ were a priori fixed with values from literature, this re-interpretation obtains $\gamma$ self-consistently from the fitting procedure. Both approaches give consider $V$ as a free parameter and its value is obtained from the minimization of the $\chi^2$-function.

The strong argument in using a set of free MOI is related to the fact that the \emph{real} moments of inertia for the nucleus are neither irrotational nor rigid, but they satisfy the relation $\mathcal{I}^\text{irr}<\mathcal{I}^\text{exp}<\mathcal{I}^\text{rig}$ (recall discussion in Chapter \ref{chapter-3} and Eq. \ref{experimental-MOI-vs-rig-irr}). A study of the three moments of inertia in regards to a possible change in magnitude at higher spins due to Coriolis or even pairing interaction is excluded within this technique as the initial quantal Hamiltonian does not contain such terms. Lastly, the $\beta_2$ quadrupole deformation parameter does not appear explicitly throughout the analytical description, since it is somewhat \emph{embedded} in the potential strength $V$. Concluding, the quality of the fitting procedures applied to each isotope will reflect the deformation of the nuclei (i.e., degree of triaxiality and prolate/oblate shape).

\section{Numerical Results in Lu Isotopes}

In the previous section, one formulated: i) the renormalization procedure that is the cornerstone in the $\mathbf{W_1}$ formalism via the Eqs. \ref{renormalized-bands-structure-TSD124} - \ref{renormalized-bands-structure-TSD3} ii) a set of analytical results of the wobbling bands for $^{163}$Lu provided by Eq. \ref{lu163-absolute-energies-tsd1234} iii) the relevant parameter set $\mathcal{P}_\text{fit}$ via Eq. \ref{fitting-parameters-p-fit} and finally iv) the actual fitting procedure that will be employed as application: the $\chi^2$-function from Eq. \ref{chi-2-fitting-function}.

In the following section, the numerical results obtained for several Lu isotopes besides $^{163}$Lu will be presented, containing data such as excitation energies but also other relevant quantities such as:
\begin{enumerate}
    \item alignments
    \item dynamic moments of inertia
    \item energies relative to a reference rotor
\end{enumerate}

Naturally, the calculations concerning the transition probabilities for each nucleus will be also provided, making a comparison with the available experimental data. The team's work with the $\mathbf{W_1}$ formalism concluded in two research papers \cite{raduta2020approach,raduta2020towards}, in which the isotopes $^{161,163,165,167}$Lu were studied.

First step is to numerically reproduce the spectrum given by Eq. \ref{tsd-bands-general-spectrum}. The set of energies sketched for $^{163}$Lu in Eq. \ref{lu163-absolute-energies-tsd1234} can be similarly adopted to the other isotopes if one knows the two wobbling phonon numbers, the quasi-particle + core coupling scheme, and the spin + parity assignments for each experimental wobbling band. The $\mathbf{W_1}$ model interprets bands TSD1 and TSD2 in $^{163}$Lu as being ground-states obtained variationally from the coupling of the same odd particle with two different cores. The remarking feature in this theory is that the same principle can be applied to the other isotopes: namely the pair $(TSD1,TSD2)$ is regarded as zero-phonon bands that originate from different even-even cores. Furthermore, some fitting rules should be taken into account for every isotope such that the numerical implementations will be consistent across the entire group of nuclei. Taking into consideration the renormalization from Eqs. \ref{renormalized-bands-structure-TSD124} - \ref{renormalized-bands-structure-TSD3}, these rules are given as follows:
\begin{itemize}
    \item $^{161}$Lu
    \begin{itemize}
        \item There are only two TSD bands, emerging from the coupling of $\mathcal{Q}_1$ with the even-even core
        \item In regards to their nature, both bands are ground-states (zero-phonon)
        \item As core states, TSD1 has even spins (i.e., $\mathscr{C}_1$) and TSD2 has odd spin spins (i.e., $\mathscr{C}_2$)
    \end{itemize}
    \item $^{163}$Lu
    \begin{itemize}
        \item Four TSD bands with the structure described in Table \ref{lu-163-table-info}
        \item three different core + odd-particle couplings: $(\mathcal{Q}_1+\mathscr{C}_1)\ ;\ TSD1$, $(\mathcal{Q}_1+\mathscr{C}_2)\ ;\ TSD2$, and $(\mathcal{Q}_2+\mathscr{C}_1)\ ;\ TSD4$
    \end{itemize}
    \item $^{165}$Lu
    \begin{itemize}
        \item The three TSD bands are created from a single quasi-particle (i.e., the proton $\mathcal{Q}_1=i_{13/2}$), which couples to a core of even states (TSD1) and a core with odd states (TSD2).
        \item $(TSD1,TSD2)$ are ground-bands
        \item TSD3 is a one-phonon band that is obtained as wobbling excitations built on top of TSD2
    \end{itemize}
    \item $^{167}$Lu
    \begin{itemize}
        \item treated in a similar fashion as the $^{161}$Lu isotope
    \end{itemize}
\end{itemize}

One can summarize the characteristics of each isotope mentioned above into a set of tables where, for the sake of completeness, the experimental data concerning spin and parity are indicated. Tables \ref{lu-161-experimental-data-table} - \ref{lu-167-experimental-data-table} contain these values. Note that besides the fourth TSD band from $^{163}$Lu, every nucleus has wobbling excitations which emerge from the coupling of the $\mathcal{Q}_1$ single-particle with an even-even core (containing either even or odd spin states). Indeed, for TSD4 the coupling involved in the triaxial structure is between the negative parity proton $\mathcal{Q}_2$ and the positive parity core $\mathscr{C}_1$. Since the previous section contain the relevant data concerning $^{163}$Lu (recall Tables \ref{lu-163-table-info} and \ref{lu-163-phonon-numbers}), it has been omitted here.
\begin{table}
    \centering
    \resizebox{\textwidth}{!}{%
    \begin{tabular}{|c|c|c|c|c|c|}
    \hline
    Band & Spins                        & $\mathcal{Q}$ & $\mathscr{C}$  & $(n_{w_1},n_{w_2})$ & $I_b$                   \\ \hline
    TSD1 & $21/2^+,25/2^+,\dots,89/2^+$ & $j_1=13/2^+$  & $4^+,6^+,8^+\dots$   & $(0,0)$             & \multirow{2}{*}{$21/2$} \\ \cline{1-5}
    TSD2 & $31/2^+,35/2^+,\dots,79/2^+$ & $j_1=13/2^+$  & $9^+,11^+,13^+\dots$ & $(0,0)$             &                         \\ \hline
    \end{tabular}%
    }
    \caption{The data concerning spin and parity assignments for $^{161}$Lu that are required for the numerical calculations of the excitation energies. The wobbling phonon numbers from Eq. \ref{phononic-term-tsd-energies} are also shown. For both bands there is only one single-particle, i.e., $\mathcal{Q}_1$. The even/odd dissimilarity between core states of the two bands are emphasized through the fourth column ($\mathscr{C}$). The band-head $I_b$ for the isotope is also given in the last column.}
    \label{lu-161-experimental-data-table}
\end{table}
\begin{table}
    \centering
    \resizebox{\textwidth}{!}{%
    \begin{tabular}{|c|c|c|c|c|c|}
    \hline
    Band & Spins                        & $\mathcal{Q}$ & $\mathscr{C}$         & $(n_{w_1},n_{w_2})$ & $I_b$                   \\ \hline
    TSD1 & $25/2^+,29/2^+,\dots,89/2^+$ & $j_1=13/2^+$  & $6^+,8^+,10^+\dots$   & $(0,0)$             & \multirow{3}{*}{$25/2$} \\ \cline{1-5}
    TSD2 & $35/2^+,39/2^+,\dots,91/2^+$ & $j_1=13/2^+$  & $11^+,13^+,15^+\dots$ & $(0,0)$             &                         \\ \cline{1-5}
    TSD3 & $41/2^+,45/2^+,\dots,81/2^+$ & $j_1=13/2^+$  & $\text{TSD2}+\Gamma^\dagger$ & $(1,0)$             &                         \\ \hline
    \end{tabular}%
    }
    \caption{The data concerning spin and parity assignments for $^{165}$Lu that are required for the numerical calculations of the excitation energies. The wobbling phonon numbers from Eq. \ref{phononic-term-tsd-energies} are also shown. For all three TSD bands there is only one single-particle, i.e., $\mathcal{Q}_1$. The even/odd dissimilarity between core states of the two bands are emphasized through the fourth column ($\mathscr{C}$). The band-head $I_b$ for the isotope is also given in the last column.}
    \label{lu-165-experimental-data-table}
\end{table}
\begin{table}
    \centering
    \resizebox{\textwidth}{!}{%
    \begin{tabular}{|c|c|c|c|c|c|}
    \hline
    Band & Spins                        & $\mathcal{Q}$ & $\mathscr{C}$         & $(n_{w_1},n_{w_2})$ & $I_b$                   \\ \hline
    TSD1 & $25/2^+,29/2^+,\dots,89/2^+$ & $j_1=13/2^+$  & $6^+,8^+,10^+\dots$   & $(0,0)$             & \multirow{2}{*}{$25/2$} \\ \cline{1-5}
    TSD2 & $35/2^+,39/2^+,\dots,91/2^+$ & $j_1=13/2^+$  & $11^+,13^+,15^+\dots$ & $(0,0)$             &                         \\ \hline
    \end{tabular}%
    }
    \caption{The data concerning spin and parity assignments for $^{167}$Lu that are required for the numerical calculations of the excitation energies. The wobbling phonon numbers (from Eq. \ref{phononic-term-tsd-energies}) are also shown. For both bands, there is only one single-particle, i.e., $\mathcal{Q}_1$. The even/odd dissimilarity between core states of the two bands are emphasized through the fourth column ($\mathscr{C}$). The band-head $I_b$ for the isotope is also given in the last column.}
    \label{lu-167-experimental-data-table}
\end{table}

Putting together the set of absolute energies from Eq. \ref{lu163-absolute-energies-tsd1234} and the discussion in reference to the excitation energy of a state $I$, one can define a general formula that will illustrate the analytical formalism for the wobbling states used in the fitting procedure:
\begin{align}
    E_{I_b}^\text{TSD1}(^A\text{Lu\ ;\ abs})=&\epsilon_{13/2}+\mathcal{H}_\text{min}^{I_b}+\mathcal{F}_{00}^{I_b}\ ,\nonumber\\
    E_{I,n_{w_1},n_{w_2}}^\text{TSDK}(^A\text{Lu\ ;\ exc})=&\left(\epsilon_{j}+\mathcal{H}_\text{min}^I+\mathcal{F}^I_{n_{w_1}n_{w_1}}\right)-E_{I_b}^\text{TSD1}(^A\text{Lu\ ;\ abs})\ ,
    \label{general-excitation-energy-fitting-model}
\end{align}
where $^A$Lu represents the isotope that is evaluated (i.e., $A\in\{161,163,165,167\}$), $I$ is the state belonging to a triaxial band $K$ from $^A$Lu ($I\in\text{TSDK}$), the labels `$\text{exc}$/$\text{abs}$' symbolize the excitation/absolute energy, respectively, and $I_b$ is the band-head from the TSD1 band of that particular nucleus, which is given explicitly in the last column of Tables \ref{lu-161-experimental-data-table} - \ref{lu-167-experimental-data-table}. Note that regarding the energy $E_{I_b}^\text{TSD1}(^A\text{Lu\ ;\ abs})$, the subscripts $n_{w_1}$ and $n_{w_2}$ have been dropped entirely because all ground-bands are zero-phonon bands. Moreover, the quasi-particle involved in the particle + rotor coupling for the ground bands is $\mathcal{Q}_1$ (Table \ref{lu-163-phonon-numbers}) across the entire mass region.

\subsection{Numerical Parameters}

With the experimental excitation energies evaluated per every nucleus, the minimization procedure of Eq. \ref{chi-2-fitting-function} can be readily applied. Note that since the variational states from TSD4 are obtained from different core + particle polarization, a separate fit for the states of this band will be made \cite{raduta2020towards}. Results are presented in Table \ref{numerical-fitting-parameters-Lu-isotopes}, where in order to appraise the quality of the fits, the root-mean-square (RMS) error for each band is also shown. Based on these numbers, it can be concluded that the description of the excitation energies is quite good across the entire mass region.
\begin{table}
    \centering
    \resizebox{\textwidth}{!}{%
    \begin{tabular}{cccccccccc}
    \hline
    \hline
    Isotope & $\mathcal{Q}$ & Bands & $\mathcal{I}_1$ $\left[\hbar^2/\text{MeV}\right]$ & $\mathcal{I}_2$ $\left[\hbar^2/\text{MeV}\right]$ & $\mathcal{I}_3$ $\left[\hbar^2/\text{MeV}\right]$ & $V$ [MeV] & $\gamma$ [$^\circ$] & n.o.s & $E_\text{rms}$ [MeV] \\ \hline
    $^{161}$Lu & $j_1=13/2$ & TSD1-2 & 87.555 & 2.773  & 22.744 & 2.933 & 20    & 29 & 0.168 \\
    $^{163}$Lu & $j_1=13/2$ & TSD1-3 & 63.2   & 20     & 10     & 3.1   & 17    & 52 & 0.264 \\
               & $j_2=9/2$  & TSD4   & 67     & 34.5   & 50     & 0.7   & 17    & 10 & 0.057 \\
    $^{165}$Lu & $j_1=13/2$ & TSD1-3 & 77.295 & 16.184 & 4.399  & 1.673 & 20    & 42 & 0.125 \\
    $^{167}$Lu & $j_1=13/2$ & TSD1-2 & 87.032 & 10.895 & 3.758  & 8.167 & 19.48 & 30 & 0.165 \\
     \hline
     \hline
    \end{tabular}%
    }
    \caption{The fitting parameters $\mathcal{P}_\text{fit}$, i.e., the moments of inertia, the single-particle potential strength, and the triaxiality $\gamma$ for each Lu isotope. Numerical results were obtained through the fitting procedure described in text. The number of wobbling states and the root-mean-square error are also given.}
    \label{numerical-fitting-parameters-Lu-isotopes}
\end{table}

To evidentiate the dependence of the three MOI with respect to the atomic mass $A$, the numerical values of $\mathcal{I}_k$ from Table \ref{numerical-fitting-parameters-Lu-isotopes} are graphically represented in Fig. \ref{MOIs-A-behavior-Lu-isotopes}, where one remarks a change in their ordering at $A=163$. Indeed, such a change can be regarded as a \emph{phase transition}. Moreover, there is a similar ordering for $^{161}$Lu and the band TSD4 from $^{163}$Lu, while the order $\mathcal{I}_2$ and $\mathcal{I}_3$ is reversed for the other nuclides.
\begin{figure}
    \centering
    \includegraphics[scale=0.65]{Chapters/Figures/fig1_moivalues.pdf}
    \caption{The fitted MOI from Table \ref{numerical-fitting-parameters-Lu-isotopes} plotted as a function of the mass number $A$. The unfilled symbols represent the band TSD4 of $^{163}$Lu.}
    \label{MOIs-A-behavior-Lu-isotopes}
\end{figure}

From the obtained MOI, one sees that the maximal value is corresponding to $1$-axis. As per the discussion regarding wobbling regime from Chapter \ref{chapter-5}, the interaction of the core with the odd particle seems to drive the system to a longitudinal-like motion. However, in the `final picture', one is unable to assert that these isotopes behave as transverse or longitudinal wobblers. This is due to the fact that within the current model, the MOI do not have angular momentum dependence, and it is impossible to pinpoint regions across the range of $I$ where transitions from transverse to longitudinal wobbling might occur. Certainly, it could be that at low rotational motion the ordering is $\mathcal{I}_2>\mathcal{I}_1>\mathcal{I}_3$ and then gradually shift to $\mathcal{I}_1>\mathcal{I}_{2,3}$ (decrease of $\mathcal{I}_2$ via the pairing interaction and increase of $\mathcal{I}_1$ due to alignment). Nevertheless, $\mathbf{W_1}$ only shows the normalized MOI as a \emph{final} and not a \emph{full} picture of the wobbling behavior. Therefore, through the current formalism, a possible transverse regime near the low-spin limit for the Lu isotopes is not excluded.

In the survey made in Chapter \ref{chapter-5} that concluded with Fig. \ref{wobbling-diagram-chart}, it was shown that $\gamma\approx 20^\circ$ for all Lu isotopes. Looking back at Table \ref{numerical-fitting-parameters-Lu-isotopes} it can be seen that except for $^{163}$Lu, the agreement between the experimental values and fitted results is quite remarkable. Also noteworthy is the fact that TSD1-2-3 and TSD4 in $^{163}$Lu have similar triaxiality, although their inertia moments differ in magnitude. The single-particle potential strength given by the fit is also graphically shown for each nucleus in Fig. \ref{fig-V-param-fitting-procedure}. There is quite a sharp increase in $V$ for $^{167}$Lu from its neighbor $^{165}$Lu, which can be explained by means of collectiveness. Indeed, it is possible that the two extra neutrons could induce a stronger quadrupole field that is \emph{experienced} by the odd-particle. Furthermore, this \emph{structural} change of the field could be enforced by the drastic evolution in the shape of the triaxial ellipsoid (i.e., the values of $\mathcal{I}_{2,3}$ decrease while $\mathcal{I}_1$ increases compared to neighboring nuclei). Additional comments concerning the single particle potential strength $V$ will be made in a separate discussion.
\begin{figure}
    \centering
    \includegraphics[scale=0.8]{Chapters/Figures/V-param-fitting.pdf}
    \caption{Fitted values for the single particle potential strength $V$ as function of the mass number $A$. The separate procedure for TSD4 in $^{163}$Lu is marked by the blue data point. The dotted line that connects each point is just to guide the eye. For an illustrative purpose, the average value across the entire mass region is also represented by the dashed horizontal line. The numerical values correspond to the ones given in Table \ref{numerical-fitting-parameters-Lu-isotopes}.}
    \label{fig-V-param-fitting-procedure}
\end{figure}

\subsection{Energies}

As discussed, the fitting procedure consists of a minimization for the $\chi^2$-function given by Eq. \ref{chi-2-fitting-function}, from which one obtains the `best' parameter set $\mathcal{P}_\text{fit}$. With this parameter set, calculating the theoretical values from Eq. \ref{general-excitation-energy-fitting-model} is straightforward. In Fig. \ref{excitation-energies-th-161Lu}, the excitation energies of $^{161}$Lu are directly compared with the available experimental data.
\begin{figure}
    \centering
    \includegraphics[width=0.49\textwidth]{Chapters/Figures/Lu-exp-energies/fig2a_lu161.pdf}
    \includegraphics[width=0.49\textwidth]{Chapters/Figures/Lu-exp-energies/fig2b_lu161.pdf}
    \caption{The experimental and theoretical excitation energies provided by Eq. \ref{general-excitation-energy-fitting-model} for the two wobbling bands in $^{161}$Lu. Experimental data are taken from Ref \cite{bringel2005evidence}.}
    \label{excitation-energies-th-161Lu}
\end{figure}

The four triaxial bands of wobbling nature in $^{163}$Lu are plotted in Fig. \ref{excitation-energies-th-163Lu}, where an agreement with the experimental data can be seen across the entire spin range. When comparing the results from here with the ones of $\mathbf{W_0}$ \cite{raduta2017semiclassical}, there is a clear improvement of the present semi-classical model.
\begin{figure}
    \centering
    \includegraphics[width=0.49\textwidth]{Chapters/Figures/Lu-exp-energies/fig3a_lu163.pdf }
    \includegraphics[width=0.49\textwidth]{Chapters/Figures/Lu-exp-energies/fig3b_lu163.pdf }
    \includegraphics[width=0.49\textwidth]{Chapters/Figures/Lu-exp-energies/fig3c_lu163.pdf }
    \includegraphics[width=0.49\textwidth]{Chapters/Figures/Lu-exp-energies/fig3d_lu163.pdf }
    \caption{The experimental and theoretical excitation energies provided by Eq. \ref{general-excitation-energy-fitting-model} for the two wobbling bands in $^{163}$Lu. Experimental data are taken from Ref \cite{reich2010nuclear}.}
    \label{excitation-energies-th-163Lu}
\end{figure}

A graphical representation showing the wobbling bands of $^{165}$Lu is depicted in Fig. \ref{excitation-energies-th-165Lu} as function of the total spin. From the comparison it can be seen that the two ground-bands and the one-wobbling-phonon band are fairly well described through the current model.
\begin{figure}
    \centering
    \includegraphics[width=0.32\textwidth]{Chapters/Figures/Lu-exp-energies/fig4a_lu165.pdf}
    \includegraphics[width=0.32\textwidth]{Chapters/Figures/Lu-exp-energies/fig4b_lu165.pdf}
    \includegraphics[width=0.32\textwidth]{Chapters/Figures/Lu-exp-energies/fig4c_lu165.pdf}
    \caption{The experimental and theoretical excitation energies provided by Eq. \ref{general-excitation-energy-fitting-model} for the three wobbling bands in $^{165}$Lu. Experimental data are taken from Ref \cite{schonwasser2003one}.}
    \label{excitation-energies-th-165Lu}
\end{figure}

Lastly, the two wobbling bands for $^{167}$Lu are shown in Fig. \ref{excitation-energies-th-167Lu} for which the experimental data are qualitatively well described by the $\mathbf{W_1}$ formalism. Based on the results summarized in Figs. \ref{excitation-energies-th-161Lu} - \ref{excitation-energies-th-167Lu}, it can be concluded that the renormalization of the wobbling structure in odd-mass nuclei was indeed a good approximation.
\begin{figure}
    \centering
    \includegraphics[width=0.49\textwidth]{Chapters/Figures/Lu-exp-energies/fig5a_lu167.pdf}
    \includegraphics[width=0.49\textwidth]{Chapters/Figures/Lu-exp-energies/fig5b_lu167.pdf}
    \caption{The experimental and theoretical excitation energies provided by Eq. \ref{general-excitation-energy-fitting-model} for the three wobbling bands in $^{167}$Lu. Experimental data are taken from Ref \cite{amro2003wobbling}.}
    \label{excitation-energies-th-167Lu}
\end{figure}

In what follows a remark about the evidence towards a transverse/longitudinal wobbling regime should be made. Indeed, if one recalls the behavior of the wobbling energy defined in the sense of Eq. \ref{eq-wobbling-energy-definition-oddA} for the $A\approx160$ region, all isotopes show a decreasing trend with respect to the total angular momentum. Such a decrease would indicate the transverse-like regime for every isotope. However, the evaluation of $E_\text{wob}$ from Eq. \ref{eq-wobbling-energy-definition-oddA} implies that the $n_w=1$ band is always the first band above the yrast line. By contrast, in this approach, the $n_w=1$ band only appears for $^{163,165}$Lu nuclides due to them having states activated by the phonon operator $\Gamma^\dagger$ from TSD2. Consequently, a direct comparison between the experimental and theoretical values can be made solely for the set of states $\text{TSD3}\to\text{TSD2}$, meaning that the standard definition for the wobbling energy is now properly adjusted to the $\mathbf{W_1}$ formalism and the graphical representation is done in Fig. \ref{experimental-wobbling-energies-Lu-aw1}. Based on the plots, it can be seen that the experimental wobbling energy is increasing from $0.144$ MeV to $0.170$ MeV within the spin range $[33/2,77/2]$ and then it decreases for the last two spin states. Interesting the fact that both isotopes have a rather similar behavior for the experimental $E_\text{wob}$, although at the high-spin limit the decrease in magnitude is unexpected because the core and the quasi-particle's a.m. must point in the same direction, thus leading to longitudinal-like motion. Additionally, the current theory predicts the increasing part and it even shows a slight quenching when $I\geq 40\hbar$. Besides that, there is an almost constant shift between the calculated and experimental values for $E_\text{wob}$ of about $\approx 0.3\ \text{MeV}$ and $\approx 0.15\ \text{MeV}$ in $^{163}$Lu and $^{165}$Lu, respectively. One may conclude that the present model properly describes the wobbling motion, which is also consistent with microscopic studies done in Ref. \cite{matsuzaki2002wobbling}.
\begin{figure}
    \centering
    \includegraphics[width=0.49\textwidth]{Chapters/Figures/Lu-exp-energies/fig6a.pdf}
    \includegraphics[width=0.49\textwidth]{Chapters/Figures/Lu-exp-energies/fig6b.pdf}
    \caption{The theoretical wobbling energies as per Eq. \ref{eq-wobbling-energy-definition-oddA} compared with the experimental ones, for the isotopes $^{163}$Lu (\textbf{left}) and $^{165}$Lu (\textbf{right}). Note that the $n=1$ and $n=0$ bands are TSD3 and TSD2 within the current description.}
    \label{experimental-wobbling-energies-Lu-aw1}
\end{figure}

\subsection{Alignment}

Going further with the study of the triaxial properties in odd-mass nuclei, other relevant quantities are calculated within the current formalism. The \emph{alignment} (or aligned angular momentum) is defined as the total spin minus a reference value. Usually the reference value is a function that is cubed in the rotational frequency $\hbar\omega$. Namely, the alignment and its reference value are given as:
\begin{align}
    i_x=&I-I_\text{ref}\ ,\nonumber\\
    I_\text{ref}=&\mathcal{I}_0\omega+\mathcal{I}_1\omega^3 \,
    \label{alignment-reference-angular-momentum}
\end{align}
where the coefficients $\mathcal{I}_0$ and $\mathcal{I}_1$ are obtained by a least square procedure fit. Within literature, these values are also known as the Harris parameters \cite{harris1965higher}. The linear term from Eq. \ref{alignment-reference-angular-momentum} is involved in the spherical symmetry and the second term is related to the axial symmetry. Consequently, the alignment gives a measure of triaxiality for the isotopes. The comparison between the theoretical and the experimental alignments is done for each isotope with the Harris parameters set to $\left(\mathcal{I}_0,\mathcal{I}_1\right)=\left(30,40\right)$ in units of $\hbar^2\text{MeV}^{-1}$, and the results can be seen in Fig. \ref{alignments-lu-163} for $^{163}$Lu, Fig. \ref{alignments-lu-165} for $^{165}$Lu, and finally Fig. \ref{alignments-lu-161-167} for $^{161}$Lu and $^{167}$Lu.
\begin{figure}
    \centering
    \includegraphics[width=0.49\textwidth]{Chapters/Figures/Lu-exp-energies/fig8a_lu163.pdf}
    \includegraphics[width=0.49\textwidth]{Chapters/Figures/Lu-exp-energies/fig8b_lu163.pdf}
    \includegraphics[width=0.49\textwidth]{Chapters/Figures/Lu-exp-energies/fig8c_lu163.pdf}
    \includegraphics[width=0.49\textwidth]{Chapters/Figures/Lu-exp-energies/fig8d_lu163.pdf}
    \caption{The theoretical and experimental alignments for $^{163}$Lu, according to Eq. \ref{alignment-reference-angular-momentum}, as function of the rotational frequency (Eqs. \ref{rotational-frequency-canonical-definition} - \ref{rotational-frequency-canonical}). Even though TSD4 was fitted separately for calculations of Eq. \ref{general-excitation-energy-fitting-model}, the same Harris parameters were employed as in TSD1-2-3.}
    \label{alignments-lu-163}
\end{figure}
\begin{figure}
    \centering
    \includegraphics[width=0.49\textwidth]{Chapters/Figures/Lu-exp-energies/fig9a_lu165.pdf}
    \includegraphics[width=0.49\textwidth]{Chapters/Figures/Lu-exp-energies/fig9b_lu165.pdf}
    \caption{The theoretical and experimental alignments for $^{165}$Lu, according to Eq. \ref{alignment-reference-angular-momentum}, as function of the rotational frequency (Eqs. \ref{rotational-frequency-canonical-definition} - \ref{rotational-frequency-canonical}). The numerical evaluation was done with the Harris parameters defined in text.}
    \label{alignments-lu-165}
\end{figure}
\begin{figure}
    \centering
    \includegraphics[width=0.49\textwidth]{Chapters/Figures/Lu-exp-energies/fig7.pdf}
    \includegraphics[width=0.49\textwidth]{Chapters/Figures/Lu-exp-energies/fig10.pdf}
    \caption{The theoretical and experimental alignments for $^{161}$Lu (\textbf{left}) and $^{167}$Lu (\textbf{right}), according to Eq. \ref{alignment-reference-angular-momentum}, as function of the rotational frequency (Eqs. \ref{rotational-frequency-canonical-definition} - \ref{rotational-frequency-canonical}). The numerical evaluation was done with the Harris parameters defined in text.}
    \label{alignments-lu-161-167}
\end{figure}

Taking a closer look at Figs. \ref{alignments-lu-163} - \ref{alignments-lu-161-167}, a good agreement between the theory and experiment can be inferred, especially for $A=167$. There are however a few discrepancies between the results for $^{163}$Lu in the high-frequency limit (i.e., $\hbar\omega\geq 0.45\ \text{MeV}$), where the experimental data shows a rather linear increase while the theoretical points seem to suffer a quenching with a slight down-bending. In fact, one might extract three regions each with its different character regarding the angular momentum: a) one linearly increasing region at small frequencies ($\hbar\omega\in[0,0.3]$), b) a saturation region at medium frequencies ($\hbar\omega\in[0.3,0.55]$), and finally c) a decreasing function at high frequencies ($\hbar\omega\geq 0.6$). It seems that $\mathbf{W_1}$ can be further improved if the alignment is properly adjusted (e.g., by amending it with a linear term having large contribution only in the upper $\hbar\omega$-limit).

Comparing the curves relative to the entire mass region, based on their similarities and the fact that they are quite close to each other indeed reflects the wobbling character. Using $\mathbf{W_0}$, some alignment values were evaluated for $^{165}$Lu and $^{167}$Lu \cite{raduta2018wobbling}, where the bands TSD2 and TSD3 were one- and two-wobbling phonons. This new approach theory shows a clear improvement over the previous model.

\subsection{Reference Energy}

Another useful quantity that could illustrate a wobbling behavior across neighboring isotopes is the excitation energy relative to a spherical rigid rotor. This is evaluated with an \emph{effective} moment of inertia and it is graphically represented as function of the total angular momentum. In principle, the reference energy is a typical rotor expression defined in terms of the squared angular momentum as $E_\text{ref}=\alpha I(I+1)$. The value of $\alpha$ can be indeed regarded as an effective inverse MOI, which is usually determined by fitting the experimental reference energies. Moreover, $\alpha$ can differ from isotope to isotope but here one kept the same value across all nuclides (for consistency), and the obtained numerical results show a fairly good agreement with the experimental data set.

Throughout the current calculations, the reference energy was fixed to $E_\text{ref}=0.0075I(I+1)\ \text{MeV}$. The results are depicted in Fig. \ref{reference-rotor-energy-lu161} for $^{161}$Lu, Fig. \ref{reference-rotor-energy-lu165} for $^{165}$Lu, and Fig. \ref{reference-rotor-energy-lu167} for $^{167}$Lu. The four wobbling bands in $^{163}$Lu are each plotted separately in Fig. \ref{reference-rotor-energy-lu163}.
\begin{figure}
    \centering
    \includegraphics[width=0.49\textwidth]{Chapters/Figures/Lu-exp-energies/fig11a_lu161.pdf}
    \includegraphics[width=0.49\textwidth]{Chapters/Figures/Lu-exp-energies/fig11b_lu161.pdf}
    \caption{The excitation energy relative to a rotor reference $E_\text{ref}=0.0075I(I+1)$ as function of the total angular momentum, for $^{161}$Lu.}
    \label{reference-rotor-energy-lu161}
\end{figure}
\begin{figure}
    \centering
    \includegraphics[width=0.49\textwidth]{Chapters/Figures/Lu-exp-energies/fig13a_lu165.pdf}
    \includegraphics[width=0.49\textwidth]{Chapters/Figures/Lu-exp-energies/fig13b_lu165.pdf}
    \caption{Comparison between theoretical and experimental excitation energy relative to a rotor reference $E_\text{ref}=0.0075I(I+1)$ for $^{165}$Lu, as function of the total angular momentum. The zero- and one-wobbling-phonon bands TSD2-3 are plotted on the same figure.}
    \label{reference-rotor-energy-lu165}
\end{figure}
\begin{figure}
    \centering
    \includegraphics[width=0.49\textwidth]{Chapters/Figures/Lu-exp-energies/fig14a_lu167.pdf}
    \includegraphics[width=0.49\textwidth]{Chapters/Figures/Lu-exp-energies/fig14b_lu167.pdf}
    \caption{Comparison between theoretical and experimental excitation energy relative to a rotor reference $E_\text{ref}=0.0075I(I+1)$ for $^{167}$Lu, as function of the total angular momentum.}
    \label{reference-rotor-energy-lu167}
\end{figure}
\begin{figure}
    \centering
    \includegraphics[width=0.49\textwidth]{Chapters/Figures/Lu-exp-energies/fig12a_lu163.pdf}
    \includegraphics[width=0.49\textwidth]{Chapters/Figures/Lu-exp-energies/fig12b_lu163.pdf}
    \includegraphics[width=0.49\textwidth]{Chapters/Figures/Lu-exp-energies/fig12c_lu163.pdf}
    \includegraphics[width=0.49\textwidth]{Chapters/Figures/Lu-exp-energies/fig12d_lu163.pdf}
    \caption{The excitation energy relative to a rotor reference $E_\text{ref}=0.0075I(I+1)$ as function of the total angular momentum, for $^{163}$Lu.}
    \label{reference-rotor-energy-lu163}
\end{figure}

Concerning the graphical representations from Figs. \ref{reference-rotor-energy-lu161} - \ref{reference-rotor-energy-lu163}, some remarks should be highlighted. Firstly, each experimental curve shows a decreasing trend with angular momentum. Some bands exhibit a stronger behavior than others. For example, TSD2 of $^{163}$Lu is almost constant in the range $I\in[15-17]\ \hbar$ and only starts to get smaller beyond when $I\geq 30\hbar$. The same can be observed for TSD3 from $^{165}$Lu. On the other hand, TSD3 from $^{163}$Lu or TSD2 from $^{167}$Lu are decreasing rather rapidly across the entire spin range. The reduction of $E-E_\text{ref}$ indicates that the contribution of the rotor part becomes more and more significant, leading to a diminishing effect of triaxiality when angular momentum reaches large values. The theoretical curves do reproduce the overall trends, showing some striking similarities for TSD4 from $^{163}$Lu or TSD3 in $^{165}$Lu. Remarking that the obtained numerical data for TSD1-2-3 from $^{163}$Lu show a decrease within a \emph{convex} manner, whereas the experimental sets behave as \emph{concave} functions. The deviation from the experimental set for TSD3 in the same nucleus increases by about 0.3 MeV for $I\geq 35\hbar$. Otherwise, it can be concluded that the current model does reproduce the dominance of a rotor-like behavior over triaxiality at large angular momentum for the studied isotopes. Improvements in the quality of the figures can be observed when comparing with the calculations made using the previous approach $\mathbf{W_0}$ \cite{raduta2018wobbling}.

\subsection{Dynamic MOI}

The dynamic moment of inertia $\mathcal{I}^{(2)}$ was introduced in Chapter \ref{chapter-3} (recall Eqs. \ref{dynamic-moi-general} - \ref{dynamic-moi-energy-levels}), where it was shown that this quantity is directly related to the energy differences for the $I+2,I,I-2$ levels, respectively. Usually $\mathcal{I}^{(2)}$ is represented as a function of the rotational frequency or even the total angular momentum. In the current model, the graphical representations are made with respect to the rotational frequency and the data for $^{161,167}$Lu can be seen in Fig. \ref{dynamic-moi-Lu-161-167}. For $^{163}$Lu, the comparison between theory and experiment concerning $\mathcal{I}^{(2)}$ is depicted in Fig. \ref{dynamic-moi-Lu-163}. Lastly, the three bands from $^{165}$Lu are sketched in Fig. \ref{dynamic-moi-Lu-165}.
\begin{figure}
    \centering
    \includegraphics[width=0.49\textwidth]{Chapters/Figures/Lu-exp-energies/fig15_lu161.pdf}
    \includegraphics[width=0.49\textwidth]{Chapters/Figures/Lu-exp-energies/fig18_lu167.pdf}
    \caption{The theoretical dynamic moment of inertia defined in Eq. \ref{dynamic-moi-general} for the isotopes $^{161}$Lu (\textbf{left}) and $^{167}$Lu (\textbf{right}) is compared with the experimental data. The rotational frequency is defined as $\hbar\omega_I=\text{d}E/\text{d}I=E_\gamma(I,I-2)/2$.}
    \label{dynamic-moi-Lu-161-167}
\end{figure}
\begin{figure}
    \centering
    \includegraphics[width=0.49\textwidth]{Chapters/Figures/Lu-exp-energies/fig16a_lu163.pdf}
    \includegraphics[width=0.49\textwidth]{Chapters/Figures/Lu-exp-energies/fig16b_lu163.pdf}
    \caption{The dynamic moment of inertia defined in Eq. \ref{dynamic-moi-general} for $^{163}$Lu. The rotational frequency is defined as $\hbar\omega_I=\text{d}E/\text{d}I=E_\gamma(I,I-2)/2$.}
    \label{dynamic-moi-Lu-163}
\end{figure}
\begin{figure}
    \centering
    \includegraphics[width=0.49\textwidth]{Chapters/Figures/Lu-exp-energies/fig17a_lu165.pdf}
    \includegraphics[width=0.49\textwidth]{Chapters/Figures/Lu-exp-energies/fig17b_lu165.pdf}
    \caption{The dynamic moment of inertia defined in Eq. \ref{dynamic-moi-general} for $^{165}$Lu showing the band TSD1 (\textbf{left}) and the bands TSD2-3 (\textbf{right}). The rotational frequency is defined as $\hbar\omega_I=\text{d}E/\text{d}I=E_\gamma(I,I-2)/2$.}
    \label{dynamic-moi-Lu-165}
\end{figure}

Usually, the dynamic MOI is sensitive to the single-particle effects, such as alignment of nucleons or even interaction between states belonging to different bands. As such, it could be a useful indicator of triaxiality and wobbling motion, since the nucleus can suffer structural changes as its angular momentum (and implicitly rotational frequency) increases. Looking at the figures with $\mathcal{I}^{(2)}$ for each isotope, some interesting features appear. Firstly, regarding the theoretical calculations, $\mathbf{W_1}$ formalism predicts constant values for this quantity, meaning that the dynamic MOI is a constant function of $\hbar\omega$ and furthermore the rotational frequency is linear in $I$. This can be indeed observed throughout the plots, where each TSD band has a constant $\mathcal{I}^{(2)}$ for the entire range of $\hbar\omega$. Nonetheless, if one `averages out' the experimental set of points to a singular value for each band, the theoretical and experimental lines would lie quite close to each other. On the other side, for the experimental data there is a staggering present (e.g., band TSD1 from $^{161}$Lu) and even sharp changes in magnitude, as it is the case for TSD2 from $^{161}$Lu and $^{165}$Lu, respectively. In $^{161}$Lu, for example, the abrupt increase is caused by the alignment of the odd-proton's a.m., showing up at $\hbar\omega=0.45\ \text{MeV}$. The staggering behavior in the triaxial bands of the isotopes, mostly occurring in the low $\hbar\omega$ region, is caused by interaction between the states from the TSD bands and their neighboring normally deformed structures. Remarkable the fact that TSD1 from $^{165}$Lu exhibits a rather constant dynamic MOI (just as the  theoretical value) except for the first two states. Comparing the present calculations with the results from Ref. \cite{raduta2018wobbling}, an overall improvement in the agreement with the experimental data can be observed.

\subsection{Electromagnetic Transitions}

In this section, the electromagnetic transitions for $^{161,163,165,167}$Lu will be calculated within the $\mathbf{W_1}$ formalism and compared with the available experimental data. For collective phenomena, one usually expects two `main' characteristics to arise: a) the transitions between neighboring bands to be predominantly of E2 character and b) the states have large quadrupole moments. In fact, as it was discussed in Chapters \ref{chapter4} and \ref{chapter-5}, the large quadrupole moments and large mixing ratios $\delta$ are regarded as essential `tests' for wobbling behavior.

For the numerical application, in the first part the electric quadrupole E2 transitions will be discussed, while in the second part the magnetic dipole M1 transitions are addressed. Concerning the states, the transitions between two states inside the same band (=intraband) as well as the transitions between adjacent bands (=interband) need to be evaluated. Because of the collective nature of the wobbling bands, their spin difference within a band is $\Delta I=2$ units of angular momentum, while two states from adjacent bands only differ by one unit $\Delta I=1$ (recall Fig. \ref{wobbling-geometry-tilting-sketch} from the discussion in Chapter \ref{chapter-5}). A schematic representation that illustrates the difference between interband/intraband transitions can be seen in Fig. \ref{schematic-interband-intraband-E2}. Therein, the E2 transitions are sketched for levels belonging to the same band and levels from different bands, and for each band the allowed spin sequences are exemplified. The magnetic transitions for wobbling states have a dipole nature, which means that they take place between states that differentiate by one unit of angular momentum. Experimental observations point out that these M1 transitions are typically quite small in comparison with the E2 ones. Moreover the `competition' between E2 and M1 for a state $I$ is reflected in the mixing ratio $\delta$. 
\begin{figure}
    \centering
    \includegraphics[scale=1.1]{Chapters/Figures/transitions-wobbling-states.pdf}
    \caption{Schematic representation with the electric quadrupole transitions that can occur in wobbling spectra of nuclei. The \emph{intraband} values ($E2_\text{in}$ ; blue) represent the transitions between an initial state $I_\text{i}$ and an final state $I_\text{f}$ belonging to the same band, which are characterized by $\Delta I=I_\text{i}-I_\text{f}=2\hbar$. On the other hand, the \emph{interband} values ($E2_\text{out}$ ; red) take place between states belonging to two contiguous bands, and they are characterized by $\Delta I=1\hbar$. The spin sequences for each band are given below the level schemes in curly brackets.}
    \label{schematic-interband-intraband-E2}
\end{figure}

The transition probabilities between states that comprise the excitation spectra can be determined from the matrix elements (m.e.) of the \emph{electromagnetic transition operators} (recall Eq. \ref{eq-general-reduced-E2} and discussion from Chapter \ref{chapter-3}). The operators of interest here are $\mathcal{M}(E2,\mu)$ for the electric quadrupole transitions and $\mathcal{M}(M1,\mu)$ for the magnetic dipole transitions. In the following subsections, each case will be treated, starting with the analytical expressions and finally obtaining some numerical results.

\subsubsection{E2 Transitions}

The calculations for E2 and M1 transitions are made working in the laboratory system, meaning that the multiple operators must be expressed in terms of the body-fixed (intrinsic) system via the Wigner functions:
\begin{align}
    \mathcal{M}(\lambda,\mu)=\sum_\nu\mathcal{D}_{\mu\nu}^\lambda\mathcal{M}(\lambda,\nu)\ ,
    \label{multipole-operator-lab}
\end{align}
where $\mathcal{M}(\lambda,\mu)$ represents the laboratory's operator and $\mathcal{M}(\lambda,\nu)$ is the intrinsic multipole operator. Obviously here $\lambda=2$ and the \emph{quadrupole transition operator} is expressed as \cite{toki1975asymmetric,raduta2020towards}:
\begin{align}
    \mathcal{M}(E2,\mu)=\left[\mathcal{D}_{\mu0}^2Q_0+\left(D_{\mu 2}^2+D_{\mu -2}^2\right)Q_2\right]+\mathrm{e}\sum_{\nu=-2,0,2}\mathcal{D}_{\mu\nu}^2r^2Y_2^\nu\ .
    \label{electric-quadrupole-operator-lab}
\end{align}

Since the triaxial system can be regarded as a collective core + a single particle, the electric transition operator defined in $\mathbf{W_1}$ can be described as an operator which is separated into a \emph{collective} term and a \emph{single-particle} term. As a matter of fact, it can be seen from Eq. \ref{electric-quadrupole-operator-lab} that indeed the first  
\begin{align}
    \mathcal{M}(E2,\mu)\equiv T_{2\mu}^\text{coll}+T_{2\mu}^\text{sp}\ ,
    \label{quadrupole-transition-operator-terms}
\end{align}

The operator $T_{2\mu}^\text{coll}$ corresponds to the core $\mathscr{C}$, and it can be expressed through the Wigner-$\mathcal{D}$ functions as:
\begin{align}
    T_{2\mu}^\text{coll}=z_\text{eff}\left[\mathcal{D}_{\mu 0}^2Q_{0}-\left(\mathcal{D}_{\mu 2}^2+\mathcal{D}_{\mu -2}^2\right)Q_{2}\right]\ ,
    \label{electric-quadrupole-T-coll}
\end{align}
where the two quadrupole moments $Q_{0}$ and $Q_{2}$ represent a measure of deformation and asymmetry of the nuclear shape. They are expressed in terms of the deformation parameters $\beta$ and $\gamma$, i.e. \cite{raduta2018wobbling}:
\begin{align}
    Q_{0}=\frac{3}{4\pi}ZR^2\beta\cos\gamma\ ,\ Q_{2}=\frac{3}{4\pi}ZR^2\beta\sin\gamma/\sqrt{2}\ ,
    \label{quadrupole-components-Q0-Q2}
\end{align}
which is in fact consistent with the discussion from Chapter \ref{chapter-3} (see Eq. \ref{quadrupole-moment-Q0}) and Chapter \ref{chapter-5} (recall Eq. \ref{quadrupole-components-q20-q22}). In terms of the nuclear charge density, $Q_{0,2}$ show to which extent its distribution deviates from spherical symmetry \cite{bohr1998nuclear}. The operator which corresponds to the contribution of the odd proton, i.e.,  $T_{2\mu}^\text{sp}$ given in Eq. \ref{quadrupole-transition-operator-terms} has the following form:
\begin{align}
    T_{2\mu}^\text{sp}=z_\text{eff}\sum_{\nu=-2,0,2}^2 \mathcal{D}_{\mu\nu}^2 r^2Y_2^\nu\ ,
    \label{electric-quadrupole-T-sp}
\end{align}

Besides the electric transition operator, the wave-functions for the wobbling bands considered in the current picture are also needed. The ground-bands are given as per the variational principle (the trial function from Eq. \ref{tdve-approach-w1}):
\begin{align}
    \ket{\Psi_{IMj}}=&\mathcal{N}\mathrm{e}^{z\hat{I}_-}\mathrm{e}^{s\hat{j}_-}\ket{IMI}\ket{jj}=\nonumber\\
    =&\sum_{K\Omega}\frac{z^{I-K}s^{j-\Omega}}{\left(1+|z|^2\right)^I\left(1+|s|^2\right)^j}\binom{2I}{I-K}^{1/2}\binom{2j}{j-\Omega}^{1/2}\ket{IMK}\ket{j\Omega}\ .
    \label{wave-function-TSD1}
\end{align}

The complex variables $(z;s)$ correspond to the classical coordinates $(r,\varphi;f,\psi)$ as per the transformations employed in Section \ref{equations-of-motion-section} (see Eqs. \ref{z-s-variables}, and \ref{changed-rho-sigma-variables} - \ref{eq-of-motion-approach-w1}). Keep in mind that the set $(r,\varphi;f,\psi)$ brought the equations of motion to a canonical form and their explicitly form was specified in Eq. \ref{eq-of-motion-explicit-coordinates}. The classical coordinates $(r,\varphi;f,\psi)$, or equivalently $z$ and $s$, were expressed as per the re-quantization procedure applied in Eqs. \ref{canonical-coordinates-quantized} - \ref{canonical-transformations-ab-qp} in terms of the creation/annihilation operators $(a^\dagger,a;b^\dagger,b)$. For the boson creation and annihilation operators $(a^\dagger,a;b^\dagger,b)$ there corresponds a vacuum state $\ket{\varnothing}_I$. Any arbitrary state $\ket{\Psi_{IMj}}$ can be spanned from the vacuum via the relation \cite{raduta2017semiclassical}:
\begin{align}
    \ket{m,n}_I=\frac{1}{\sqrt{m!n!}}\left(a^\dagger\right)^m\left(b^\dagger\right)^n\ket{\varnothing}_I\ ,\ n,k=1,2,3,\dots\ .
\end{align}

It was determined that the CEF (Eq. \ref{full-classical-energy-function}) is minimal in the point $p_0$, which was given Eq. \ref{cef-minimum-point-p0}. Evaluating the wave-function from Eq. \ref{wave-function-TSD1} around this minimum point, the expression of $\ket{\Psi_{IMj}}$ becomes:
\begin{align}
    \ket{\Psi_{IMj}}|_{p_0}=\frac{1}{2^{I+j}}\sum_{K\Omega}\binom{2I}{I-K}^{1/2}\binom{2j}{j-\Omega}^{1/2}\ket{IMK}\ket{j\Omega}\ket{\varnothing}_I\ ,
    \label{wave-function-TSD1-p0}
\end{align}

The wave-functions which need to be used in the calculus of transition probabilities can be developed from the following exploit: expansion up to first-order (abbreviated `1st or.') in the coordinates $(r,\varphi;f,\psi)$ and then perform re-quantization of the coordinates as per the aforementioned recipe. Thus, from the wave-function of Eq. \ref{wave-function-TSD1-p0} one obtains:
\begin{align}
    \ket{\Psi_{IMj}}|_{p_0}^{\text{1st or.}}\equiv&\ket{\Psi_{IMj}^{(1)}}=\frac{1}{2^{I+j}}\sum_{K\Omega}\binom{2I}{I-K}^{1/2}\binom{2j}{j-\Omega}^{1/2}\ket{IMK}\ket{j\Omega}\times\nonumber\\
                           &\times\left\{1+\frac{\iu}{\sqrt{2}}\left[\left(\frac{K}{I}k_1+\frac{I-K}{k_1}\right)a^\dagger+\left(\frac{\Omega}{j}k_2+\frac{j-\Omega}{k_2}\right)b^\dagger\right]\right\}\ket{\varnothing}_I\ .
    \label{wave-function-TSD1-p0-first-order}
\end{align}

Looking at Eq. \ref{wave-function-TSD1-p0-first-order} indeed, it can be seen that the first-order expansion is reflected through the presence of $a^\dagger$ and $b^\dagger$ as independent variables (that is, no mixed or quadratic terms). The canonicity factors $k_1,k_2$ defined in Eq. \ref{canonicity-factors} also show up for each coordinate. Lastly, $K,\Omega$ are the projections of the total and single-particle angular momentum vectors in the intrinsic coordinate system, respectively.

The wave-function $\ket{\Psi_{IMj}^{(1)}}$ can be applied for the transitions concerning bands TSD1-2 and TSD4 in all the isotopes, since these are ground-bands obtained variationally. However, transitions in and out of TSD3 from $^{163}$Lu and $^{165}$Lu require the one-wobbling-phonon band description. Considering the most general one-phonon band which emerges from the $\mathbf{W_1}$ picture, this can be inferred from the set of Eqs. \ref{tsd-bands-compressed-spectrum} - \ref{tsd-bands-general-spectrum}. Hence, a one-phonon band is obtained by activating states $I-1$ from the neighboring zero-phonon partner with $n_{w_1}$ phonon operators $\Gamma_1^\dagger$ and $n_{w_2}$ phonon operators $\Gamma_2^\dagger$, each applied on the vacuum states $|\varnothing)_{I-1}$ and $|\varnothing)_j$:
\begin{align}
    \ket{\Phi_{IMj}}=\ket{\Psi_{I-1M-1j}^{(1)}}\left\{\frac{1}{\sqrt{n_{w_1}!}}\left(\Gamma_1^\dagger\right)^{n_{w_1}}|\varnothing)_{I-1}\otimes\frac{1}{\sqrt{n_{w_2}!}}\left(\Gamma_2^\dagger\right)^{n_{w_2}}|\varnothing)_j\right\}\ ,
    \label{wave-function-TSD-excited-phonon-state}
\end{align}
where the two vacuum states are defined in Appendix D from Ref. \cite{raduta2017semiclassical}.

As it was discussed in the previous section, the renormalization given in Eqs. \ref{renormalized-bands-structure-TSD124}-\ref{renormalized-bands-structure-TSD3} was defined only in terms of the phonon operator $\Gamma_1^\dagger$. From the spectrum depicted in Eq. \ref{lu163-absolute-energies-tsd1234} and the collection of wobbling phonon numbers provided in Tables \ref{lu-163-phonon-numbers} - \ref{lu-167-experimental-data-table}, it occurs that only the first phonon (corresponding to the motion of the core) will generate excited phonon states. This results in the straightforward structure of the bands TSD3 from $^{163,165}$Lu:
\begin{align}
    \ket{\Phi_{IMj}}=\ket{\Psi_{I-1M-1j}^{(1)}}\left(\Gamma_1^\dagger\right)^{n_{w_1}}|\varnothing)_{I-1}\ .
    \label{wave-function-TSD3}
\end{align}

With the expressions of the E2 transition operator granted by Eq. \ref{electric-quadrupole-transition-operator} and wave-functions defined according to Eqs. \ref{wave-function-TSD1-p0-first-order} and \ref{wave-function-TSD3}, the matrix elements of $\mathcal{M}(E2,\mu)$ can be finally constructed, where the superscript `(1)` from $\ket{\Psi_{I-1M-1j}^{(1)}}$ will be dismissed hereafter. The reduced transition probabilities were firstly depicted in Eq. \ref{eq-general-reduced-E2} from Chapter \ref{chapter-3} and here they are defined as transitions from an initial state $I_i$ to a final state $I_f$:
\begin{align}
    B(E2,\mu;(I_i,n_{w_1})\to(I_f,n'_{w_1}))=&\sum_{M_iM_f\mu}\left|\bra{\Psi_{I_iM_ij}}\mathcal{M}(E2,\mu)\ket{\Psi_{I_fM_fj}}\right|^2=\nonumber\\
                                            =&\left|\bra{\Psi_{I_i}}\left|\mathcal{M}(E2)\right|\ket{\Psi_{I_f}}\right|^2\ ,
    % B(E2_\text{in};(I_i,n_{w_1})\to(I_f,n_{w_1}))=&\bra{\Psi_{I_iM_ij}}\left|\mathcal{M}(E2)\right|\ket{\Psi_{I_fM_fj}}^2\ ,
    % B(E2_\text{out};(I_i,n_{w_1})\to(I_f,n'_{w_1}))\stackrel{TSD2\to TSD1}{=}&\bra{\Psi_{I_iM_ij}}\left|\mathcal{M}(E2)\right|\ket{\Psi_{I_fM_fj}}^2\ ,
\end{align}
where the Rose convention is adopted throughout the calculations \cite{rose1995elementary}:
\begin{align}
    \bra{JM}T_{\lambda\mu}\ket{J'M'}=C_{M'\mu M}^{J'\lambda J}\bra{J}|T_\lambda|\ket{J'}\ .
    \label{rose-convention}
\end{align}

The matrix elements for the collective and single-particle components of $\mathcal{M}(E2)$ are given analytically in Appendix D from \cite{raduta2017semiclassical}. These matrix elements are also needed in order to obtain the mixing ratio that shows the \emph{competition} between E2 and M1 transitions from states belonging to TSD2 and TSD1 \cite{krane1970determination,toki1975asymmetric}:
\begin{align}
    \delta=8.87\cdot 10^{-4}E_{if}\frac{\bra{I_i}|\mathcal{M}(E2)|\ket{I_f}}{\bra{I_i}|\mathcal{M}(M1)|\ket{I_f}}\ .
    \label{mixing-ratio-Lu-isotopes}
\end{align}

Regarding Eq. \ref{mixing-ratio-Lu-isotopes}, the E2 matrix element has units of $\mathrm{e}\cdot\text{fm}^2$ and M1 matrix element has units of $\mathrm{e}\cdot\text{fm}$. The transition energy between the initial and final state is represented by $E_{if}$ and its unit is $\text{MeV}$. Details on the calculation of the M1 matrix elements will be provided in the following section, which is dedicated for magnetic transitions.

% Because the formalism $\mathbf{W_1}$ adopts a set of MOI which are determined freely from Eq. \ref{chi-2-fitting-function}, one cannot infer their actual nature (that is neither hydrodynamical nor rigid). Moreover, from the calculated parameters $\mathcal{P}_\text{fit}$ shown in Table \ref{numerical-fitting-parameters-Lu-isotopes} it was concluded that the maximal MOI is $\mathcal{I}_1$ for each isotope. Accordingly, the quadrupole components $Q_{0}$ and $Q_{2}$ give the degree of elongation and asymmetry of the charge distribution with respect to this axis. In order to keep a consistent numerical procedure, the two quadrupole moments should also be considered as free quantities, meaning that the analytical expressions from Eq. \ref{quadrupole-components-Q0-Q2} do not have to be employed explicitly within calculations. In fact, the two components are determined per each isotope by fixing one intraband transition from TSD1 and one interband transition $\text{TSD2}\to\text{TSD1}$. With this method, the values $Q_0=18.43\ \mathrm{e}b$ and $Q_2=19.81\ \mathrm{e}b$ are obtained for $^{163}$Lu. The results for the other isotopes are graphically represented in Fig. \ref{quadrupole-moments-fit-numerical-results}, where a change of 

% ordering between the two quadrupole components can be observed at $^{165}$Lu. 

% Besides the change of ordering, the magnitude of the components drops quite low compared to the neighboring nuclei.

Remarking the fact that the isotope $^{167}$Lu has a set of quadrupole moments which are almost identical.
\begin{figure}
    \centering
    \includegraphics[scale=0.65]{Chapters/Figures/Lu-exp-energies/fig19.pdf}
    \caption{The calculated quadrupole moments $Q_0$ and $Q_2$, which enter in Eq. \ref{electric-quadrupole-transition-operator}. The unit for $Q$ is $\mathrm{e}b$. See text for details on their numerical determination.}
    \label{quadrupole-moments-fit-numerical-results}
\end{figure}

% The \emph{quadrupole moment} can also be evaluated for a state $I$ and it gives a measure of nuclear charge distribution associated with rotational motion of collective nature.
% \begin{align}
%     Q_I=\sqrt{\frac{16\pi}{5}}C_{I0I}^{I2I}\bra{I}|\mathcal{M}(E2)\ket{I}
% \end{align}

\subsubsection{M1 Transitions}

For the magnetic transitions within wobbling states, the corresponding operator is also expressed as a collective plus a single particle components, keeping thus a consistency with Eq. \ref{quadrupole-transition-operator-terms}:
\begin{align}
    \mathcal{M}(M1;\mu)=M_{1\mu}^\text{coll}+M_{1\mu}^\text{sp}\ .
\end{align}