\chapter{A New Theoretical Formalism}

In this chapter, a formalism that describes the wobbling properties in odd-$A$ nuclei will be presented. The model was developed recently by the current team (i.e., Raduta and Poenaru) and applied to $^{135}$Pr \cite{raduta2020new}, and the `Lu family' with $^{161,165,167}$Lu \cite{raduta2020approach} and $^{163}$Lu \cite{raduta2020approach,raduta2020towards,poenaru2021parity,poenaru2021extensive1,poenaru2021extensive2}. This framework is an original contribution to the field of nuclear structure, with focus on the theoretical aspects of collective phenomena in nuclei.

\section{Previous Work - Foundation}

From a development standpoint, it is instructive to review the previous `stages' that lead to the current work, Since the description of odd-$A$ nuclei achieved in here is strongly connected with former calculations done by the team. Consequently, in this section a brief overview of the wobbling motion in $^{163}$Lu done by Raduta et al. in 2017 \cite{raduta2017semiclassical} will be given. Indeed, by using a \emph{semi-classical} approach, the wobbling properties of this odd-mass isotope were accurately described. 

The semi-classical approaches tend to be very useful when applied to problems having a \emph{quantal} Hamiltonian that contains quantities which behave as in the \emph{classical limit} when certain constraints or approximations are made. Moreover, these methods always keep the dynamics of the system in close contact with the classical features, which in principle are easier to interpret. 