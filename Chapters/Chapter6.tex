\chapter{Conclusions}
\label{chapter-8-conclusions}

This thesis represents the work of several publications focused on the topic of Nuclear Structure. The study of the nucleonic matter that lacks any axial symmetry was the main objective of the team. Nuclear triaxiality became a hot topic over the last decade due to its challenges of measuring it experimentally. Moreover, the theoretical description of triaxially deformed nuclei requires certain methods or approximations, which can become quite cumbersome.

Starting with Section \ref{section-nuclear-shapes} from Chapter \ref{chapter-2-theoretical-aspects}, the nuclear surface is introduced in Eq. \ref{nuclear-shape}, which was parametrized in terms of the collective coordinates and spherical harmonics. The relevant excitation mode for triaxiality is given by the quadrupole deformation, having $\lambda=2$. The quadrupole deformation introduces two parameters that give an insight with respect to the elongation and departure from axial symmetry of a nucleus, by means of the quadrupole deformation parameter $\beta_2$ and triaxiality parameter $\gamma$, which are provided in Eq. \ref{bohr-deformation-params}. The two parameters dictate the stretching of the nuclear axes, and this was shown in Fig. \ref{nuclear-radius-elongation}. From the representation of a general ellipsoid in terms of $\beta_2$ and $\gamma$, all the possible shapes that posses axial symmetry emerge at certain values of $\gamma$, while the unique triaxial region is found in the region $\gamma\in(0,60)$ (see Fig. \ref{beta-gamma-plane}).

The theoretical study of deformed nuclei is performed in Chapter \ref{chapter-2-theoretical-aspects}, where the Nilsson Model is employed (recall Section \ref{subsection-nilsson-model}). The single-particle energies are obtained as a sum between the anisotropic harmonic oscillator, a spin-orbit term, and the centrifugal term. The last two terms are defined with the strength parameters $\kappa$ and $\mu$, which are specific to this theory. The collective model is described in the same chapter (see Section \ref{subsection-collective-model}), and it emphasized the behavior of the nuclear shapes in terms of the moments of inertia, i.e., the irrotational and rigid MOI provided in Eq. \ref{eq-irrotational-rigid-mois}. These quantities are crucial to the development of the model. Their behavior with respect to the asymmetry parameter $\gamma$ was depicted in Fig. \ref{fig-irrotational-rigid-mois}. In terms of nuclear rotations and vibrations, several experimental spectra are graphically shown in Fig. \ref{rotational-bands-odd-a} from Appendix \ref{appendix:ral-dal-signature-scheme}, Figs. \ref{energy-levels-120Te-virbational-band} - \ref{energy-levels-63Cu-virbational-band}, and Fig. \ref{rotational-bands-even-even}. The spectra of nuclear rotations and vibrations are important for understanding wobbling nuclei with respect to the angular momentum (the nuclear vibrations and rotations are treated in Sections \ref{subsection-nuclear-vibration} and \ref{subsection-nuclear-rotation}, respectively). 

Important quantities for collective phenomena used in the numerical application, such as the moments of inertia (including the kinematical in Eq. \ref{kinematic-moi-general} and dynamical in Eq. \ref{dynamic-moi-general}), the quadrupole moment (measure of deformation or departure of the nuclear shape away from spherical symmetry) were discussed in detail throughout Section \ref{c3-collective-quantities}. The Hamiltonian depicted in Eq. \ref{eq-triaxial-prm-full-hamiltonian} marks the onset of the theoretical model that is adopted in this work. Namely. from the Triaxial PRM Hamiltonian, energy spectra and transition probabilities are obtained for odd-mass Lu isotopes.

Chiral bands and wobbling motion are unique fingerprints of triaxiality. Any experimental identification of these two effects is the decisive test that can pinpoint triaxial nuclei across the chart of nuclides. For the chiral case, it was shown in Section \ref{subsection-chiral-motion} that the spectra emerge from the coupling of three angular momenta: a proton, a neutron, and the core, leading to a trihedral system lacking chiral symmetry. Some examples of spectra are shown in Fig. \ref{chiral-bands-1} - \ref{chiral-bands-2}. The schematic from Fig. \ref{chiral-geometry} shows the geometry of the mutual coupling between the angular momenta. Also in Chapter \ref{chapter-2-theoretical-aspects} the Triaxial Particle + Rotor Model has been analytically treated (section \ref{tprm-model}), since that Hamiltonian was the foundation of this current work. Firstly the Hamiltonian for the symmetric case is considered, starting with Eq. \ref{general-rotor-hamiltonian}, then a single-particle term is added via Eq. \ref{triaxial-prm-general-hamiltonian}. This term represents the motion of a valence nucleon within a quadrupole deformed mean-field generated by an even-even core. As a matter of fact, this single particle term is given by the Nilsson's deformed shell model (Eq. \ref{single-particle-nilsson-defored-potential} represents such a term). It is remarkable that the single-particle potential strength $V$ is used to parametrize the interaction. This parameter is furthermore employed within the numerical implementation of the energy spectrum. As a purely theoretical application, the matrix elements for the single-particle energies of protons and neutrons are calculated according to Eq. \ref{single-particle-energies-hpn}. A set of graphical representations are made, showing their behavior with respect to the quadrupole deformation parameter $\beta_2$ and triaxiality $\gamma$. 

%  given as a combined precession and oscillation of the total angular momentum around a fixed position 
Triaxial nuclei rotate around any of the three principal axes, with the main rotation about the axis with largest MOI. The contribution from the other two axes has a vibrational character, and through a first approximation, this kind of motion can be described analytically by a harmonic-like Hamiltonian. In Chapter \ref{chapter-3} it is shown that the wobbling motion differs from the even-$A$ to odd-$A$ nuclei. Indeed, for systems with even number of nucleons, the energy spectrum is achieved in the so-called Harmonic approximation by Eq. \ref{eq-wobbling-energy-evenA}, where the wobbling phonon number represents a tilting strength of the total angular momentum away from the axis with largest MOI. The final energy spectrum is characterized by a rotational motion around this MOI and a frequency of oscillation of the nucleus. The behavior follows Eq. \ref{wobbling-frequency-even-A} and it is depicted in Fig. \ref{fig-even-even-wobbling-energies}. The Harmonic approximation is tested for $^{130}$Ba nucleus, which has two wobbling bands. Experimental data is numerically well reproduced via a fitting procedure, where the free parameters are the three moments of inertia. Section \ref{ba-130-numerical-calculations} shows the actual workflow regarding the numerical algorithm. The wobbling energies and transition probabilities are quantitatively well reproduced (see Figs. \ref{plot-ba130-excitation-energies} - \ref{BE2out-transitions-130ba}) with results comparable with alternative methods from the literature (e.g., Chen et al. \cite{chen2019transverse}). It is worth mentioning that the results given throughout Section \ref{ba-130-numerical-calculations} are unique to this present work. 

For the wobbling motion in odd-mass nuclei, a so-called \emph{Frozen Approximation} is illustrated, showing that the spectrum contains a similar harmonic-like term, but the wobbling frequency has a different behavior. The behavior is dictated by the alignment of the odd-particle with the even-even triaxial core. Depending on whether the particle's a.m. aligns itself along or perpendicular to the axis of largest MOI, two wobbling scenarios emerge: longitudinal and transverse. The workflow diagrams \ref{advanced-quasiparticle-coupling-1} - \ref{advanced-quasiparticle-coupling-3} show how these two wobbling regimes can occur based on considerations of density overlap between the density distributions of the core and the single-particle. These diagrams also illustrate the precessional + oscillatory behavior of the total a.m., which is the first geometrical representation of such a precessional cone in the literature. The wobbling frequency in odd nuclei was analyzed via Eq. \ref{wobbling-frequency-odd-A-MOI}, showing that it depends on the three MOI with a behavior presented in Fig. \ref{wobbling-freq-oddA}. Remarkable to this research is also the catalogue that contains all the known wobblers, where for each nucleus the experimental wobbling energy is shown (as defined in Eq. \ref{eq-wobbling-energy-definition-oddA}), together with details on the number of wobbling bands and deformation parameters ($\beta_2$ and $\gamma$). This chart can be seen in Fig. \ref{wobbling-diagram-chart}.

%encapsulate the conclusions from chapter 5 6 and 7 into a single bit
The Chapters \ref{chapter-3}, \ref{chapter-4-aw1-formalism}, and \ref{chapter-5-novel} do also have separate sets of conclusions and discussions, so one can refer to the individual sections \ref{chapter-3-concluding-remarks}, \ref{chapter-4-concluding-remarks}, \ref{chapter-5-concluding-remarks}, respectively. For this reason, only a brief revision of the emerging characteristics is mentioned here.
\begin{itemize}
    \item an initial quantal Hamiltonian of Triaxial Particle-Rotor type is employed as the foundational tool for describing wobbling motion in odd-mass triaxial nuclei
    \item Hamiltonian is brought the a classical form by means of a variational principle (see Eq. \ref{tdve-approach-w1})
    \item in the classical view, the dynamics are described by two sets of coordinates: one for the even-even core and one for the single-particle (Eqs. \ref{changed-rho-sigma-variables} - \ref{eq-of-motion-approach-w1})
    \item excitation energies are analytically given in terms of five free parameters (Eq. \ref{fitting-parameters-p-fit})
    \item Chapter \ref{chapter-4-aw1-formalism} employs a re-normalization of the wobbling bands (Eqs. \ref{renormalized-bands-structure-TSD124} - \ref{renormalized-bands-structure-TSD3}) by applying the variational principle to not only the ground-state, but also other bands. This is called $\mathbf{W_1}$ formalism throughout the thesis. The TSD1 and TSD2 bands in $^{161,163,165,167}$Lu are signature partners, meaning that the phonon numbers are $(n_{w_1},n_{w_2})=(0,0)$ for both bands
    \item two fitting procedures are employed for $^{163}$Lu, as the fourth triaxial band is obtained from the coupling of a different valence proton ($h_{9/2}$) than the other three wobbling bands. Results regarding energies and transition probabilities verify the experimental data very well
    \item in Chapter \ref{chapter-5-novel} the novel approach called $\mathbf{W_2}$ is employed, treating TSD2 and TSD4 in $^{163}$Lu as parity partners: \emph{a set of bands with opposite parity that emerge from the same single-particle alignment (the $i_{13/2}$ proton), but the core is different}
    \item The formalism $\mathbf{W_2}$ adopts a unified fitting procedure of the entire spectrum of $^{163}$Lu, and the experimental data are very well reproduced (see Figs. \ref{results-parity-partners-163lu-1} - \ref{results-parity-partners-163lu-2})
    \item geometrical interpretations of the wobbling motion are realized within the space generated by the angular momentum components, showing the classical trajectories of the $\mathbf{I}$ (recall the set of Figs. \ref{classical-trajectory-TSD1-plot} - \ref{classical-trajectory-TSD4-plot})
    \item classical trajectories and the the identification of stable/unstable wobbling motion (see Figs. \ref{contour-cef-polar-tsd1} - \ref{contour-cef-polar-tsd4}) are unique concepts that emerge from this research, and they provide a clear picture of the dynamics of a triaxial system
    \item for a given spin state within the wobbling spectra, the nucleus can execute precessional motion up to a certain energy (critical point), and beyond this value, a phase transition in the rotational regime emerges, where the total angular momentum changes the axis it precesses about
\end{itemize}

% TO-DO: add summary + conclusions for Boson Description chapter
Chapter \ref{extra-chapter-new-boson} is dedicated to a completely different description of the wobbling mechanism, via a boson description.

\emph{Based on all the discussions and results presented in this work, the semi-classical analysis of the wobbling motion in triaxial nuclei proves to be an efficient and remarkable tool, giving realistic results that are on par with alternative description, which are more complex and difficult. Keeping a close contact with the classical dynamics is indeed a remarkable characteristic of the developed model.}

%ending phrase
%TO-DO: update the number of publications after latest chapter is added
{\color{red}\textbf{XXX}} publications are summarized herein, namely two research papers that introduce the re-normalization in terms of Signature Partner Bands for the first two triaxial bands in $^{161,163,165,167}$Lu (i.e., Refs. \cite{raduta2020approach,raduta2020towards}), two more papers that extends this formalism with the Parity Partner Bands in $^{163}$Lu (that is Refs. \cite{poenaru2021parity,poenaru2021extensive1}), and lastly a paper devoted to the geometry of the wobbling mode in odd-mass nuclei (i.e., Ref. \cite{poenaru2021extensive2}). In addition, two more papers published by the team (namely Refs. \cite{raduta2020new} and \cite{raduta2022simultaneous}) that cover odd-mass nuclei give unique results concerning the unified description of the wobbling + chiral phenomena, which are treated on an equal footing.