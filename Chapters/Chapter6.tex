\chapter{A New Theoretical Formalism}

In this chapter, a formalism that describes the wobbling properties in odd-$A$ nuclei will be presented. The model was developed recently by the current team (i.e., Raduta and Poenaru) and applied to $^{135}$Pr \cite{raduta2020new}, and the `Lu family' with $^{161,165,167}$Lu \cite{raduta2020approach} and $^{163}$Lu \cite{raduta2020approach,raduta2020towards,poenaru2021parity,poenaru2021extensive1,poenaru2021extensive2}. This framework is an original contribution to the field of nuclear structure, with focus on the theoretical aspects of collective phenomena in nuclei.

\section{Previous Work - Foundation}

From a development standpoint, it is instructive to review the previous `stages' that lead to the current work, Since the description of odd-$A$ nuclei achieved in here is strongly connected with former calculations done by the team. Consequently, in this section a brief overview of the wobbling motion in $^{163}$Lu done by Raduta et al. in 2017 \cite{raduta2017semiclassical} will be given. Indeed, by using a \emph{semi-classical} approach, the wobbling properties of this odd-mass isotope were accurately described. 

The semi-classical approaches tend to be very useful when applied to problems having a \emph{quantal} Hamiltonian that contains quantities which behave as in the \emph{classical limit} when certain constraints or approximations are made. Moreover, these methods always keep the dynamics of the system in close contact with the classical features, which in principle are easier to interpret (i.e., a clear physical meaning). For this reason, the collective properties of $^{163}$Lu were characterized by such a method. Dealing with an odd nucleus it makes sense to adopt a Particle + Rotor Model in which the odd quasi-particle couples to an even-even core. From the total Hamiltonian $H=H_\text{rot}+H_\text{sp}$ (recall discussion on the PRM model \ref{triaxial-prm-general-hamiltonian} and also QTR Hamiltonian \ref{oddA-QTR-general-hamiltonian}), one obtained the energy spectrum for the four wobbling bands in this nucleus. Remarking the fact that there is another debate regarding the `true' nature of the fourth triaxial strongly-deformed band. For example, Jensen et al. \cite{jensen2004coexisting} suspect that this band is built from single-particle excitations, meaning that the states to not show wobbling mechanism. However, Tanabe et al. \cite{tanabe2008selection} is in favor of attributing $n_w=3$ for TSD4. More details on the interpretation of TSD4 will be made in a future section.

From the seminal work from 2017 done by the team, several features are emphasized:
\begin{itemize}
    \item The four TSD bands are considered as zero-, one-, two-, and three-wobbling phonon bands for TSD1, TSD2, TSD3, and TSD4, respectively
    \item Each excited band is obtained by acting on the yrast (TSD1) band with one-, two-, and three-phonons, respectively (e.g., a state $I$ from TSD2 is obtained by acting with the wobbling-phonon operator on a state $I-1$ from TSD1)
    \item Wobbling structure for the group TSD1-2-3 emerged from a proton $i_{13/2}$ ($\mathcal{Q}_p$) where all spin states have positive parity
    \item The band TSD4 has spin states with negative parity, and it is built on a proton from the $h_{9/2}$ orbital (also a $\mathcal{Q}_p$)
    \item In the expression of the rotor Hamiltonian, the rigid-like MOI were adopted (see Eq. \ref{eq-irrotational-rigid-mois}) which depend on $\gamma$ and $\mathcal{I}_0$
    \item Analytical expressions for the energies were expressed in terms of total spin and wobbling phonon numbers
    \item Experimental data was reproduced through a fitting procedure, with the free parameters $\mathcal{I}_0^{-1}$ (rotor part) and a \emph{scaling factor} $s=V\cdot \mathcal{I}_0$ (single-particle part)
    \item Deformation parameters $\beta_2$ and $\gamma$ were a priori fixed (taken from literature)
\end{itemize}