\chapter{Wobbling Motion Study via a Boson Description}
\label{extra-chapter-new-boson}

The last part will be focused on the same wobbling phenomenon but with a different approach than the ones employed in Chapters \ref{chapter-4-aw1-formalism} and \ref{chapter-5-novel}, as this method does not share the same foundational concepts as the $\mathbf{W_1}$ and $\mathbf{W_2}$ techniques. The results shown here correspond to a recent publication made by the team in Ref. \cite{raduta2020new}, this entire chapter consisting of a summary with all the unique features.

The $\mathbf{W_0}$ formalism developed by the team in Ref. \cite{raduta2017semiclassical} described the wobbling motion for even-even nuclei using a \emph{boson expansion} of the angular momentum components. The initial problem consisted in a triaxial rigid rotor with moments of inertia $\mathcal{I}k$, which described by the known rotor Hamiltonian (recall Section \ref{trm-model} and Eq. \ref{general-rotor-hamiltonian}):
\begin{align}
    \hat{H}_R=\frac{\hat{R}_1}{2\mathcal{I}_1}+\frac{\hat{R}_2}{2\mathcal{I}_2}+\frac{\hat{R}_3}{2\mathcal{I}_3}\ ,
\end{align}
and angular momentum components $\hat{R}_k$. From this initial quantal problem, a variational principle (similar recipe as the one described in Section \ref{variational-principle-section}, but only for one set of coordinates) brings the system into a classical view, where the eigenvalue problem becomes a system of classical equations in a phase space (see Section II.A and II.B from Ref. \cite{raduta2017semiclassical}). Summarizing the calculations from Ref. \cite{raduta2017semiclassical}, one obtained a pair of conjugate variables (i.e., $r$ and $\varphi$) and two equations of motion (recall discussion from Section \ref{equations-of-motion-section} and Eq. \ref{eq-of-motion-approach-w1}) as:
\begin{align}
    \frac{\partial\mathcal{H}}{\partial r}=\dot{\varphi}\ ,\ \frac{\partial\mathcal{H}}{\partial\varphi}=-\dot{r}\ .
\end{align}

If the average values of the angular momentum components are expressed in terms of the phase space coordinates $(r,\varphi)$, then the following equations emerge:
\begin{align}
    J_+^\text{cls}&\equiv\langle\hat{R}_+\rangle=\sqrt{r(2I-r)}e^{\iu\varphi}\ ,\nonumber \\
    J_-^\text{cls}&\equiv\langle\hat{R}_-\rangle=\sqrt{r(2I-r)}e^{-\iu\varphi}\ ,\nonumber \\
    J_3^\text{cls}&\equiv\langle\hat{R}_3\rangle=r-I\ ,
    \label{classical-am-components-boson}
\end{align} 
with their algebra being governed by the Poisson brackets:
\begin{align}
    \left\{J_+^\text{cls},J_-^\text{cls}\right\}=-2\iu J_3^\text{cls}\ ,\ \left\{J_\pm^\text{cls},J_3^\text{cls}\right\}=\pm\iu J_{\pm}^\text{cls}\ .
\end{align} 

A pair of complex functions $f$ and $g$, which are defined in this classical phase space, will have their associated Poisson bracket defined in the following way:
\begin{align}
    \left\{f,g\right\}=\frac{\partial f}{\partial \varphi}\frac{\partial g}{\partial r} - \frac{\partial f}{\partial r}\frac{\partial g}{\partial \varphi}
\end{align} 

Furthermore, the classical coordinates belonging to that phase space were re-quantized through the \emph{Dyson boson expansion} \cite{dyson1956general}. Firstly, the quantization of complex coordinates  pair of complex functions defined on the phase space  

Starting with two canonical complex variables:
\begin{align}
    \mathcal{C}=\sqrt{2I}\sqrt{\frac{2I-r}{r}}e^{-\iu\varphi}\ ,\nonumber\\
    \mathcal{B}=\frac{1}{\sqrt{2I}}\sqrt{(2I-r)(r)}e^{\iu\varphi}\ ,
\end{align}


Remarkable that this is the first time within literature when this kind of representation is successfully applied to odd-$A$ nuclei.