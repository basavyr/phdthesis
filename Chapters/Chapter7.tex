\chapter{General Conclusions}
\label{chapter-8-conclusions}

This thesis presents a comprehensive analysis of several publications on the subject of Nuclear Structure. The primary objective was to investigate nucleonic matter that exhibits a lack of axial symmetry. In recent years, nuclear triaxiality has gained prominence as a topic of significant interest due to the challenges associated with its experimental measurement. Furthermore, the theoretical description of triaxially deformed nuclei necessitates the use of specific methods or approximations that can be quite complex.
% This thesis represents the work of several publications focused on the topic of Nuclear Structure. The study of the nucleonic matter that lacks any axial symmetry was the main objective of the team. Nuclear triaxiality became a hot topic over the last decade due to its challenges of measuring it experimentally. Moreover, the theoretical description of triaxially deformed nuclei requires certain methods or approximations, which can become quite cumbersome.


Starting with Section \ref{section-nuclear-shapes} from \textbf{Chapter} \ref{chapter-2-theoretical-aspects}, the nuclear surface is introduced in Eq. \eqref{nuclear-shape}, which was parametrized in terms of the collective coordinates and spherical harmonics. The relevant excitation mode for triaxiality is given by the quadrupole deformation, having $\lambda=2$. The quadrupole deformation introduces two parameters that give an insight with respect to the elongation and departure from axial symmetry of a nucleus, by means of the quadrupole deformation parameter $\beta_2$ and triaxiality parameter $\gamma$. The two parameters, which are provided in Eq. \eqref{bohr-deformation-params}, dictate the stretching of the nuclear axes, and this was shown in Fig. \ref{nuclear-radius-elongation}. From the representation of a general ellipsoid in terms of $\beta_2$ and $\gamma$, all the possible shapes that posses axial symmetry appear at certain values of $\gamma$, while the unique triaxial region is found in the region $\gamma\in(0,60)$ (see Fig. \ref{beta-gamma-plane}).

The theoretical study of deformed nuclei is performed in Chapter \ref{chapter-2-theoretical-aspects}, where the Nilsson Model is employed (recall Section \ref{subsection-nilsson-model}). The single-particle energies are obtained as a sum between the anisotropic harmonic oscillator, a spin-orbit term, and the centrifugal term. The last two terms are defined with the strength parameters $\kappa$ and $\mu$, which are specific to this theory. The collective model is described in the same chapter (see Section \ref{subsection-collective-model}), and it emphasized the behavior of the nuclear shapes in terms of the moments of inertia, i.e., the irrotational and rigid MOI provided in Eq. \eqref{eq-irrotational-rigid-mois}. These quantities are crucial to the development of the model. Their behavior with respect to the asymmetry parameter $\gamma$ was depicted in Fig. \ref{fig-irrotational-rigid-mois}. In terms of nuclear rotations and vibrations, several experimental spectra are graphically shown in Fig. \ref{rotational-bands-odd-a} from Appendix \ref{appendix:ral-dal-signature-scheme}, Figs. \ref{energy-levels-120Te-virbational-band} - \ref{energy-levels-63Cu-virbational-band}, and Fig. \ref{rotational-bands-even-even}. The spectra of nuclear rotations and vibrations are important for understanding wobbling nuclei with respect to the angular momentum (the nuclear vibrations and rotations are treated in Sections \ref{subsection-nuclear-vibration} and \ref{subsection-nuclear-rotation}, respectively). 

Important quantities for collective phenomena used in the numerical application, such as the moments of inertia (including the kinematic and dynamical MOI from Eqs. \eqref{kinematic-moi-general} - \eqref{dynamic-moi-general}) and the quadrupole moment (measure of deformation or departure of the nuclear shape away from spherical symmetry) were discussed in detail throughout Section \ref{c3-collective-quantities}. The Hamiltonian depicted in Eq. \eqref{eq-triaxial-prm-full-hamiltonian} marks the onset of the theoretical model adopted in this work, and from the Triaxial PRM Hamiltonian, the energy spectra and transition probabilities are obtained for odd-mass Lu isotopes.

\emph{Chiral bands} and \emph{wobbling motion} are unique fingerprints of triaxial deformation. Any experimental identification of these two effects is a decisive test that can pinpoint triaxial nuclei. For the chiral case, it was shown in Section \ref{subsection-chiral-motion} that the spectra emerge from the coupling of three angular momenta: a proton, a neutron, and the core, leading to a trihedral system lacking chiral symmetry. Some examples of spectra are shown in Fig. \ref{chiral-bands-1} - \ref{chiral-bands-2}. In Fig. \ref{chiral-geometry} the geometry of the mutual coupling between the angular momenta is shown. Also in Chapter \ref{chapter-2-theoretical-aspects}, the Triaxial Particle + Rotor Model has been analytically treated (section \ref{tprm-model}), since that Hamiltonian was the foundation of this work. Firstly the Hamiltonian for the symmetric case is considered, starting with Eq. \eqref{general-rotor-hamiltonian}, then a single-particle term is added via Eq. \eqref{triaxial-prm-general-hamiltonian}. This term represents the motion of a valence nucleon within a quadrupole deformed mean-field generated by an even-even core. As a matter of fact, this single-particle term is given by the Nilsson's deformed shell model (Eq. \eqref{single-particle-nilsson-defored-potential}). The single-particle potential strength $V$ is used to parametrize the interaction and it is incorporated into the numerical implementation of the energy spectrum. As a purely theoretical application, the matrix elements for the single-particle energies of protons and neutrons are calculated according to Eq. \eqref{single-particle-energies-hpn}. A set of graphical representations are made, showing their behavior with respect to the quadrupole deformation parameter $\beta_2$ and triaxiality $\gamma$. 

Triaxial nuclei rotate around any of the three principal axes, with the main rotation about the axis with largest MOI. The contribution from the other two axes has a vibrational character, and through a first approximation, this kind of motion can be described analytically by a harmonic-like Hamiltonian. In \textbf{Chapter} \ref{chapter-3} it is shown that the wobbling motion differs from the even-$A$ to odd-$A$ nuclei. Indeed, for systems with even number of nucleons, the energy spectrum is achieved in the so-called \emph{Harmonic Approximation} by Eq. \eqref{eq-wobbling-energy-evenA}, where the wobbling phonon number represents a tilting strength of the total angular momentum away from the axis with largest MOI. The final energy spectrum is characterized by a rotational motion around this MOI and a frequency of oscillation of the nucleus. The behavior follows Eq. \eqref{wobbling-frequency-even-A} and it is depicted in Fig. \ref{fig-even-even-wobbling-energies}. The \emph{Harmonic Approximation} is tested for $^{130}$Ba nucleus, which has two wobbling bands. The experimental data are well reproduced by a fitting procedure that has the three moments of inertia as free parameters. The actual workflow regarding the numerical algorithm for calculating wobbling energies and transition probabilities is provided in Section \ref{ba-130-numerical-calculations} (see results in Figs. \ref{plot-ba130-excitation-energies} - \ref{BE2out-transitions-130ba}). While the results can be compared to alternative methods found in literature such as Chen et al. \cite{chen2019transverse}, they are distinct and specific to this work.

For the wobbling motion in odd-mass nuclei, a \emph{Frozen Approximation} is illustrated, showing that the spectrum contains a similar harmonic-like term, but with a different behavior for the wobbling frequency. This behavior is dictated by the alignment of the odd-particle with the even-even triaxial core. Two wobbling scenarios may arise depending on whether the particle’s angular momentum aligns itself along (\emph{longitudinal}) or perpendicular (\emph{transverse}) to the axis of largest moment of inertia. The workflow diagrams \ref{advanced-quasiparticle-coupling-1} - \ref{advanced-quasiparticle-coupling-3} show how these two wobbling regimes can occur, based on an overlap between the density distributions of the core and the single-particle. These diagrams also illustrate the precessional + oscillatory behavior of the total a.m., which is the first geometrical representation of such a precessional cone in the literature. The wobbling frequency in odd nuclei was analyzed via Eq. \eqref{wobbling-frequency-odd-A-MOI}, showing that it depends on the three MOI as in Fig. \ref{wobbling-freq-oddA}. In addition, the catalogue from Fig. \ref{wobbling-diagram-chart} containing all known wobblers is another remarkable result of this research. Therein, the experimental wobbling energy, the number of wobbling bands, and deformation parameters ($\beta_2$ and $\gamma$) are shown for each nucleus.

\textbf{Chapters} \ref{chapter-3}, \ref{chapter-4-aw1-formalism}, and \ref{chapter-5-novel} do also have separate sets of conclusions and discussions, so one can refer to the individual sections \ref{chapter-3-concluding-remarks}, \ref{chapter-4-concluding-remarks}, \ref{chapter-5-concluding-remarks}, respectively. For this reason, only a brief revision of the emerging characteristics is mentioned here:
\begin{itemize}
    \item an initial quantal Hamiltonian of Triaxial Particle-Rotor type is employed as the foundational tool for describing wobbling motion%in odd-mass triaxial nuclei
    \item the Hamiltonian is brought to a classical form by means of the \emph{variational principle} (according to Eq. \eqref{tdve-approach-w1})
    \item in the classical view, the dynamics are described by two sets of coordinates, i.e., one for the even-even core and one for the single-particle (Eqs. \eqref{changed-rho-sigma-variables} - \eqref{eq-of-motion-approach-w1})
    \item the excitation energies can be analytically determined through the utilization of five free parameters, as indicated in Equation \eqref{fitting-parameters-p-fit}
    \item \textbf{Chapter} \ref{chapter-4-aw1-formalism} employs a re-normalization of the wobbling bands (Eqs. \eqref{renormalized-bands-structure-TSD124} - \eqref{renormalized-bands-structure-TSD3}) by applying the variational principle to not only the ground-state, but also other bands. This is called $\mathbf{W_1}$ formalism throughout the thesis. The TSD1 and TSD2 bands in $^{161,163,165,167}$Lu are signature partners, meaning that the phonon numbers are $(n_{w_1},n_{w_2})=(0,0)$
    \item two fitting procedures are employed for $^{163}$Lu, as the fourth triaxial band is obtained from the coupling of a different valence proton ($h_{9/2}$). Results regarding energies and transition probabilities verify the experimental data very well
    \item \emph{the novel approach} called $\mathbf{W_2}$ is employed in \textbf{Chapter} \ref{chapter-5-novel}, treating TSD2 and TSD4 in $^{163}$Lu as \emph{parity partners}: \emph{a set of bands with opposite parity that emerge from the same single-particle alignment (the $i_{13/2}$ proton) but different core}
    \item The formalism $\mathbf{W_2}$ adopts a unified fitting procedure for the spectrum of $^{163}$Lu, and the experimental data are very well reproduced (see Figs. \ref{results-parity-partners-163lu-1} - \ref{results-parity-partners-163lu-2})
    \item geometrical interpretations of the wobbling motion are realized in the space generated by the angular momentum components, showing the classical trajectories of the $\mathbf{I}$ (recall the set of Figs. \ref{classical-trajectory-TSD1-plot} - \ref{classical-trajectory-TSD4-plot})
    \item classical trajectories and the identification of \emph{stable}/\emph{unstable} wobbling motion (see Figs. \ref{contour-cef-polar-tsd1} - \ref{contour-cef-polar-tsd4}) are unique concepts of this research and they provide a clear picture with the dynamics of a triaxial system
    \item for a given spin state belonging to the wobbling spectra, the nucleus can execute precessional motion up to a certain energy (critical point). Beyond this value a phase transition occurs, where the total angular momentum changes the axis it precesses about.
\end{itemize}

The last chapter is dedicated to a completely different description for the wobbling motion, but also studied in the framework of odd-mass nuclei. Through a unique method that requires a boson description of the angular momenta and also the help of the Jacobi Elliptic Functions \cite{jacobi1829fundamenta}, the wobbling motion in $^{135}$Pr nucleus is treated in a classical fashion, obtaining results that agree with the experimental measurements. More conclusions are available in Section \ref{new-boson-concluding-remarks} within the chapter.

\emph{The semi-classical analysis of wobbling motion in triaxial nuclei, as portrayed in this work, has proven to be a remarkable and efficient tool. It yields realistic results that are comparable to those obtained through more complex and difficult alternative descriptions. A notable characteristic of the developed model is its close congruence with classical dynamics.}
% \emph{Based on all the discussions and results presented in this work, the semi-classical analysis of the wobbling motion in triaxial nuclei proves to be an efficient and remarkable tool, giving realistic results that are on par with alternative description, which are more complex and difficult. Keeping a close contact with the classical dynamics is indeed a remarkable characteristic of the developed model.}

To summarize the entire work encapsulated in this thesis, \textbf{six publications} are presented: \textbf{two} research papers that introduce the re-normalization in terms of Signature Partner Bands for the triaxial bands in $^{161,163,165,167}$Lu (i.e., Refs. \cite{raduta2020approach,raduta2020towards}), \textbf{two} more papers that extend this formalism with Parity Partner Bands in $^{163}$Lu (that is Refs. \cite{poenaru2021parity,poenaru2021extensive1}), \textbf{one} paper devoted to the geometry of the wobbling mode in odd-mass nuclei (i.e., Ref. \cite{poenaru2021extensive2}), and lastly, \textbf{one} paper that covers odd-mass nuclei and give a unified description of the wobbling + chiral phenomena (i.e., Ref. \cite{raduta2020new}). %TO-DO: update this paragraph with reference to a dedicated appendix where each paper is given in detail

%TO-DO: add another reference to the appendix with details about the development of the thesis itself.