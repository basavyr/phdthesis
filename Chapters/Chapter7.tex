\chapter{Novel Description of the Wobbling Bands}
\label{chapter-7-novel}

In this chapter, a unique picture for the description of wobbling bands will be developed. This formalism is based on the $\mathbf{W_1}$ model, which was systematically treated in the previous chapter, showing its validity through the comparison with the experimental measurements. It will also follow a semi-classical picture, and the main goal is to fully describe the wobbling spectra of nuclei in which admixtures of positive- and negative-parity bands occur. Consequently, the interpretation will be tested on $^{163}$Lu, since the fourth wobbling band (TSD4) consists of states $I^\pi$ where $\pi=-1$.

The work that will be depicted here is in fact based on three papers that the current team published. Indeed, in Ref. \cite{poenaru2021parity} the key concept that stands behind this interpretation were pointed out and new results concerning the excitation energies were obtained, while in Refs. \cite{poenaru2021extensive1,poenaru2021extensive2}, the authors took the model even further to calculate other quantities such as Routhians, rotational frequencies, and wobbling energy (in the sense of Eq. \ref{eq-wobbling-energy-definition-oddA}). Moreover, the geometrical interpretation of the \emph{Classical Energy Function} (CEF) was extensively studied in terms of the behavior near the critical points, meaning that an interest is devoted in finding stable/unstable wobbling motion. On top of that, graphical representations with two constants of motion, i.e., the total energy and the total angular momentum are realized for this nucleus, pointing out the \emph{allowed} classical trajectories of the system. As it will be shown, the information that can be retrieved from these figures proves to be an efficient way of understanding this collective phenomenon unique to triaxial nuclei.