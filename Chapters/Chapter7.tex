\chapter{Novel Description of the Wobbling Bands}
\label{chapter-7-novel}

In this chapter, a unique picture for the description of wobbling bands will be developed. This formalism is based on the $\mathbf{W_1}$ model, which was systematically treated in the previous chapter, showing its validity through the comparison with the experimental measurements. It will also follow a semi-classical picture, and the main goal is to fully describe the wobbling spectra of nuclei in which admixtures of positive- and negative-parity bands occur. Consequently, the new method will be tested on $^{163}$Lu, since the fourth wobbling band (TSD4) consists of states $I^\pi$ where $\pi=-1$.

The work that will be depicted here is in fact based on three papers published by the team. Indeed, in Ref. \cite{poenaru2021parity}, the key concept standing behind this interpretation were pointed out and new results concerning the excitation energies were obtained, while in Refs. \cite{poenaru2021extensive1,poenaru2021extensive2}, the authors took the model even further to calculate other quantities such as Routhians, rotational frequencies, and wobbling energy (in the sense of Eq. \ref{eq-wobbling-energy-definition-oddA}). Moreover, the geometrical interpretation of the \emph{Classical Energy Function} (CEF) was extensively studied in terms of its behavior near the critical points, meaning that an interest is devoted in finding stable/unstable wobbling motion. On top of that, graphical representations with two constants of motion, i.e., the total energy and the total angular momentum are realized for this nucleus, pointing out the \emph{allowed} classical trajectories of the system. As it will be shown, the information that can be retrieved from these figures proves to be an efficient way of understanding this collective phenomenon unique to triaxial nuclei. The classical study of the wobbling picture for $^{163}$Lu in \cite{poenaru2021extensive2} is a remarking characteristics of the model, as the alternative treatments within literature are based on quantal pictures that do not have a clear and `easy-to-grasp' physical meaning regarding system dynamics. In addition, the study of collective motion in terms of stability diagrams in relation to a classical set of coordinates is another first in the literature.

This chapter will be structured as follows: $1)$ a part that will cover the energy spectrum of $^{163}$Lu (as per Ref. \cite{poenaru2021extensive1}) and $2)$ a part that will be focused on the classical view of the stability and trajectories (according to the investigations from Ref. \cite{poenaru2021extensive2}). More concisely, part $1)$ will cover:
\begin{itemize}
    \item introduction of \emph{Signature Partner} + \emph{Parity Partner Bands}
    \item new analytical formula for the wobbling spectrum for $^{163}$Lu
    \item numerical calculations for the excitation energies
    \item interpretation of the free parameters in relation to other studies
\end{itemize}
while part $2)$ will be focused on:
\begin{itemize}
    \item formalism of the energy function employed in the context of Parity Partner Bands
    \item analytical expressions for CEF in the critical regions
    \item contour plots are constructed with the obtained CEF(s)
    \item \emph{nuclear trajectories} (i.e., intersection curves of the energy and angular momentum) 
\end{itemize}

Obviously, for both parts some general discussions will be made, emphasizing the main results that emerge from the considerations.

\section{Parity Partner Bands}

Recalling the main features that were adopted in the $\mathbf{W_1}$ formalism for the odd$-A$ $^{163}$Lu nucleus, all its wobbling properties were described through the TDVE (Eq. \ref{tdve-approach-w1}), which was used in the bands TSD1-2 and TSD4, respectively. The TSD3 band was the one-wobbling-phonon band built as an excitation on top of TSD2. The set TSD1-2 were considered Signature Partner Bands, where the former has the favored and the latter has the unfavored signature quantum number. These two bands are formed by the odd proton $i_{13/2}$ (that is, $\mathcal{Q}_1$ from Table \ref{lu-163-phonon-numbers}), where the valence proton will couple with an even-even core $\mathscr{C}_1=0,2,4,\dots$ for TSD1 and TSD2 is generated by the coupling of the proton with the $\mathscr{C}_2=1,3,5,\dots$ core. For consistency reasons, the same notations used in $\mathbf{W_1}$ will be kept here. Both $\mathscr{C}_1$ and $\mathscr{C}_2$ are of positive parity (as per the re-normalizations from Eq. \ref{renormalized-bands-structure-TSD124}).

Although the variational principle was applied to TSD4 (ground-state with zero-wobbling-phonon numbers), taking into account the negative parity states, a different odd quasi-particle was considered as particle + core alignment of the nucleus. More precisely, the $h_{9/2}$ proton of negative parity (i.e., $\mathcal{Q}_2$ from Table \ref{lu-163-phonon-numbers}) coupled with the core $\mathscr{C}_2$ of odd spin states. Obviously, the total parity was given in terms of the positive parity of the core states and the negative proton. In terms of parities of the wobbling bands in $^{163}$Lu, they can be summarized as:
\begin{align}
    \pi_\text{TSD1}=\pi_{\mathscr{C}_1}\pi_{\mathcal{Q}_1}=+1\ ,\nonumber\\
    \pi_\text{TSD2}=\pi_{\mathscr{C}_2}\pi_{\mathcal{Q}_1}=+1\ ,\nonumber\\
    \pi_\text{TSD3}=\pi_\text{TSD2}\pi_{\Gamma^\dagger}=+1\ ,\nonumber\\
    \pi_\text{TSD4}=\pi_{\mathscr{C}_2}\pi_{\mathcal{Q}_2}=-1\ ,
\end{align}
where for the band TSD3, the positive parity is given by the fact that the phonon operator applied to TSD2 does not change the parity, even though the spins are increased by one unit.