\chapter{Novel Description of the Wobbling Bands}
\label{chapter-7-novel}

In this chapter, a unique picture for the description of wobbling bands will be developed. This formalism is based on the $\mathbf{W_1}$ model, which was systematically treated in the previous chapter, showing its validity through the comparison with the experimental measurements. It will also follow a semi-classical picture, and the main goal is to fully describe the wobbling spectra of nuclei in which admixtures of positive- and negative-parity bands occur. Consequently, the new method will be tested on $^{163}$Lu, since the fourth wobbling band (TSD4) consists of states $I^\pi$ where $\pi=-1$.

The work that will be depicted here is in fact based on three papers published by the team. Indeed, in Ref. \cite{poenaru2021parity}, the key concept standing behind this interpretation were pointed out and new results concerning the excitation energies were obtained, while in Refs. \cite{poenaru2021extensive1,poenaru2021extensive2}, the authors took the model even further to calculate other quantities such as Routhians, rotational frequencies, and wobbling energy (in the sense of Eq. \ref{eq-wobbling-energy-definition-oddA}). Moreover, the geometrical interpretation of the \emph{Classical Energy Function} (CEF) was extensively studied in terms of its behavior near the critical points, meaning that an interest is devoted in finding stable/unstable wobbling motion. On top of that, graphical representations with two constants of motion, i.e., the total energy and the total angular momentum are realized for this nucleus, pointing out the \emph{allowed} classical trajectories of the system. As it will be shown, the information that can be retrieved from these figures proves to be an efficient way of understanding this collective phenomenon unique to triaxial nuclei. The classical study of the wobbling picture for $^{163}$Lu in \cite{poenaru2021extensive2} is a remarking characteristics of the model, as the alternative treatments within literature are based on quantal pictures that do not have a clear and `easy-to-grasp' physical meaning regarding system dynamics. In addition, the study of collective motion in terms of stability diagrams in relation to a classical set of coordinates is another first in the literature.

This chapter will be structured as follows: $1)$ a part that will cover the energy spectrum of $^{163}$Lu (as per Ref. \cite{poenaru2021extensive1}) and $2)$ a part that will be focused on the classical view of the stability and trajectories (according to the investigations from Ref. \cite{poenaru2021extensive2}). More concisely, part $1)$ will cover:
\begin{itemize}
    \item introduction of \emph{Signature Partner} + \emph{Parity Partner Bands}
    \item new analytical formula for the wobbling spectrum for $^{163}$Lu
    \item numerical calculations for the excitation energies
    \item interpretation of the free parameters in relation to other studies
\end{itemize}
while part $2)$ will be focused on:
\begin{itemize}
    \item formalism of the energy function employed in the context of Parity Partner Bands
    \item analytical expressions for CEF in the critical regions
    \item contour plots are constructed with the obtained CEF(s)
    \item \emph{nuclear trajectories} (i.e., intersection curves of the energy and angular momentum) 
\end{itemize}

Obviously, for both parts some general discussions will be made, emphasizing the main results that emerge from the considerations.

\section{Parity Partner Bands}

Recalling the main features that were adopted in the $\mathbf{W_1}$ formalism for the odd$-A$ $^{163}$Lu nucleus, all its wobbling properties were described through the TDVE (Eq. \ref{tdve-approach-w1}), which was used in the bands TSD1-2 and TSD4, respectively. The TSD3 band was the one-wobbling-phonon band built as an excitation on top of TSD2. The bands TSD1-2 were considered Signature Partner Bands, where the former has the favored and the latter has the unfavored signature quantum number. These are formed by the odd proton $i_{13/2}$ (that is, $\mathcal{Q}_1$ from Table \ref{lu-163-phonon-numbers}), where the valence proton will couple with an even-even core $\mathscr{C}_1=0^+,2^+,4^+,\dots$ for TSD1 and $\mathscr{C}_2=1^+,3^+,5^+,\dots$ core for TSD2. For consistency reasons, the same notations used in $\mathbf{W_1}$ will be kept here. Both $\mathscr{C}_1$ and $\mathscr{C}_2$ are of positive parity (as per the re-normalizations from Eq. \ref{renormalized-bands-structure-TSD124}).

Although the variational principle was applied to TSD4 as ground-state with zero-wobbling-phonon numbers, a different valence nucleon was coupled with the rotational core. More precisely, the $h_{9/2}$ proton (i.e., $\mathcal{Q}_2$ from Table \ref{lu-163-phonon-numbers}) coupled with the core $\mathscr{C}_2$. Obviously, the total nucleus' parity was given in terms of the positive parity of the core states and the negative proton. In the wobbling bands of $^{163}$Lu, the parities can be summarized as:
\begin{align}
    \pi_\text{TSD1}=&\pi_{\mathscr{C}_1}\pi_{\mathcal{Q}_1}=+1\ ,\nonumber\\
    \pi_\text{TSD2}=&\pi_{\mathscr{C}_2}\pi_{\mathcal{Q}_1}=+1\ ,\nonumber\\
    \pi_\text{TSD3}=&\pi_\text{TSD2}\pi_{\Gamma^\dagger}=+1\ ,\nonumber\\
    \pi_\text{TSD4}=&\pi_{\mathscr{C}_2}\pi_{\mathcal{Q}_2}=-1\ ,
    \label{aw1-parity-list-TSD-bands}
\end{align}
where for the band TSD3, the positive parity is given by the fact that the phonon operator applied to TSD2 does not change the parity, even though the spins are increased by one unit. 

Despite the fact that the $\mathbf{W_1}$ model describes the wobbling phenomenon in triaxial nuclei successfully, it still encounters a major inconsistency within its formalism: \emph{two different quasi-particles arise in the coupling mechanism that is typical to a Particle-Rotor-Model}. Although there are other studies where such couplings are required (i.e., Ref. \cite{nandi2020first} also employs the $\mathcal{Q}_{1,2}$ quasi-particles in order to study the wobbling mechanism in $^{183}$Au), achieving a unified coupling scheme would make this model a `robust' tool for odd-mass triaxial nuclei. Namely, it is worth investigating if instead of dealing with two valence nucleons (one for TSD1-3 and one for TSD4), all bands created by the same valence nucleon. This would (as it will be shown later) result in dealing with only one fitting procedure for the excited spectrum of this isotope. Since the intruder $i_{13/2}$ is shown to cause the triaxiality in TSD1-3, and because of the similarities between this group and the fourth band TSD4, the quasi-particle $\mathcal{Q}_1$ seems to be the proper candidate to the new coupling re-normalization. Indeed, adopting only the proton from $i$-orbital with $j=13/2$, the coupling schemes in $^{163}$Lu will be expressed in a compact form as \cite{poenaru2021extensive1}:
\begin{enumerate}
    \item Coupling $C_1'$: the $\mathcal{Q}_1$ proton aligns itself with the (positive) core of even spin states $\mathscr{C}_1=0^+,2^+,4^+,\dots$.
    \item Coupling $C_2'$: the $\mathcal{Q}_1$ proton couples with the (positive) core of odd spin states $\mathscr{C}_2^+=1^+,3^+,5^+,\dots$.
    \item Coupling $C_3'$: the $\mathcal{Q}_1$ proton couples with the (negative) core of odd spin states $\mathscr{C}_2^-=1^-,3^-,5^-,\dots$.
\end{enumerate}

Here, the two cores used in $C_2'$ and $C_3'$, respectively, are labelled with the superscripts $+$ and $-$ in order to distinguish them by the opposite parity. Obviously, these new couplings $\left\{C_1',C_2',C_3'\right\}$ preserve the rules outlined in Eq. \ref{aw1-parity-list-TSD-bands}. It is conspicuous that $C_1'$ corresponds to TSD1, $C_2'$ to the band TSD2, and lastly $C_3'$ defines the band TSD4. No changes are applied to TSD3, which is the one-wobbling-phonon band. Hereafter, this new formalism will be referred to as $\mathbf{W_2}$, to be differentiated from $\mathbf{W_1}$.

Even though TSD4 is now considered a zero-phonon band generated by the quasi-particle $\mathcal{Q}_1$, a relationship between it and its other neighboring bands must be established. In the $\mathbf{W_1}$ approach it was proven \cite{raduta2020approach} that signature (recall Eq. \ref{signature-quantum-number}) is a good quantum number and that the wave-function admits states with both positive and negative signatures. This lead to TSD1 and TSD2 being signature partners, since their similar properties and spin differences pointed towards such a consideration. Going further and looking at the the two sequences $\mathscr{C}_2^+$ and $\mathscr{C}_2^-$ that were assigned as triaxial even-even cores to TSD2 and TSD4, it could suggest that they are \emph{Parity Partner Bands} \cite{poenaru2021parity}. This idea assumes that two rotational structures with energy states following a $\propto I(I+1)$ trend and having opposite parities can co-exist near the same deformation region. Additionally, the two partners have $\Delta I=2$ between states belonging to the same band and $\Delta = 1$ for adjacent states.

\subsection{Parity of the wave-function}

Since the re-normalization of the band TSD4 is based on the idea that it is the parity partner of TSD2, a discussion about the parity quantum number within this semi-classical picture should be. Namely, one has to check how the trial function employed in the VP (Eq. \ref{tdve-approach-w1}) behaves under a parity transformation. The parity transformation in quantum mechanics (also called \emph{space reflection}) is the operation through which the coordinate axes change sign. An example of such a general transformation can be given in terms of the Cartesian system:
\begin{align}
    (x,y,z)\stackrel{\hat{P}}{\longrightarrow}(-x,-y,-z)
\end{align}

The wave-function can at most change a sign under the parity transformation, meaning that one can have:
\begin{align}
    \hat{P}\Psi(\mathbf{r})=\Psi(-\mathbf{r})=\Psi(\mathbf{r})
\end{align}
for the \emph{positive parity states} and:
\begin{align}
    \hat{P}\Psi(\mathbf{r})=\Psi(-\mathbf{r})=-\Psi(\mathbf{r})
\end{align}
for the \emph{negative parity states}.

In the case of the classical energy function defined throughout this formalism, the parity transformation (parity operator) is defined as the product between the complex conjugation operator and a rotation of angle $\pi$ around the quantization axis. Moreover, the total parity is defined as the product between the parity operator of the core and the single-particle one:
\begin{align}
    \hat{P}_T=\hat{P}_\text{core}\otimes\hat{P}_\text{sp}\ .
\end{align}

\subsection{Redefined Energy Spectrum}
