\chapter{Novel Description of the Wobbling Bands}
\label{chapter-7-novel}

In this chapter, a unique picture for the description of wobbling bands will be developed. This formalism is based on the $\mathbf{W_1}$ approach, which was systematically treated in the previous chapter, showing its validity through the comparison with the experimental measurements. It will also follow a semi-classical picture, and the main goal is to fully describe the wobbling spectra of nuclei in which admixtures of positive- and negative-parity bands occur. Consequently, the new method will be tested on $^{163}$Lu, since the fourth wobbling band (TSD4) consists of states $I^\pi$ where $\pi=-1$.

The work that will be depicted here is in fact based on three papers published by the team. Indeed, in Ref. \cite{poenaru2021parity}, the key concept standing behind this interpretation were pointed out and new results concerning the excitation energies were obtained, while in Refs. \cite{poenaru2021extensive1,poenaru2021extensive2}, the authors took the model even further to calculate other quantities such as Routhians, rotational frequencies, and wobbling energies. Moreover, the geometrical interpretation of the \emph{Classical Energy Function} (CEF) was extensively studied in terms of its behavior near the critical points, meaning that an interest is devoted in finding stable/unstable wobbling motion. On top of that, graphical representations with two constants of motion, i.e., the total energy and the total angular momentum are realized for this nucleus, pointing out the \emph{allowed} classical trajectories of the system. As it will be shown, the information retrieved from these figures proves to be an efficient way of understanding this collective phenomenon unique to triaxial nuclei. The classical study of the wobbling picture for $^{163}$Lu in \cite{poenaru2021extensive2} is a remarking characteristic of the model, as the alternative treatments from the literature are based on quantal pictures that do not have a clear and `easy-to-grasp' physical meaning regarding the system dynamics. In addition, the study of collective motion in terms of stability diagrams in relation to a classical set of coordinates is another first within literature.

This chapter will be structured as follows: $1)$ a part that will cover the energy spectrum of $^{163}$Lu (as per Ref. \cite{poenaru2021extensive1}) and $2)$ a part that will be focused on the classical view of the stability and trajectories (according to the investigations from Ref. \cite{poenaru2021extensive2}). More concisely, part $1)$ will cover:
\begin{itemize}
    \item introduction of \emph{Signature Partner} + \emph{Parity Partner Bands}
    \item new analytical formula for the wobbling spectrum for $^{163}$Lu
    \item numerical calculations for the excitation energies
    \item interpretation of the free parameters in relation to other studies
\end{itemize}
while part $2)$ will be focused on:
\begin{itemize}
    \item formalism of the energy function employed in the context of Parity Partner Bands
    \item analytical expressions for CEF in the critical regions
    \item contour plots are constructed with the obtained CEF(s)
    \item \emph{nuclear trajectories} (i.e., intersection curves of the energy and angular momentum) 
\end{itemize}

Obviously, for both parts some general discussions will be made, emphasizing the main results that emerge from the considerations.

\section{Parity Partner Bands}
\label{parity-partners-renormalizaion}

Recalling the main features that were adopted in the $\mathbf{W_1}$ formalism for the odd$-A$ $^{163}$Lu nucleus, all its wobbling properties were described through the TDVE (Eq. \ref{tdve-approach-w1}), which was used in the bands TSD1-2 and TSD4, respectively. The TSD3 band was the one-wobbling-phonon band built as an excitation on top of TSD2. The bands TSD1-2 were considered Signature Partner Bands, where the former has the favored and the latter has the unfavored signature quantum number. These are formed by the odd proton $i_{13/2}$ (that is, $\mathcal{Q}_1$ from Table \ref{lu-163-phonon-numbers}), where the valence proton will couple with an even-even core $\mathscr{C}_1=0^+,2^+,4^+,\dots$ for TSD1 and $\mathscr{C}_2=1^+,3^+,5^+,\dots$ core for TSD2. For consistency reasons, the same notations used in $\mathbf{W_1}$ will be kept here. Both $\mathscr{C}_1$ and $\mathscr{C}_2$ are of positive parity (as per the re-normalizations from Eq. \ref{renormalized-bands-structure-TSD124}).

Although the variational principle was applied to TSD4 as ground-state with zero-wobbling-phonon numbers, a different valence nucleon was coupled with the rotational core. More precisely, the $h_{9/2}$ proton (i.e., $\mathcal{Q}_2$ from Table \ref{lu-163-phonon-numbers}) coupled with the core $\mathscr{C}_2$. Obviously, the total nucleus' parity was given in terms of the positive parity of the core states and the negative proton. In the wobbling bands of $^{163}$Lu, the parities can be summarized as:
\begin{align}
    \pi_\text{TSD1}=&\pi_{\mathscr{C}_1}\pi_{\mathcal{Q}_1}=+1\ ,\nonumber\\
    \pi_\text{TSD2}=&\pi_{\mathscr{C}_2}\pi_{\mathcal{Q}_1}=+1\ ,\nonumber\\
    \pi_\text{TSD3}=&\pi_\text{TSD2}\pi_{\Gamma^\dagger}=+1\ ,\nonumber\\
    \pi_\text{TSD4}=&\pi_{\mathscr{C}_2}\pi_{\mathcal{Q}_2}=-1\ ,
    \label{aw1-parity-list-TSD-bands}
\end{align}
where for the band TSD3, the positive parity is given by the fact that the phonon operator applied to TSD2 does not change the parity, even though the spins are increased by one unit. 

Despite the fact that the $\mathbf{W_1}$ model describes the wobbling phenomenon in triaxial nuclei successfully, it still encounters a major inconsistency within its formalism: \emph{two different quasi-particles arise in the coupling mechanism that is typical to a Particle-Rotor-Model}. Although there are other studies where such couplings are required (i.e., Ref. \cite{nandi2020first} also employs the $\mathcal{Q}_{1,2}$ quasi-particles in order to study the wobbling mechanism in $^{183}$Au), achieving a unified coupling scheme would make this model a `robust' tool for odd-mass triaxial nuclei. Namely, it is worth investigating if instead of dealing with two valence nucleons (one for TSD1-3 and one for TSD4), all bands created by the same valence nucleon. This would (as it will be shown later) result in dealing with only one fitting procedure for the excited spectrum of this isotope. Since the intruder $i_{13/2}$ is shown to cause the triaxiality in TSD1-3, and because of the similarities between this group and the fourth band TSD4, the quasi-particle $\mathcal{Q}_1$ seems to be the proper candidate to the new coupling re-normalization. Indeed, adopting only the proton from $i$-orbital with $j=13/2$, the coupling schemes in $^{163}$Lu will be expressed in a compact form as \cite{poenaru2021extensive1}:
\begin{enumerate}
    \item Coupling $C_1'$: the $\mathcal{Q}_1$ proton aligns itself with the (positive) core of even spin states $\mathscr{C}_1=0^+,2^+,4^+,\dots$.
    \item Coupling $C_2'$: the $\mathcal{Q}_1$ proton couples with the (positive) core of odd spin states $\mathscr{C}_2^+=1^+,3^+,5^+,\dots$.
    \item Coupling $C_3'$: the $\mathcal{Q}_1$ proton couples with the (negative) core of odd spin states $\mathscr{C}_2^-=1^-,3^-,5^-,\dots$.
\end{enumerate}

Here, the two cores used in $C_2'$ and $C_3'$, respectively, are labelled with the superscripts $+$ and $-$ in order to distinguish them by the opposite parity. Obviously, these new couplings $\left\{C_1',C_2',C_3'\right\}$ preserve the rules outlined in Eq. \ref{aw1-parity-list-TSD-bands}. It is conspicuous that $C_1'$ corresponds to TSD1, $C_2'$ to the band TSD2, and lastly $C_3'$ defines the band TSD4. No changes are applied to TSD3, which is the one-wobbling-phonon band. Hereafter, this new formalism will be referred to as $\mathbf{W_2}$, to be differentiated from $\mathbf{W_1}$.

Even though TSD4 is now considered a zero-phonon band generated by the quasi-particle $\mathcal{Q}_1$, a relationship between it and its other neighboring bands must be established. In the $\mathbf{W_1}$ approach it was proven \cite{raduta2020approach} that signature (recall Eq. \ref{signature-quantum-number}) is a good quantum number and that the wave-function admits states with both positive and negative signatures. This lead to TSD1 and TSD2 being signature partners, since their similar properties and spin differences pointed towards this consideration. Going further and looking at the the two sequences $\mathscr{C}_2^+$ and $\mathscr{C}_2^-$ that were assigned as triaxial even-even cores to TSD2 and TSD4, it could suggest that they are \emph{Parity Partner Bands} \cite{poenaru2021parity}. This idea assumes that two rotational structures with energy states following a $\propto I(I+1)$ trend and having opposite parities can co-exist near the same deformation region. Additionally, the two partners have $\Delta I=2$ between states belonging to the same band and $\Delta = 1$ for adjacent states.

\subsection{Parity of the wave-function}

Since the re-normalization of the band TSD4 is based on the idea that it is the parity partner of TSD2, a discussion about the parity quantum number within this semi-classical picture should be. Namely, one has to check how the trial function employed in the VP (Eq. \ref{tdve-approach-w1}) behaves under a parity transformation. The parity transformation in quantum mechanics (also called \emph{space reflection}) is the operation through which the coordinate axes change sign. An example of such a general transformation can be given in terms of the Cartesian system:
\begin{align}
    (x,y,z)\stackrel{\hat{P}}{\longrightarrow}(-x,-y,-z)
\end{align}

The wave-function can at most change a sign under the parity transformation, meaning that one can have:
\begin{align}
    \hat{P}\Psi(\mathbf{r})=\Psi(-\mathbf{r})=\Psi(\mathbf{r})\ ,
\end{align}
for the \emph{positive parity states} and:
\begin{align}
    \hat{P}\Psi(\mathbf{r})=\Psi(-\mathbf{r})=-\Psi(\mathbf{r})\ ,
\end{align}
for the \emph{negative parity states}. In the case of this formalism, the parity transformation (parity operator) is defined as the product between the complex conjugation operator and a rotation of angle $\pi$ around the quantization axis. Moreover, the total parity of an odd-mass nucleus can be defined as the product between the parity operator of the core and the single-particle one:
\begin{align}
    \hat{P}_T=\hat{P}_\text{core}\otimes\hat{P}_\text{sp}\ .
    \label{parity-operator-aw2}
\end{align}

The trial function was defined in terms of the set of variables $(z,s)$  through Eq. \ref{z-s-variables}, which were further expressed using $(r,\varphi;f,\psi)$ according to the transformations Eq. \ref{changed-rho-sigma-variables}. When acting with the parity operator defined in Eq. \ref{parity-operator-aw2} on the this trial function, the change of coordinates follows the rule:
\begin{align}
    \hat{P}_T\Psi_{IM;j}(r,\varphi;f,\psi)=\Psi_{IM;j}(r,\varphi+\pi;f,\psi+\pi)\stackrel{not.}{=}\bar{\Psi}\ ,
\end{align}
and the invariance of the classical energy function to rotations with $\pi$ around the quantization axis gives:
\begin{align}
    \mathcal{H}(r,\varphi;f,\psi)=\mathcal{H}(r,\varphi+\pi;f,\psi+\pi)\ .
\end{align}

The remarking feature of the last two equations is that the wave-function describing the triaxial system and its image through the action of $\hat{P}_T$ form a set of two linearly dependent functions, which differ only by a multiplicative constant $p$, where $|p|=1$. Consequently, $p$ can either be $+1$ or $-1$ resulting in the following relationship:
\begin{align}
    \bar{\Psi}=&p\Psi\nonumber\ ,\\
    \bar{\Psi}(r,\varphi;f,\psi)=&\pm\Psi(r,\varphi;f,\psi)\ .
\end{align}

This important result shows that the triaxial rotor admits eigenfunctions with positive and negative parity, so the trial function could describe wobbling bands with both parities. For the case of $\Psi(r,\varphi;f,\psi)$ and the Hamiltonian describing $^{163}$Lu, \emph{TSD4 band can emerge from the coupling of the same odd proton as the other three bands TSD1-3}. Concluding this discussion, it was shown that parity is a good quantum number and the two bands TSD2 and TSD4 can indeed be considered parity partners, i.e., they are generated by the coupling of a triaxial even-even core with an identical odd quasi-particle. Another interesting characteristic coming out from this work is related to the energies in both bands. Concisely, TSD2 states lie lower than those of its negative partner. New experimental data on wobbling nuclei comprising both positive and negative parity bands could encourage pursuing a study on the relative energy spacings between the partners and see if a general pattern can be inferred.

\subsection{Redefined Energy Spectrum}

The Hamiltonian of the triaxial system remains unchanged in $\mathbf{W_2}$ and the Variational Principle from Eq. \ref{tdve-approach-w1} keeps the same structure. However, instead of obtaining an energy spectrum  for the TSD bands as the one defined in Eqs. \ref{tsd-bands-general-spectrum}, \ref{lu163-absolute-energies-tsd1234}, a different class of equations has to be used. From the phonon term $\mathcal{F}_{n_{w_1}n_{w_2}}^I$ (Eq. \ref{phononic-term-tsd-energies}) and the minimal energy $\mathcal{H}_\text{min}^I$ (Eq. \ref{minimal-energy-term-hmin}), the modified spectrum of $^{163}$Lu is given as \cite{poenaru2021parity}:
\begin{align}
    E_{I,0,0}^\text{TSD1}=&\epsilon_{13/2}+\mathcal{H}_\text{min}^I+\mathcal{F}_{00}^I\nonumber\ ,\ I^\pi=13/2^+,17/2^+,21/2^+\dots\ ,\\
    E_{I,0,0}^\text{TSD2}=&{\color{red}\epsilon_{13/2}^1}+\mathcal{H}_\text{min}^I+\mathcal{F}_{00}^I\nonumber\ ,\ I^\pi=27/2^+,31/2^+,35/2^+\dots\ ,\\
    E_{I,1,0}^\text{TSD3}=&\epsilon_{13/2}+\mathcal{H}_\text{min}^{I-1}+\mathcal{F}_{10}^{I-1}\nonumber\ ,\ I^\pi=33/2^+,37/2^+,41/2^+\dots\ ,\\
    E_{I,0,0}^\text{TSD4}=&{\color{blue}\epsilon_{13/2}^2}+\mathcal{H}_\text{min}^I+\mathcal{F}_{00}^I\ ,\ I^\pi=47/2^-,51/2^-,55/2^-\dots\ ,
    \label{tsd-energies-lu163-parity-partners}
\end{align}
where one kept a consistent notation as in the previous approach. The analytical terms from Eq. \ref{tsd-energies-lu163-parity-partners} contain different single-particle energies, which is in contrast to $\mathbf{W_1}$. This re-normalization implies a change in the particle + core interaction, in the sense that the single-particle mean-field of the unfavored partner states (TSD2) and negative partner states (TSD4) are affected by the parity change $\pi=+1\to \pi=-1$ for TSD4 (signified by blue color in Eq. \ref{tsd-energies-lu163-parity-partners}) and the alternating signature between TSD1-2 (signified by the red color in Eq. \ref{tsd-energies-lu163-parity-partners}) \cite{poenaru2021parity,poenaru2021extensive1,poenaru2021extensive2}. Actually, the two corrections come at the cost of the \emph{unified} fitting procedure that is now applied to $^{163}$Lu, as compared to $\mathbf{W_1}$. Looking at the spectrum of Eq. \ref{tsd-energies-lu163-parity-partners}, it is obvious that the same fitting parameters from Eq. \ref{fitting-parameters-p-fit} will be used here.

\section{Numerical Results}

Proceeding with the workflow illustrated in Fig. \ref{fitting-workflow-fig}, a new numerical implementation will be performed in this section. The results together with discussions will be presented herein. Because $^{163}$Lu is the only isotope to have positive and negative parity bands in the same collective structure, the model will be tested on this nucleus. Firstly, the normalization adopted in $\mathbf{W_2}$ is sketched in Table \ref{lu163-table-info}, and with the information from this table, the minimization procedure for the $\chi^2$-function can be carried out.
\begin{table}
    \centering
      \begin{tabular}{llllll}
      \hline
      Band & $n_s$ & $\mathbf{j}_\mathcal{Q}$ & $\mathbf{R}_\mathscr{C}$ - Sequence & $\mathbf{I}$ - Sequence & Coupling  \\
      \hline
      \hline
        TSD1 & $21$ & $\mathcal{Q}_1$ & $\mathscr{C}_1=0^+,2^+,4^+,\dots$   & $13/2^+,17/2^+,21/2^+,\dots$ & $C'_1$        \\
        TSD2 & $17$ & $\mathcal{Q}_1$ & $\mathscr{C}_2^+=1^+,3^+,5^+,\dots$ & $27/2^+,31/2^+,35/2^+,\dots$ & $C'_2$        \\
        TSD3 & $14$ & $\mathcal{Q}_1$ & 1-phonon exc.                       & $33/2^+,37/2^+,41/2^+,\dots$ & \\
        TSD4 & $11$ & $\mathcal{Q}_1$ & $\mathscr{C}_2^-=1^-,3^-,5^-,\dots$ & $47/2^-,51/2^-,55/2^-,\dots$ & $C'_3$        \\
      \hline
    \end{tabular}
    \caption{The number of energy states $n_s$ within each wobbling band of $^{163}$Lu, the valence nucleon that is $j^\pi=13/2^+$, the core's a.m. $\mathbf{R}_\mathscr{C}$, the nucleus' a.m. $\mathbf{I}$, and the corresponding coupling scheme that was established according to the $\mathbf{W_2}$ model as per Section \ref{parity-partners-renormalizaion}.}
    \label{lu163-table-info}
\end{table}

In what follows, the spectrum of excitation energies will be analyzed. Table \ref{lu163-parameters-parity-fitting} shows the results of the fitting method in the $\mathbf{W_2}$ formalism, where the three moments of inertia, the single particle potential strength, and the triaxiality parameter are provided. The root mean square error for the excitation energy is $E_\text{rms}\approx 79\ \text{keV}$, while the formalism $\mathbf{W_1}$ gave $E_\text{rms}\approx 240$ keV \cite{raduta2020towards}. In fact, this is the first semi-classical study of $^{163}$Lu within the literature achieving an agreement of under $100\ \text{keV}$ with the experimental measurements for the wobbling spectrum \cite{poenaru2021extensive1}. Note that a three-fold improvement of the overall deviation is achieved for the theoretical calculations, as opposed to $\mathbf{W_1}$.
\begin{table}
    \centering
    \begin{tabular}{lllll}
        \hline
        $\mathcal{I}_1$ [$\hbar^2$/\text{MeV}] & $\mathcal{I}_2$ [$\hbar^2$/\text{MeV}]& $\mathcal{I}_3$ [$\hbar^2$/\text{MeV}] & $\gamma$ [deg. ] & $V$ [\text{MeV}] \\
        \hline
        \hline
        72              & 15              & 7               & 22       & 2.1\\
        \hline
    \end{tabular}
    \caption{The parameter set $\mathcal{P}$ of Eq. \ref{fitting-parameters-p-fit} obtained by minimizing the $\chi^2$-function for $^{163}$Lu in the re-normalization of $\mathbf{W_2}$. A unique fitting was applied for all four TSD bands of the isotope.}
    \label{lu163-parameters-parity-fitting}
\end{table}

In reference to the excitation energies of Eq. \ref{general-excitation-energy-fitting-model}, when the band-head $E_{13/2}^\text{TSD1}$ is subtracted, the single-particle mean-field corrections present in Eq. \ref{tsd-energies-lu163-parity-partners} become $\epsilon_{13/2}^1-\epsilon_{13/2}$ and $\epsilon_{13/2}^2-\epsilon_{13/2}$ for TSD2 and TSD4, respectively, and they were slightly varied such that a consistency with the measured data is maintained. In the current computations the two quantities are $\epsilon_{13/2}^1-\epsilon_{13/2}=0.3$ MeV and $\epsilon_{13/2}^2-\epsilon_{13/2}=0.6$ MeV each. Evaluating the signature splitting between the first two states and the terminus states in TSD2 (i.e., $E^\text{TSD2}_{27/2}-E^\text{TSD2}_{25/2}$ and $E^\text{TSD2}_{91/2}-E^\text{TSD2}_{89/2}$) with the parameter set $\mathcal{P}_\text{fit}$, the values $0.492\ \text{MeV}$ and $0.936\ \text{MeV}$ are obtained, which agree with the estimates made by Jensen et al. \cite{jensen2002wobbling}. The obtained excitation energies are graphically represented in Figs. \ref{results-parity-partners-163lu-1} - \ref{results-parity-partners-163lu-2}, where for each band only the first and last spin values are labelled. Remarkable is the very small differences across the states belonging to TSD1-3, while for TSD4 there is a small downward shift in the region $47/2\leq I\leq 55/2$ and an upward shift when $I\geq 75/2$. A possible reason for the over-estimation of high-spin states might be the \emph{ad-hoc} mean-field correction that was adopted for TSD4 (recall Eq. \ref{tsd-energies-lu163-parity-partners}). On the other hand, the classical energy function provided by the VP seems to have a steep minimum, causing the system to be more cranked near $I=47/2$.
\begin{figure}
    \centering
    \includegraphics[width=0.5\textwidth]{Chapters/Figures/parity-partners-plots/tsd1.pdf}
    \includegraphics[width=0.49\textwidth]{Chapters/Figures/parity-partners-plots/tsd2.pdf}
    \caption{The excitation energies (Eq. \ref{general-excitation-energy-fitting-model}) of $^{163}$Lu for the bands TSD1 and TSD2, obtained in the $\mathbf{W_2}$ formalism using Eq. \ref{tsd-energies-lu163-parity-partners}. Calculations were done with the parameters presented in Table \ref{lu163-parameters-parity-fitting}.}
    \label{results-parity-partners-163lu-1}
\end{figure}
\begin{figure}
    \centering
    \includegraphics[width=0.49\textwidth]{Chapters/Figures/parity-partners-plots/tsd3.pdf}
    \includegraphics[width=0.49\textwidth]{Chapters/Figures/parity-partners-plots/tsd4.pdf}
    \caption{The excitation energies (Eq. \ref{general-excitation-energy-fitting-model}) of $^{163}$Lu for the bands TSD3 and TSD4 , obtained in the $\mathbf{W_2}$ formalism using Eq. \ref{tsd-energies-lu163-parity-partners}. Calculations were done with the parameters presented in Table \ref{lu163-parameters-parity-fitting}.}
    \label{results-parity-partners-163lu-2}
\end{figure}

Also worth discussing is the difference $\delta_{42}=E_I^\text{TSD4}-E_I^\text{TSD2}$, where the almost constant value $\delta_{42}\approx 0.3\ \text{MeV}$ is observed, suggesting that the states with the same angular momentum from TSD2 and TSD4 might emerge from the parity projection of a single wave-function lacking reflection symmetry (hence the change $\pi=+1\to\pi=-1$). For the sake of completeness, the two wobbling frequencies $\Omega_{1,2}$ that comprise the phonon term $\mathcal{F}^I_{n_{w_1}n_{w_2}}$ defined in Eq. \ref{phononic-term-tsd-energies} are also represented as function of the total angular momentum in Fig. \ref{lu163-wobbling-frequencies-parity}. It can be seen that the solution $\Omega_2$ is much larger over the entire spin range, indicating that the odd proton's a.m. exhibits a more pronounced precessional motion than that of the core. Indeed, the difference is only about $1.5-2\ \text{MeV}$ for low $I$ and it goes to $3.5-4\ \text{MeV}$ in the large spin limit. Nevertheless, this is expected because the condition $\Omega_2>\Omega_1$ was fixed in Eq. \ref{omega-1-2-3-4-solutions}.
\begin{figure}
    \centering
    \includegraphics[scale=0.8]{Chapters/Figures/parity-partners-plots/wobblingFrequency.pdf}
    \caption{The wobbling frequencies, i.e., the two solutions of Eq. \ref{wobbling-freq-oddA} as function of the total spin. Calculations are made with the parameter set provided in Table \ref{lu163-parameters-parity-fitting}.}
    \label{lu163-wobbling-frequencies-parity}
\end{figure}

\subsection{Parameter set - Interpretation}

The set $\mathcal{P}_\text{fit}$ attained after the minimization procedure of the $\chi^2$-function is given in Table \ref{lu163-parameters-parity-fitting}. However, the fitting method only gives the numerical values, without providing context on the physical interpretation of the results themselves. As such, in this section one should make a quantitative analysis for the five parameters and see how the current model compares with existing studies. Concerning the magnitudes of the three moments of inertia, it is clear that the system rotates around the $1$-axis, as the largest MOI corresponds to this axis. Moreover, the $1$-axis MOI is much larger than the other two, signifying the nature of small-amplitude vibrations that are typical to a triaxial rotor (i.e., large rotational motion around $\mathcal{I}_1$ and a fluctuation generated by the anisotropy of $\mathcal{I}_{2,3}$). A graphical representation is shown in Fig. \ref{moi-comparison-w1-w2-formalisms} where the three moments of inertia are compared with the $\mathbf{W_1}$ formalism. Looking at the plot, it can be seen that $\mathcal{I}_1$ is the largest across all the fitting procedures, although $\mathbf{W_2}$ provides the largest one. It should be pointed out that these are \emph{effective} MOI of the system, that is the particle + triaxial-rotor. There is no spin dependence inferred for the three quantities, so a possible change in their ordering cannot be studied with the current description. Since the current approach is not a microscopic one, no presumptions on that causes the MOI ordering can be stated (e.g., orbital mixing and so on). As a matter of fact, the nature of the three MOI is neither rigid nor irrotational (see discussion from Chapter \ref{chapter-3} and Eq. \ref{experimental-MOI-vs-rig-irr}).
\begin{figure}
    \centering
    \includegraphics[scale=0.75]{Chapters/Figures/parity-partners-plots/W1-W2-Mois.pdf}
    \caption{Comparison between the MOI obtained in the two approaches $\mathbf{W_1}$ (as per Chapter \ref{chapter-6-aw1-formalism}) and $\mathbf{W_2}$ (current chapter). Note that for the previous formalism, one had a separate minimization of the $\chi^2$-function, hence the label `W1;TSD4'. The numerical values for $\mathcal{I}_{1,2,3}$ are taken from Tables \ref{numerical-fitting-parameters-Lu-isotopes} and \ref{lu163-parameters-parity-fitting}.}
    \label{moi-comparison-w1-w2-formalisms}
\end{figure}

Taking the moments of inertia given by the $\mathbf{W_2}$ model, a direct comparison with the rigid and hydrodynamical (irrotational) ones is made in Fig. \ref{w2-mois-comparison}. The rigid values are calculated using the first formula from Eq. \ref{eq-irrotational-rigid-mois} and the hydrodynamical values are attained using the second formula. It is noteworthy that the MOI provided by the fit suggest an irrotational-like behavior rather than a rigid-like for the triaxial system, since there is a larger discrepancy between the MOI of the $2$- and $3$-axes in the rigid case.
\begin{figure}
    \centering
    \includegraphics[width=0.49\textwidth]{Chapters/Figures/parity-partners-plots/rigid-mois-fit.pdf}
    \includegraphics[width=0.49\textwidth]{Chapters/Figures/parity-partners-plots/hydrodynamic-mois-fit.pdf}
    \caption{The rigid (\textbf{left}) and hydrodynamical (\textbf{right}) MOI as function of $\gamma$. The scaling factors for both types of MOI are determined by fixing $\mathcal{I}^\text{rig}_1$ and $\mathcal{I}_1^\text{irr}$ to the fitted $\mathcal{I}_1$ with the deformation parameters $(\beta,\gamma)=(0.38,22^\circ)$. The black, red, and blue points represented the numerical values of $\mathcal{I}_1$, $\mathcal{I}_2$, and $\mathcal{I}_3$ from Table \ref{lu163-parameters-parity-fitting}, respectively. The value $\gamma=22^\circ$ from the fit is marked with the dashed magenta line just to guide the eye.}
    \label{w2-mois-comparison}
\end{figure}

Going further with the analysis of the parameters, a value of $\gamma=22^\circ$ is obtained. The agreement with the predicted minimum of $^{163}$Lu \cite{jensen2002wobbling,jensen2004coexisting} is quite good, as the experimental minimum for the TSD structures is identified at $(\beta,\gamma)=(0.38,20^\circ)$ (recall Fig. \ref{pes-example-set-2}). Other studies that describe the wobbling properties of $^{163}$Lu take the triaxiality parameter fixed a priori (see Refs. \cite{tanabe2006algebraic,tanabe2017stability} for example), but the current approach determines $\gamma$ self-consistently. Remarkable the fact that the obtained value is slightly bigger than the one from $\mathbf{W_1}$ ($\gamma=17^\circ$). This might be caused by the larger ratios for $\mathcal{I}_1/\mathcal{I}_2=4.8$ and $\mathcal{I}_1/\mathcal{I}_3=10.2$ in $\mathbf{W_2}$ than the ones from $\mathbf{W_1}$ ($\mathcal{I}_1/\mathcal{I}_2=3.1$ and $\mathcal{I}_1/\mathcal{I}_3=6.3$, respectively), which also indicate a larger triaxiality.

Lastly, the single-particle potential strength should be discussed. It has a value of $V=2.1\ \text{MeV}$ and in $\mathbf{W_1}$ this parameter was $V^{\text{TSD1,2,3}}=3.1\ \text{MeV}$ and $V^\text{TSD4}=0.7\ \text{MeV}$. An explanation for its decrease in the present case might be due to the upward shift in the energy caused by the unfavored partner, or due to the energetic shift of the parity partner. Nevertheless, a quenching effect on the quadrupole deformation of the triaxial system due to either the signature splitting or the parity symmetry breaking is observed. Still, the obtained value seems to be consistent with the previous calculations, as $V$ is close to the average value of the two $V$'s from $\mathbf{W_1}$. Other interpretations developed using a similar single-particle term in the Hamiltonian adopted values around $V=1.6\ \text{MeV}$ \cite{tanabe2017stability}, however, that was for an isotope with smaller quadrupole deformation $\beta_2=0.18$.

According to the structure described in the beginning of this chapter, the above sections conclude part $1)$ of the research, which was supposed to describe the concept of Parity Partners, to re-define the excitation energy spectrum and apply it to $^{163}$Lu, and finally to give a some context on the interpretation of the free parameters. Since all points were properly covered, and one can go further to part $2)$ of this study, where the classical energy function must be re-evaluated under the assumptions of $\mathbf{W_2}$ approach.

\section{Classical Energy Function (CEF)}

From the Hamiltonian given by Eq. \ref{total-ham-approach-w1} (i.e., sum of the single-particle term in Eq. \ref{single-particle-ham-approach-w1} and the rotor term in Eq. \ref{rotor-ham-approach-w1}), one obtained the classical energy function by taking the Hamiltonian's average on the trial function (Eq. \ref{trial-function-appeoach-w1}). The structure of the CEF was firstly given in Section \ref{classical-energy-function-subsection} and even a qualitative analysis of its sub-terms has been employed in Figs. \ref{fig-A-varphi-canonical} - \ref{fig-A-gamma-canonical}. The minimum point of $\mathcal{H}$ was found to be $p_0=(0,I;0,j)$ (recall Eq. \ref{cef-minimum-point-p0}). These considerations are still important here. Adopting the polar coordinate system $(\theta,\varphi)$, the Cartesian components of the total angular momentum can be expressed as:
\begin{align}
    \mathbf{I}=&\left\{I_1,I_2,I_3\right\}\stackrel{not.}{=}\left\{x_1,x_2,x_3\right\}\ ,\nonumber\\
    x_1=I\sin\theta\cos\varphi\ ,&\ x_2=I\sin\theta\sin\varphi\ ,\ x_3=I\cos\theta\ ,
\end{align}
where the quantization axis is chosen as the $3$-axis. Going back to the expression of $\mathcal{H}$ from Eq. \ref{full-classical-energy-function} and evaluating it in the minimum point using the new coordinate system, a compact form is achieved \cite{poenaru2021extensive2}:
\begin{align}
    \left. \mathcal{H}\ \right\vert_{p_0}=I\left(I-\frac{1}{2}\right)\sin^2\theta\cdot \mathcal{A}_\varphi-2A_1Ij\sin\theta+T_\text{core}+T_\text{sp}\ ,
    \label{CEF-minimum-point-Tcore-Tsp}
\end{align}
where the last two terms are independent of the polar angles $(\theta,\varphi)$ and they are defined as \cite{poenaru2021extensive2}:
\begin{align}
    T_\text{core}&=\frac{I}{2}(A_1+A_2)+A_3I^2\ ,\\
    T_\text{s.p.}&=\frac{j}{2}(A_2+A_3)+A_1j^2-V\frac{2j-1}{j+1}\sin\left(\gamma+\frac{\pi}{6}\right)\ .
    \label{core-sp-sub-terms}
\end{align}

It would be instructive to see how the core factor from Eq. \ref{core-sp-sub-terms} behaves under different $A_k$ orderings and also how the single-particle term evolves for different $j$. Two such comparisons are depicted in Fig \ref{fig-t-core-sp-terms}. The variations of $T_\text{sp}$ at different $j$ values are not significant. On the other hand, the $A_k$ ordering affects $T_\text{core}$ with a large amount, especially for higher spins.
\begin{figure}
    \centering
    \includegraphics[width=0.48\textwidth]{Chapters/Figures/parity-partners-plots/t-core.pdf}
    \includegraphics[width=0.495\textwidth]{Chapters/Figures/parity-partners-plots/t-sp.pdf}
    \caption{\textbf{Right:} The term $T_\text{core}$ from Eq. \ref{core-sp-sub-terms} as function of the angular momentum $I$ at different orderings for $A_k$ (the arbitrary values $0.05, 0.025, 0.011\ \hbar^{-2}\text{MeV}$ were interchanged between each factor). The magenta line is for the inertia factors provided by the fit. \textbf{Left:} The term $T_\text{sp}$ from Eq. \ref{core-sp-sub-terms}, evaluated for three different single-particle angular momenta using the potential strength and the three MOI from Table \ref{lu163-parameters-parity-fitting}.}
    \label{fig-t-core-sp-terms}
\end{figure}

\subsection{CEF - Stability Regions}

The classical energy function from Eq. \ref{CEF-minimum-point-Tcore-Tsp} is expressed as using the classical coordinates that describe the dynamics of the core (i.e., $r,\varphi$) and the valence nucleon (i.e., $f,\psi$). However, $\mathcal{H}(r,\varphi;f,\psi)$ is nothing else than the average of $\hat{H}$ on the trial function. Going back into the quantal picture, the physical interpretation of Eq. \ref{CEF-minimum-point-Tcore-Tsp} is thus: \emph{the expectation value of the Hamiltonian} or \emph{the allowed energy states that the total system can have}. For this reason, analyzing the CEF around the critical points is crucial for identifying stability of the system concerning its the rotational motion. In this subsection, such calculations will be performed, namely the critical points of $\mathcal{H}$ will be identified and then graphical representations in the polar plane $(\theta,\varphi)$ will be made. If there are extremal points with minimum character present, then the so-called regions of stability could be pinpointed. These regions should indicate if there are any allowed rotational states for the system.

Firstly, all the critical points with minimum character for $\mathcal{H}$ are given in Table \ref{critical_points_h}. Furthermore, by fixing the intervals of variation for $\theta$ and $\varphi$ to be $\theta\in[0,\pi]$ and $\varphi\in[0,2\pi]$, respectively, some numerical evaluations for the CEF can be realized in this coordinate space. These are finally sketched as \emph{contour plots} for different spins belonging to the four triaxial bands in $^{163}$Lu. For the sake of completeness, contour plots for two spin states from each band will be constructed.
\begin{table}
    \centering
    \begin{tabular}{cccc}
    \hline
    Minimal point & $\theta$ [rad] & $\varphi$ [rad] & $A_k$ ordering \\
    \hline
    \hline
    $m_1$ & $\pi/2$ &   $0$     &   $A_3>A_2>A_1$   \\
    $m_2$ & $\pi/2$ &   $\pi$   &   $A_3>A_2>A_1$   \\
    $m_3$ & $\pi/2$ &   $2\pi$  &   $A_3>A_2>A_1$   \\
    \hline
    \end{tabular}
    \caption{The minimum points of $\mathcal{H}$ evaluated for the constraints from $\mathcal{P}_\text{fit}$.}
    \label{critical_points_h}
\end{table}
\begin{figure}
    \centering
    \includegraphics[width=0.49\textwidth]{Chapters/Figures/parity-partners-plots/contour-tsd1-1.pdf}
    \includegraphics[width=0.49\textwidth]{Chapters/Figures/parity-partners-plots/contour-tsd1-2.pdf}
    \caption{Classical energy function (Eq. \ref{CEF-minimum-point-Tcore-Tsp}) expressed in terms of the polar coordinates for two states from TSD1 (i.e., $I=25/2^+$ and $52/2^+$). The calculations are done using the parameters obtained by the fitting procedure (see Table \ref{lu163-parameters-parity-fitting}). The minimum points of $\mathcal{H}$ are also marked by the red dots, which are surrounded by the contours.}
\end{figure}
\begin{figure}
    \centering
    \includegraphics[width=0.49\textwidth]{Chapters/Figures/parity-partners-plots/contour-tsd1-1.pdf}
    \includegraphics[width=0.49\textwidth]{Chapters/Figures/parity-partners-plots/contour-tsd1-2.pdf}
    \caption{Classical energy function (Eq. \ref{CEF-minimum-point-Tcore-Tsp}) expressed in terms of the polar coordinates for two states from TSD2 (i.e., $I=25/2^+$ and $52/2^+$). The calculations are done using the parameters obtained by the fitting procedure (see Table \ref{lu163-parameters-parity-fitting}). The minimum points of $\mathcal{H}$ are also marked by the red dots, which are surrounded by the contours.}
\end{figure}