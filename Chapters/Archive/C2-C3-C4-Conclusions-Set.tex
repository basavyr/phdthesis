
% from Chapter 2
In this chapter, the concept of nuclear shape was introduced through the description of a nuclear radius parametrized in terms of collective coordinates and spherical harmonics, as per Eq. \ref{nuclear-shape}. The relevant excitations of the nuclear surface were emphasized, and the importance of $\lambda=2$ mode was illustrated in Section \ref{section:quadrupole-deformation}. A set of deformation parameters that describe the nuclear shape in terms of elongation and asymmetry, i.e., $\beta$ and $\gamma$ are also attained by means of geometrical transformations from Eq. \ref{bohr-deformation-params}. These two parameters are crucial quantities for the development of the theoretical models that will be presented in the later chapters. Lastly, a general characterization of the nuclear shapes in terms of $\beta$ and $\gamma$ parameters is depicted in Fig. \ref{beta-gamma-plane}, where both axial and also non-axial nuclei can be visualized. This will constitute a first step in achieving a geometrical interpretation of triaxial nuclei, which will be further described in Chapter \ref{chapter-7-novel} by means of a novel formalism uniquely attributed to this work.

% From Chapter 3

Concluding this chapter, all the relevant models required for understanding the collective phenomena in nuclei were systematically described, starting with the shell model characteristics, then going further with the representation of the single-particle states in deformed nuclei, and finally reaching the collective model. A consistent portrayal for the nuclear rotations and vibrations was given, with the emergence of a general Hamiltonian that describes the nucleus in terms of the $\beta$ and $\gamma$ degrees of freedom. Importance of these two deformation parameters was also emphasized throughout the sections. Lastly, the important quantities indicating collectiveness or deformations were introduced, as they will play a crucial role for testing any newly developed theory.