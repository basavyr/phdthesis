\chapter{Conclusions}
\label{chapter-8-conclusions}

This thesis represents the work of several publications focused on the topic of Nuclear Structure, but on the theoretical side. The study of the nucleonic matter that lacks any axial symmetry was the main objective of the team. Nuclear triaxiality became a hot topic over the last decade due to its challenges of measuring it experimentally. Moreover, the theoretical description of triaxially deformed nuclei requires certain methods or approximations, which can become quite cumbersome.

Starting with Chapter \ref{chapter-2}, the nuclear surface is introduced in Eq. \ref{nuclear-shape} and it was parametrized in terms of the collective coordinates and spherical harmonics. The relevant excitation mode for triaxiality is given by the quadrupole deformation, having $\lambda=2$. The quadrupole deformation introduces two parameters that give an insight with respect to the elongation and departure from axial symmetry of a nucleus, by means of the quadrupole deformation parameter $\beta_2$ and triaxiality parameter $\gamma$, which are provided in Eq. \ref{bohr-deformation-params}. The two parameters dictate the stretching of the nuclear axes, and this was shown in Fig. \ref{nuclear-radius-elongation}. From the representation of a general ellipsoid in terms of $\beta_2$ and $\gamma$, all the possible shapes that posses axial symmetry emerge at certain values of $\gamma$, while the unique triaxial region is found in the region $\gamma\in(0,60)$ (see Fig. \ref{beta-gamma-plane}).

The study of deformed nuclei is performed in Chapter \ref{chapter-3}, where the Nilsson Model is employed (recall Section \ref{nilsson-model-section}). The single-particle energies are obtained as a sum between the anisotropic harmonic oscillator, a spin-orbit term and the centrifugal term. The last two terms are defined with the strength parameters $\kappa$ and $\mu$, which are specific to this theory.

The Chapters \ref{chapter-5}, \ref{chapter-6-aw1-formalism}, and \ref{chapter-7-novel} do also have separate sets of conclusions and discussions, so one can refer to the individual sections. 

%ending phrase
Five publications are summarized herein, namely two research papers that introduce the re-normalization in terms of Signature Partner Bands for the first two triaxial bands in $^{161,163,165,167}$Lu (i.e., Refs. \cite{raduta2020approach,raduta2020towards}), two more papers that extends this formalism with the Parity Partner Bands in $^{163}$Lu (that is Refs. \cite{poenaru2021parity,poenaru2021extensive1}), and lastly a paper devoted to the geometry of the wobbling mode in odd-mass nuclei (i.e., Ref. \cite{poenaru2021extensive2}).