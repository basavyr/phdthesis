\chapter{Conclusions}
\label{chapter-8-conclusions}

This thesis represents the work of several publications focused on the topic of Nuclear Structure, but on the theoretical side. The study of the nucleonic matter that lacks any axial symmetry was the main objective of the team. Nuclear triaxiality became a hot topic over the last decade due to its challenges of measuring it experimentally. Moreover, the theoretical description of triaxially deformed nuclei requires certain methods or approximations, which can become quite cumbersome.

Starting with Chapter \ref{chapter-2}, the nuclear surface is introduced in Eq. \ref{nuclear-shape} and it was parametrized in terms of the collective coordinates and spherical harmonics. The relevant excitation mode for triaxiality is given by the quadrupole deformation, having $\lambda=2$. The quadrupole deformation introduces two parameters that give an insight with respect to the elongation and departure from axial symmetry of a nucleus, by means of the quadrupole deformation parameter $\beta_2$ and triaxiality parameter $\gamma$, which are provided in Eq. \ref{bohr-deformation-params}. The two parameters dictate the stretching of the nuclear axes, and this was shown in Fig. \ref{nuclear-radius-elongation}. From the representation of a general ellipsoid in terms of $\beta_2$ and $\gamma$, all the possible shapes that posses axial symmetry emerge at certain values of $\gamma$, while the unique triaxial region is found in the region $\gamma\in(0,60)$ (see Fig. \ref{beta-gamma-plane}).

The study of deformed nuclei is performed in Chapter \ref{chapter-3}, where the Nilsson Model is employed (recall Section \ref{nilsson-model-section}). The single-particle energies are obtained as a sum between the anisotropic harmonic oscillator, a spin-orbit term and the centrifugal term. The last two terms are defined with the strength parameters $\kappa$ and $\mu$, which are specific to this theory. The collective model that is described in the same chapter gave insight to the behavior of the nuclear shapes in terms of the moments of inertia. It was shown that two types of MOI can describe the behavior of the nucleus, i.e., the irrotational and rigid MOI provided in Eq. \ref{eq-irrotational-rigid-mois}. These quantities are crucial to the development of the model. Their behavior with respect to the asymmetry parameter $\gamma$ was depicted in Fig. \ref{fig-irrotational-rigid-mois}. In terms of nuclear rotations and vibrations, several experimental spectra are graphically shown in Fig. \ref{rotational-bands-even-even}, Fig. \ref{rotational-bands-odd-a}, and Figs. \ref{energy-levels-120Te-virbational-band} - \ref{energy-levels-63Cu-virbational-band}. The spectra of nuclear rotations and vibrations are important for understanding the behavior of the wobbling nuclei with respect to the angular momentum. Relevant quantities that will be measured and tested within the formalism are presented in Section \ref{c3-collective-quantities}. These are moments of inertia (including the kinematical in Eq. \ref{kinematic-moi-general} and dynamical in Eq. \ref{dynamic-moi-general}). The quadrupole moment is a measure of deformation or departure of the nuclear shape away from spherical symmetry. The most general expression of this quantity was shown in Eq. \ref{general-quadrupole-moment-Q0-charge}. The Hamiltonian depicted in Chapter \ref{chapter4} from Eq. \ref{eq-triaxial-prm-full-hamiltonian} marks the onset of the theoretical model that will be adopted in this work. Namely. from the Triaxial PRM Hamiltonian, energy spectra and transition probabilities are obtained for odd-mass Lu isotopes.

Chiral bands and wobbling motion are unique fingerprints of triaxiality, and experimental identification of these two effects is the decisive test that can pinpoint triaxial nuclei across the chart of nuclides. It was shown in Section \ref{chiral-section} that the spectra emerge from the coupling of three angular momenta: for a proton, for a neutron, and for the even core, leading to a trihedral system that lacks chiral symmetry. The typical energy spectrum consists of pairs of bands, and examples are shown in Fig. \ref{chiral-bands-1} - \ref{chiral-bands-2}. The schematic from Fig. \ref{chiral-geometry} shows the geometry of the mutual coupling between the angular momenta. Also in Chapter \ref{chapter4} the Triaxial Particle + Rotor Model has been analytically treated, since its Hamiltonian will be at the foundation of the theoretical formalism of this current work. Firstly the Hamiltonian for the symmetric case is considered, starting with Eq. \ref{general-rotor-hamiltonian}, then a single-particle term is added via Eq. \ref{triaxial-prm-general-hamiltonian}. This term represents the motion of a valence nucleon within a quadrupole deformed mean-field generated by an even-even core. As a matter of fact, this single particle term is given by the Nilsson's deformed shell model, and Eq. \ref{single-particle-nilsson-defored-potential} represents such a term. It is remarkable that the single-particle potential strength $V$ is used to parametrize the interaction. This parameter will be further employed within the numerical implementation of the energy spectrum. As a purely theoretical application, the matrix elements for the single-particle energies of protons and neutrons according to Eq. \ref{single-particle-energies-hpn} are calculated, and a set of graphical representations are made, showing their behavior with respect to the quadrupole deformation parameter $\beta_2$ and triaxiality $\gamma$. 

Triaxial nuclei rotate around any of the three principal axes, with the main rotation about the axis with largest MOI. The contribution from the other two axes has a vibrational phonon character, and through a first approximation, this kind of motion can be described analytically by a harmonic-like Hamiltonian. In Chapter \ref{chapter-5} it is shown that the wobbling motion given as a combined precession and oscillation of the total angular momentum around a fixed position differs from the even-$A$ to odd-$A$ nuclei.

%encapsulate the conclusions from chapter 5 6 and 7 into a single bit
The Chapters \ref{chapter-5}, \ref{chapter-6-aw1-formalism}, and \ref{chapter-7-novel} do also have separate sets of conclusions and discussions, so one can refer to the individual sections \ref{c5-concluding-remarks}, \ref{c6-concluding-remarks}, \ref{c7-concluding-remarks}, respectively.

%ending phrase
Five publications are summarized herein, namely two research papers that introduce the re-normalization in terms of Signature Partner Bands for the first two triaxial bands in $^{161,163,165,167}$Lu (i.e., Refs. \cite{raduta2020approach,raduta2020towards}), two more papers that extends this formalism with the Parity Partner Bands in $^{163}$Lu (that is Refs. \cite{poenaru2021parity,poenaru2021extensive1}), and lastly a paper devoted to the geometry of the wobbling mode in odd-mass nuclei (i.e., Ref. \cite{poenaru2021extensive2}).