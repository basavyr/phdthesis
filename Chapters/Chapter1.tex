\chapter{Introduction}

Ground-state nuclear shapes with spherical symmetry or axial symmetry are predominant across the char of nuclides. Near closed shells, the deformation is indeed sufficient that models based on spherical symmetries can be used to describe nuclear properties (e.g., energies, quadrupole moments, and so on). Besides the spherical and axially-symmetric shapes, the existence of triaxial nuclear deformation was theoretically predicted a long time ago \cite{bohr1998nuclear}. The rigid triaxiality of nuclei is defined by the asymmetry parameter $\gamma$, giving rise to unique quantum phenomena (this parameter will be characterized later on). The quantum mechanical properties of the rigid triaxial shapes drew a lot of attention within the nuclear physics community lately, since the description of nuclear properties for the deformed nuclei represents a great challenge from both an experimental and a theoretical standpoint (e.g., a great progress for the experimental evidence of strong nuclear deformation has only been possible after the 2000s). It is worth mentioning that some experiments concerning alpha-$\alpha$ particle reactions induced in heavy nuclei in the early 1960s (e.g., \cite{morinaga1963gamma}) helped producing 

The physics of \emph{high-spin} states have been studied from the early 1950s, with the major breakthrough on the theoretical side made by Bohr and Mottelson \cite{bohr1998nuclear}. The elusive properties of nuclear rotation were described in terms of the rotational degrees of freedom associated with other nuclear degrees of freedom (e.g, particle-vibration, quadrupole-quadrupole, parity and so on). 