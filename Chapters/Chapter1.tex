\chapter{Introduction}

Ground-state nuclei possessing spherical or axial symmetry are predominant across the char of nuclides. Near closed shells, the deformation is sufficient that models based on spherical symmetries can be used to describe nuclear properties (e.g., energies, quadrupole moments, and so on). Besides the spherical and axially-symmetric shapes, the existence of triaxial nuclear deformation was theoretically predicted a long time ago \cite{bohr1998nuclear}. The rigid triaxiality of nuclei is defined by the asymmetry parameter $\gamma$, giving rise to a unique behavior concerning the system dynamics. Lately, triaxial nuclei drew a lot of attention within the nuclear physics community, since the description of nuclear properties represents a real challenge from an experimental and also theoretical standpoint. Great progress towards experimental setups being able to perform measurements of high-spin has only been possible after the 2000s. It is worth mentioning that some experiments regarding alpha-$\alpha$ particle reactions induced in heavy nuclei in the early 1960s (e.g., \cite{morinaga1963gamma}) helped to produce a decent amount of data for the rotational in the high-spin region ($\geq 20 \hbar$).

The physics of \emph{high-spin} states has been studied since the early 1950s, with the breakthrough on the theoretical side made by Bohr and Mottelson \cite{bohr1998nuclear}. The nuclear rotation was described in terms of the rotational degrees of freedom associated with other nuclear degrees of freedom such as particle-vibration, quadrupole-quadrupole, parity, and so on. The spherical shell-model only describes nuclei near the closed shells. On the other hand, for nuclei that are far from closed shells, a deformed potential must be utilized. In the case of even-even nuclei, unique band structures resulting from vibrations and rotations of the nuclear surface appear in the energy range $0-2\ \text{MeV}$.

Even though triaxiality has an elusive character, two phenomena, i.e., wobbling motion and chiral bands are uniquely attributed to triaxial shapes. Consequently, these two were intensively searched by using advanced techniques. In this work, the focus is given exclusively on the first effect, although it is worth specifying that the current team provides a unified description of both phenomena in their most recent work \cite{raduta2022simultaneous}, which stands out as the first-ever theoretical treatment describing wobbling and chirality on an equal footing.

Quantized wobbling modes were first discovered experimentally in $^{163}$Lu, where the presence of the odd-proton $i_{13/2}$ grants the apparition of multiple rotational bands to appear around the yrast line. This experiment consisted of populating high-spin states with a $^{29}$Si beam interacting with a thin target of $^{139}$La. Later on, other odd-$A$ nuclei were discovered as exhibiting wobbling excitations, and all the experimental findings will be mentioned in the following chapters. The nuclei having a triaxial shape can rotate about any of the principal axes, causing rich collective spectra to emerge. The family of rotational bands is described in terms of vibrational excitations. As a classical analog, nuclear wobbling motion corresponds to the spinning motion of an asymmetric top. The experimental fingerprints for wobbling motion indicate that the energy spectrum behaves as $\sim I(I+1)$ concerning the angular momentum, there is a clear dominance of electric transitions over the magnetic ones, and the nuclei have large quadrupole moments. All these quantities will be exploited in detail in the coming chapters.

\section{Aim}

The objective of this research is two-fold. On one side, the theoretical description of wobbling motion is treated in detail, starting from the required nuclear models specific to deformed nuclei, and reaching a set of key properties of the phenomenon. Differences between wobbling that occurs in odd- but also even-mass nuclei are depicted since each situation manifests in remarkable ways. Once the general formalism behind this effect is presented, an inventory of all the currently identified nuclei will be made, providing clear explanations for the band structure and also the relevant parameters describing deformation. The complete overview of all existing wobbling nuclei is encapsulated in a unified informative chart, which is one of the unique features of the research, and a first within the literature.

The second objective of this present work is to describe the wobbling mechanism by means of a novel semi-classical approach. Indeed, the important quantities related to collective excitations are properly reproduced by the, i.e., excitation energies, quadrupole moments, transition probabilities, and many more. This model starts from an initial quantal Hamiltonian that is dequantized through a variational method. A set of classical equations of motion that describe triaxial nuclei are obtained, and a classical energy function is granted by the approach. This function is a remarking feature of the developed framework, as this fully analytical expression (containing only classical variables that were obtained via the dequantization) will provide an insight into multiple analyses: energy spectrum, stability of the wobbling motion, critical regions, phase transitions and even possible changes in the wobbling regime. 

Another remarking feature of the current model is the geometrical interpretation of the rotational motion specific to triaxial nuclei, which is described in both a two- and three-dimensional space. Also unique to this research is the introduction of two concepts that are related to the band structures of odd-mass nuclei. These are the Signature Partner Bands and Parity Partner Bands and they can be considered hallmarks of the theory. Finally, it should be noted that this work consistently employs graphical representations such as workflow diagrams, schematics, and charts to facilitate a clearer understanding of the underlying mechanisms where possible.

\section{Motivation}

Over the years, many theoretical interpretations were suggested for the description of the wobbling motion and its main features. The Triaxial Rotor Model \cite{bohr1998nuclear,davydov1958rotational} and the Particle Rotor Model \cite{hamamoto2002wobbling} are quantal models that are solved exactly in the laboratory frame. Other investigations are based on mean-field theories, such as the Random Phase Approximation \cite{shimizu1995nuclear}, the Angular Momentum Projection \cite{oi2000wobbling}, and also the Collective Hamiltonian \cite{chen2014collective} were adopted.

This research aimed at a semi-classical description of the wobbling phenomenon due to its advantage in keeping close contact with the `classical picture' of the system dynamics. Certainly, working with a set of classical equations of motion is much easier than having to deal with quantum mechanical objects that do not have a clear one-to-one correspondence with classical mechanics. It will be shown that the rotational motion of a triaxial nucleus can be approximated quite well with that of a rigid rotator, meaning that the energy spectrum could be accurately described through quantities that have concise physical meaning (i.e., moments of inertia, angular momentum, angular frequency). Moreover, the analytical spectrum that is achieved by solving the equations for an odd-mass nucleus is indeed remarkable, since it will be described by separated degrees of freedom associated with an even-even core and a valence nucleon that interacts with the core. The variational method proves to be an efficient tool in accurately depicting the energy spectra and transition probabilities of several odd-$A$ nuclei within the $A=160$ mass region.

The lack of studies of geometrical treatments for the wobbling motion encouraged the team to pursue such an analysis. A two-dimensional evaluation shows if regions of stability exist, meaning that one can identify the energies at which the total angular momentum exhibits stable precessional motion. Taking the formalism a step further, the wobbling motion is explored within the space generated by the three components of the total angular momentum. The two constants of motion, i.e., the total energy and the total spin are graphically represented in the same picture, and their intersection signifies the allowed trajectories that the angular momentum precesses around. Each trajectory corresponds to a particular set of spin and energies, meaning that the entire spectrum of a wobbling nucleus can be interpreted. Using the classical view of the angular momentum and the total energy for a triaxial ellipsoid represents the onset of a fully unified description of nuclear deformation. Moreover, this phenomenological and semi-classical model gives results that are on par with fully microscopic or quantal descriptions (much more complex), making it a successful tool in describing collective phenomena.

\section{Thesis Structure}

This thesis is structured in the following way. \textbf{Chapter} \ref{chapter-2-theoretical-aspects} begins with the study of the nuclear surface using the expansion of the radius $R$ in terms of collective coordinates and spherical harmonics. The multipole modes are presented and a focus on the quadrupole $\lambda=2$ excitation mode is considered. This is the main vibrational mode causing triaxial deformation of the nucleonic matter. Within the approximation of the nuclear surface, the deformation parameters $\beta_2$ and $\gamma$ are introduced, which give a direct insight into the degree of elongation and asymmetry between the ellipsoid axes. Furthermore, the theoretical models necessary to describe the deformed nuclei are sketched in Section \ref{section-nuclear-models} within the same chapter. Starting with the Spherical Shell-Model, a step-by-step procedure is considered and more complete models such as the Deformed Shell, Nilsson, or even the Collective model are constructed. The last section of the chapter (Section \ref{section-triaxial-signatures}) presents two fingerprints of triaxiality, i.e., chiral motion and wobbling motion. Concerning the chiral bands, some experimental data are presented, and nucleonic configurations that lead to the apparition of chiral motion are geometrically represented. 

Wobbling motion, which is the `core' topic of this research is treated in \textbf{Chapter} \ref{chapter-3}, where its physical meaning is given. The difference between wobbling in odd- and even-mass nuclei is provided, and the two wobbling regimes that emerge in the odd-$A$ nuclei are portrayed. A quantal approximation made by Frauendorf in Ref. \cite{frauendorf2014transverse} shows that based on the alignment of the valence nucleon and the even-even core, two possible wobbling modes occur: transverse and longitudinal. The modes are properly outlined giving geometrical interpretations and showing key differences between the involved quantities. Using an approximation for an initial Hamiltonian of a rigid triaxial rotor, a proper analytical expression is obtained for the even-$A$ nuclei. The energy spectrum is defined as the sum between a quadratic term in angular momentum and a harmonic-like term. With this formula, the experimental wobbling bands in $^{130}$Ba are successfully reproduced, together with the transition probabilities of the collective states. Also in Chapter \ref{chapter-3}, all known experimental findings concerning wobbling excitations are specified, finishing with a chart that contains relevant parameters for each nucleus in particular. 

In \textbf{Chapter} \ref{chapter-4-aw1-formalism}, the theoretical formalism for describing odd-mass nuclei is employed. This semi-classical model starts from a Time-Dependent Variational Equation applied on a Particle-Rotor Hamiltonian. From this, a classical energy function is attained, and it is used two describe the excited spectra of triaxial nuclei. The energy spectrum is parametrized in terms of the moments of inertia, the triaxiality $\gamma$, and the interaction strength between the core and the odd-particle. These five free parameters are determined by fitting the excitation energies of $^{161,163,165,167}$Lu to the experimental data. Other quantities such as transition probabilities, alignments, and dynamical moments of inertia are calculated for each isotope with the obtained parameter set. A key feature of the approach depicted in Chapter \ref{chapter-4-aw1-formalism} is the renormalization of the wobbling band structure in terms of Signature Partner Bands. The method is contrasted with a prior description that utilized an alternative mechanism for generating wobbling excitations.

A novel method for studying wobbling motion in odd-mass nuclei is discussed in \textbf{Chapter} \ref{chapter-5-novel}, starting from the considerations made in the previous chapter. Herein, through the concept of Parity Partner Bands, it is shown that the wave-function describing the triaxial nuclei admits states of both positive and negative parity, which makes it possible to use an even simpler analytical spectrum of $^{163}$Lu. Another fitting method is used for the entire spectrum, including a rotational band that previously was treated separately. Consequently, a set of results that are very close to the real values is obtained. For a classical description, these results are quite impressive, as they are the first within the literature that achieve deviations under $100\ \text{keV}$ across the entire spectrum of $^{163}$Lu. Identification of phase transitions, where the nucleus could change its rotational mode, but also regions where rotational motion is unstable, are obtained through graphical representations within a space generated by dequantized variables that describe the rotational degrees of freedom for the particle + rotor system. 

\textbf{Chapter} \ref{extra-chapter-new-boson} presents a specialized description of wobbling motion in odd-mass nuclei, based on a recent case study conducted by the team on $^{135}$Pr. Utilizing a boson representation for the angular momentum components and the Hamiltonian yielded accurate results for both the energy spectrum and transition probabilities. The numerical implementation of this approach involved fitting experimental data and utilizing a set of well-established trigonometric functions - specifically, the Jacobi Elliptic Functions. These functions will offer a valuable understanding of the potential energy, wobbling frequency, and stability of odd-mass triaxial nuclei.

In \textbf{Chapter} \ref{chapter-8-conclusions}, general concluding remarks are provided that summarize the key findings and contributions of this work to the topic of Nuclear Structure. These conclusions highlight the significance of the research and its implications for future studies in this field. It should be noted that several chapters contain dedicated conclusions providing further insights into their specific topics.%It should be noted that several chapters also contain their dedicated set of conclusions, offering more insights into the specific topics covered in those chapters.