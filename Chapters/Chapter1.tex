\chapter{Introduction}

Ground-state nuclear shapes with spherical or axial symmetry are predominant across the char of nuclides. Near closed shells, the deformation is indeed sufficient that models based on spherical symmetries can be used to describe nuclear properties (e.g., energies, quadrupole moments, and so on). Besides the spherical and axially-symmetric shapes, the existence of triaxial nuclear deformation was theoretically predicted a long time ago \cite{bohr1998nuclear}. The rigid triaxiality of nuclei is defined by the asymmetry parameter $\gamma$, giving rise to unique phenomena. Lately, the quantum mechanical properties of triaxial nuclei drew a lot of attention within the nuclear physics community, since the description of nuclear properties represents a real challenge from both an experimental and a theoretical standpoint. Indeed, a great progress for the experimental setups being able to perform measurements of high-spin has only been possible after the 2000s. It is worth mentioning that some experiments concerning alpha-$\alpha$ particle reactions induced in heavy nuclei in the early 1960s (e.g., \cite{morinaga1963gamma}) helped to produce decent amount of data related to the rotational in the high-spin region ($\geq 20 \hbar$).
% Within the experimental studies made by Morinaga et al., the alpha reactions which were induced in the nuclei were generated by the formation of a so-called \emph{compound nucleus}. This system may exist at a large spin value due to the absorption of the angular momentum from the incident particle (i.e., spin values up to $\approx25\ \hbar$ can be obtained from a 50 MeV alpha particle energy - relative to the target nucleus \cite{morinaga1963gamma}).

The physics of \emph{high-spin} states have been studied from the early 1950s, with the major breakthrough on the theoretical side made by Bohr and Mottelson \cite{bohr1998nuclear}. The elusive properties of nuclear rotation were described in terms of the rotational degrees of freedom associated with other nuclear degrees of freedom (e.g., particle-vibration, quadrupole-quadrupole, parity, and so on). The spherical shell-model only describes nuclei near the closed shells. On the other side, for the nuclei that lie far from closed shells, a deformed potential must be employed. In the case of even-even nuclei, unique band structures resulting from the vibrations and rotations of the nuclear surface appear in the energy range 0-2 MeV. 
%This as proposed by Bohr and Mottelson \cite{bohr1998nuclear} in the Geometric Collective Model - GCM. Within the GCM, the nucleus is described as a classical charged liquid drop. For the low-lying energy spectrum, usually, the compression of nuclear matter and the nuclear skin thickness are neglected. This results in the final picture of a liquid drop with a constant nuclear density and a sharp surface \cite{greiner1996nuclear}.

With the difficulties of studying triaxial nuclei, some fingerprints of this elusive phenomenon should be inferred. Fortunately, Wobbling Motion (WM) represents such a feature. Namely, WM is an effect that is uniquely attributed to non-axial deformation in nuclei. 

\section{Aim}



\section{Motivation}



