\chapter{Introduction}

Ground-state nuclear shapes with spherical or axial symmetry are predominant across the char of nuclides. Near closed shells, the deformation is indeed sufficient that models based on spherical symmetries can be used to describe nuclear properties (e.g., energies, quadrupole moments, and so on). Besides the spherical and axially-symmetric shapes, the existence of triaxial nuclear deformation was theoretically predicted a long time ago \cite{bohr1998nuclear}. The rigid triaxiality of nuclei is defined by the asymmetry parameter $\gamma$, giving rise to a unique behavior concerning the system dynamics. Lately, triaxial nuclei drew a lot of attention within the nuclear physics community, since the description of nuclear properties represents a real challenge from an experimental and also a theoretical standpoint. Indeed, a great progress for the experimental setups being able to perform measurements of high-spin has only been possible after the 2000s. It is worth mentioning that some experiments concerning alpha-$\alpha$ particle reactions induced in heavy nuclei in the early 1960s (e.g., \cite{morinaga1963gamma}) helped to produce decent amount of data related to the rotational in the high-spin region ($\geq 20 \hbar$).

The physics of \emph{high-spin} states has been studied from the early 1950s, with the major breakthrough on the theoretical side made by Bohr and Mottelson \cite{bohr1998nuclear}. The nuclear rotation was described in terms of the rotational degrees of freedom associated with other nuclear degrees of freedom such as particle-vibration, quadrupole-quadrupole, parity, and so on. The spherical shell-model only describes nuclei near the closed shells. On the other side, for the nuclei that lie far from closed shells, a deformed potential must be employed. In the case of even-even nuclei, unique band structures resulting from the vibrations and rotations of the nuclear surface appear in the energy range 0-2 MeV. 

Even though triaxiality has an elusive character, two phenomena, i.e., wobbling motion and chiral bands are uniquely attributed to triaxial shapes. Consequently, these two were intensively searched by using the advanced techniques. In this work, the focus is given exclusively on the first effect, although it is worth specifying that current team provides a unified description of both phenomena in their most recent work \cite{raduta2022simultaneous}, which stands out as the first ever theoretical treatment describing wobbling and chirality on an equal footing.

Quantized wobbling modes were firstly discovered experimentally in $^{163}$Lu, where the presence of the odd-proton $i_{13/2}$ grants the apparition of multiple rotational bands to appear around the yrast line. This experiment consisted of populating high-spin states with a $^{29}$Si beam interacting with a thin target of $^{139}$La. Later on, other odd-$A$ nuclei were discovered as exhibiting wobbling excitations, and all the experimental findings will be mentioned in the following chapters. The nuclei having a triaxial shape can rotate about any of the principal axes, causing rich collective spectra to emerge. The family of rotational bands is described in terms of vibrational excitations. As a classical analog, nuclear wobbling motion is corresponds to the spinning motion of an asymmetric top. The experimental fingerprints for wobbling motion indicate that the energy spectrum behaves as $\sim I(I+1)$ with respect to the angular momentum, there is a clear dominance of electric transitions over the magnetic ones, and the nuclei have large quadrupole moments. All these quantities will be exploited in detail in the coming chapters.

\section{Aim}
 
Over the years, many theoretical interpretations were suggested for the description of the wobbling motion and its main features. 

\section{Motivation}


