\chapter{Nuclear Models}
\label{chapter-3}

In the following chapter, a discussion will be made about the nuclear models that are used in order to describe phenomena specific to rotating nuclei. Since the focus of this work emerges from a \emph{class} of properties that usually apply to the high-spin region, it makes sense to give an insight in the tools that fit the best the underlying effects.

\section{Shell model}

The idea that an atomic nucleus can have a structure that behaves rather similarly as the atom itself has been enforced by the experimental observations that were done across time. The sharp and discrete discontinuities of nuclear properties, such as the nucleon separation energy, point out that nucleus can be explained through the existence of \emph{shells}. Some examples of observations indicating this are:
\begin{itemize}
    \item When adding a nucleon to a nucleus, there are certain places where the \emph{binding energy} of the next nucleon becomes considerably smaller than the previous one. 
    \item Separation energies for both the protons and neutrons suffer drastic changes, having strong deviations from the predictions of the semi-empirical mass formula \cite{weizsacker1935theorie}, the discontinuities being represented by major shell closures (complete filling) \cite{krane1991introductory}.
    \item The neutron absorption cross-section has a substantial decrease in value at the neutron magic numbers
    \item Great abundance of nuclides where $Z$ and $N$ are magic numbers.
\end{itemize}

The sudden discontinuities occur at specific values of the proton $Z$ and neutron $N$ numbers: these are called \emph{magic numbers}. Currently, these magic numbers correspond to $Z$ or $N=2,8,20,28,50,82,126$, and they represent the major shells. There are also two \emph{weakly magic numbers}: 40 and 64. One can examine the values for the first excited states $2^+$ that are shown in Figs. \ref{e2plus_proton}, \ref{e2plus_neutron}. Indeed, these values show some peaks, each peak corresponding to a particular magic number. This results are part of the work of Raman et al. \cite{raman2001transition}, where the transition probabilities from the ground state to the first-excited $2^+$ state in even-even nuclei were evaluated.
\begin{figure}
    \centering
    \includegraphics[width=0.99\textwidth]{Chapters/Figures/E2plus_proton.pdf}
    \caption{The first excited energy states $2^+$ of nuclei with even $Z$ and $N$ are graphically represented with respect to the proton number. Each connecting line represents a set of isotopes. Figure taken from Ref. \cite{matta2017exotic}.}
    \label{e2plus_proton}
\end{figure}
\begin{figure}
    \centering
    \includegraphics[width=0.99\textwidth]{Chapters/Figures/E2plus_neutron.pdf}
    \caption{The first excited energy states $2^+$ of nuclei with even $Z$ and $N$ are graphically represented with respect to the neutron number. Each connecting line represents a set of isotopes. Figure taken from Ref. \cite{matta2017exotic}.}
    \label{e2plus_neutron}
\end{figure}

The shell model starts from the basic assumption that the nucleus is a \emph{mean-field potential}, where the motion of a single nucleon is caused by all the other nucleons. In other words, the nucleon is moving inside an average potential generated by all the other constituents of the nucleus. Of course that all the nucleons under the influence of this mean field occupy the energy levels which correspond to a series of sub-shells verifying the \textit{Pauli exclusion principle}. Having a general expression for the potential that properly reproduces all the magic numbers and the observed nuclear properties is therefore crucial. Since the model starts from the concept of independent (non-interacting) particle motion within an average potential, finding each energy will be equivalent of solving the Schrödinger equation:
\begin{align}
    -\frac{\hbar^2}{2m}\nabla ^2\psi_i(r)+V(r)\psi_i(r)=e_i\psi_i(r)\, 
    \label{schrodinger-single-particle-eq}
\end{align}
where $e_i$ represents the energy (eigenvalue), $\psi_i$ represents the wave-function (eigenstates), and $V(r)$ is the nuclear potential whose expression must be evaluated. The choice of $V(r)$ will be dictated by the reproduction of various experimental data (such as nuclear saturation, scattering, nuclear reactions, and so on). For the motion of an independent particle, an obvious first attempt would be the \emph{simple harmonic oscillator} (SHO), which has the known expression:
\begin{align}
    V(r)=\frac{1}{2}m(\omega_i r)^2\ ,
    \label{harmonic-potential}
\end{align}
with $\omega_i$ as the frequency of the basic harmonic-like motion of the particle in the nucleus. With Eq. \ref{harmonic-potential}, the motion of the nucleon has a straightforward expression:
\begin{align}
    \frac{\hbar^2}{2m}\nabla^2\psi_i(r)+\frac{1}{2}m(\omega r)^2\psi_i(r)=e_i\psi_i(r)\ .
\end{align}

This Schrödinger equation has its energy eigenvalues under to form:
\begin{align}
    e_N=\left(N+\frac{3}{2}\right)\hbar\omega\ ,
\end{align}
where $N$ is the number of oscillator quanta which describes each major shell (also called the \emph{principal quantum number}). One should keep in mind that such an expression is typical for a three-dimensional and isotropic harmonic oscillator. The principal quantum number $N$ is furthermore defined as:
\begin{align}
    N=2(n-1)+l\ ,
\end{align}
with $n$ and $l$ being the \emph{radial} quantum number and \emph{orbital angular momentum} quantum number, respectively, taking values $n=1,2,3,\dots$ and $l=0,1,2,\dots,n-1$. In this first approximation, all the levels with the same principal quantum number $N$ are \emph{degenerate}, with a maximal degeneracy given by $2(2l+1)$. However, by using only the SHO term as the expression of $V(r)$, only the first three magic numbers are reproduced, meaning that some additional term(s) might be needed in order to consistently obtain the series of magic numbers.

Furthermore, the steepness of the SHO can be corrected with an \emph{attractive} term proportional to $l$-squared. This acts as a centrifugal term which provides an angular momentum barrier, lifting the degeneracy between the levels with the same principal quantum number $N$ and different values for the orbital angular momentum $l$. This SHO+$l^2$ adjustment is still not enough, such that a so-called \emph{spin-orbit} coupling term of the form $\vec{l}\cdot\vec{s}$ must be also added. This term comes from the consideration that the nucleon-nucleon interaction has a spin dependence, and the potential depends on the intrinsic spin $s$ ($\vec{s}$) and the orbital angular momentum $l$ ($\vec{l}$) of a nucleon. Since $\vec{j}=\vec{l}+\vec{s}$, two possible states emerge from a single value of $l$ (depending on wether $\vec{s}$ is parallel or anti-parallel to $\vec{l}$). The final will consist of the \emph{Modified Harmonic Oscillator} (MHO).
\begin{align}
    V(r)=\frac{1}{2}(\omega r)^2+B\ \vec{l}^2+A\ \vec{l}\cdot\vec{s}\ .
    \label{modified-harmonic-oscillator-eq}
\end{align}

For the sake of simplicity, the centrifugal term will be denoted within formulas without the vector symbol. Since the intrinsic spin of a nucleon is $s=1/2$, for a given value of $l$, there can be two values for the \emph{total angular momentum} (a.m.) $j=l\pm1/2$: one for each spin orientation with respect to the direction of the orbital a.m. Moreover, for every value of $l=0,1,2,3,4,\dots$, there is a similar notation $l=s,p,d,f,g,\dots$, respectively. Regarding the spectroscopic notation, usually, the value of $j$ is considered as a subscript; $nl_j$ (for example $1p_{1/2}$ and $1p_{3/2}$). For high enough shells, there can be splittings between $j+1/2$ and $j-1/2$ that are large enough to lower the $j+1/2$ state from one oscillator shell $n$ to one located below $n-1$. These types of levels are called \emph{intruder states}, and they have opposite parity $\pi=(-1)^l$ with respect to the shell that these levels will occupy.

Going back to the expression of the $\vec{l}\cdot\vec{s}$ term from Eq. \ref{modified-harmonic-oscillator-eq} and denoting it with $V_{ls}(r)$, its contribution to the total potential can be regarded as a surface effect and it can be expressed as a function that depends on the radial coordinate as such \cite{casten2000nuclear}:
\begin{align}
    V_{ls}(r)=-a_{ls}\frac{\partial V(r)}{\partial r}\vec{l}\cdot\vec{s}\ ,
\end{align}
where $V(r)$ is the expression for a central potential and $a_{ls}$ is a strength constant.

The nuclear potential is now able to reproduce all the magic numbers. It is also possible to formulate the total energy of a single-particle within the average potential. Thus, the Hamiltonian of this simple system (i.e., the MHO) can be formulated as such:
\begin{align}
    H&=-\frac{\hbar^2}{2m}\nabla^2+V_\text{SHO}+(l^2)_\text{term}+(\vec{l}\cdot\vec{s})_\text{term}=-\frac{\hbar^2}{2m}\nabla^2+V_\text{MHO} , \nonumber\\
    H&=-\frac{\hbar^2}{2m}\nabla^2+\frac{1}{2}m(\omega r)^2+Bl^2+A\vec{l}\cdot\vec{s}\ .
\end{align}

The evolution from a SHO, then SHO+$l^2$, and finally SHO+$l^2+\vec{l}\cdot\vec{s}$ or modified oscillator potential is illustrated in Fig. \ref{energy-levels-mho}, where it can be seen how each extra term removes a degeneracy, with the complete reproduction of the magic numbers in the third column. The \emph{intruder} levels can also be observed, where states with $j=l+1/2$ from a particular $n$ are so low, that they lie below an $n-1$ adjacent level.
\begin{figure}
    \centering
    \includegraphics[width=0.99\textwidth]{Chapters/Figures/SM_level_scheme.png}
    \caption{The energy levels obtained via calculation of the shell model potential using the simple oscillator (SHO), the SHO amended with a centrifugal term $l^2$, and finally the modified oscillator (MHO) that contains a spin-orbit term. The `correct' magic numbers are the ones in the right-most column. Figure is adapted from Refs. \cite{krane1991introductory,matta2017exotic}.}
    \label{energy-levels-mho}
\end{figure}

Another, more realistic potential that can be used in order to reproduce the specific shell model calculation is the so-called Woods-Saxon potential. Because of the short-range character of the strong nuclear force, it is safe to assume that this potential should behave in the same manner as the density distribution of the nucleons. Since for medium and heavy nuclei, the Fermi-like functions (distributions) are the ones that best fit the experimentally measured data, this potential should have the following form \cite{woods1954diffuse}:
\begin{align}
    V_\text{ws}(r)=-\frac{V_0}{1+e^{\frac{r-R_0}{a}}}\ ,
    \label{woods-saxon-potential}
\end{align}
where $V_0$ represents the depth of the potential ($\approx 50$ MeV, in order to reproduce the experimental separation energies for the nucleons), the surface thickness $a$ (also called the diffuseness parameter, giving information about how fast the potential drops to zero) with a value of approximately $0.5$ fm, $R_0$ is the nuclear radius ($R_0=1.2A^{1/3}$). The nature of this potential is of \emph{central type} and Eq. \ref{woods-saxon-potential} in its pure form is not enough the reproduce the higher magic numbers. As such, the addition of a spin-orbit term, similarly as in the case of MHO potential, is required \cite{martin2017particle}: 
\begin{align}
    V_\text{total}=V_\text{ws}^\text{central}+V_{ls}(r)\vec{l}\cdot\vec{s}\ .
    \label{woods-saxon-so-potential}
\end{align}
The only good quantum numbers in the case of the WS potential are the total a.m. $j$ and the parity $\pi=(-1)^l$.
The expectation value of the spin-orbit term $\vec{l}\cdot\vec{s}$ can be given as:
\begin{align}
    \langle ls \rangle=\hbar^2\begin{cases}
        \frac{l}{2} \quad &\text{for} j=l+\frac{1}{2}\\
        -\frac{l+1}{2} &\text{for} j=l-\frac{1}{2}\ \\
    \end{cases}\ .
\end{align}
and the spacing between two levels can be furthermore expressed as \cite{martin2017particle}:
\begin{align}
    \Delta E_{ls}=\frac{2l+1}{2}\hbar^2\langle V_{ls}\rangle\ .
\end{align}
The experimental evidence points to the fact that $V_{ls}(r)$ is negative, meaning that states with $j=l-1/2$ are shifted higher than $j=l+1/2$. Some characteristics of the WS potential are the following:
\begin{enumerate}
    \item It increases with the increase of $R$, meaning that it has an \emph{attractive nature}
    \item It flattens out for large enough $A$ in the center of the nucleus
    \item It rapidly goes to zero as $R$ increases (given by the diffuseness parameter), indicating its short-range nature
    \item When $R=R_0$ (that is for the nucleons near the surface), a large force towards the center of the nucleus is experienced by the these nucleons.
\end{enumerate}
\begin{figure}
    \centering
    \includegraphics[scale=0.2]{Chapters/Figures/ws_potential_plot.png}
    \caption{The shape of the Woods-Saxon potential, defined by Eq. \ref{woods-saxon-potential}. The parameters are arbitrarily chosen as: $V_0=50$ MeV, $R=5.57$ fm, and $a=0.5$ fm.}
    \label{woods-saxon-plot}
\end{figure}

The Hamiltonian that describes the motion of the nucleon within the mean-field potential is given by:
\begin{align}
    %H&=-\frac{\hbar^2}{2m}\nabla^2+V_\text{ws}^\text{central}+(\vec{l}\cdot\vec{s})_\text{term}\ ,\\
    H&=-\frac{\hbar^2}{2m}\nabla^2-\frac{V_0}{1+e^{\frac{r-R_0}{a}}}+A\vec{l}\cdot\vec{s}\ ,
\end{align}
and the shape of a typical Woods-Saxon potential is shown in Fig. \ref{woods-saxon-plot}. A comparison between the Woods-Saxon potential, a SHO, and the square-well-like potential is made in Fig \ref{shell-model-functional-potentials}. The difference between the pure form of the Woods-Saxon potential and the total potential amended with the spin-orbit contribution can be seen in Fig. \ref{woods-saxon-energy-levels}.
\begin{figure}
    \centering
    \includegraphics[scale=0.2]{Chapters/Figures/functional-potentials-shell-model.png}
    \caption{A schematic representation with the three kind of potentials used to describe the shell model: harmonic oscillator, Woods-Saxon, and for completeness, the square-well.}
    \label{shell-model-functional-potentials}
\end{figure}
\begin{figure}
    \centering
    \includegraphics[width=0.99\textwidth]{Chapters/Figures/energy_levels_WS.png}
    \caption{The energy levels calculated for the Woods-Saxon potential given by Eq. \ref{woods-saxon-potential} (left-side), and the single-particle energies with the spin-orbit correction added, as in Eq. \ref{woods-saxon-so-potential} (right-side). Figure adapted from Ref. \cite{lewis2019lifetime}.}
    \label{woods-saxon-energy-levels}
\end{figure}

So far, the general discussion concerning nuclear models was made for the case where each nucleon is treated as an \emph{independent} particle moving in an average (mean-field) potential. However, such an assumption is not accurate enough (especially for the nuclei that lie far away from the closed shells), and this problem should be treated within a \emph{many-body} approach: considering the mutual interaction between the nucleons. These interactions are also called \emph{residual interactions} \cite{casten2000nuclear,bertulani2007nuclear}. With these residual interactions, an accurate depiction of the nucleus might be achieved. The \emph{Deformed Shell Model} will be employed in the following sections, reaching to the famous Nilsson model of describing the nucleus.

\section{Deformed Shell Model}

In the previous section, an approximate description of the independent motion of a nucleon within an average potential was given. The potential is generated by the interaction of that nucleon with all the remaining nucleons from the compounding nucleus. The formalism works really well for spherical nuclei and it is a successful tool in reproducing and predicting the properties of nuclear states. The theory also proves to be efficient for excited states having nucleonic configurations that are dominated by a single nucleon or a very small number of `extra' nucleons.

For nuclei that are even in both the proton number and the neutron number, the nuclear ground-state has a spin and parity that are properly reproduced by the \emph{spherical shell model}: $I^\pi=0^+$. In a nucleus with complete shells, the \emph{net spin} must be zero. On the other hand, a nucleus with one nucleon missing from a complete shell closure, the ground-state spin should be equal to the a.m. value of the orbital occupied by the hole. Moreover, the parity of the ground-state for a given nucleus is determined by the orbital a.m. value $l$:
\begin{align}
    \pi=(-)^l\to
    \begin{cases}
        +1 &\text{for even-}l\ \text{levels}\\
        -1 &\text{for odd-}l\ \text{levels}\ .
    \end{cases}
\end{align}

For odd-odd nuclei, one can find the ground-state (g.s.) spin and parity via the last two valence nucleons \cite{krane1991introductory,bertulani2007nuclear}. The coupling rules that are allowed in the odd-odd nuclei were determined more than 50 years ago by Gallagher et al. \cite{gallagher1958coupling}:
\begin{align}
    I&=j_p+j_n\ \text{if}\ j_p=l_p\pm\frac{1}{2}\ \text{and}\ j_n=l_n\pm\frac{1}{2}\ ,\\
    I&=|j_p-j_n|\ \text{if}\ j_p=l_p\pm\frac{1}{2}\ \text{and}\ j_n=l_n\mp\frac{1}{2}\ .
\end{align}

\subsection{Deformed Shell Model - Nilsson Model}
\label{nilsson-model-section}

The idea that some nuclei are deformed in their ground-state was enforced experimentally a long time ago when measuring quantities such as density distributions, nuclear quadrupole moments \cite{casten2000nuclear}. The non-spherical shapes are given by the existence of nucleonic configurations that lie away from the major shell closure. In Chapter \ref{chapter-2} the description of the nuclear shapes was treated, using the well-known formula for the parametrization of the nuclear radius in terms of the collective coordinates (see Eq. \ref{nuclear-shape}), resulting in the \emph{spherical}, \emph{axially-symmetric}, and \emph{axially-asymmetric} nuclear shapes.

Developed by Nilsson in 1955 \cite{nilsson1955binding} for treating the \emph{deformed nuclei}, it is a modified shell model which allows for deformations to be taken into account by the use of the \emph{anisotropic harmonic oscillator} (AHO). Similarly as for the basic shell model, the goal is to obtain an expression for the single-particle energies of a nucleon. The basic Hamiltonian corresponding to this kind of system is shown below \cite{bertulani2007nuclear}:
\begin{align}
    H=H_0+a_1\vec{l}\cdot\vec{s}+a_2l^2\ ,
    \label{nilsson-simple-hamiltonian}
\end{align}
where $H_0$ is the AHO term. The general expression for this kind of oscillator is:
\begin{align}
    H_\text{AHO}\equiv H_0=-\frac{\hbar^2}{2m}\nabla^2+\frac{1}{2}m(\omega_x^2x^2+\omega_y^2y^2+\omega_z^2z^2)\ .
\end{align}

In the expression of the single-particle Hamiltonian, the constants $a_1$ and $a_2$ are usually determined via adjustments to the experimental results. It can be seen that both the centrifugal-like term $l^2$, which simulates a flattening of the oscillator potential, and the $\vec{l}\cdot\vec{s}$ term are also present here, as it was the case for the spherical shell model. However, the explicit form of Eq. \ref{nilsson-simple-hamiltonian} is as follows:
\begin{align}
    H_\text{Nil}=-\frac{\hbar^2}{2m}\nabla^2+&\frac{1}{2}m(\omega_x^2x^2+\omega_y^2y^2+\omega_z^2z^2)-2\kappa\hbar\omega_0(\vec{l}\cdot\vec{s})\nonumber\\&-2\kappa\hbar\omega_0\mu\left(l^2-\langle l^2\rangle_N\right)\ .
    \label{eq-full-nilsson-ham}
\end{align}

The parameters $\kappa$ and $\mu$ act as strength parameters for the spin-orbit coupling term and the centrifugal term, respectively. The last term is a correction, which was introduced by Gustafson et al. \cite{gustafson1967nuclear}.
%The last term is a correction, which was originally considered as $\mu l^2$, but it was pointed by Gustafson et al. \cite{gustafson1967nuclear} that the shift in energy is way too large for big values of $N$ (principal quantum number). As a result, taking the current expression for the correction term helps to compensate.
The three \emph{oscillator frequencies} are chosen to be inversely proportional to the semi-axis lengths of the deformed ellipsoid (denoted by $a_x$, $a_y$, and $a_z$) such that:
\begin{align}
    \omega_r=\omega_0\frac{R_0}{a_r}\ ,\ r=x,y,z\ .
\end{align}

For the spherical case, the oscillator frequency $\hbar\omega_0$ is set to $41A^{-1/3}$ MeV (calculation for this value arise from the shell model with SHO \cite{bertulani2007nuclear}). For the axially-symmetric case, one can choose the $z$-axis as symmetry axis, implying that the oscillator frequencies along the $x$ and $y$ axes are equivalent (that is $\omega_x=\omega_y\equiv\omega_\perp$). Following the calculations done in \cite{bertulani2007nuclear}, one can express the two relevant oscillator frequencies in terms of a deformation parameter $\epsilon_2$ (whose dependence on the quadrupole deformation parameter $\beta_2$ has been shown in Eq. \ref{epsilon-beta-relation}) as such:
\begin{align}
    \omega_\perp^2=\omega_0^2\left(1+\frac{2}{3}\epsilon_2\right)\ ,\\
    \omega_z^2=\omega_0^2\left(1-\frac{4}{3}\epsilon_2\right)\ .
    \label{oscillator-frequencies-nilsson}
\end{align}

Moreover, a dependence on the deformation parameter itself is employed for the frequency $\omega_0$ that appears in the expressions for $\omega_\perp$ and $\omega_z$, respectively:
\begin{align}
    \omega_0=\left(1-\frac{4}{3}\epsilon_2^2-\frac{16}{27}\epsilon_2^3\right)^{-1/6}\ ,
    \label{omega-0-oscillator-frequency}
\end{align}
where $\bar{\omega}_0$ can be considered a constant written as $\bar{\omega}_0=(\omega_x\omega_y\omega_z)^{1/3}=\text{const}$. This is coming from the harmonic oscillator at zero deformation and by considering the conservation of the nuclear volume. The energy eigenvalues $\epsilon_q$ for the nucleonic state $\psi_q$ belonging to a deformed nucleus can be determined by solving the Schrödinger equation associated to each nucleon in particular:
\begin{align}
    H_\text{Nil}\psi_q=\epsilon_q\psi_q\ ,
    \label{nilsson-schrodiner-equation}
\end{align}
where the index $q$ denotes a set with all the relevant quantum numbers. This set is also called the \emph{asymptotic quantum numbers}, which are used to specify a \emph{Nilsson orbital}. The well-known notation is as follows (still considering the $z$-axis as the symmetry axis):
\begin{align}
    \Omega^\pi\left[Nn_z\Lambda\right]\ .
    \label{nilsson-notation}
\end{align}
\begin{itemize}
    \item $\Lambda$ is the projection of the particle's orbital a.m. along the symmetry axis (component of $l$ along $z$)
    \item $N$ the principal quantum number of the major shell. It also determines the parity as $\pi=(-1)^N$, making the notation from Eq. \ref{nilsson-notation} somewhat redundant
    \item $n_z$ is the number of oscillator quanta along the symmetry axis. More precisely, it gives the number of nodes for the wave-function along the direction of the $z$-axis
    \item $\Omega$ is the projection of the particle's total a.m. along the symmetry axis (i.e., $\mathbf{j}$). Moreover, the projection of the intrinsic spin of a nucleon onto the symmetry axis can have the values $\Sigma=\pm\frac{1}{2}$, so that $\Omega=\Lambda+\Sigma=\pm\frac{1}{2}$.
\end{itemize}

Fig. \ref{fig-nilsson-quantum-numbers} shows the geometrical meaning of the asymptotic quantum numbers for the Nilsson model. Indeed, for a single nucleon orbiting a deformed core, the vector $\mathbf{R}$ represents the angular momentum of a \emph{rotating nucleus}, the vector $\mathbf{I}$ represents the total a.m. of the entire nucleus, $\mathbf{j}$ is the total a.m. of the single-particle (that is $\mathbf{j}=\mathbf{l}+\mathbf{s}$). However, two more quantum numbers should be mentioned: the projection of the total a.m. $\mathbf{I}$ onto the symmetry axis, denoted by $K$, and the projection of the same vector onto the laboratory axis, referred to as $M$.
\begin{figure}
    \centering
    \includegraphics[width=0.99\textwidth]{Chapters/Figures/nilsson_quantum_numbers.pdf}
    \caption{A schematic showing a geometrical interpretation of the Nilsson's asymptotic quantum numbers. This figure is inspired from Ref. \cite{garnsworthy2007neutron}}
    \label{fig-nilsson-quantum-numbers}
\end{figure}

Regarding the quantum numbers sketched in Fig. \ref{fig-nilsson-quantum-numbers}, there is an important aspect which needs to be specified about the projections $K$ and $\Omega$, respectively. Indeed, it is clear that compared to the spherical case, where different orientations are irrelevant to the energy spectrum of nucleons, in the deformed case different directions in space lead to different energies. The orientation is in fact specified by the \emph{magnetic sub-state} of the nucleon, i.e., the projection of the total angular momentum on the symmetry axis. This projections is denoted by $\Omega$ for the single-particle, however, because the rotational angular momentum $\mathbf{R}$ in the axially deformed is perpendicular to the symmetry axis for low-lying states, then it will have no contribution to $K$, meaning that one can use $\Omega$ and $K$ interchangeably.

\subsection{Single-particle states in deformed nuclei}

It is instructive to go into detail about the quantum numbers defined in Eq. \ref{nilsson-notation}, since the orbits characterizing the nucleons point out the nature of deformations that take place. The quantum numbers $N$, $n_z$ and $\Lambda$ are good quantum numbers only when the nuclear deformation is large, meaning that $\epsilon$ (or equivalently $\beta$) tends to infinity. In fact this is the main reason why they are called asymptotic quantum numbers. However, the numbers $\Omega$ and $\pi$ remain good quantum numbers even for low and moderate deformations for the nucleus. It should be noted that if $N$ is even, then $(\Lambda+n_z)$ is also even. Similarly, if $N$ is odd, then the sum of the other two quantum numbers must also be odd \cite{casten2000nuclear}.

Since the eigenvalues of the Hamiltonian $H_\text{Nil}$ ultimately depend on the deformation parameter $\epsilon$, each nucleon will have an orbit (energy) that is deformation dependent. At no deformation, all the energy levels for a single-particle state will have a $2j+1$ degeneracy. This translates to the fact that all $2j+1$ possible orientations of $\vec{j}$ are equivalent. On the other side, when the potential is deformed, this will no longer hold, i.e., the energy levels in the deformed potential will depend on the spatial orientation of the orbit itself.
% the energy depends on the component of $\vec{j}$ along the symmetry axis of the core. 

As an example, a nucleon from the $f_{7/2}$ shell will be considered. This nucleon can have eight possible components in the range $\Omega=[-\frac{7}{2},\frac{7}{2}]$. Because of the reflection symmetry, the positive components of $\Omega$ will have the same energy as the negative ones, leading to a degeneracy of the levels. Additionally, the single-particle $f_{7/2}$ state will split up into four new states when deformation emerges: $\Omega=\frac{1}{2},\frac{3}{2},\frac{5}{2},\frac{7}{2}$ (all of of negative parity). In Figs. \ref{nillson-orbits-prolate-projections} - \ref{nillson-orbits-oblate-projections} the different orbits of the odd particle are given for both prolate and oblate deformations.
\begin{figure}
    \centering
    \includegraphics[scale=1]{Chapters/Figures/nillson_SP_orbits.pdf}
    \caption{A simple sketch showing the single-particle orbits for the $j=7/2$ nucleonic state, along the symmetry axis for a \emph{prolate} deformation. The actual projections are $\Omega_1=\frac{1}{2}$, $\Omega_2=\frac{3}{2}$, $\Omega_3=\frac{5}{2}$, and $\Omega_4=\frac{7}{2}$. The figure was inspired from Ref. \cite{krane1991introductory}.}
    \label{nillson-orbits-prolate-projections}
\end{figure}
\begin{figure}
    \centering
    \includegraphics[scale=1]{Chapters/Figures/nillson_SP_orbits_2.pdf}
    \caption{A simple sketch showing the single-particle orbits for the $j=7/2$ nucleonic state, along the symmetry axis for an \emph{oblate} deformation. The actual projections are $\Omega_1=\frac{1}{2}$, $\Omega_2=\frac{3}{2}$, $\Omega_3=\frac{5}{2}$, and $\Omega_4=\frac{7}{2}$. The figure was inspired from Ref. \cite{krane1991introductory}.}
    \label{nillson-orbits-oblate-projections}
\end{figure}

From Figs. \ref{nillson-orbits-prolate-projections} - \ref{nillson-orbits-oblate-projections}, it can be seen that the first orbit (denoted by orbit $1$) lies closest to the core in the prolate case, while in the oblate case this is true for orbit $4$. This will influence the interaction strength, meaning that for the prolate case, the orbit $1$ will interact the strongest with the \emph{core}, while in the oblate case, it is the orbit $4$ which has the strongest interaction with the bulk core. Moreover, the strength of interaction indicates the magnitude of the energies for each projection: the stronger the interaction between the orbit and the core, the more tightly bound these states are and lie lower in energy. For prolate deformations, the orbits with smallest $\Omega$ `prefer' to lie lower in energy (interacting strongly with the core). For oblate deformations, the situation is opposite: orbits with the maximal $\Omega$ have the strongest core interactions and therefore lie lowest in energy.

Another way of looking at the coupling of the single-particle with the bulk core can be given in terms of overlaps of their corresponding wave-functions (eigenstates). Indeed, a nucleon lying in the lowest $\Omega$ orbit will have a \emph{maximum} wave-function overlap with a prolate core. On the other hand, nucleons lying in the highest $\Omega$ orbits will have maximum overlap with the oblate core. The overlap gives the overall binding energy between the two systems (i.e., core and particle) as explained in the previous paragraph. Discussion about the wave-function overlap and the nuclear density distribution \cite{frauendorf2014transverse, das2018nuclear} will be made in the following chapters. The induced degeneracy due to deformation for a particle state $l_j$ is shown in Fig. \ref{nillson-orbits-splittings}, for the same example of the nucleon with the orbit $f_{7/2}$.
\begin{figure}
    \centering
    \includegraphics[width=0.99\textwidth]{Chapters/Figures/nillson_SP_splittings.pdf}
    \caption{The effect of deformation for the particle state $f_{7/2}$. It can be seen that indeed, as it was mentioned within the text, $\Omega_1$ component lies lowest in energy for the oblate deformation, and $\Omega_4$ component lies the lowest in energy for an oblate deformation.}
    \label{nillson-orbits-splittings}
\end{figure}

Obviously, the sketch shown in Fig. \ref{nillson-orbits-splittings} is just an instructive example, and it does not represent an accurate description of the single-particle energies for deformed nuclei. In fact, if the potential is deformed, the quantum numbers $l$ and $j$ are not valid anymore (a.m. is no longer a constant of motion for non-spherical potentials). A proper description of the single-particle orbits are represented by the so-called \emph{Nilsson diagrams}, where the energy for each state is represented as a function of the deformation parameter. Remember that the energies are in fact the eigenvalues of the Schrödinger equation associated to the initial Nilsson deformed Hamiltonian (see Eq. \ref{nilsson-schrodiner-equation}).

The spectrum of one-particle orbits plays an invaluable role within the nuclear structure and the study of deformed nuclei: the picture of one-particle motion in deformed potential works for deformed nuclei much better than the case of single-particle motion in spherical potentials for spherically shaped nuclei. Multiple quantitative analyses have been performed on experimental data of well-deformed (especially odd-$A$) nuclei, from light ($^{25}$Mg, $^{25}$Al) to heavy ($^{169}$Tm, $^{175}$Yb, $^{177}$Yb) \cite{hamamoto2016interplay}. Examples with this kind of diagrams are shown in Figs. \ref{nillson-diagram} - \ref{nillson-diagram-2}. 
\begin{figure}
    \centering
    \includegraphics[width=0.99\textwidth]{Chapters/Figures/nillson_diagram.png}
    \caption{A Nilsson diagram for protons or neutrons, with $Z$ or $N\leq50$. Picture taken from Ref. \cite{ragnarsson2005shapes}.}
    \label{nillson-diagram}
\end{figure}
\begin{figure}
    \centering
    \includegraphics[width=0.99\textwidth]{Chapters/Figures/nillson_diagram_2.png}
    \caption{A Nilsson diagram for neutrons, with $82\leq N\leq126$. Picture taken from Ref. \cite{ragnarsson2005shapes}.}
    \label{nillson-diagram-2}
\end{figure}

It can be seen that each state within a Nilsson diagram is represented as a solid line or a dashed line, depending on its parity (remember that the parity quantum number is given by $(-1)^N$ or, equivalently, by $(-1)^l$). The orbital labelling from the Figs. \ref{nillson-diagram} - \ref{nillson-diagram-2} is consistent with the one defined in the previous subsection. Another important aspect which can be seen in the Nilsson diagrams (for some orbits) is the `crossing' between states with different quantum numbers. In order to fully understand this concept the \emph{two-state mixing} concept is explained in Appendix \ref{appendix:two-state}.

\subsection{Nilsson orbitals}

Recalling Fig. \ref{nillson-orbits-splittings}, the splitting of an orbital into $j+1/2$ magnetic sub-states can be viewed as a set of energies where the nucleon is \emph{orbiting} around the bulk nucleus with an orbit that has a certain \emph{tilt} angle $\theta$ (see the orbits depicted in Figs. \ref{nillson-orbits-prolate-projections} - \ref{nillson-orbits-oblate-projections}). The tilting angle is conceptually shown in Fig. \ref{fig-nilsson-tilting-angle}. For that particular orbit, the angle is given by the expression \cite{krane1991introductory,casten2000nuclear}:
\begin{align}
    \sin\theta&=\frac{\Omega}{j}\ , \nonumber\\
    \theta&=\arcsin(\frac{\Omega}{j})\ .
    \label{theta-tilting-angle}
\end{align}
\begin{figure}
    \centering
    \includegraphics[scale=1]{Chapters/Figures/nilsson_tilting_angle.pdf}
    \caption{The orbit of a single particle orbiting the deformed nucleus, defined by the projection of the particle's a.m. $\Omega$ (on the symmetry axis) and the tilting angle $\theta$. Figure inspired from Ref. \cite{casten2000nuclear}}
    \label{fig-nilsson-tilting-angle}
\end{figure}

The change of $\theta$ is rather slow for low $\Omega$ projections, while rapid changes take place at high $\Omega$ values. %It should be noted that this discussion applies to a single-particle total a.m. projection, but as it was discussed previously, using the projection $K$ within calculations is equivalent (since the axially deformed potentials keep $\mathbf{R}$ oriented perpendicular to the symmetry axis).
As a numerical example, the $\theta$ variation is studied for the orbits $j=\{9/2,11/2,13/2\}$, with their corresponding projections. The evolution with $\Omega$ for different orbits can be seen in Fig. \ref{fig-tilting-angle-shape}.
\begin{figure}
    \centering
    \includegraphics[scale=0.7]{Chapters/Figures/tilted_theta_shape.pdf}
    \caption{The change in $\theta$ angle defined in Eq. \ref{theta-tilting-angle} with increasing values of $\Omega$, for a few orbitals $j$.}
    \label{fig-tilting-angle-shape}
\end{figure}

The splitting of an orbital $j$ into multiple sub-states (see Fig. \ref{nillson-orbits-splittings}) emerges from the considerations regarding the change in tilting angle and also the fact that the difference in energy is rather smooth (high) depending on low (high) $\Omega$ values. Based on this discussion, it is clear that a full Nilsson diagrams is constructed with the configuration mixing of different $j$ values.
% The configuration is superimposed on state-splitting via the $\Omega$ projections. With this idea, one can state that
Remarkable, \emph{no two lines in the Nilsson diagram with similar $\Omega$ values can cross each other}. When two such orbits come close to each other, they must repel as shown in Fig. \ref{fig-non-crossing}. Explaining the behavior of the lines that appear in the Nilsson diagrams \ref{nillson-diagram} - \ref{nillson-diagram-2} is straightforward: each line represents a Nilsson state, starting out straight and then sloping downward or upward, depending on the angle of the orbit relative to the bulk nucleus. The \emph{curving} of an orbit starts when it approaches another level with the same quantum number $\Omega$ and parity $\pi$. Thus, the structure of any Nilsson diagram relies on three main features \cite{casten2000nuclear}:
\begin{itemize}
    \item the $\Omega$ splitting
    \item repulsion between two levels
    \item single-particle shell model energies
\end{itemize}

Taking a closer look at the Nilsson from Fig. \ref{nillson-diagram-2}, there are two orbits within the $82$-$126$ neutron gap that can be analyzed in terms of \emph{mixing}: $f_{7/2}$ and $h_{9/2}$, respectively. Obviously, the lack of deformation implies a degeneracy of these orbits, but when deformation occurs, splitting kicks in. The angle of the orbital orientation $\theta$ depends on the ratio $\Omega/j$ (recall $\theta=\arcsin(\Omega/j)\approx\Omega/j$ at low $\Omega$). Small tilting angles will occur due to i) small values of $\Omega$ or ii) high $j$. As a result, the energies for the orbits $\Omega=1/2,3/2,5/2$ belonging to the $h_{9/2}$ shell are decreasing in energy faster with deformation than those from $f_{7/2}$ orbit. Consequently, the different rates of decrease for the Nilsson energies will overcome any small spherical energy separation $f_{7/2}-h_{9/2}$, making the orbits with low $\Omega$ to approach each other, leading to a more pronounced mixing. However, as discussed in \ref{appendix:two-state}, two orbits defined by the same quantum numbers cannot cross, causing the apparition of an \emph{inflection point}. The points can be seen when looking at the $\Omega=5/2$ and $\Omega=7/2$ orbits, which correspond to $f_{7/2}$ and $h_{9/2}$.

% Lastly, an alternative form of the Nilsson Hamiltonian should be expressed, taking into consideration the already studied nuclear radius (see Eq. \ref{nuclear-shape} which describes the shape of the nuclear surface) and the fact that until now, only the \emph{quadrupole} effects have been relevant to the discussion about deformed potentials in nuclei. Indeed, for quadrupole deformations, the nuclear radius can be simplified to:
% \begin{align}
%     R(\theta,\varphi)=R_0\left(1+\beta Y_2^0(\theta,\varphi)\right)\ .
%     \label{simple-quadrupole-nuclear-surface}
% \end{align}

The single-particle Hamiltonian can be written in the general form, starting from the expression Eq. \ref{eq-full-nilsson-ham}:
\begin{align}
    H_\text{Nil}=-\frac{\hbar^2}{2m}+\frac{1}{2}m(\omega_0r)^2-&\frac{4}{3}\sqrt{\frac{\pi}{5}}m(\omega_0r)^2\epsilon Y_2^0(\theta,\varphi)-2\kappa\hbar\omega_0(\vec{l}\cdot\vec{s})\nonumber\\
    &-2\kappa\hbar\omega_0\mu\left(l^2-\langle l^2\rangle_N\right)\ .
    \label{eq-nilsson-ham-spherical-harmonics}
\end{align}

The expression for the oscillator frequencies were already given defined as functions of the deformation parameter $\epsilon$, and they keep the same form  as Eqs. \ref{oscillator-frequencies-nilsson} - \ref{omega-0-oscillator-frequency}. It is worth mentioning that both forms of $H_\text{Nil}$ from Eq. \ref{eq-full-nilsson-ham} and Eq. \ref{eq-nilsson-ham-spherical-harmonics} are equivalent. Moreover, they describe the structure of the deformed nuclei in the limits of large deformations (via Eq. \ref{eq-full-nilsson-ham}) and small deformations (via Eq. \ref{eq-nilsson-ham-spherical-harmonics}). Within literature, $\kappa\approx 0.06$ and $\mu=0\sim 0.7$. As previously shown, the relationship between the $\epsilon$ and $\beta$ deformation parameters is given by $\epsilon=3/4\sqrt{5/\pi}\beta$.

When the deformations are small, $j$ is a good quantum number, and the Eq. \ref{eq-nilsson-ham-spherical-harmonics} represents a Hamiltonian for the AHO plus a \emph{perturbation} that is proportional to $\epsilon r^2Y_2^0$. Therefore, one can consider the eigenstates of the Hamiltonian as states labelled by the quantum numbers $Nlj$ and $m$ typical to the spherical case. Casten shows that if the angular part $Y_2^0$ is treated as a perturbation, it is possible to obtain a shift in energies relative to $\epsilon=0$ \cite{casten2000nuclear}:
\begin{align}
    \Delta E_{Nljm}=-\frac{4}{3}\sqrt{\frac{\pi}{5}}m\omega_2^0\epsilon\bra{Nljm}r^2Y_2^0\ket{Nljm}\ .
\end{align}

Furthermore, one can perform a separation of the radial and the angular parts while using the known relation for a harmonic oscillator potential:
\begin{align}
    \frac{1}{2}m\omega_0^2\bra{Nljm}r^2\ket{Nljm}=\frac{1}{2}\hbar\omega_0\left(N+\frac{3}{2}\right)\ ,
\end{align}
and together with the evaluation of the matrix elements for spherical harmonics, the final expression for the energy shift at small deformations is:
\begin{align}
    \Delta E_{Nljm}=-\frac{2}{3}\hbar\omega_0\left(N+\frac{2}{3}\right)\epsilon\frac{\left[3K^2-j(j+1)\right]\left[\frac{3}{4}-j(j+1)\right]}{(2j-1)j(j+1)(2j+1)}\ ,
    \label{nilsson-energy-shifts}
\end{align}
where the projection $m$ was replaced with the total angular momentum projection onto $z$-axis $K$. Based on Eq. \ref{nilsson-energy-shifts}, the following properties for a Nilsson diagram emerge:
\begin{itemize}
    \item There is a $K^2$ dependence for the energy shifts
    \item The quadrupole deformation parameter (albeit $\epsilon$ or $\beta$) shows a clear linear dependence for $\Delta E_{Nljm}$
    \item Another linear dependence for the shifts is induced by the principal (oscillator) quantum number $N$
    \item When the deformation parameter is positive, there are more downward sloping orbits than upward ones (example discussed below)
\end{itemize}

For a value $j>1/2$, the terms $\left[3K^2-j(j+1)\right]$ and $3/4-j(j+1)$ are negative, resulting in the following types of orbits: \cite{krane1991introductory}:
\begin{align}
    \text{downward sloping:}&\ K<\sqrt{\frac{j(j+1)}{3}}=\frac{j}{1.8}\approx 0.65j\ ,\\
    \text{upward sloping:}&\ K>0.65j\ .
\end{align}

It was already shown that the angular orientation (i.e., the tilting angle $\theta$) of an orbit is given by $\theta=\arcsin(K/j)$. For a ratio $K/j=0.65$, the tilting angle is $\theta=40^\circ$. The physical implication is that any larger \emph{tilt} of an orbit within a prolate quadrupole deformation is energetically unfavorable. In Fig. \ref{fig-nilsson-delta-E-shift} two different $j$ orbits, namely $h_{9/2}$ and $i_{13/2}$ are studied in terms of their energy shifts according to Eq. \ref{nilsson-energy-shifts}. It can be seen that there are more downward sloping orbitals, since the quadrupole deformation parameter has been set to a positive value $\epsilon=0.22$.
\begin{figure}
    \centering
    \includegraphics[scale=0.65]{Chapters/Figures/energy_shift_nilssonDeltaE.pdf}
    \caption{The energy shift $\Delta E$ for two orbits: $h_{9/2}$ and $i_{13/2}$ for a given deformation $\epsilon=0.22$. The dashed vertical lines (colored) represent the value for $K$ where the `change' from downward sloping curves to upward sloping curves takes place (that is $K\approx 0.65j$). This is just an illustrative example inspired from the discussion regarding single-particle orbits in Ref. \cite{casten2000nuclear}}
    \label{fig-nilsson-delta-E-shift}
\end{figure}

Based on the principal quantum number $N$, there is another important physical consequence. The dependence on $N$ will imply that the slopes of any Nilsson energy level will be \emph{steeper} for larger values of $N$. Thus, heavier nuclei will tend to deform much easier than lighter ones. The explanation was done in Refs. \cite{bohr1998nuclear,krane1991introductory,casten2000nuclear}. Shortly, a nucleon belonging to a high oscillator shell will have a large average radius (the expectation value of $r^2$ was provided in Eq. \ref{nilsson-energy-shifts} via the expression $\langle r^2 \rangle=(N+3/2)$ \cite{bertulani2007nuclear}). As the nucleus deforms, the density distribution of the nuclear matter will approach that orbit. The effect on the orbiting nucleon to decrease its energy rapidly as the nuclear matter comes closer to the orbit is due to the \emph{attractive} nature of the nuclear force. 
Clearly, this effect is less obvious for a particle in a lower oscillator shell that is already very close to the rest of the nuclear matter.

The centrifugal $\vec{l}^2$ and spin-orbit $\vec{l}\cdot\vec{s}$ terms from Eq. \ref{eq-nilsson-ham-spherical-harmonics} will become negligible in the limit of \emph{large deformation}, such that the Nilsson Hamiltonian will become just like an AHO. In this special case, the motion will separate into \emph{independent} oscillations in the direction of the symmetry axis and the perpendicular plane (i.e., in the direction of $z$-axis and $xy$ plane). Consequently, the good quantum numbers for this kind of situation are the $n_z$ and $(n_x+n_y)$ oscillator quantum numbers. Since the eigenvalues for a one-dimensional (and, implicitly for the three-dimensional) harmonic oscillator are established, the energy spectrum for single-particle orbits in the regime of large $\epsilon$ will be given by \cite{casten2000nuclear}:
\begin{align}
    E_{n_x,n_y,n_z}=\hbar\omega_x(N-n_z+1)+\hbar\omega_z\left(n_z+\frac{1}{2}\right)\ .
\end{align}

The remarking feature of the Hamiltonian is its invariance to rotations about the $z$ axis. The projections for the particle's orbital and spin a.m. are constants of motion. As it was discussed, the sum of the two projections $\Lambda$ and $\Sigma$ is indeed $\Omega$ or, equivalently, $K$ in the case of $\vec{R}$ being perpendicular to the $z$-axis. Concluding, the importance of the Nilsson Deformed Model was made clear enough in this section. Its importance within the rest of the present work will be justified later on, where based on a Particle-Rotor-Model \cite{bohr1998nuclear,davydov1958rotational}, it will play a crucial role in determining the Hamiltonians specific to the nuclei of interest.

\section{Collective Model}

Although the previous single-particle model is able to successfully treat many nuclei, the single-nucleon motion within a deformed potential is not enough to describe phenomena such as nuclear fission and quantities like quadrupole moments of multiple deformed nuclei \cite{townes1949nuclear}. Moreover, lifetime measurements through single-particle calculations fail to reproduce experimental data on some gamma-ray transitions of quadrupole nature \cite{goldhaber1951classification}. 

The collective model is one of the most `complete' tools in describing the nuclear phenomena across the chart of nuclides. It brought tremendous progress in nuclear community by validating and predicting the nuclear behavior at different spin ranges. A major feature of this model is the introduction of the so-called \emph{rotational bands}. Developed by Bohr and Mottelson \cite{bohr1953collective,bohr1998nuclear} more than 50 years ago, the Nuclear Collective Model is based on the Liquid-Drop-Model \cite{meitner1939disintegration,bohr1939mechanism,myers1974nuclear}. Moreover, the predictions for nuclear deformation made by Rainwater \cite{rainwater1950nuclear} played another fundamental role in the development. The basic assumption is that the nuclear density distribution can be approximated as a droplet of nuclear matter with shape-specific degrees of freedom. This droplet is also capable of vibrating and rotating. According to the discussion from Chapter \ref{chapter-2}, the nuclear radius was described in terms of a set of \emph{collective coordinates}, which dictate the shape evolution with time.

\subsection{Bohr Hamiltonian}

As a first step, the concept of a nuclear liquid drop is used to construct the Hamiltonian. The droplet exhibits shape and surface oscillations, which have a dynamical character. These shape oscillations are illustrated in Fig. \ref{fig-nuclear-vibration}, where a vibrating nucleus is represented, having a spherical equilibrium shape. Since the collective coordinates are time-dependent, at each moment in time, the nuclear radius $R$ will be defined in the direction given by the radial coordinates $\theta,\varphi$. By using Eq. \ref{nuclear-shape} that characterizes the \emph{vibrations} of a nuclear surface, one can give the Hamiltonian of \emph{collective} nature as:
\begin{align}
    H_\text{coll}\equiv T+V=\frac{1}{2}\sum_{\lambda\mu}\left[B_\lambda\left|\frac{\text{d}\alpha_{\lambda\mu}}{\text{d}t}\right|^2+C_\lambda|\alpha_{\lambda\mu}|^2\right]\ .
    \label{collective-hamiltonian-stiffness-inertia}
\end{align}
\begin{figure}
    \centering
    \includegraphics[width=0.99\textwidth]{Chapters/Figures/shape_oscillations.pdf}
    \caption{The vibration of a nucleus whose equilibrium shape is a spheroid, with nuclear radius $R_0$ (\emph{average nuclear radius}) and the nuclear radius $R$ at a different moment in time. The equilibrium shape is represented by the red color, while the surface vibration is represented by the blue color. The oscillations are depicted through the double arrows.}
    \label{fig-nuclear-vibration}
\end{figure}

The Hamiltonian is invariant under rotations and also invariant under time reversal \cite{messiah2014quantum}. The real numbers $B_\lambda$ and $C_\lambda$ represent the \emph{inertial} and \emph{stiffness} parameters. After a canonical quantization, the spectrum of such a Hamiltonian will have a harmonic-like structure, depending on the value of $\lambda$ (see calculations in Refs. \cite{ring2004nuclear,bertulani2007nuclear}). Indeed, by looking at the expression from Eq. \ref{collective-hamiltonian-stiffness-inertia}, one can see that it can be brought to a form:
\begin{align}
    H_\text{coll}^\text{osc}=\frac{p^2}{2m}+\frac{1}{2}kr^2\ ,
    \label{eq-bohr-hamiltonian-oscillator-simple}
\end{align}
which is typical to a harmonic oscillator Hamiltonian. The frequency of oscillation is given by the relation $\omega=\sqrt{k/m}$. The vibrations can now be understood in terms of a sum of harmonic oscillator frequencies, where each frequency is given by $\lambda$:
\begin{align}
\omega_\lambda=\sqrt{\frac{C_\lambda}{B_\lambda}}\ .
\end{align}

The inertia term $B_\lambda$ (also called the \emph{mass parameter}) has the following expression \cite{ring2004nuclear}:
\begin{align}
    B_\lambda=\rho \frac{mR_0^5}{\lambda}=\frac{3}{4\pi\lambda}AmR_0^2\ ,
    \label{inertia-parameters-B}
\end{align}
showing a quadratic dependence with the average nuclear radius. The nuclear density of nucleons with mass $m$ is given by $\rho$. The stiffness parameter $C_\lambda$ is usually expressed in terms of the \emph{surface tension} $\sigma$, assuming that the nuclear matter exhibits an irrotational flow and the nuclear charge is distributed uniformly over the entire volume \cite{ring2004nuclear}:
\begin{align}
    C_\lambda=(\lambda-1)(\lambda+2)\sigma R_0^2-\frac{3}{2\pi}\frac{\lambda-1}{2\lambda+1}\frac{(Ze)^2}{R_0}\ .
    \label{stiffness-parameters-C}
\end{align}

With these ingredients, one can sketch a collective Hamiltonian similar to Eq. \ref{eq-bohr-hamiltonian-oscillator-simple} in the following manner:
\begin{align}
    H_\text{coll}=\sum_{\lambda\mu}\hbar\omega_\lambda\left(N_{\lambda\mu}+\frac{1}{2}\right)\ ,
\end{align}
where indeed, the typical harmonic oscillator spectrum emerges. The recipe for further manipulation of the Bohr's collective Hamiltonian from Eq. \ref{collective-hamiltonian-stiffness-inertia} is a lengthy process \cite{bohr1998nuclear,ring2004nuclear}, with several considerations beyond the scope of the current work. Shortly, for the quadrupole deformed nuclei, the expression of the potential term $V$ will be a function that depends on the parameters $(\beta,\gamma)$. The potential will be defined as a quadratic approximation in the vicinity of a deformed minimum point $p_0|_\text{min}=(\beta_,\gamma_0)$. This starts from the general assumption that the nucleus has the deformation $p_0$ in the ground state, and excitations around the \emph{equilibrium} point. Thus $V(\beta,\gamma)$ can be written as:
\begin{align}
    V(\beta,\gamma)=\frac{1}{2}C_{0}\left(\alpha_{20}(\beta,\gamma)-\alpha_{20}^0\right)^2+\frac{1}{2}C_{2}\left(\alpha_{22}(\beta,\gamma)-\alpha_{22}^0\right)\ .
    \label{bohr-collective-potential}
\end{align}

Choosing the body-fixed axis as a reference system will simplify the results, since the system's axes coincide with the principal axes of the ellipsoid itself. Assuming this coordinate system and an axial ellipsoid, the kinetic term from $H_\text{coll}$ can be decomposed into a \emph{rotational} and a \emph{vibrational} part, i.e., $T=T_\text{vib}+T_\text{rot}$. Their expressions are given in terms of the deformation parameters $(\beta,\gamma)$, the mass parameters $B_2$ (for the quadrupole deformations) and the stiffness parameters $C_2$. Thus, the vibrational term is \cite{li2022model}:
\begin{align}
    T_\text{vib}=\frac{1}{2}B_2\left(\dot{\beta}^2+\beta^2\dot{\gamma}^2\right)\ ,
    \label{kinetic-vibrational-energy-collective}
\end{align}
and the rotational term is \cite{li2022model}:
\begin{align}
    T_\text{rot}=\frac{1}{2}\sum_i^3\mathcal{I}_i\omega_i^2\ ,
    \label{kinetic-rotational-energy-collective}
\end{align}
where the three principal axes of the ellipsoid are indexed by $i=1,2,3$. In the expression of $T_\text{rot}$, two crucial physical quantities arise, namely the angular velocities around each body-fixed axis and the functions $\mathcal{I}_k$ that will eventually play the role of \emph{moments of inertia}. 
%The moments of inertia are functions $\mathcal{I}_k(\beta,\gamma$) that are defined more generally as \cite{ring2004nuclear}:
% \begin{align}
%     \mathcal{I}_k=4B_2\beta^2\sin^2\left(\gamma-\frac{2\pi}{3}k\right)\ .
% \end{align}
% In the case of fixed deformation, then the rotational kinetic term represents the energy of a rotor with the moments of inertia $\mathcal{I}_{1,2,3}$, i.e., a \emph{pure rotor}. When the deformation parameters are changing, the rotational and vibrational degrees of freedom will become coupled by the deformation dependence of $\mathcal{I}_k$, leading to a situation that is not specific to a pure rotor. In fact, the functions $\mathcal{I}_k$ will not represent the moments of inertia (MOI) for a rigid rotor anymore. However, a comparison between $\mathcal{I}$, the rigid-like $\mathcal{I}_k^\text{rig}$ MOI, and irrotational-like MOI $\mathcal{I}_k^\text{irr}$ is done experimentally, from determinations of the energy spacing between the first excited states. 
There are two types of moments of inertia that describe the nuclear rotation: the rigid-like MOI and the irrotational (hydrodynamical) MOI, having the following expressions \cite{bohr1954rotational,ring2004nuclear}:
\begin{align}
    \mathcal{I}_k^\text{rig}&=\frac{2}{5}mAR_0^2\left(1-\sqrt{\frac{5}{4\pi}\beta\cos\left(\gamma-\frac{2\pi}{3}k\right)}\right)\ ,\\
    \mathcal{I}_k^\text{irr}&=\frac{3}{2\pi}mAR_0^2\beta^2\sin^2\left(\gamma-\frac{2\pi}{3}k\right)\ .
    \label{eq-irrotational-rigid-mois}
\end{align}

The dependence of the two types of MOI on the triaxiality parameter $\gamma$ (and fixed $\beta$) can be seen in Fig. \ref{fig-irrotational-rigid-mois}. The differences between the rigid and irrotational MOI are as follow:
\begin{itemize}
    \item The irrotational MOI vanish when the ellipsoid has axial symmetry
    \item $\mathcal{I}^\text{irr}$ is much more sensitive to the deformation $\beta$, while the rigid MOI have most of the contribution coming from a typical rigid sphere
    \item The \emph{experimental} MOI of well-deformed nuclei show that the \emph{real} MOI within a nucleus are neither irrotational, nor rigid-like, but in fact they follow:
    \begin{align}
        \mathcal{I}^\text{irr}<\mathcal{I}^\text{exp}<\mathcal{I}^\text{rig}\ ,
        \label{experimental-MOI-vs-rig-irr}
    \end{align}
\end{itemize}
\begin{figure}
    \centering
    \includegraphics[width=0.99\textwidth]{Chapters/Figures/mois_rig_irr.pdf}
    \caption{Comparison between the rigid-like and irrotational-like MOI defined in Eq. \ref{eq-irrotational-rigid-mois}, which are typical for a rigid rotator or an irrotational motion of a fluid. The deformation parameter $\beta$ is fixed $\beta=0.3$.}
    \label{fig-irrotational-rigid-mois}
\end{figure}

\subsection{Nuclear Vibration}

The energy spectrum is composed of levels built from the oscillator frequencies. The absorption or emission of vibration-like energy quanta (i.e., phonons) when the nucleus transitions to a higher or a lower energy state will be in terms of dipole, quadrupole, octupole, etc. vibrating phonons. These excitations will generate the ground and excited states, resulting in a `collection' of \emph{vibrational bands}. Since the quadrupole effects are of interest within the current work, a spectrum specific to quadrupole vibrations $\lambda=2$ phonons can be seen in Fig. \ref{fig-vibrational-bands}.
\begin{figure}
    \centering
    \includegraphics[width=0.99\textwidth]{Chapters/Figures/vibrational_states.pdf}
    \caption{Illustration of vibrational bands built as excitations of multiple phonons on a ground state. The left side reflects an ideal harmonic vibrator, with the degenerate spin states indicated for each level. Right side shows non-degenerate vibrational levels that exist in nuclei. Keep in mind that each phonon level is built as multiple excitations of a \emph{quadrupole phonon}.}
    \label{fig-vibrational-bands}
\end{figure}

% A general rule that is used to construct vibrational bands is related to the concept of \emph{symmetrized states}. Since the excited quanta are represented by phonons (identical bosons) having integer angular momentum (e.g., $\lambda=2$ for the quadrupole vibrations), the only spin states than can be observed in such spectra are the ones for which the final angular momenta couple to give symmetric combinations. This is the reason why in the energy state created by exciting the ground state $0^+$ with two phonons, only the states $0^+,2^+,4^+$ appear; the spin states $1^+$ and $3^+$ do not give symmetric combinations. A more generic method for determining which sequence of spins can be obtained by coupling angular momenta of vibrational phonons can be seen in Ref. \cite{ring2004nuclear}.

A quantity often used within the measurements of nuclear properties is the ratio between the second and the first excited states within a band. These ratios are usually denoted by $E(4^+)/E(2^+)$ (the state $4^+$ belongs to the triplet phonon state depicted in Fig. \ref{fig-vibrational-bands}), and the theoretical value for the vibrational model gives a value of $2$. However, some experimental results point out to a value close to $2.2$ for nuclei below $A=150$, and a constant value of $3.3$ for $150<A<190$ (refer to Fig. \ref{4state-2state-ratio}). It should be noted that for the latter case, the value of the ratio is specific to another kind of nucleonic motion: \emph{rotation}, which will be discussed in the following section.
Example of experimental \emph{vibrational bands} for several even-$A$ nuclei are shown in Fig. \ref{energy-levels-120Te-virbational-band}.
\begin{figure}
    \centering
    \includegraphics[width=0.32\textwidth]{Chapters/Figures/120Te_vib_experimental.pdf}
    \includegraphics[width=0.32\textwidth]{Chapters/Figures/60Ni_vib_experimental.pdf}
    \includegraphics[width=0.32\textwidth]{Chapters/Figures/44Ca_vib_experimental.pdf}
    \caption{Vibrational bands in even-$A$ nuclei. \textbf{Left:} The experimental energy levels for $^{120}$Te. The triplet states that correspond to $\lambda=2$ can be seen, together with the quintuplet formed by adding three phonons to the $0^+$ ground state. Experimental data are taken from \cite{kitao2002nuclear}. \textbf{Middle:} The experimental data for $^{60}$Ni. For simplicity, only the first two phonon states are represented. Experimental data were taken from Ref. \cite{browne2013nuclear}. \textbf{Right}: The experimental data for a vibrational-like structure in $^{44}$Ca \cite{chen2011nuclear}. Note the highest energy level coming from the vibrational motion corresponding to an \emph{octupole} mode ($\lambda=3$).}
    \label{energy-levels-120Te-virbational-band}    
\end{figure}

There can also be vibrational bands in odd-$A$ nuclei. Indeed, if one considers the nucleus as a spherical even-even core plus an extra nucleon, the \emph{final} nuclear states are formed by coupling an individual $j$ orbit with the vibrational states of the core. An example is $^{63}$Cu, which has the ground state $3/2^{-}$. The g.s. for this nucleus is given by the last \emph{uncoupled} nucleon that occupies a shell. In fact, for this particular nucleus, it is the $2p_{3/2}$ proton that will give the final spin and parity of the nucleus. The vibrational spectrum is regarded as coupling the aforementioned proton with the $^{62}$Ni core. Indeed, by taking a $2^+$ vibrational phonon from the even-$A$ nucleus, then the (odd-proton + phonon) system can generate a sequence of energy states with angular momenta $I=1/2,\ 3/2,\ 5/2,\ 7/2$ and negative parity states. The energy levels depicted in Fig. \ref{energy-levels-63Cu-virbational-band} show the particle-core coupling effects on the vibrational structure in odd-mass nuclei.
\begin{figure}
    \centering
    \includegraphics[width=0.8\textwidth]{Chapters/Figures/63Cu_vib_experimental.png}
    \caption{The experimental data of the vibrational states in $^{63}$Cu \cite{erjun2001nuclear}. The $2^+$ phonon state in $^{62}$Ni is also shown. The experimental data for $^{62}$Ni are taken from Ref. \cite{nichols2012nuclear}. The blue rectangle tries to emphasize that the quadruplet in $^{63}$Cu is formed by the coupling of the odd proton to the $2^+$ vibrational state from the neighboring nucleus.}
    \label{energy-levels-63Cu-virbational-band}
\end{figure}

Since the quadrupole vibrations carry an angular momentum of $\lambda=2$, there can be two types of vibrations in a deformed nucleus: one corresponding to $K=0$ and one for $K=2$. The $K=0$ vibrational motion is called $\beta$-vibration, and this induces a small fluctuations in the quadrupole deformation parameter, but it preserves the axial symmetry of the nucleus (the vibration is aligned with the deformed axis). The $K=02$ vibration is called the $\gamma$-vibration which is responsible for fluctuations in the triaxiality parameter $\gamma$. A qualitative description for such an oscillation can be explained in terms of an american football \cite{krane1991introductory}: $\gamma$-vibrations is a simultaneous pushing and pulling of the ball on its sides, while the $\beta$ vibrations correspond to continuous pushing and pulling on the ends of the football. An illustration explaining the geometrical meaning of the $\beta$ and $\gamma$ vibrations in nuclei can be seen in Fig. \ref{rotation-vibration-geometrics}.
\begin{figure}
    \centering
    \includegraphics[width=0.99\textwidth]{Chapters/Figures/rotationsVibrations_Rotations.pdf}
    \caption{The $\beta$ and $\gamma$ vibrations occurring in nuclei. The initial nucleus in this example is of prolate type. Each mode of vibration is visualized from the side-view and top-view, respectively. This figure was inspired from Ref. \cite{li2022model}.}
    \label{rotation-vibration-geometrics}
\end{figure}

\subsection{Nuclear Rotation}

The rotational-kinetic term from $H_\text{coll}$ is called the \emph{rotational energy}, and its expression is given by \cite{corrigan1976exact}:
\begin{align}
    \hat{T}_\text{rot}=\frac{\hat{I}_1^2}{2\mathcal{I}_1}+\frac{\hat{I}_2^2}{2\mathcal{I}_2}+\frac{\hat{I}_3^2}{2\mathcal{I}_3}\ .
    \label{eq-rotational-energy-quantized}
\end{align}

The three operators that appear in Eq. \ref{eq-rotational-energy-quantized} are the projections of the total angular momentum $\mathbf{I}$ on the body-fixed axes (see Fig. \ref{rotational-coupling-schematic} for reference). Note that throughout the text, the angular momentum will be interchangeably represented with an arrow or by bold letters.

An important conclusion emerging from the work of Bohr and Mottelson \cite{bohr1954rotational} (also see discussion in Ref. \cite{greiner1996nuclear}) is that no rotations about the symmetry axis are possible for spherical nuclei or axially deformed nuclei. 
%Consequently, a prior deformation (e.g., of quadrupole type) and a rotating axis which does not coincide with the symmetry one are required for the appearance of rotational levels.
Indeed, every nucleus which contains energy states that are generated by the rotational motion has these bands due to the rotation about an axis that is different from the symmetry axis, and its shape is either axially-symmetric or axially-asymmetric \cite{hamamoto2016interplay}. In its motion, a nucleus can generate angular momentum in two ways:
\begin{itemize}
    \item \emph{collectively}: via combined rotations and vibrations of the nuclear droplet (the rotation + vibration spectrum of a nucleus will be shown in the next section)
    \item \emph{single-particle excitations}: unpaired nucleons can rearrange themselves in such a way that they account to the nuclear spin
\end{itemize}

The coupling of the droplet's rotational a.m. $\mathbf{R}$ with the single-particle angular momentum of a valence nucleon $\mathbf{j}$ can be seen in Fig. \ref{rotational-coupling-schematic}. Based on the expression of $T_\text{rot}$ from Eq. \ref{kinetic-rotational-energy-collective}, its quantized form given in Eq. \ref{eq-rotational-energy-quantized}, and the coupling scheme depicted in Fig. \ref{rotational-coupling-schematic}, it is possible to construct a Hamiltonian that corresponds to rotating nucleus with no valence particle. More precisely, the $\mathbf{j}$ term is neglected, such that only the pure collective motion of a nucleus is emphasized ($\mathbf{I}=\mathbf{R}$). The energy spectrum for an even-even nucleus is thus expressed as \cite{davydov1958rotational}:
% This approximation is however only valid for even-even nuclei and for the low-lying states \cite{bertulani2007nuclear,davydov1958rotational}:
% \begin{align}
%     H&=H_\text{rot}+H_\text{intr}=\sum_i\frac{\hbar^2}{2\mathcal{I}_i}R_i^2\ ,
% \end{align}
% where the second term $H_\text{intr}$ represents some effects coming from the internal structure of the rotator itself. However, most of the time these terms can be neglected, leading to a rather simple energy spectrum for the even-even nuclei:
\begin{align}
    E_\text{rot}(I)=\frac{\hbar^2}{2\mathcal{I}_\perp}I(I+1)\ .
    \label{eq-simple-rotor-spectrum}
\end{align}
\begin{figure}
    \centering
    \includegraphics[scale=0.6]{Chapters/Figures/SCHEMATIC_COUPLING_ROTATIONAL.pdf}
    \caption{The coupling of the collective angular momentum with the angular momentum of a single-particle for an axially deformed nucleus that is rotating about an axis perpendicular to the deformation axis.}
    \label{rotational-coupling-schematic}
\end{figure}

In the above expression, $\mathcal{I}_\perp$ corresponds to an axis that is perpendicular to the symmetry axis (i.e., either $1$-axis or the $2$-axis) of the ellipsoid. In the case of axial symmetry both moments are equivalent, such that the general notation $\mathcal{I}_\perp$ is made. As an observation, since there is no single-particle contribution, the quantized angular momentum $I$ is equivalent to $R$. The spectrum defined by Eq. \ref{eq-simple-rotor-spectrum} leads to energy spacings $\propto I(I+1)$, which is also met in the molecular spectra. The ground-state will always be the state $0^+$, and due to the \emph{mirror symmetry} that is required for the wave-functions describing the even-even nuclei, every other excited state will be represented by an even value of the spin: $2^+,4^+,\dots$ \cite{ring2004nuclear}. The ratio $E(4^+)/E(2^+)$ for these bands is $3.33$. This is indeed the best signature for the rotational phenomena in nuclei, indicating a clear presence of deformation + rotational motion. In a previous work by the current team (Raduta et al. \cite{raduta2017semiclassical}), some spectra of even-even $^{158}$Er were studied, and agreement with the observed experimental data was impressive. The energy spectra obtained from Eq. \ref{eq-simple-rotor-spectrum} has a classical counterpart known within literature as the \emph{symmetric top}. Fig \ref{rotational-bands-even-even} shows some examples of rotational bands in even-$A$ nuclei.
\begin{figure}
    \centering
    \includegraphics[scale=0.7]{Chapters/Figures/Er158-Rotational-Bands.pdf}
    \includegraphics[scale=0.7]{Chapters/Figures/Hf180-Rotational-Bands.pdf}
    \caption{Experimental data showing the rotational bands in even-even nuclei. Note the spacing between the states that increases with $I$ via the rule from Eq. \ref{eq-simple-rotor-spectrum}. Experimental data are taken from Refs. \cite{nica2017nuclear,mccutchan2015nuclear}.}
    \label{rotational-bands-even-even}
\end{figure}

The spectra of a simple rigid rotator assumes that the projection $K$ of the total angular momentum for the ground-state of even-even nuclei is $K=0$. So the next step is to consider more general cases of nuclear rotation. There are two general cases of rotational bands that can occur, and both require the coupling scheme of a rotor with a valence nucleon, such that the angular momentum will be $\mathbf{I}=\mathbf{R}+\mathbf{j}$ (so both situations will correspond to odd-$A$ nuclei). The two situations are called \emph{Deformation aligned bands} and \emph{Rotation aligned bands} \cite{uwitonze2015assignment}. These are detailed in Appendix \ref{appendix:ral-dal-signature-scheme}.

\subsection{Superimposed Rotations and Vibrations}

Up until now, the collective rotations and vibrations were discussed separately. The small vibrations of the nuclear surface lead to band structures constructed via the quadrupole phonons carrying $\lambda=2$ units of angular momentum. The proper picture of a rotating deformed nucleus consists of a stable \emph{equilibrium shape} that is determined by $i)$ the \emph{rapid internal motion of the nucleons} within the nuclear potential and $ii)$ their entire distribution doing a \emph{slow rotation} having a negligible effect on the nuclear structure or on the individual nucleonic orbits.

Besides the two (separated) degrees of freedom for the nuclear system (i.e., collective rotation and vibrations), they can also be superimposed on each other. Starting from the expression of the \emph{Collective Bohr's Hamiltonian} (calculations were presented in Eq. \ref{collective-hamiltonian-stiffness-inertia} and Eqs. \ref{bohr-collective-potential} - \ref{kinetic-rotational-energy-collective}), the spectrum can be described by the general Hamiltonian:
\begin{align}
    H_\text{coll}=T+V=T_\text{vib}+T_\text{rot}+V
\end{align}

A separation of the Hamiltonian terms in three main components will be made: one associated to the $\beta$-vibration, one for $\gamma$-vibrational mode, and the third term is the rotation of a rigid rotor characterized by a total spin $I$ and well-defined MOI. Following the separation, the spectrum will be \cite{ring2004nuclear,li2022model}:
\begin{align}
    E_{n_\beta n_\gamma IK}=\hbar\omega_\beta\left(n_\beta+\frac{1}{2}\right)+\hbar\omega_\gamma\left(n'_\gamma+\frac{1}{2}|K|\right)+\frac{\hbar^2}{2\mathcal{I}}\left[I(I+1)-K^2\right]\ .
    \label{collective-rotation-vibration-energy-spectrum}
\end{align}

Note the two harmonic-like solutions making the vibrational mode present within the energy formula. The two `frequencies' are given in terms of the stiffness and inertia parameters, which were previously defined via Eqs. \ref{inertia-parameters-B} - \ref{stiffness-parameters-C}:
\begin{align}
\omega_\beta=\sqrt{\frac{C_0}{B_2}}\ ,\ \omega_\gamma=\sqrt{\frac{C_2}{B_2}}\ ,   
\end{align}
and the two quantum numbers are as follows \cite{li2022model}: 
\begin{align}
    n_\beta&=0,1,\dots\ ,\\
    n'_\gamma&=2n_\gamma+1\ ,\ n_\gamma=0,1,\dots\ .
\end{align}
The allowed values for the quantum numbers are as \cite{li2022model}:
\begin{align}
    K&=0,2,4,\dots\ ,\nonumber\\
    I&=\begin{cases}
        K,K+1,K+2,\dots &\text{for} K\neq 0\\
        0,2,4,\dots &\text{for} K=0\\
   \end{cases}\ .
\end{align}

Consequently, the spectrum of a typical nucleus in which both vibration and rotation occur has a set of bands that are characterized by the quantum numbers $K,n_\beta,n_\gamma$. This collective spectrum is exemplified in Fig. \ref{collective-rotation-vibration-energy-levels}, and the main bands are:
\begin{enumerate}
    \item the ground-state band, having states with even $I$, where excitation energies are constructed from the rotor term
    \item the $\beta$ band, starting above the g.s. band with $\hbar\omega_\beta$ and containing only even spins
    \item the $\gamma$ band, corresponding to $K=2$ (as explained in a previous section). It is distinguished from the $\beta$ band because it starts with $2^+$ as first state and it contains both odd and even spins.
    \item the higher-level bands are the $n_\gamma=1$ and $K=4$ bands for $\gamma$-vibrational mode, and the $\beta$ band with $n_\beta=2$.
\end{enumerate}
\begin{figure}
    \centering
    \includegraphics[width=0.99\textwidth]{Chapters/Figures/types_collective_bands.pdf}
    \caption{The energy spectrum specific to the Collective Model with \emph{Rotations + Vibration}. Each quantum number is also shown at the bottom of the bands. This figure is taken from Ref. \cite{li2022model}.}
    \label{collective-rotation-vibration-energy-levels}
\end{figure}

\subsection{Collective Quantities}
\label{c3-collective-quantities}

In this section, some important quantities that are strictly related to the collective nature of nuclei will be described. Indeed, one can understand nuclear deformation, energy spectra, and behavior of nuclei with respect to spin by studying quantities such as \emph{rotational frequencies}, \emph{moments of inertia}, \emph{quadrupole moments}, and so on. The comparison with experimental data for these quantities can help validate the theoretical assertions that are initially made, which represents the crucial test of any developed model.

\subsubsection{R - Energy Ratio}

As discussed before, the energy ratio between the first excited $4^+$ state to the first excited $2^+$ state is a very good test of rotational or vibrational spectra of nuclei. Casten et al. \cite{casten2000nuclear} shows the evolution of this ratio across the mass number $A$, and a classification between \emph{vibrational} vs. \emph{rotational} character is made in Fig. \ref{4state-2state-ratio}.
\begin{figure}
    \centering
    \includegraphics[width=0.99\textwidth]{Chapters/Figures/vibrations_rotations_E42-ratio.pdf}
    \caption{The experimental ratio $R_{4^+/2^+}$ in even-$Z$ and even-$N$ nuclei. Each line is connecting sequences of isotopes. Note the two important values for $R_{4^+/2^+}$, namely $2$ and $3.33$ given for a perfect vibrator and a pure rotator, respectively. Text with magenta color marks the magic numbers for $Z$ or $N$. This plot was adapted from Ref. \cite{casten2000nuclear}.}
    \label{4state-2state-ratio}
\end{figure}

\subsubsection{Rotational Frequencies}

In the classical limit the angular frequency:
\begin{align}
    \omega=l_\text{cls.}/\mathcal{I}\ ,
\end{align}
describes the kinetic energy of a rotating object. Indeed, $\omega$ is frequency of rotation around a particular direction, and its quadratic behavior gives the energy:
\begin{align}
    E=\frac{1}{2}\mathcal{I}\omega^2\ ,
\end{align}

The above expression can also be given in terms of the angular momentum $l_\text{cls.}$, such that the final energy becomes $E_\text{cls.}=l_\text{cls.}^2/(2\mathcal{I})$. Quantum mechanically, it was shown that $l^2$ is expressed as $\hat{l}^2_\text{quantum}=\hbar^2 l(l+1)$. Bengtsson et al. \cite{bengtsson1979quasiparticle} calculated the so-called \emph{Routhians} (single-particle energies within the rotating frame of reference), and they found a \emph {canonical relation} between the energies and rotational frequencies:
\begin{align}
    \omega=\frac{\text{d}E(I)}{\text{d}I_x}\ ,
    \label{rotational-frequency-canonical-definition}
\end{align}
where the term $I_x$ is called the \emph{aligned angular momentum}, and usually it denotes the experimental spin of every state minus a reference value (see Harris \cite{harris1965higher}). The signature property discussed in Appendix \ref{appendix:ral-dal-signature-scheme} plays a pivotal role in determining $I_x(I)$, since the projection $K$ of the angular momentum onto the deformation axis must be taken into account:
\begin{align}
    I_x(I)=\sqrt{\left(I+\frac{1}{2}\right)^2-K^2}\ ,
    \label{aligned-angular-momentum}
\end{align}
which for the $K=0$ bands reaches the simplified form $I_x^2=(I+\frac{1}{2})^2$. The value of $K$ is typically the band-head's angular momentum \cite{bengtsson1979quasiparticle,bengtsson1984signature}. Alternatively, for a sequence of states where $\Delta I=2\hbar$, the rotational frequency is usually calculated as:
\begin{align}
    \omega\stackrel{not}{=}\hbar\omega_\text{rot}(I)=\frac{E_\gamma(I\to I-2)}{2}\ .
    \label{rotational-frequency-canonical}
\end{align}

% A more concise definition for the rotational frequency is related to the transition between two consecutive states $I+1\to I-1$ within a rotational band: a unique value of $\omega$ is attributed to the spin $I$, which is defined as a mean value of the two angular momenta from the corresponding transition. Such a construction will yield a set of discrete points $\omega(I)$, and one can obtain a `continuous' function $\omega(I)$ together with its inverse $I(\omega)$, thus making the term $I$ from Eq. \ref{aligned-angular-momentum} to be in fact $I(\omega)$.

The rotational frequency is used to represent many quantities which characterize collective motion and nuclear deformation. By representing the total or the aligned angular momenta as functions of $\omega$ can show wether multiple bands have the same nature. Moreover, representing the MOI as function of $\omega$ will also give an insight on the intrinsic structure of the nucleus. Another interesting feature that is present in high-spin spectra of nuclei is the so-called \emph{backbending} phenomena. This phenomenon appears due to the Coriolis effect: nucleons can suffer a de-pairing that leads makes their a.m. to align with the rotational axis, causing a sharp increase in the MOI \cite{ring2004nuclear,kvasil2004backbending}. These kind of effects are correlated to high deformation, increased rotation, and change in the nucleonic alignment.

\subsubsection{Moments of Inertia}

This is a crucial quantity that describes the degree of deformation and asymmetry of the nuclear shape (remember discussion from Chapter \ref{chapter-2}). It is possible to retrieve an \emph{experimental} value for the MOI by inferring the energy spacing between consecutive levels of a collective spectrum. A classification of types of MOI was done in Eq. \ref{eq-irrotational-rigid-mois}, where the MOI dependence on the deformation parameters and even the mass parameter $B_\lambda$ was shown. The most general expression for the MOI can be written as \cite{ahmad2021backbending}:
\begin{align}
    \mathcal{I}=\frac{\hbar^2}{2}\left(\frac{\text{d}E}{\text{d}J(J+1)}\right)^{-1}\ ,
\end{align}
where the classical angular momentum $J$ is related to its quantum equivalent via the correction $J=I+1/2$ (see Eq. \ref{aligned-angular-momentum}). Practically, the derivative can be expressed in terms of the aligned angular momentum $I_x$. Moreover, there are two types of MOI describing the characteristics of the rotational bands: the \emph{kinematical} and \emph{dynamical} moments of inertia. The kinematic MOI is given by \cite{wu1992relation}:
\begin{align}
    \mathcal{I}^{(1)}=\frac{\hbar I_x}{\omega}=\hbar^2 I_x\left(\frac{\text{d}E}{\text{d}I_x}\right)^{-1}\ ,
    \label{kinematic-moi-general}
\end{align}
% One can see why the aligned angular momentum is important, since its variation w.r.t. the energies lead to theoretical determinations of the kinematic MOI. From the observed intraband $E2$ transitions one can extract the (kinematic) moment of inertia via the rule:
% \begin{align}
%     \mathcal{I}^{(1)}(I-1)=\hbar^2\frac{2I-1}{E_\gamma(I,I-2)}\ ,
%     \label{kinematic-moi-energy-levels}
% \end{align}
% where $E_\gamma(I,I-2)$ represents the energy difference between two consecutive levels $E(I)$ and $E(I-2)$. The dependence on $I$ for this type of MOI makes its experimental determination to require some spin assignments to each state of the excited spectrum.
while the dynamic moment of inertia is expressed as \cite{wu1992relation}:
\begin{align}
    \mathcal{I}^{(2)}(I)=\hbar\frac{\text{d}I_x}{\text{d}\omega}=\hbar^2\left(\frac{\text{d}^2E}{\text{d}I_x^2}\right)^{-1}\ ,
    \label{dynamic-moi-general}
\end{align}
or, equivalently, as:
\begin{align}
    \mathcal{I}^{(2)}(I)=\hbar^2\frac{4}{\Delta E_\gamma(I)}=\hbar^2\frac{4}{E_\gamma(I+2,I)-E_\gamma(I,I-2)}\ .
    \label{dynamic-moi-energy-levels}
\end{align}

%Note that any calculations for the dynamical MOI does not require prior knowledge about the spin assignments. These two types of MOI are usually represented as function of the rotational frequency $\omega$. Since the total spin $I$ can be expressed as a function of rotational frequency, plotting the kinematic/dynamic MOI as function of angular momentum is also preferred. In the present work, these quantities are of high interest (graphical representations for different nuclei will be shown in a future chapter), their relative behavior will help characterize bands with similar nucleonic structure.

% An alternative description for these MOI can be done through the so-called $ab$ formula \cite{wu1992relation,wu1992spin}, where the energies corresponding to the rotational spectrum are parametrized as:
% \begin{align}
%     E(I)=a\left(\sqrt{1+bI(I+1)}-1\right)
%     \label{ab-formula}
% \end{align}

% Fitting the experimental data will produce a set of parameters $a$ and $b$ that will be used to get expressions for the kinematic and dynamic MOI:
% \begin{align}
%     \mathcal{I}^{(1)}=\mathcal{I}_0\left[1-\frac{(\hbar\omega)^2}{a^2b}\right]^{-1/2}\ ,\\
%     \mathcal{I}^{(2)}=\mathcal{I}_0\left[1-\frac{(\hbar\omega)^2}{a^2b}\right]^{-3/2}\ ,
% \end{align}
% where $\mathcal{I}_0$ is defined as the \emph{band-head} moment of inertia: $\mathcal{I}_0=\hbar^2/(ab)$. In fact, such an approach of determining the MOIs of several odd-$A$ nuclei will consist in a future work of the same team. 

As mentioned, the sharp or abrupt irregularities of the MOI with respect to the increase in rotational frequency is known as backbending. The Figs. \ref{fig-hfNuclei-mois} - \ref{fig-ErYbnuclei-mois} show experimental MOI as function of the squared rotational frequency.
\begin{figure}
    \centering
    \includegraphics[scale=0.51]{Chapters/Figures/mois_Hf162-164.pdf}
    \includegraphics[scale=0.52]{Chapters/Figures/mois_Hf166.pdf}
    \caption{The moment of inertia as function of rotational frequencies for three even-even nuclei. \textbf{Left:} The MOI for $^{162,164}$Hf nuclei, with their corresponding rotational ground-state bands. \textbf{Right:} The MOI for the first two rotational bands in $^{166}$Hf. Experimental data for these two nuclei are taken from Ahmad et al. \cite{ahmad2021backbending}.}
    \label{fig-hfNuclei-mois}
\end{figure}
\begin{figure}
    \centering
    \includegraphics[scale=0.5]{Chapters/Figures/mois_Er158.pdf}
    \includegraphics[scale=0.5]{Chapters/Figures/mois_Yb160Er182.pdf}
    \caption{\textbf{Left:} The MOI for $^{158}$Er nucleus, with the ground-state band $K^\pi=0^+$ and the $\beta$-vibrational band with the same quantum numbers. Experimental data are taken from \cite{nica2017nuclear}. \textbf{Right:} The MOI for $^{160}$Yb compared with $^{162}$Er. Experimental data are taken from \cite{nica2021nuclear} ($A=158$) and \cite{reich2007nuclear} ($A=162$).}
    \label{fig-ErYbnuclei-mois}
\end{figure}

Indeed, by looking at the evolution of $\mathcal{I}$ from Figs. \ref{fig-hfNuclei-mois} - \ref{fig-ErYbnuclei-mois}, some sharp increases are noted. These are usually attributed to the centrifugal stretching in the rotational model \cite{davydov1960rotation}. Moreover, the constant increase in $\mathcal{I}$ is considered to occur due to the slow and constant quenching of the pairing correlations between nucleons \cite{mottelson1960effect}. The abrupt changes are explained as rapid phase transitions of the nucleonic matter \cite{krumlinde1974effect} exclusively due to the Coriolis Anti-Pairing effect. On the other hand, backbending can also be explained via the band-crossing of two intersecting bands having different moments of inertia. Certainly, the \emph{non-crossing} effect (see Appendix \ref{appendix:two-state}) makes the bands approach each other as much as the interaction strength allows \cite{ring2004nuclear}. 
%Nevertheless, it is clear that the theoretical investigations point out the reduction of pairing correlations, which will affect the MOI at low spin. Finally, the alignment of nucleons will eventually cause backbending \cite{Stephens1974BackbendingAR,Stephens1972CoriolisEI}. 

% The band crossing is shown in Fig. \ref{bands-crossing-backbending}. This phenomenon starts from the idea of two bands with different moments of inertia: if one analyzes the plot of the energies w.r.t. the angular momentum $I$, at one point (a \emph{critical value} for $I$), the second band would intersect the first one, crossing it. But such a thing is forbidden, so there will be a change in behavior for the parabolas (left side plot from Fig. \ref{bands-crossing-backbending}), causing a simultaneous increase in total angular momentum and decrease in rotational frequency. This change in behavior is superimposed with the change in structure of the bands themselves, making the sharp transition possible. For small interaction strength between the two bands, the backbending will be quite `strong', while for large values the transition region is very broad, making the backbending non-occurring.
% \begin{figure}
%     \centering
%     \includegraphics[scale=0.56]{Chapters/Figures/backbending_crossing_1.pdf}
%     \includegraphics[scale=0.56]{Chapters/Figures/backbending_crossing_2.pdf}
%     \caption{An illustrative example with two bands that are interacting with each other, leading to backbending. This sketch was inspired from the work of Ring et al. \cite{ring2004nuclear}.}
%     \label{bands-crossing-backbending}
% \end{figure}

Usually, the graphical representations of the MOI (kinematic or dynamic) as functions of $\omega$ (or, equivalently, $I$) are useful when comparing multiple spectra in the same nucleus. Based on their behavior w.r.t. rotational frequency or spin, one can determine if the bands belong to the same intrinsic structure.
%For example, after spin and parity assignments of each energy state within the collective spectra of the odd-$A$ nucleus $^{163}$Lu, it is of interest to see how many sequences have \emph{normal deformation}, which single-particle + core couplings are preferred, if the bands have enhanced (strong) deformation, and so on. In the following chapters, a detailed overview with the evolution of the kinematic and dynamic MOI as a function of rotational frequencies for odd-$A$ around $A\approx 160$ nuclei will be made, since it plays a major role in the study of highly deformed nuclei. 
%The experimental data concerning $^{163}$Lu isotope allows one to check the experimental rotational frequencies $\hbar\omega(I)$, the \emph{alignment} $I_x$, and the two types of MOI which were discussed. These calculations are shown in Figs.
% A discussion about the backbending effect needs to be made, since it strictly connects to the paragraph just discussed above. Actually, within the region $10\sim20$ units of angular momentum this strange effect is observed, especially in the ground-state band (i.e., the \emph{yrast} band). The yrast line corresponds to the set of lowest energies for a given set of spin states \cite{bohr1954rotational}. In the work of Mariscotti el al. \cite{Mariscotti1969PhenomenologicalAO}, it was shown that for the lowest two orders, the moments of inertia can be approximated as $\mathcal{I}=\mathcal{I}_0+\mathcal{I}_1\omega^2$.

\subsubsection{Electric Quadrupole Moment}
\label{intro-EM-chapter3}

An important indicator of nuclear deformation is the \emph{electric quadrupole moment} \cite{hamamoto2016interplay}, which measures the `departure' of the nuclear shape away from spherical symmetry (through elongation and asymmetry). The most general expression of the intrinsic quadrupole moment for a rotating nucleus is given in terms of its \emph{charge density distribution} \cite{casten2000nuclear}:
\begin{align}
    Q_0=\int(3z^2-r^2)\rho(r)_\text{charge}\text{d}v\ .
    \label{general-quadrupole-moment-Q0-charge}
\end{align}

This shows how the nuclear charge distribution inside the nucleus plays a pivotal role in determining the nuclear deformation. A relationship between the deformation parameter $\beta$ and the quadrupole moment itself can be approximated (in second order of $\beta$) as \cite{krane1991introductory}:
\begin{align}
    Q_0=\frac{3}{\sqrt{5\pi}}R^2Z\beta(1+0.16\beta)\ ,
    \label{quadrupole-moment-Q0}
\end{align}
where $R$ is given as $R=R_0A^{1/3}$ and $R_0=1.2\ \text{fm}$. For values of $\beta$ that correspond to \emph{strongly deformed} nuclei (i.e., $\beta\approx 0.3$), higher order terms are not necessary. According to the discussion concerning the nuclear shapes, $\beta$ describes the eccentricity of the deformed ellipsoid (albeit prolate or oblate). THe difference between a prolate and oblate ellipsoid is that for prolate (oblate) case there is an extension in one (two) direction and a squeezing in the other two (one). Depending on the value of $\beta$, the quadrupole moments for nuclei will be positive (indicating a prolate deformation) or negative (giving an oblate deformation). The experimental values of $\beta_2$ are shown in Fig. \ref{fig-quadrupole-beta-nuclides} for rare-earth nuclei. Note that for $\beta_2$ the subscript `2' will be used or dismissed freely throughout this work, but it signifies the same quantity. 
\begin{figure}
    \centering
    \includegraphics[width=0.99\textwidth]{Chapters/Figures/quadrupole_Deformation_rareEarth.pdf}
    \caption{Experimental values of the quadrupole deformation parameter $\beta_2$ as a function of the mass number $A$, for a few isotopes in the rare earth region. The values of $\beta_2$ were determined from the experimental transition probabilities $0^+\to 2^+$. The figure is reproduced from Ref. \cite{casten2000nuclear}.}
    \label{fig-quadrupole-beta-nuclides}
\end{figure}

In order to understand the behavior of $\beta_2$ shown in Fig. \ref{fig-quadrupole-beta-nuclides}, some shell-model considerations need to be taken into account. Firstly, when deformation kicks in, the individual $j$ orbits within a major shell are nearly empty, resulting in positive values for the quadrupole moments of the nucleons from these orbits. With increasing deformation, large and positive values $Q(\beta)$ are present. Furthermore, as the shells start to fill, contributions from individual $j$ orbits to the total quadrupole moment will accumulate, making its value to decrease, vanish, and eventually becoming negative near the shell closure (see inset 3 in Fig. 5.4 from \cite{casten2000nuclear}).

The \emph{observed} quadrupole moment (also known as spectroscopic or measured) can be obtained via a transformation to the laboratory frame applied to $Q_0$, meaning that the spectroscopic quadrupole moment has the result \cite{casten2000nuclear}:
\begin{align}
    Q=\left[\frac{3K^2-I(I+1)}{(I+1)(2I+3)}\right]Q_0\ ,
    \label{quadrupole-moment-spectro}
\end{align}
where the quantum number $K$ is the projection of $I$ onto the symmetry (deformation) axis. The dependence of $Q$ on both $K$ and $I$ emphasizes the fact that the observed shape of a rotating nucleus is not equivalent to the shape in the intrinsic frame of reference. Consider the case of a prolate nucleus rotating about an axis that is perpendicular to the symmetry axis. Then the \emph{averaged} density distribution of the nuclear matter will look more like an oblate shape (see Fig. \ref{fig-averaged-prolate-density}). As a result, when the intrinsic quadrupole moment is positive, the observed one will have a negative value: for the condition $I(I+1)>3K^2$ employed in Eq. \ref{quadrupole-moment-spectro}.
\begin{figure}
    \centering
    \includegraphics[scale=0.7]{Chapters/Figures/averaged_nuclearMatter_prolate.pdf}
    \caption{The average flattened (oblate) density distribution generated by the rotation of a prolate nucleus. As the `initial' prolate nucleus with its nuclear density (represented by the gray ellipse with dashed borders) being distributed along the deformed axis exhibits rotation, the rotated shape will generate an averaged oblate disk along the rotational axis (represented by the blue). This is why the observed quadrupole moment $Q$ will have a negative sign if $Q_0>0$.}
    \label{fig-averaged-prolate-density}
\end{figure}

Another important quantity used as a `test' for collectivity and deformation is the \emph{reduced electric quadrupole transition probability}, or $B(E2)$. This transition probability $B(E2)$ can be given in terms of the quadrupole moment introduced in Eq. \ref{general-quadrupole-moment-Q0-charge} through the following form \cite{bohr1998nuclear}:
\begin{align}
    B(E2;\ I_i\to I_f)=\frac{5}{16\pi}e^2Q_0^2\bra{I_iK20}\ket{I_fK}^2\ ,
    \label{reduced-E2-clebsch-gordan}
\end{align}
where the squared factor $\bra{I_iK20}\ket{I_fK}^2$ is the Clebsch-Gordan coefficient, also written as $\bra{I_iK20}\ket{I_fK}\equiv C^{I_i2I_f}_{K0K}$. In the case of $0^+\to 2^+$ transition, the reduced probability will be given by \cite{casten2000nuclear}:
\begin{align}
    B(E2;\ 0^+\to 2^+)=\frac{5}{16\pi}e^2Q_0^2\ .
    \label{reduced-E2-0Plus-2Plus-Transition}
\end{align}

The transition probability provided in Eq. \ref{reduced-E2-clebsch-gordan} is in fact a special case that can be applied to nuclei possessing \emph{axial symmetry}. This is because the involved transitions are not affected by a change of the projection $K$.
%A more general treatment of $B(E2)$ also requires prior knowledge of the so-called \emph{electric quadrupole transition operator}, which are not adopted here yet. This will be properly employed in a future chapter when discussing the transition probabilities in several odd-mass nuclei. 
From the quadratic dependence of the intrinsic quadrupole moment on $\beta$, high values (e.g. $\beta\approx 0.3$) will lead to $B(E2)$ that are one or even two orders of magnitude higher than those specific to nearly spherical nuclei $\beta_\text{sph}\approx 0.05$. Usually in nuclei, the valence nucleons are causing some core polarizations, which will affect the `final' structure of the electric quadrupole moment. For example, in an odd-$Z$ and even-$N$ nucleus, the total quadrupole moment $Q$ (also referred to as \emph{single-particle quadrupole moment} $Q_\text{s.p.}$) is given by the following expression \cite{bertulani2007nuclear}:
\begin{align}
    Q=-\langle r^2\rangle\frac{2j-1}{2j+2}\frac{e_\text{eff}}{e}\equiv Q_\text{s.p.}\ ,
    \label{single-particle-quadrupole-moment}
\end{align}
where the need for an \emph{effective charge} $e_\text{eff}$ is discussed in \cite{heyde1994nuclear}. The mean squared radius corresponds to the radial function of the particle within that $j$ orbital. 
%Stable nuclear deformation will require that the nuclear energy should be minimal. This can be achieved either if the overlap of the core with the valence particle is maximal, which for  a particle+core interaction will produce an oblate polarization, driving the nucleus to a final oblate deformed state. The opposite is true for the hole+core coupling (see discussion made in \cite{neugart2006nuclear}).
An interesting characteristic emerging from Eq. \ref{single-particle-quadrupole-moment} is that the odd-$A$ nuclei having $I=1/2$ will give a vanishing quadrupole moment $Q$. Although this situation concerns the measured (i.e., observed) moment, the odd-$A$ nucleus with $I=1/2$ does not necessarily imply that the system has no quadrupole moment. In fact, according to Eq. \ref{quadrupole-moment-spectro}, the spectroscopic moment is expressed in terms of the intrinsic component $Q_0$, so it is possible to have a vanishing $Q$ but $Q_0$ different from zero. The same circumstance is also met for the case of even-even nuclei.

The experimental data from Fig. \ref{experimental-Q-odd-nuclei} show the magnitude of $Q$ that would correspond to the value given by Eq. \ref{single-particle-quadrupole-moment}. One can see sharp increases with the nucleonic number for some nuclei (e.g., $^{167}$Er or $^{175}$Lu) but also very strong decreases to negative values (e.g., $^{123}$Sn). These alternations between positive (prolate) and negative (oblate) $Q$ values are also located near the magic numbers. Moreover, when the odd-particle is a neutron, the nucleus still exhibits a quadrupole moment that is different from zero, meaning that to some extent, the last nucleon is not the sole player regarding the quantitative behavior of $Q$.
\begin{figure}
    \centering
    \includegraphics[scale=0.55]{Chapters/Figures/Exp_quadrupoleMoments.pdf}
    \caption{The measured quadrupole moments $Q$ as per Eq. \ref{single-particle-quadrupole-moment}, given in units of $ZR^2$, as function of the odd proton and neutron numbers $Z$,$N$, respectively. Arrows signify the magic numbers. Experimental data are taken from Ref. \cite{bertulani2007nuclear}.}
    \label{experimental-Q-odd-nuclei}
\end{figure}

The experimental data for $Q$ (the spectroscopic quadrupole moment in units of \emph{barn} $[b]$) for the lowest $2^+$ states of even-$Z$ and even-$N$ nuclei are graphically represented in Fig. \ref{experimental-quadrupole-2Plus-states}, where both positive and negative values are observed. As already explained, a negative $Q$ value ($Q=-2b$ for nuclei with permanent deformation belonging to the mass range $150\leq A \leq 190$) will correspond to an intrinsic quadrupole moment $Q_0=7b$, which results in deformation parameter $\beta\approx 0.29$. Such values indicate substantial eccentricities of the nuclear matter.
\begin{figure}
    \centering
    \includegraphics[scale=0.25]{Chapters/Figures/2Plus_spectroscopicQ.pdf}
    \caption{The measured quadrupole moment for the first excited $2^+$ states in even-even nuclei. The expression of $Q$ was defined in Eq. \ref{quadrupole-moment-spectro}. The lines between data-points connect the isotope sequences. The figure was reproduced with the experimental data taken from \cite{krane1991introductory}.}
    \label{experimental-quadrupole-2Plus-states}
\end{figure}

Concluding this chapter, all the relevant models required for understanding the collective phenomena in nuclei were systematically described, starting with the shell model characteristics, then going further with the representation of the single-particle states in deformed nuclei, and finally reaching the collective model. A consistent portrayal for the nuclear rotations and vibrations was depicted, with the emergence of a general Hamiltonian that describes the nucleus in terms of the $\beta$ and $\gamma$ degrees of freedom. Importance of these two deformation parameters was also emphasized throughout the subsections. Lastly, the important quantities indicating collectiveness or deformations were introduced, as they will play a crucial role for testing any newly developed theory.