\chapter{Nuclear Models}

\section{Introduction}

In the following, it is worth to make a discussion about the nuclear models that are used by theoreticians in order to describe phenomena that are specific to rotating nuclei and high-spin regime. Since the focus of this work emerges from a \emph{class} of properties that usually apply to the high-spin region (but this does not necessarily also imply a high-energy region), it makes sense to give an insight in the tools that fit the best the underlying effects.

\subsection{Shell model}

The fact that an atomic nucleus can have a structure that behaves rather similarly as its \emph{parent} (i.e., the atom) in terms of changing the number of constituents, has been enforced by the experimental observations that were done across time. The sharp and discrete discontinuities of nuclear properties, such as the nucleon separation energy, point to the fact that nucleus can be explained through the existence of \emph{shells}. Some examples of observations which indicate this are:
\begin{itemize}
    \item When adding a nucleon to a nucleus, there are certain places where the \emph{binding energy} of the next nucleon becomes considerably smaller than the previous one. 
    \item Separation energies for both the protons and neutrons suffer drastic changes, having strong deviations from the predictions of the semi-empirical mass formula
\end{itemize}