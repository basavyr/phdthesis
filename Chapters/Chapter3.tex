\chapter{Nuclear Models}

\section{Introduction}

In the following, it is worth briefly discussing the nuclear models that are used by theoreticians in order to describe phenomena that are specific to nuclei. Since the focus of this work emerges from a \emph{class} of properties that usually apply to the high-spin region (but this does not necessarily also imply a high-energy region), it makes sense to give an insight in the tools that fit the best.

Nuclei are mostly described by two general types of models. The first kind is based on the assumption that nucleus interact strongly inside the nucleus and that their mean free path is small.