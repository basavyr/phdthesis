\documentclass[a4paper, 12pt, oneside]{Thesis}


\usepackage{float}
\usepackage{amsmath}
\usepackage{amssymb}
\usepackage{multirow}
\usepackage{xcolor}
\usepackage{lipsum}
\usepackage{siunitx}
\usepackage{physics}
\usepackage{graphicx}
% \graphicspath{Figures/}  % Location of the graphics files (set up for graphics to be in PDF format)
\usepackage{mathrsfs}
\usepackage{verbatim}  % Needed for the "comment" environment to make LaTeX comments
\usepackage{vector}  % Allows "\bvec{}" and "\buvec{}" for "blackboard" style bold vectors in maths
\usepackage[square, numbers, comma, sort&compress]{natbib}

\hypersetup{urlcolor=blue, colorlinks=true}  % Color hyperlinks in blue, but this can be distracting if there are many links.

%% --------------------------------- tikz ---------------------------------------
\usepackage{pgfplots}
\usepackage{tikz}
\pgfplotsset{compat=1.16}
\usepgfplotslibrary{fillbetween}
\usetikzlibrary{cd,arrows,calc,fit,positioning,matrix,backgrounds}
\usetikzlibrary{shapes.geometric,shapes.multipart,decorations.pathreplacing}
\tikzstyle{startstop} = [rectangle, rounded corners, minimum width=3cm, minimum height=1cm, text centered, draw=black, fill=white]
\tikzstyle{arrow1}=[ultra thick,->,>=stealth]
\tikzstyle{arrow}=[thick,->,>=stealth]
\tikzset{
    shift left/.style ={commutative diagrams/shift left={#1}},
    shift right/.style={commutative diagrams/shift right={#1}}
}

% define the gaussian function
\newcommand\gauss[2]{1/(#2*sqrt(2*pi))*exp(-((x-#1)^2)/(2*#2^2))} % Gauss function, parameters mu and sigma

%define proper imaginary unit within math-mode
\newcommand{\iu}{\mathrm{i}\mkern1mu}

% define the thesis title
\newcommand\mytitle{A Systematic Description of the Wobbling Motion in Odd-Mass Nuclei Within a Semi-Classical Formalism}


\bibliographystyle{unsrt}
%%-------------------------------------------------------MAIN DOCUMENT---------------------------------------------------%%
% ------------------------------------------------------------------------------------------------------------------------
\begin{document}

\frontmatter
\begin{titlepage}
    \begin{center}
        \textsc{ \LARGE{University of Bucharest \\}}
        \textsc{ \LARGE{Faculty of Physics\\ }}
        \vspace{40mm}
        \textup{\LARGE{\mytitle}}
    \end{center}
    
    \vspace{25mm}
    
    \begin{minipage}[t]{0.47\textwidth}
        \textnormal{\large{\bf Scientific Coordinator:\\}}
        {\large Prof. Dr. Em. A. A. Raduta}
    \end{minipage}\hfill\begin{minipage}[t]{0.47\textwidth}\raggedleft
        \textnormal{\large{\bf PhD Student:\\}}
        {\large Robert Poenaru}
    \end{minipage}
    
    \vspace{40mm}
    
    \centering{\large{A thesis submitted for the degree of \\ \emph{Doctor of Philosophy}}}
    
    \vspace{35mm}
    
    \centering{\large{\today}}
\end{titlepage}

\title{\mytitle}

% ------------ OLD TITLE -----------------
% ----------------------------------------
% \title{A Systematic Semi-Classical Description of the Wobbling Motion in Nuclei}
% \authors  {\texorpdfstring
%             {\href{robert.poenaru@drd.unibuc.ro}{Robert Poenaru}}
%             {Robert Poenaru}
%             }
% \addresses  {\groupname\\\deptname\\\univname}  % Do not change this here, instead these must be set in the "Thesis.cls" file, please look through it instead
% \date       {\today}
% \subject    {}
% \keywords   {}
% \maketitle
% \setstretch{1.3}  % It is better to have smaller font and larger line spacing than the other way round
% % Define the page headers using the FancyHdr package and set up for one-sided printing
% \fancyhead{}  % Clears all page headers and footers
% \rhead{\thepage}  % Sets the right side header to show the page number
% \lhead{}  % Clears the left side page header
% \pagestyle{fancy}  % Finally, use the "fancy" page style to implement the FancyHdr headers
% ----------------------------------------

% The Abstract Page
% \addtotoc{Abstract}  % Add the "Abstract" page entry to the Contents
% \abstract{
% \addtocontents{toc}{\vspace{1em}}  % Add a gap in the Contents, for aesthetics
% \textbf{Keywords}: \textit{nuclear shape, nuclear deformation, collective parameters, triaxiality, wobbling}.
% sheeesh...
% }
% \clearpage

\setstretch{1.3}  % Reset the line-spacing to 1.3 for body text (if it has changed)

% The Acknowledgements page, for thanking everyone
\acknowledgements{
\addtocontents{toc}{\vspace{1em}}  % Add a gap in the Contents, for aesthetics
It has been quite a journey. Firstly, I would like to express my gratitude to Prof. A. A. Raduta, my coordinator and also mentor. His patience, work ethic, and constant support gave me the necessary motivation to keep going and also improve in every aspect specific to academic research. I would also like to give big thanks to my family for the tremendous emotional help and life advices offered when needed.

These four years showed me that many results can be achieved through hard work and dedication, but also much more can lie ahead of us in terms of new projects, new discoveries, and innovative theories.

Huge thanks and eternal appreciation to my \emph{medical team}, consisting of Dr. Demian, Dr. Mester, Dr. Radu, Dr. Zarma, and Aurica, without whom I could not be here in a good mental and physical state. Finishing my studies and implicitly this thesis was only possible through their careful attention. % Same amount of respect and gratitude are also expressed to Aurica for explaining to me and my family that time is always the crucial factor when it comes to health issues.

Finally, I would like to believe that by reading this thesis, our work and achieved results will be appreciated.
}
\clearpage

\tableofcontents
\pagestyle{fancy}  % Return the page headers back to the "fancy" style
%% ----------------------------------------------------------------
\mainmatter	  % Begin normal, numeric (1,2,3...) page numbering

% chapters
\chapter{Introduction}

Ground-state nuclear shapes with spherical or axial symmetry are predominant across the char of nuclides. Near closed shells, the deformation is sufficient that models based on spherical symmetries can be used to describe nuclear properties (e.g., energies, quadrupole moments, and so on). Besides the spherical and axially-symmetric shapes, the existence of triaxial nuclear deformation was theoretically predicted a long time ago \cite{bohr1998nuclear}. The rigid triaxiality of nuclei is defined by the asymmetry parameter $\gamma$, giving rise to a unique behavior concerning the system dynamics. Lately, triaxial nuclei drew a lot of attention within the nuclear physics community, since the description of nuclear properties represents a real challenge from an experimental and also a theoretical standpoint. A great progress for towards experimental setups being able to perform measurements of high-spin has only been possible after the 2000s. It is worth mentioning that some experiments regarding alpha-$\alpha$ particle reactions induced in heavy nuclei in the early 1960s (e.g., \cite{morinaga1963gamma}) helped to produce decent amount of data for the rotational in the high-spin region ($\geq 20 \hbar$).

The physics of \emph{high-spin} states has been studied from the early 1950s, with the major breakthrough on the theoretical side made by Bohr and Mottelson \cite{bohr1998nuclear}. The nuclear rotation was described in terms of the rotational degrees of freedom associated with other nuclear degrees of freedom such as particle-vibration, quadrupole-quadrupole, parity, and so on. The spherical shell-model only describes nuclei near the closed shells. On the other side, for the nuclei that lie far from closed shells, a deformed potential must be employed. In the case of even-even nuclei, unique band structures resulting from the vibrations and rotations of the nuclear surface appear in the energy range 0-2 MeV. 

Even though triaxiality has an elusive character, two phenomena, i.e., wobbling motion and chiral bands are uniquely attributed to triaxial shapes. Consequently, these two were intensively searched by using the advanced techniques. In this work, the focus is given exclusively on the first effect, although it is worth specifying that current team provides a unified description of both phenomena in their most recent work \cite{raduta2022simultaneous}, which stands out as the first ever theoretical treatment describing wobbling and chirality on an equal footing.

Quantized wobbling modes were firstly discovered experimentally in $^{163}$Lu, where the presence of the odd-proton $i_{13/2}$ grants the apparition of multiple rotational bands to appear around the yrast line. This experiment consisted of populating high-spin states with a $^{29}$Si beam interacting with a thin target of $^{139}$La. Later on, other odd-$A$ nuclei were discovered as exhibiting wobbling excitations, and all the experimental findings will be mentioned in the following chapters. The nuclei having a triaxial shape can rotate about any of the principal axes, causing rich collective spectra to emerge. The family of rotational bands is described in terms of vibrational excitations. As a classical analog, nuclear wobbling motion is corresponds to the spinning motion of an asymmetric top. The experimental fingerprints for wobbling motion indicate that the energy spectrum behaves as $\sim I(I+1)$ with respect to the angular momentum, there is a clear dominance of electric transitions over the magnetic ones, and the nuclei have large quadrupole moments. All these quantities will be exploited in detail in the coming chapters.

\section{Aim}
 
The objective of this research is two-fold. On one side, the theoretical description of wobbling motion is treated in detail, starting from the required nuclear models specific to deformed nuclei, and reaching a set of key properties of the phenomenon. Differences between wobbling that occurs in odd- but also even-mass nuclei are depicted, since each situation manifests in remarkable ways. Once the general formalism behind this effect is presented, an inventory of all the currently identified nuclei will be made, providing clear explanations for the band structure and also the relevant parameters describing deformation. The experimental measurements will be addressed for each nucleus in particular, concluding with a chart that contains these nuclides. The complete overview of all existing wobbling nuclei is in fact one of the unique features of the research and a first within literature.

The second objective of this present work is to describe the wobbling mechanism by means of a novel semi-classical approach. Indeed, the important quantities related to collective excitations are properly reproduced by the, i.e., excitation energies, quadrupole moments, transition probabilities and many more. This model starts from an initial quantal Hamiltonian that is dequantized through a variational method. A set of classical equations of motion that describe triaxial nuclei are obtained, and a classical energy function is granted by the approach. This function is a remarking feature of the developed framework, as this fully analytical expression (containing only classical variables that were obtained via the dequantization) will provide an insight into multiple analyses: energy spectrum, stability of the wobbling motion, critical regions, phase transitions and even possible changes in the wobbling regime. 

Another remarking feature of the current model is the geometrical interpretation of the rotational motion specific to triaxial nuclei, which is described in both a two- and also three-dimensional space. Also unique to this research are the introduction of two concepts that are related to the band structures of odd-mass nuclei, namely the Signature Partner Bands and Parity Partner Bands. These two characteristics, which will be detailed in dedicated chapters, can be considered as hallmarks of the theory. Lastly, it is worth mentioning that throughout this entire work, a consistent graphical representation is adopted, giving workflow diagrams, schematics, and charts, with the purpose of providing clearer explanations (where possible) of the underlying mechanisms.

\section{Motivation}

Over the years, many theoretical interpretations were suggested for the description of the wobbling motion and its main features. The Triaxial Rotor Model \cite{bohr1998nuclear,davydov1958rotational} and the Particle Rotor Model \cite{hamamoto2002wobbling} are quantal models that are solved exactly in the laboratory frame. Other investigations are based on mean-field theories, such as the Random Phase Approximation \cite{shimizu1995nuclear}, the Angular Momentum Projection \cite{oi2000wobbling}, and also the Collective Hamiltonian \cite{chen2014collective} were adopted.

In this research, one aimed at a semi-classical description of the wobbling phenomenon due to its advantage in keeping a close contact with the `classical picture' of the system dynamics. Certainly, working with a set of classical equations of motion is much easier than having to deal with quantum mechanical objects that to not have a clear one-to-one correspondence with classical mechanics. It will be shown that the rotational motion of a triaxial nucleus can be approximated quite well with that of a rigid rotator, meaning that the energy spectrum could be accurately described through quantities that have concise physical meaning (i.e., moments  of inertia, angular momentum, angular frequency). Moreover, the analytical spectrum that is achieved by solving the equations for an odd-mass nucleus is indeed remarkable, since it will be described by separated degrees of freedom associated to an even-even core and a valence nucleon that interacts with core. The variational method that is employed proves to be an efficient tool in accurately depict the energy spectra and transition probabilities of several odd-$A$ nuclei within the $A=160$ mass region.

The lack of studies of geometrical treatments for the wobbling motion encouraged the team to pursue such an analysis. A two-dimensional evaluation shows if regions of stability exists, meaning that one can identify the energies at which the total angular momentum exhibits stable precessional motion. Taking the formalism a step further, the wobbling motion is explored within the space generated by the three components of the total angular momentum. The two constants of motion, i.e., the total energy and the total spin are graphically represented in the same picture, and their intersection signify the allowed trajectories that the angular momentum precesses around. Each trajectory corresponds to a particular set of spin and energies, meaning that the entire spectrum of a wobbling nucleus can be interpreted. Using the classical view of the angular momentum and the total energy for a triaxial ellipsoid represents the onset of a fully unified description for nuclear deformation. Moreover, this phenomenological and semi-classical model gives results that are on par with fully microscopic or quantal descriptions (much more complex), making it a successful tool in describing collective phenomena.

This thesis is structured in the following way. Starting of, in Chapter \ref{chapter-2} the nuclear surface is described using the expansion in terms of collective coordinates and spherical harmonics. The relevant multipole modes are presented and a focus is given on the quadrupole $\lambda=2$ excitation mode. This is the main vibrational mode that causes deformation of the nucleonic matter. Within the approximation of the nuclear surface, the deformation parameters $\beta_2$ and $\gamma$ are introduced, which give a direct insight with respect to the degree of elongation and asymmetry between the ellipsoid axes.  % Introduction
\chapter{Deformed Nuclei}

\section{Nuclear deformation}

Most of the nuclei across the nuclide chart are spherical or symmetric in their ground state. Moreover, for the axially symmetric nuclei (i.e, either \emph{oblate} or \emph{prolate}), there is a prolate over oblate dominance.
% \begin{figure}[ht]
%     \centering
%     \includegraphics[scale=0.3]{Chapters/Figures/nuclear_shapes.png}
%     \caption{Nuclear Shapes.}
%     \label{nuclear_shapes}
% \end{figure}

The spherical shell model only describes nuclei near the closed shells. On the other side, for the nuclei that lie far from closed shells, a deformed potential must be employed. 
\par In the case of even-even nuclei, unique band structures resulting from the vibrations and rotations of the nuclear surface (as proposed by Bohr and Mottelson \cite{bohr1998nuclear} in the \emph{Geometric Collective Model} - GCM) appear in the energy range 0-2 MeV.

Within the GCM, the nucleus is described as a classical charged liquid drop. For the low-lying energy spectrum, usually, the compression of nuclear matter and the nuclear skin thickness are neglected. This results in the final picture of a liquid drop with a constant nuclear density and a sharp surface \cite{greiner1996nuclear}.

\subsection{Collective coordinates}

The nuclear surface can be described via an expansion of the spherical harmonic functions with some time-dependent parameters as \emph{expansion coefficients}. The expression of the nuclear shape is shown below \cite{greiner1996nuclear}:
\begin{align}
    R(\theta,\varphi,t)=R_0\left(1+\sum_{\lambda=0}^\infty\sum_{-\lambda}^\lambda\alpha_{\lambda\mu}(t)Y_\lambda^\nu(\theta,\varphi)\right)\ .
    \label{nuclear-shape}
\end{align}

In \ref{nuclear-shape}, $R$ denotes the nuclear radius as a function of the spherical coordinates $\theta,\varphi)$ expressing the direction, and the time $t$, while $R_0$ is the radius of the spherical nucleus when all the expansion coefficients vanish. It is worth mentioning that the expansion coefficients $\alpha_{\lambda\mu}$ act as \emph{collective coordinates} since the time-dependent amplitudes describe the vibrations of the nuclear surface.

\subsection{Nuclear radius under rotation}

To get a grasp at the physical meaning behind the deformation parameters that are used to describe the nuclear surface, it is instructive to see what happens when the system undergoes a rotation transformation.

The function $R(\theta,\varphi)$ describes the original (non-rotated) nuclear shape. Rotating the system will result in the change of the angular coordinates $(\theta,\varphi)$ to $(\theta',\varphi')$, which will correspond to a new function $R'(\theta',\varphi')$. Moreover, both nuclear surfaces (i.e., the non-rotated and the rotated one) must hold the equality:
\begin{align}
    R'(\theta',\varphi')=R(\theta,\varphi)
\end{align}

The rotational invariance of $R$ employs that $R'(\theta,\varphi)$ must have the same functional form, but the expansion coefficients $\alpha_{\lambda\mu}$ must be rotated, meaning:
\begin{align}
    \sum_{\lambda\mu}\alpha_{\lambda\mu}'Y'_{\lambda\mu}(
        \theta,\varphi)
\end{align} % Chapter 2
\chapter{Nuclear Models}

\section{Introduction}

In the following, it is worth to make a discussion about the nuclear models that are used by theoreticians in order to describe phenomena that are specific to rotating nuclei and high-spin regime. Since the focus of this work emerges from a \emph{class} of properties that usually apply to the high-spin region (but this does not necessarily also imply a high-energy region), it makes sense to give an insight in the tools that fit the best the underlying effects.

\section{Shell model}

The fact that an atomic nucleus can have a structure that behaves rather similarly as its \emph{parent} (i.e., the atom) in terms of changing the number of constituents, has been enforced by the experimental observations that were done across time. The sharp and discrete discontinuities of nuclear properties, such as the nucleon separation energy, point to the fact that nucleus can be explained through the existence of \emph{shells}. Some examples of observations which indicate this are:
\begin{itemize}
    \item When adding a nucleon to a nucleus, there are certain places where the \emph{binding energy} of the next nucleon becomes considerably smaller than the previous one. 
    \item Separation energies for both the protons and neutrons suffer drastic changes, having strong deviations from the predictions of the semi-empirical mass formula \cite{weizsacker1935theorie}, the discontinuities being represented by major shell closures (complete filling) \cite{krane1991introductory}.
    \item The neutron absorption cross-section has a substantial decrease in value at the neutron magic numbers
    \item Great abundance of nuclides where $Z$ and $N$ are magic numbers.
\end{itemize}

The sudden discontinuities occur at specific values of the proton $Z$ and neutron $N$ numbers: these are called \emph{magic numbers}. Currently, these magic numbers correspond to $Z$ or $N=2,8,20,28,50,82,126$, and they represent the so-called major shells. There are also two \emph{weakly magic numbers}: 40 and 64.

One can examine the values for the first excited states $2^+$ that are shown in Figs. \ref{e2plus_proton}, \ref{e2plus_neutron}. Indeed, these values show some peaks, each peak corresponding to a particular magic number.

\begin{figure}
    \centering
    \includegraphics[scale=0.33]{Chapters/Figures/E2plus_proton.pdf}
    \caption{The first excited energy states $2^+$ of nuclei with even $Z$ and $N$ graphically represented with respect to the proton number. Each line represents a set of isotopes. Figure taken from Ref. \cite{matta2017exotic}.}
    \label{e2plus_proton}
\end{figure}

\begin{figure}
    \centering
    \includegraphics[scale=0.33]{Chapters/Figures/E2plus_neutron.pdf}
    \caption{The first excited energy states $2^+$ of nuclei with even $Z$ and $N$ graphically represented with respect to the neutron number. Each line represents a set of isotopes. Figure taken from Ref. \cite{matta2017exotic}.}
    \label{e2plus_neutron}
\end{figure}

The shell model starts from the basic assumption that the nucleus is a \emph{mean-field potential}, that is a potential for which the motion of a single nucleon is caused by all the other nucleons (the nucleon is moving inside an average potential generated by all the other constituents of the nucleus). Of course, all the nucleons that are under the influence of such a mean field potential occupy the energy levels which correspond to a series of (sub)shells that agree with the Paul exclusion principle. Having a general expression for the potential that properly reproduces all the magic numbers (and the observed nuclear properties) is the main goal.

Since the model starts from the concept of independent (non-interacting) particle motion within an average potential, finding each energy will be equivalent of solving the Schrödinger equation:
\begin{align}
    -\frac{\hbar^2}{2m}\nabla ^2\psi_i(r)+V(r)\psi_i(r)=e_i\psi_i(r)\, 
    \label{schrodinger-single-particle-eq}
\end{align}
where $e_i$ represents the energy (eigenvalue) and $\psi_i$ represents the wave-function (eigenstates), while $V(r)$ is the nuclear potential whose expression must be evaluated.

The choice of $V(r)$ will be dictated by the reproduction of various experimental data (such as nuclear saturation, scattering, nuclear reactions, and so on). For the motion of an independent particle, an obvious first attempt would be the \emph{simple harmonic oscillator} (SHO), which has the known expression:
\begin{align}
    V(r)=\frac{1}{2}m(\omega_i r)^2\ ,
    \label{harmonic-potential}
\end{align}
with $\omega_i$ as the frequency of the basic harmonic-like motion of the particle in the nucleus. With Eq. \ref{harmonic-potential}, the motion of the nucleon has a straightforward expression:
\begin{align}
    \frac{\hbar^2}{2m}\nabla^2\psi_i(r)+\frac{1}{2}m(\omega r)^2\psi_i(r)=e_i\psi_i(r)\ .
\end{align}

This Schrödinger equation has its energy eigenvalues under to form:
\begin{align}
    e_N=\left(N+\frac{3}{2}\right)\hbar\omega\ ,
\end{align}
where $N$ is the number of oscillator quanta which describes each major shell (also called the \emph{principal quantum number}). One should keep in mind that such an expression is typical for a three-dimensional and isotropic harmonic oscillator. The principal quantum number $N$ is furthermore defined as:
\begin{align}
    N=2(n-1)+l\ ,
\end{align}
with $n$ and $l$ being the \emph{radial} quantum number and \emph{orbital angular momentum} quantum number, respectively, taking values $n=1,2,3,\dots$ and $l=0,1,2,\dots,n-1$. In this first approximation, all the levels with the same principal quantum number $N$ are \emph{degenerate}, with a maximal degeneracy given by $2(2l+1)$. However, by using only the SHO term as the expression of $V(r)$, only the first three magic numbers are reproduced, meaning that some additional term(s) might be needed in order to consistently obtain the series of magic numbers.

A next step is to use the fact that the experimentally observed short range of the strong nuclear force: the steepness of the SHO can be corrected with an \emph{attractive} term proportional to $l$-squared. This acts as a centrifugal term which provides an angular momentum barrier, lifting the degeneracy between the levels with the same principal quantum number $N$ and different values for the orbital angular momentum $l$. This SHO+$l^2$ step is still not enough though. The last step is to add a so-called \emph{spin-orbit} coupling term of the form $\vec{l}\cdot\vec{s}$. 
% This last term is enough to reproduce all the magic numbers and the experimentally measured quantities that are relevant to the shell model itself.
This term comes from the consideration that the nucleon-nucleon interaction has a spin dependence, and the potential depends on the intrinsic spin $s$ ($\vec{s}$) and the orbital angular momentum $l$ ($\vec{l}$) of a nucleon. Since $\vec{j}=\vec{l}+\vec{s}$, two possible states emerge from a single value of $l$ (depending on wether $\vec{s}$ is parallel or anti-parallel to $\vec{l}$). The final expression of the terms SHO+$\vec{l}^2$+$\vec{l}\cdot\vec{s}$ will consist in the \emph{Modified Harmonic Oscillator} (HMO).
\begin{align}
    V(r)=\frac{1}{2}(\omega r)^2+B\ \vec{l}^2+A\ \vec{l}\cdot\vec{s}\ .
    \label{modified-harmonic-oscillator-eq}
\end{align}

For the sake of simplicity, the centrifugal term will be denoted within formulas without the vector symbol. Since the intrinsic spin of a nucleon is $s=1/2$, for a given value of $l$, there can be two values for the \emph{total angular momentum} (a.m.) $j=l\pm1/2$: one for each spin orientation with respect to the direction of the orbital a.m. Moreover, for each value of $l=0,1,2,3,4,\dots$, there is a similar notation $l=s,p,d,f,g,\dots$, respectively. Regarding the spectroscopic notation, usually, the value of $j$ is considered as a subscript; $nl_j$ (for example $1p_{1/2}$ and $1p_{3/2}$). What it is worth mentioning is that for high enough shells, there can be splittings between $j+1/2$ and $j-1/2$ that are large enough to lower the $j+1/2$ state from one oscillator shell $n$ to one located below $n-1$. These types of levels are called \emph{intruder states} and they have opposite parity $\pi=(-1)^l$ with respect to the shell that these levels will occupy.

Going back to the expression of the $\vec{l}\cdot\vec{s}$ term from Eq. \ref{modified-harmonic-oscillator-eq} and denoting it with $V_{ls}(r)$, it is shown by Casten \cite{casten2000nuclear} that its contribution to the total potential can be regarded as a surface effect. Due to this, its form can be expressed as a function that depends on the radial coordinate as such \cite{casten2000nuclear}:
\begin{align}
    V_{ls}(r)=-a_{ls}\frac{\partial V(r)}{\partial r}\vec{l}\cdot\vec{s}\ ,
\end{align}
where $V(r)$ is the expression for a central potential and $a_{ls}$ is a strength constant.
% Indeed, in the work of Casten \cite{casten2000nuclear}, it is stated that if in the nucleus, the spin-orbit forces were large enough, then there should be an overall preference for nucleons with spins aligned parallel their orbital a.m. other than the opposite alignment, making the nucleons with anti-parallel spins to not be surrounded by an equal number of nucleons with all spin orientations.

Now that an expression for the nuclear potential that is able to reproduce all the magic numbers has been formulated, it is also possible to formulate the total energy of a single-particle within the average potential that is generated by all the other nucleons within the nucleus. Thus, the Hamiltonian of this simple system (the modified oscillator) can be formulated as such:
\begin{align}
    H&=-\frac{\hbar^2}{2m}\nabla^2+V_\text{SHO}+(l^2)_\text{term}+(\vec{l}\cdot\vec{s})_\text{term}=-\frac{\hbar^2}{2m}\nabla^2+V_\text{MHO} , \nonumber\\
    H&=-\frac{\hbar^2}{2m}\nabla^2+\frac{1}{2}m(\omega r)^2+Bl^2+A\vec{l}\cdot\vec{s}\ .
\end{align}

The evolution from each term in the shell-model potential (that is the first approximation as a SHO, then SHO+$l^2$, and finally SHO+$l^2+\vec{l}\cdot\vec{s}$ or modified oscillator potential) is illustrated in Fig. \ref{energy-levels-mho}, where it can be seen how each extra term introduces a new degeneracy within the energy states, with the complete reproduction of the magic numbers in the third column. Moreover, the \emph{intruder} levels can be observed, where levels with $j=l+1/2$ from a particular $n$ are so low, that they lie below an $n-1$ adjacent level.
\begin{figure}
    \centering
    \includegraphics[scale=0.12]{Chapters/Figures/SM_level_scheme.png}
    \caption{The energy levels obtained via calculation of the shell model potential using the simple oscillator (SHO), the SHO amended with a centrifugal term $l^2$, and finally the modified oscillator (MHO) that contains a spin-orbit term. The `correct' magic numbers are the ones in the right-most column. Figure is adapted from Refs. \cite{matta2017exotic},\cite{krane1991introductory}.}
    \label{energy-levels-mho}
\end{figure}

Another, more realistic potential that can be used in order to reproduce the specific shell model calculation is the so-called Woods-Saxon potential. Because of the short-range character of the strong nuclear force, it is safe to assume that this potential should behave in the same manner as the density distribution of the nucleons. Since for medium and heavy nuclei, the Fermi-like functions (distributions) are the ones that best fit the experimentally measured data, this potential should have the following form \cite{woods1954diffuse}:
\begin{align}
    V_\text{ws}(r)=-\frac{V_0}{1+e^{\frac{r-R_0}{a}}}\ .
    \label{woods-saxon-potential}
\end{align}
This mean-field potential contains the terms $V_0$ that represents the depth of the potential ($\approx 50$ MeV, in order to reproduce the experimental separation energies for the nucleons), the surface thickness $a$ (also called the diffuseness parameter, giving information about how fast the potential drops to zero) with a value of approximately $0.5$ fm, while $R_0$ is the nuclear radius with $R_0=r_0A^{1/3}$ and $r_0=1.2$ fm. The nature of this potential is of \emph{central type} and, unfortunately, Eq. \ref{woods-saxon-potential} in its pure form is not enough the reproduce the higher magic numbers. As such, the addition of a spin-orbit term, similarly as in the case of MHO potential, is required \cite{martin2017particle}: 
\begin{align}
    V_\text{total}=V_\text{ws}^\text{central}+V_{ls}(r)\vec{l}\cdot\vec{s}\ .
    \label{woods-saxon-so-potential}
\end{align}
The only good quantum numbers in the case of the WS potential are the total a.m. $j$ and the parity $\pi=(-1)^l$.
The expectation value of the spin-orbit term $\vec{l}\cdot\vec{s}$ can be given as:
\begin{align}
    \langle ls \rangle=\hbar^2\begin{cases}
        \frac{l}{2} \quad &\text{for} j=l+\frac{1}{2}\\
        -\frac{l+1}{2} &\text{for} j=l-\frac{1}{2}\ \\
   \end{cases}\ .
\end{align}
and the spacing between two levels can be furthermore expressed as \cite{martin2017particle}:
\begin{align}
    \Delta E_{ls}=\frac{2l+1}{2}\hbar^2\langle V_{ls}\rangle\ .
\end{align}
The experimental evidence points to the fact that $V_{ls}(r)$ is negative, meaning that states with $j=l-1/2$ are shifted higher than $j=l+1/2$. Some characteristics of the WS potential are the following:
\begin{enumerate}
    \item It increases with the increase of $R$, meaning that it has an \emph{attractive nature}
    \item It flattens out for large enough $A$ in the center of the nucleus
    \item It rapidly goes to zero as $R$ increases (given by the diffuseness parameter), indicating its short-range nature
    \item When $R=R_0$ (that is for the nucleons near the surface), a large force towards the center of the nucleus is experienced by the these nucleons.
\end{enumerate}
\begin{figure}
    \centering
    \includegraphics[scale=0.2]{Chapters/Figures/ws_potential_plot.png}
    \caption{The shape of the Woods-Saxon potential, defined by Eq. \ref{woods-saxon-potential}. The parameters are arbitrarily chosen as: $V_0=50$ MeV, $R=5.57$ fm, and $a=0.5$ fm.}
    \label{woods-saxon-plot}
\end{figure}

Fig. \ref{woods-saxon-plot} shows the shape of a typical Woods-Saxon potential. Aiming at a final Hamiltonian which describes the motion of the nucleon within the mean-field potential, the formula can be readily obtained:
\begin{align}
    H&=-\frac{\hbar^2}{2m}\nabla^2+V_\text{ws}^\text{central}+(\vec{l}\cdot\vec{s})_\text{term}\ ,\\
    H&=-\frac{\hbar^2}{2m}\nabla^2-\frac{V_0}{1+e^{\frac{r-R_0}{a}}}+A\vec{l}\cdot\vec{s}\ .
\end{align}
In addition to the shape of the Woods-Saxon potential, a comparison between it, a SHO,and the square-well-like potential is made in Fig \ref{shell-model-functional-potentials}.
\begin{figure}
    \centering
    \includegraphics[scale=0.2]{Chapters/Figures/functional-potentials-shell-model.png}
    \caption{A schematic representation with the three kind of potentials used to describe the shell model: harmonic oscillator, Woods-Saxon, and for completeness, the square-well.}
    \label{shell-model-functional-potentials}
\end{figure}

The difference between the pure form of the Woods-Saxon potential and the total potential, where the spin-orbit contribution is amended, can be seen in Fig. \ref{woods-saxon-energy-levels}.
\begin{figure}
    \centering
    \includegraphics[scale=0.46]{Chapters/Figures/energy_levels_WS.png}
    \caption{The left side represents the energy levels calculated for the Woods-Saxon potential given by Eq. \ref{woods-saxon-potential}, and the right side shows the single-particle energies with the spin-orbit correction added, as in Eq. \ref{woods-saxon-so-potential}. Figure adapted from Ref. \cite{lewis2019lifetime}.}
    \label{woods-saxon-energy-levels}
\end{figure}

So far, the general discussion concerning the nuclear models was for the case where each nucleon is treated as an \emph{independent} particle moving in an average potential (mean-field potential) which represents an \emph{effective} interaction of all the other nucleons with the nucleon under study. However, such an assumption is not accurate enough (especially for the nuclei that lie far away from the closed shells), and this problem should be treated within a \emph{many-body} approach: considering the mutual interaction between the nucleons. These interactions are also called \emph{residual interactions} \cite{casten2000nuclear,bertulani2007nuclear}. With these residual interactions, an accurate depiction of the nucleus might be achieved, and in the following sections, the \emph{Deformed Shell Model} will be employed, reaching to the famous Nilsson model/theory of describing the nucleus.

\section{Deformed Shell Model}

In the previous section, the discussion was focused on an approximate description of the (independent) motion of a nucleon within an average potential. That potential is generated by the interaction of that nucleon with all the remaining nucleons within the compounding nucleus. Indeed, for spherical nuclei, the model described previously works really well and it is a successful tool in reproducing and predicting the properties of nuclear states, especially if the excited states have nucleonic configurations that are dominated by a single nucleon or a very small number of `extra' nucleons.

For nuclei that are even in both the proton number and the neutron number (i.e., even-even nuclei), the nuclear ground-state has a spin and parity that are properly reproduced by the \emph{spherical shell model}: $I^\pi=0^+$. In a nucleus with complete shells, the \emph{net spin} must be zero while for the nucleus with one nucleon missing from a complete shell closure (a hole), that ground-state spin should equal to the total a.m. $j$ value of the orbital which that particular hole is occupying. Moreover, the parity of the ground-state for a given nucleus is determined by the orbital a.m. value $l$:
\begin{align}\pi=(-)^l\to
    \begin{cases}
        +1 &\text{for even-}l\ \text{levels}\\
        -1 &\text{for odd-}l\ \text{levels}\\
   \end{cases}\ .
\end{align}
For odd-odd nuclei, one can find the ground-state (g.s.) spin and parity via the coupling of the spin and parity of the last two valence nucleons \cite{krane1991introductory,bertulani2007nuclear}. The coupling rules that are allowed in the odd-odd nucleus were determined more than 50 years ago by Gallagher et al. \cite{gallagher1958coupling}:
\begin{align}
    I&=j_p+j_n\ \text{if}\ j_p=l_p\pm\frac{1}{2}\ \text{and}\ j_n=l_n\pm\frac{1}{2}\ ,\\
    I&=|j_p-j_n|\ \text{if}\ j_p=l_p\pm\frac{1}{2}\ \text{and}\ j_n=l_n\mp\frac{1}{2}\ .
\end{align}

\subsection{Deformed Shell Model - Nilsson Model}

The idea that some nuclei are deformed in their ground-state was enforced experimentally a long time ago by measuring quantities such as density distributions, nuclear quadrupole moments \cite{casten2000nuclear} and so on. The non-spherical shapes are given by the existence of nucleonic configurations that lie away from the major shell closure. In Chapter \ref{chapter-2} the description of the nuclear shapes was treated, using the well-known formula for the parametrization of the nuclear radius in terms of the collective coordinates (see Eq. \ref{nuclear-shape}), resulting in the known nuclear shapes: \emph{spherical}, \emph{axially-symmetric} (that is prolate or oblate), and \emph{axially-asymmetric} (or triaxial). 

Developed by Nilsson in 1955 \cite{nilsson1955binding} for treating the \emph{deformed nuclei} proved to be a big pillar within the nuclear community, especially for the study of medium and heavy nuclei. In essence, this tools is a modified shell model which allows for deformations to be taken into account by the use of the \emph{anisotropic harmonic oscillator} (AHO). Similarly as for the basic shell model, the goal is to obtain an expression for the single-particle energies of a nucleon. The basic Hamiltonian corresponding to this kind of system is shown below \cite{bertulani2007nuclear}:
\begin{align}
    H=H_0+a_1\vec{l}\cdot\vec{s}+a_2l^2\ ,
    \label{nilsson-simple-hamiltonian}
\end{align}
where $H_0$ is a Hamiltonian for the AHO. The general expression for this kind of oscillator is of the form:
\begin{align}
    H_\text{AHO}\equiv H_0=-\frac{\hbar^2}{2m}\nabla^2+\frac{1}{2}m(\omega_x^2x^2+\omega_y^2y^2+\omega_z^2z^2)\ .
\end{align}
In the general expression of the single-particle Hamiltonian, the constants $a_1$ and $a_2$ are usually determined via adjustments to the experimental results. It can be seen that the centrifugal-like term $l^2$, which simulates a flattening of the oscillator potential, and the $\vec{l}\cdot\vec{s}$ term are still present here, as it was the case for the spherical shell model. However, the explicit form of Eq. \ref{nilsson-simple-hamiltonian} is as follows:
\begin{align}
    H_\text{Nil}=-\frac{\hbar^2}{2m}\nabla^2+&\frac{1}{2}m(\omega_x^2x^2+\omega_y^2y^2+\omega_z^2z^2)-2\kappa\hbar\omega_0(\vec{l}\cdot\vec{s})\nonumber\\&-2\kappa\hbar\omega_0\mu\left(l^2-\langle l^2\rangle_N\right)\ .
\end{align}
Obviously, the parameters $\kappa$ and $\mu$ act as strength parameters for the spin-orbit coupling term and the centrifugal term, respectively. The last term is a correction, which was originally considered as $\mu l^2$, but it was pointed by Gustafson et al. \cite{gustafson1967nuclear} that the shift in energy is way too large for big values of $N$ (principal quantum number). As a result, taking the current expression for the correction term helps to compensate. The three \emph{oscillator frequencies} are chosen to be inversely proportional to the semi-axis lengths of the deformed ellipsoid (denoted by $a_x$, $a_y$, and $a_z$) such that:
\begin{align}
    \omega_r=\omega_0\frac{R_0}{a_r}\ ,\ r=x,y,z\ .
\end{align}
For the spherical case, the oscillator frequency $\hbar\omega_0$ is set to $41A^{-1/3}$ MeV (calculation for this value arise from the shell model with SHO \cite{bertulani2007nuclear}). For the axially-symmetric case, one can choose the $z$-axis as symmetry axis, implying that thw oscillator frequencies along the $x$ and $y$ axes are equivalent (that is $\omega_x=\omega_y\equiv\omega_\perp$).

Following the calculations done in \cite{bertulani2007nuclear}, one can express the two relevant oscillator frequencies in terms of a deformation parameter $\epsilon_2$ (whose dependence on the quadrupole deformation parameter $\beta_2$ has been shown in Eq. \ref{epsilon-beta-relation}) as such:
\begin{align}
    \omega_\perp^2=\omega_0^2\left(1+\frac{2}{3}\epsilon_2\right)\ ,\\
    \omega_z^2=\omega_0^2\left(1-\frac{4}{3}\epsilon_2\right)\ .
\end{align}
Moreover, a dependence on the deformation parameter itself is employed for the frequency $\omega_0$ that appears in the expressions for $\omega_\perp$ and $\omega_z$, respectively:
\begin{align}
    \omega_0(\epsilon_2)=\bar{\omega}_0\left(1-\frac{4}{3}\epsilon_2^2-\frac{16}{27}\epsilon_2^3\right)^{-1/6}\ ,
\end{align}
where $\bar{\omega}_0$ can be considered a constant written as $\bar{\omega}_0=(\omega_x\omega_y\omega_z)^{1/3}=\text{const}$ (coming from the harmonic oscillator at zero deformation and also considering the conservation of the nuclear volume).

The energy eigenvalues $\epsilon_q$ for the nucleonic state $\psi_q$ belonging to a deformed nucleus can be determined within the Nilsson model by solving the Schrödinger equation associated to each nucleon in particular:
\begin{align}
    H_\text{Nil}\psi_q=\epsilon_q\psi_q\ ,
    \label{nilsson-schrodiner-equation}
\end{align}
where the index $q$ denotes a set with all the relevant quantum numbers. This set is also called the \emph{asymptotic quantum numbers}, and they are used to specify a \emph{Nilsson orbital}. The well-known notation is as follows (still considering the $z$-axis as the symmetry axis):
\begin{align}
    \Omega^\pi\left[Nn_z\Lambda\right]\ .
    \label{nilsson-notation}
\end{align}
\begin{itemize}
    \item $\Lambda$ is the projection of the particle's orbital a.m. along the symmetry axis (component of $l$ along $z$)
    \item $N$ the principal quantum number of the major shell. It also determines the parity as $\pi=(-1)^N$, making the notation from Eq. \ref{nilsson-notation} somewhat redundant in terms of explicitly specifying it
    \item $n_z$ is the number of oscillator quanta along the symmetry axis. More precisely, it gives the number of nodes for the wave-function of that particle
    \item $\Omega$ is the projection of the particle's total a.m. along the symmetry axis. Moreover, the projection of the intrinsic spin of a nucleon onto the symmetry axis can have the values $\Sigma=\pm\frac{1}{2}$, so that $\Omega=\Lambda+\Sigma=\pm\frac{1}{2}$.
\end{itemize}

\subsection{Single-particle states in deformed nuclei}

It is instructive to go into detail about the quantum numbers defined in Eq. \ref{nilsson-notation} since the orbits which characterize the nucleons with such numbers help to point out the nature of nuclear deformations that take place.

The quantum numbers $N$, $n_z$ and $\Lambda$ are good quantum numbers only when the nuclear deformation is large, meaning that $\epsilon$ (or equivalently $\beta$) tends to infinity: also the reason why they are called asymptotic quantum numbers. However, the numbers $\Omega$ and $\pi$ remain good quantum numbers even for low and moderate deformations for the nucleus. It should be noted that if $N$ is even, then $(\Lambda+n_z)$ is also even. Similarly, if $N$ is odd, then the sum of the other two quantum numbers must also be odd.

Since the eigenvalues of the Hamiltonian $H_\text{Nil}$ ultimately depend on the deformation parameter $\epsilon$, each nucleon will have an orbit (energy) that is deformation dependent. At no deformation (i.e., the spherical case), all the energy levels for a single-particle state will have a $2j+1$ degeneracy. This translates to the fact that all $2j+1$ possible orientations of $\vec{j}$ are equivalent, when referring to any arbitrary axis of choice. On the other side, when the potential is deformed, this will no longer hold: the energy levels in the deformed potential will depend on the spatial orientation of the orbit itself: the energy depends on the component of $\vec{j}$ along the symmetry axis of the core.

As an example, a nucleon from the $f_{7/2}$ shell will be considered. This nucleon can have eight possible components for $\vec{j}$, this is the range $\Omega=[-\frac{7}{2},\frac{7}{2}]$. Because of the reflection symmetry for nuclei for either of the two possible directions of the symmetry axis, the positive components of $\Omega$ will have the same energy as the negative ones: leading to a degeneracy of the levels. Now, the single-particle $f_{7/2}$ state will split up into four new states when deformation emerges: $\Omega=\frac{1}{2},\frac{3}{2},\frac{5}{2},\frac{7}{2}$ and all have negative parity. In Fig. \ref{nillson-orbits-prolate-projections} an illustration with the different orbits of the odd particle is given, for both the prolate deformed nuclei as well as for oblate ones. Similarly, the orbits of the same state are pictorially represented in Fig. \ref{nillson-orbits-oblate-projections}.

\begin{figure}
    \centering
    \includegraphics[scale=1]{Chapters/Figures/nillson_SP_orbits.pdf}
    \caption{A simple sketch showing the single-particle orbits for the $j=7/2$ nucleonic state, along the symmetry axis for a \emph{prolate} deformation. The actual projections are $\Omega_1=\frac{1}{2}$, $\Omega_2=\frac{3}{2}$, $\Omega_3=\frac{5}{2}$, and $\Omega_4=\frac{7}{2}$. The figure was inspired from Ref. \cite{krane1991introductory}.}
    \label{nillson-orbits-prolate-projections}
\end{figure}

\begin{figure}
    \centering
    \includegraphics[scale=1]{Chapters/Figures/nillson_SP_orbits_2.pdf}
    \caption{A simple sketch showing the single-particle orbits for the $j=7/2$ nucleonic state, along the symmetry axis for a \emph{oblate} deformation. The actual projections are $\Omega_1=\frac{1}{2}$, $\Omega_2=\frac{3}{2}$, $\Omega_3=\frac{5}{2}$, and $\Omega_4=\frac{7}{2}$. The figure was inspired from Ref. \cite{krane1991introductory}.}
    \label{nillson-orbits-oblate-projections}
\end{figure}

From Figs. \ref{nillson-orbits-prolate-projections} - \ref{nillson-orbits-oblate-projections}, it can be seen that the first orbit (denoted by orbit $1$) lies closest to the core in the prolate case, while in the oblate case this is true for orbit $4$. This plays the role in the interaction strength, meaning that for the prolate case, the orbit $1$ will interact the strongest with the \emph{core}, while in the oblate case, it is the orbit $4$ which has the strongest interaction with the bulk core. Moreover, the strength of interaction indicates the magnitude of the energies for each projection: the stronger the interaction between the orbit and the core, the more tightly bound these states are and lie lower in energy. For prolate deformations, the orbits with smallest $\Omega$ `prefer' to lie lower in energy (interacting strongly with the core). For oblate deformations, the situation is opposite: orbits with the maximal $\Omega$ have the strongest core interactions and therefore lie lowest in energy.

Another way of looking at the coupling of the single-particle with the bulk core can be given in terms of overlaps of their corresponding wave-functions (eigenstates). Indeed, a nucleon lying in the lowest $\Omega$ orbit will have a \emph{maximum} wave-function overlap with a prolate core. On the other hand, nucleons lying in the highest $\Omega$ orbits will have maximum overlap of the wave-function with an oblate core. The overlap gives the overall binding energy between the two systems (i.e., core and particle) as explained in the previous paragraph. Discussion about the wave-function overlap and the nuclear density distribution \cite{frauendorf2014transverse, das2018nuclear} will be made in the following chapters.

The induced degeneracy due to deformation for a particle state $l_j$ is shown in Fig. \ref{nillson-orbits-splittings}, for the same example of the nucleon with the orbit $f_{7/2}$.

\begin{figure}
    \centering
    \includegraphics[scale=0.95]{Chapters/Figures/nillson_SP_splittings.pdf}
    \caption{The effect of deformation for the particle state $f_{7/2}$. It can be seen that indeed, as it was mentioned within the text, $\Omega_1$ component lies lowest in energy for the oblate deformation, and $\Omega_4$ component lies the lowest in energy for an oblate deformation.}
    \label{nillson-orbits-splittings}
\end{figure}

Obviously, the sketch shown in Fig. \ref{nillson-orbits-splittings} is just an instructive example, and it does not represent an accurate description of the single-particle energies for deformed nuclei. In fact, if the potential is deformed, the quantum numbers $l$ and $j$ are not valid anymore (that is, the angular momentum is no longer a constant of motion for non-spherical potentials). A proper description of the single-particle orbits are represented by the so-called \emph{Nilsson diagrams}, where the energy for each state is represented as a function of the deformation parameter. Remember that the energies are in fact the eigenvalues of the Schrödinger equation associated to the initial Nilsson deformed Hamiltonian (see Eq. \ref{nilsson-schrodiner-equation}).

The spectrum of one-particle orbits plays an invaluable role within the nuclear structure and the study of deformed nuclei: the picture of one-particle motion in deformed potential works for deformed nuclei much better than the case of single-particle motion in spherical potentials for spherically shaped nuclei. Multiple quantitative analyses have been performed on experimental data of well-deformed (especially odd-$A$) nuclei, from light ($^{25}$Mg, $^{25}$Al) to heavy ($^{169}$Tm, $^{175}$Yb, $^{177}$Yb) \cite{hamamoto2016interplay}. Examples with this kind of diagrams are shown in Figs. \ref{nillson-diagram} - \ref{nillson-diagram-2}. 

\begin{figure}
    \centering
    \includegraphics[scale=0.185]{Chapters/Figures/nillson_diagram.png}
    \caption{A Nilsson diagram for protons or neutrons, with $Z$ or $N\leq50$. Picture reproduced from Ref. \cite{ragnarsson2005shapes}.}
    \label{nillson-diagram}
\end{figure}

\begin{figure}
    \centering
    \includegraphics[scale=0.185]{Chapters/Figures/nillson_diagram_2.png}
    \caption{A Nilsson diagram for neutrons, with $82\leq N\leq126$. %This diagram also shows the orbits of neutrons from nuclei such as the Lu isotopes which will be studied in this work.
    Picture reproduced from Ref. \cite{ragnarsson2005shapes}.}
    \label{nillson-diagram-2}
\end{figure}

It can be seen that each state within a Nilsson diagram is represented as a solid line or a dashed line, depending on its parity (remember that the parity quantum number is given by $(-1)^N$ or, equivalently, by $(-1)^l$). The labelling from the Figs. \ref{nillson-diagram} - \ref{nillson-diagram-2} is consistent with the one defined in the previous subsection.

Another important aspect which can be seen in the Nilsson diagrams (for some orbits) is the `crossing' between states with different quantum numbers. In order to fully understand this concept, it is instructive to go into detail about \emph{two-state mixing}.

\subsubsection{Two-state mixing}

In the work of Casten \cite{casten2000nuclear}, an analytical approach is given for treating the mixing of two different states (energy levels). It starts from the basic idea of two initial levels, each with its corresponding energy $E_1$ and $E_2$, and their associated wave-functions (denoted here with $\psi_1$ and $\psi_2$). Any interaction between them results in the mixing matrix element $\bra{\psi_1}V_\text{int}\ket{\psi_2}$, where $V_\text{int}$ is the arbitrary interaction between the states. This is sketched in Fig. \ref{two-state-mixing-scheme}.

\begin{figure}
    \centering
    \includegraphics[scale=0.95]{Chapters/Figures/two-state-mixing.pdf}
    \caption{Defining the mixing between two different states, with two corresponding energies and wave-functions. Interaction is illustrated via the curved line and $V_\text{int}$ term.}
    \label{two-state-mixing-scheme}
\end{figure}

This problem can be solved by finding the final energies and wave-functions, this being done via the diagonalization procedure of a $2\times 2$ matrix, where the main diagonal contains the two energies and off-diagonal terms represent the interaction itself. The final states will be denoted here with ($E_I$, $E_{II}$) for the energies and ($\psi_I$, $\psi_{II}$) for the wave-functions. As a general rule, the mixing depends on the initial separation $\Delta E_{21}=(E_2-E_1)$ and the matrix element $\bra{\psi_1}V_\text{int}\ket{\psi_2}$. Given a large spacing, the effect of a given matrix element will be quenched. Moreover, even for a small matrix element, it can introduce a large mixing if the energy separation between the states is small (that is, the unperturbed states are lie close in energy). 

A reduction from these two parameters can be performed, obtaining a single universal mixing expression that is valid for any arbitrary interaction and any initial spacing. As a first step, one should define the ratio between the spacing of the unperturbed states ($Delta E_{21}$) and the strength of the matrix element:
\begin{align}
    R=\frac{\Delta E_{21}}{V_\text{int}}\ .
\end{align}
With this quantity, the newly perturbed energies $E_I$ and $E_{II}$ are readily obtained \cite{casten2000nuclear}:
\begin{align}
    E_I&=\frac{1}{2}(E_1+E_2)+\frac{\Delta E_{21}}{2}\sqrt{1+\frac{4V_\text{int}^2}{\Delta E_{21}^2}}\ ,\\
    E_{II}&=\frac{1}{2}(E_1+E_2)-\frac{\Delta E_{21}}{2}\sqrt{1+\frac{4V_\text{int}^2}{\Delta E_{21}^2}}\ .
    \label{eq-two-state-mixing-energies}
\end{align}

Even more useful would be to find the amount by which each energy is shifted after the interaction. This is denoted in Fig. \ref{two-state-mixing-scheme} by $\Delta E_S$ and its expression depends on $\Delta E_{12}$ as such:
\begin{align}
    |\Delta E_S|=|E_{II}-E_2|=|E_{I}-E_1|=\frac{\Delta E_{21}}{2}\left[\sqrt{1+\frac{4}{R^2}}-1\right]\ .
    \label{eq-shift-mixed-states}
\end{align}
The two perturbed wave functions are as follow:
\begin{align}
    \psi_I&=\alpha\psi_1+\beta\psi_2\ ,\nonumber\\
    \psi_{II}&=-\beta\psi_1+\alpha\psi_2\ ,
\end{align}
where the two amplitudes $\alpha$ and $\beta$ must verify the condition $\alpha^2+\beta^2=1$ and:
\begin{align}
    \beta=\frac{1}{\left\{1+\left[\frac{R}{2}+\sqrt{\frac{R^2}{4}+1}\right]^2\right\}^{1/2}}
    \label{eq-beta-mixing-amplitude}
\end{align}

It is noteworthy to point out that the amplitude $\beta$ is in fact a function that only depends on $R$ (i.e., the ratio between the unperturbed energy splitting and the interaction strength). Similarly, by dividing the shift in energy $\Delta E_S$ to the initial splitting $\Delta E_{21}$, one will obtain an expression that is independent of the initial level spacing:
\begin{align}
    \frac{|\Delta E_S|}{\Delta E_{21}}=\frac{|E_{II}-E_{2}|}{\Delta E_{21}}=\frac{|E_{I}-E_{1}|}{\Delta E_{21}}=\frac{1}{2}\left[\sqrt{1+\frac{4}{R^2}}-1\right]
\end{align}

The importance of these formula will be now emphasized through a numerical example. First of all, the evolution of the ratio of the unperturbed shift and the interaction can be graphically represented as a function of the small mixing amplitude $\beta$ by the use of Eq. \ref{eq-beta-mixing-amplitude}. The graphical representation is shown in Fig. \ref{fig-beta-mixing-amplitude}. Following this analysis, in Fig. \ref{mixing-energy-shift-shape} the shape of $R$ as a function of the energy shift of the perturbed states can be visualized.

\begin{figure}
    \centering
    \includegraphics[scale=0.5]{Chapters/Figures/beta_mixing_amplitude.pdf}
    \caption{The dependence of $R$ (see text) on the mixing amplitude $\beta$.}
    \label{fig-beta-mixing-amplitude}
\end{figure}

For an arbitrary case where two initial states are separated by, say $\Delta_{21}=0.07$ MeV, and they become \emph{perturbed} via the interaction with a strength $V_\text{int}=0.03$ MeV, this gives a value of $R=3.5$ and, moreover, the mixing amplitude is $beta=0.256$. The two states will both be shifted by only $\Delta E_S=5.31$ keV (accounting for about $7.6 \%$ of the initial separation). Indeed, for this particular example, the perturbation results in an energy shift that is rather small compared to the initial state spacing.

\begin{figure}
    \centering
    \includegraphics[scale=0.5]{Chapters/Figures/energy_shift_mixing_shape.pdf}
    \caption{The dependence of $R$ (see text) on the energy shift of the perturbed states ($\Delta E_S$).}
    \label{mixing-energy-shift-shape}
\end{figure}

Besides the numerical example discussed above, there are also two extremely important limiting situations when two states interact via a perturbation. The first one is the so-called \emph{strong mixing limit}, when the two initial states are degenerate (i.e., there is practically no spacing between them and $\Delta_{21}=0$). In this situation, the analytical expressions from Eq. \ref{eq-shift-mixed-states} fail to provide a quantitative analysis, but from Eq. \ref{eq-two-state-mixing-energies} a small adjustment of the expression will give rise to the following:
\begin{align}
    E_{I,II}=\frac{1}{2}\left[(E_1+E_2)\pm2V_\text{int}\right]=E_0\pm V_\text{int}\ ,
\end{align}
where the initial (common) energy of the two degenerate states is denoted by $E_0$. The above equation indicates the important fact that the energy shift which the two states suffer via the perturbation is only given by the \emph{mixing matrix element}. This means that the final separation energy for a two-state isolated system can never be closer than twice the interaction strength ($2V$). In the degeneracy case, the values for $\beta$ and $\alpha$ are readily obtained: ($\beta=\alpha=\frac{1}{\sqrt{2}}=0.707$), such that the states are completely mixed. Consequently, the mixed wave-functions of two (initially) degenerate states do not depend on the strength $V_\text{int}$ between them.

The second limiting case is called \emph{weak mixing limit}, corresponding to a very large value of $R$ (meaning that the initial separation of the states is very large compared to the magnitude of the interaction itself). The shift in energy of the perturbed states in this case is given by:
\begin{align}
    |\Delta E_S|=\frac{1}{R^2}\ .
\end{align}

As a final step in the analysis of the two-state mixing, it is worth mentioning a corner-case which will help to get a better grasp of the Nilsson orbitals. Consider two states (say $\psi_1$ and $\psi_2$) whose energies are parametrized in terms on some argument $c_\text{nuc}$ which is relevant for the nuclear structure of that system (e.g., $c_\text{nuc}$ could be a quadrupole deformation and the two initial states are in fact Nilsson orbits). The remarking `feature' of this hypothesis is that if there indeed exists mixing between the two states, they can never cross each other. The two mixed states will always repel and they can never be closer than twice the mixing matrix element $V_\text{int}$ after mixing occurs. % Chapter 3
\chapter{Triaxial Nuclei and Their Signatures}

In the following chapter, some theoretical background that is necessary for understanding triaxiality will be presented, with examples from literature and also some results obtained by this team. It is also instructive to realize why the nuclear community focuses their attention to the highly deformed nuclei, and moreover, nuclei which depart so much from the spherical shapes that they become \emph{triaxial}.

Presenting the theoretical formalism that is used to describe triaxial nuclei, and mention the \emph{fingerprints} of nuclear triaxiality is that last step before diving into the recently developed framework for odd-$A$ nuclei which the team created, and showing the results. 

\section{Non-axial nuclei}

The discussion regarding excitation energies of a rigid rotator from the previous chapter was focused on pure rotators or nuclei with axially symmetric shapes (i.e,, prolate or oblate). Remember that deformation is still required in order to define a collective spectra with rotational character. Moreover, the relevant quantities that are involved in the rotational motion for a deformed nucleus are the moments of inertia corresponding to the principal axes of the deformed ellipsoid: $\mathcal{I}_{1,2,3}$, and within previous calculations, two moments of inertia were supposed to be identical.

Even though calculations are performed with the rigid-like MOI or the irrotational-like, their dependence on the deformation parameters $\beta_2$ and $\gamma$ is present (recall expressions given in Eq. \ref{eq-irrotational-rigid-mois}). Taking a closer look at their evolution with $\gamma$, one can see that indeed, identical MOI can only occur at certain values (see Fig. \ref{fig-irrotational-rigid-mois}). As such, the nuclei can be regarded (when referring to their ground state) as such:
\begin{itemize}
    \item \textbf{Spherical:} all MOI are identical and no deformations are present
    \item \textbf{Axially-symmetric:} two identical MOI and only the $\beta_2$ parameter plays a role in the collective phenomena of these nuclei
    \item \textbf{Triaxial (Axially-asymmetric):} all three MOI are different (usually one of them is very large when compared to the other two), the quadrupole deformation parameter as well as the triaxiality parameter $\gamma$ are present
\end{itemize}

Now, across the chart of nuclides, most of the isotopes are either spherical either symmetric in their ground state \cite{budaca2018tilted}. % Chapter 4
\chapter{Wobbling Motion in Nuclei}

The pioneering work of Bohr and Mottelson \cite{bohr1998nuclear} which was done more than 50 years ago lead to some interesting features regarding the collective phenomena in triaxial nuclei. Namely, they pointed out that a specific precessional motion of the nucleus's spin will take place when the rotational energy is sufficient. The angular momentum for triaxial nuclei is not aligned any of the principal axes of the ellipsoid, but it \emph{precesses} and \emph{wobbles} around one of these axes. They called this phenomenon \textbf{wobbling motion} (w.m.).
This combined motion comes from a consequence regarding the MOI. Indeed, the asymmetry of the three MOI makes the quantum mechanical nature of rotation to be possible around any of the three axes. As such, a \emph{main} rotation around the axis with the largest MOI will be the most energetically favorable, but the other two directions can \emph{quantum mechanically disturb} this main rotation, leading to this unique characteristic of triaxial nuclei.

The non-uniformity nature of w.m. was firstly studied for the `pure' rigid-rotators that correspond to the even-even nuclei. In this case, the w.m. can be treated as small amplitude oscillations of the total angular momentum $\mathbf{I}$ around the axis corresponding to the largest MOI.

\section{Wobbling Motion in Even-Even Nuclei}

The analytical expressions for wobbling excitations were firstly evaluated by Bohr and Mottelson using the so-called \emph{Harmonic Approximation} (HA). This can be described as a small-amplitude limit for the Triaxial Rigid Rotor Hamiltonian that was discussed in Chapter \ref{chapter4} (see Section \ref{trm-model}). In this limit, the projection of the total angular momentum onto the axis with largest MOI can be approximated $I_3\approx I$, meaning that the nucleus will do most of its rotation around this axis, with some `disturbance' from the other two principal of the triaxial rotor. 

For the description of this simple wobbler, one can consider the case when the $3$-axis has the largest MOI, and the following order holds true:
\begin{align}
    \mathcal{I}_3>\mathcal{I}_2>\mathcal{I}_1\ ,
\end{align}
or equivalently:
\begin{align}
    A_3<A_2<A_1\ .
\end{align}
The Hamiltonian can be written as:
\begin{align}
    \hat{H}_\text{rot}={\color{red}A_3I_3^2}+{\color{blue}\left(A_1I_1^2+A_2I_2^2\right)}
    \label{general-rotor-ham-evenA}
\end{align}

The different colors from Eq. \ref{general-rotor-ham-evenA} try to emphasize the fact that a Hamiltonian for the simple wobbler can be regarded as a main rotation around $3$-axis (represented by {\color{red}red color}) and the (precession + oscillation) of the total angular momentum (represented by {\color{blue}blue} color.)

The wobbling excitations which cause oscillations with small amplitudes for $\mathbf{I}$ about the $3$-axis are assumed to have a harmonic-like behavior, meaning that the final energy spectrum of a simple wobbler (even-even nucleus) will have the typical $\hbar\omega(n+1/2)$ behavior. Since this oscillator motion can be explained as `vibrations' of the total angular momentum around a steady position, where each wobbling excitation consists of an additional vibrating phonon, one can express the Hamiltonian in terms of \emph{boson} creation and annihilation operators. As such, the quantum mechanical treatment implies \cite{bohr1998nuclear}:
\begin{align}
    b^\dagger=\frac{1}{\sqrt{2I}}I_+\ ,\ b=\frac{1}{\sqrt{2I}}I_-\ ,\ \left[b,b^\dagger\right]\approx 1\ .
\end{align}

This initial quantization allows one to write Eq. \ref{general-rotor-ham-evenA} as a rotational term and a wobbling-specific one:
\begin{align}
    \hat{H}_\text{rot}&={\color{red}A_3I_3^2}+{\color{blue}H_w}\ ,\label{rot-ham-non-diagonal} \\
    {\color{blue}H_w}&={\color{blue}t_1\left(n+\frac{1}{2}\right)+\frac{1}{2}t_2(b^\dagger b^\dagger + bb)}\ ,
    \label{wob-ham-non-diagonal}
\end{align}
where the \emph{number of boson excitations} is denoted by $n$ and it is given by $n=b^\dagger b$. Each wobbling quanta will carry an angular momentum of one unit less with respect to the $3$-axis. The two factors $t_{1,2}$ are expressed in terms of the inertial parameters as \cite{bohr1998nuclear}:
\begin{align}
    t_1&=I(A_2+A_1-2A_3)\ , \\
    t_2&=I(A_2-A_1)\ .
\end{align}

Notice the linear dependence of the two parameters on the total angular momentum $I$. Moreover, depending on the values of $A_k$, the contribution of $t_{1,2}$ can be negative. Their behavior is shown within the right inset of Fig. \ref{fig-even-even-wobbling-energies}. Although the Hamiltonian $H_w$ is considered to have an oscillator-like behavior, its general expression does not look like a typical harmonic Hamiltonian. For this, the Hamiltonian given in Eq. \ref{wob-ham-non-diagonal} can be brought to a diagonalized form by introducing a new set of boson creation and annihilation operators. These operators will be written as linear combinations of $(b^\dagger,b)$:
\begin{align}
    c^\dagger=w_1b^\dagger-w_2b\ ,\\
    c=w_1b-w_2b^\dagger\ ,
\end{align}
where the two coefficients $w_{1,2}$ are defined in terms of $t_{1,2}$ as:
\begin{align}
    w_1&=\left[\frac{1}{2}\left(\frac{t_1}{\sqrt{t_1^2-t_2^2}}+1\right)\right]^{1/2}\ ,\nonumber\\
    w_2&=\left[\frac{1}{2}\left(\frac{t_1}{\sqrt{t_1^2-t_2^2}}-1\right)\right]^{1/2}\ .
    \label{eqs-w1-w2-terms-wobbling}
\end{align}

The terms $w_{1,2}$ are verify the condition $w_1^2-w_2^2=1$ and they make the `dangerous' products ($b^\dagger b^\dagger$,$bb$) disappear in this new representation \cite{oi2006semi}. Note that there is no spin dependence inferred in Eq. \ref{eqs-w1-w2-terms-wobbling} such that $w_{1,2}$ are constant functions of spin, unlike the coefficients $t_{1,2}$. Moreover, introducing a number operator $\hat{n}=c^\dagger c$ and the excitation quanta $\hbar\omega_w$ defined as:
\begin{align}
    \hbar\omega_w=\sqrt{t_1^2-t_2^2}=2I\sqrt{(A_1-A_3)(A_2-A_3)}\ ,
    \label{wobbling-frequency-even-A}
\end{align}
then a final expression of $H_w$ can be expressed, which has a behavior typical to the \emph{harmonic oscillator}:
\begin{align}
    H_w=\hbar\omega_w\left(\hat{n}+\frac{1}{2}\right)\ .
    \label{wob-ham-diagonal}
\end{align}

In this expression, the excitation quanta $\hbar\omega_w$ which was defined in Eq. \ref{wobbling-frequency-even-A} in terms of $t_{1,2}$ is called \emph{wobbling frequency} and its increasing linearly with the total angular momentum. Accordingly, Eq. \ref{rot-ham-non-diagonal} can be re-written with the wobbling Hamiltonian defined in Eq. \ref{wob-ham-diagonal}:
\begin{align}
    \hat{H}_\text{rot}=A_3I(I+1)+\hbar\omega_w\left(\hat{n}+\frac{1}{2}\right)\ .
    \label{rot-wob-ham-diagonal}
\end{align}

Thus, in the HA, the eigenvalues of the rotor Hamiltonian can be expressed in terms of a \emph{wobbling phonon number} $n_w$ (which is the eigenvalue of the number operator $\hat{n}=c^\dagger c$) and a \emph{wobbling frequency} (defined in Eq. \ref{wobbling-frequency-even-A}):
\begin{align}
    E_{I,n}={\color{red}A_3I(I+1)}+{\color{blue}\hbar\omega_w\left(n_w+\frac{1}{2}\right)}\ .
    \label{eq-wobbling-energy-evenA}
\end{align}

The spectrum for an even-even wobbling nucleus is thus represented by Eq. \ref{eq-wobbling-energy-evenA}. Notice again the two colored terms that illustrate the energy coming from the rotation around the $3$-axis and the disturbed motion with small oscillations around the other two axes. Consequently, the wobbling character of the system will be generated by the latter harmonic term. The wobbling phonon number $n_w$ is related to the `strength' of the tilting for $\mathbf{I}$, indicating the fact that an increasing number for $n_w$ will result in oscillations with larger amplitudes around the other two axes. The phonon number takes values $n_w=0,1,\dots$. In inset $b)$ from Fig. \ref{wobbling-geometry-tilting-sketch}, a sketch which shows the tilting effect that the wobbling phonon number has on the total angular momentum vector is drawn. For completeness, the collective structure of two wobbling bands generated through phonon excitations is exemplified in inset $a)$ from Fig. \ref{wobbling-geometry-tilting-sketch}.

\begin{figure}
    \centering
    \includegraphics[scale=0.72]{Chapters/Figures/wobbling_n_schematic-1.pdf}
    \includegraphics[scale=0.6]{Chapters/Figures/wobbling_n_schematic-2.pdf}
    \caption{\textbf{Left:} A typical wobbling structure for even-even nuclei. The yrast band contains even values of spins since the band has signature $\alpha=0$, while the first excited band has odd spins and $\alpha=1$. The intraband states differ by 2 units of angular momenta, while the interband ones differ with only one unit. \textbf{Right:} The increase of tilting angle between the rotational axis (the $3$-axis in this case) and the total angular momentum $\mathbf{I}$. With each wobbling phonon number, the total angular momentum tilts more and more, generating a `stronger' precessional motion (illustrated by the colored ellipses).}
    \label{wobbling-geometry-tilting-sketch}    
\end{figure}

An alternative way of depicting the wobbling term $H_w$ from Eq. \ref{rot-ham-non-diagonal} would be to express it more generally, in terms of $I_2$ and $I_3$. When doing so, one achieves the following form (assuming that rotation is around the $3$-axis) \cite{oi2006semi}:
\begin{align}
    % H_w={\color{magenta}(A_1-A_3)I_1^2}+{\color{orange}(A_2-A_3)I_2^2}={\color{magenta}T_\text{kin}}+{\color{orange}T_\text{pot}}\ ,
    H_w=(A_1-A_3)I_1^2+(A_2-A_3)I_2^2=T_\text{kin}+T_\text{pot}\ ,
\end{align}
where according to Ref. \cite{wen2015wobbling}, one can regard these two factors as a \emph{kinetic} and a \emph{potential} term. This way of expressing $H_w$ is instructive since it keeps a close contact with the `classical' picture of understanding the total energy of a system.

As a quantitative analysis of the wobbling frequency and the rotor energy, one can take three arbitrary values for the moments of inertia (and, implicitly, the inertia factors $A_k$) and see the behavior of both $E_{I,n}$ and $\hbar\omega_w$ with increasing angular momentum and wobbling phonon number. Keep in mind that depending on the value of the wobbling phonon number, different spin sequences will be allowed. More precisely, from the invariance of the rotor w.r.t. rotations by $\pi$ about the principal axes for even-even nuclei, the signature quantum number $\alpha$ can take the values 0 and 1. Each wobbling band will have an alternating signature, starting with $\alpha=0$ for $n_w=0$ then $\alpha=1$ for $n_w=1$ and so on: even spin sequences appear for even values of $n_w$ and odd spin sequences appear for odd values of $n_w$ (see inset $b)$ from Fig. \ref{wobbling-geometry-tilting-sketch}).

The rotor energy from Eq. \ref{eq-wobbling-energy-evenA} is graphically represented for an arbitrary set of moments of inertia as a function of the nuclear angular momentum $I$ in Fig. \ref{fig-even-even-wobbling-energies}. This pedagogical example contains rotational bands up to $n_w=5$ in the wobbling phonon number. From Fig. \ref{fig-even-even-wobbling-energies}, one can see the linear dependence on the total angular momentum and, moreover, the wobbling energy and frequency both are increasing with spin.

\begin{figure}
    \centering
    \includegraphics[scale=0.7]{Chapters/Figures/wobbling-evenA.pdf}
    \includegraphics[scale=0.74]{Chapters/Figures/wobblingFreq-evenA.pdf}
    \caption{\textbf{Left:} The energy spectrum for an even-even nucleus with three different moments of inertia, with the main rotation around the $3$-axis, according to Eq. \ref{eq-wobbling-energy-evenA}. Each wobbling band has alternating signature number $\alpha$ (starting with $\alpha=0$ for the ground state $n_w=0$ band). Notice the even/odd spin sequences for each band. \textbf{Right:}: The wobbling frequency plotted together with the linear terms $t_1$ and $t_2$ that are used to express $\hbar\omega_w$. Same set of MOI were used across both figures and the unit for $\mathcal{I}_i$ is $\hbar^2\ \text{MeV}^{-1}$.}
    \label{fig-even-even-wobbling-energies}
\end{figure}


Another instructive analysis would be the evolution of the components of $\mathbf{I}$ as functions of the polar and azimuthal angles $\theta,\varphi$. Indeed, expressing the three angular momentum components as:
\begin{align}
    I_1&=I'\sin\theta\cos\varphi\ ,\\
    I_2&=I'\sin\theta\sin\varphi\ ,\\
    I_3&=I'\cos\theta\ ,
    \label{angular-momentum-polar-components}
\end{align}
where $I'=\sqrt{I(I+1)}$, one can make a graphical representation for them, by letting $\theta$ and $\varphi$ vary within their corresponding intervals. In Fig. \ref{figs-angular-momentum-components-polar}, the quantities $I_1$ and $I_2$ are represented in the $(\theta,\varphi)$ plane for a fixed spin value $I=10\hbar$. Since the third component is independent of the azimuthal angle $\varphi$, it has been dismissed.

\begin{figure}
    \centering
    \includegraphics[scale=0.66]{Chapters/Figures/angular_components-TRM-1.pdf}
    \includegraphics[scale=0.66]{Chapters/Figures/angular_components-TRM-2.pdf}
    \caption{The geometrical representation of the first and second component of the total angular momentum $\mathbf{I}$ as functions of the polar angles, according to Eq. \ref{angular-momentum-polar-components}.}
    \label{figs-angular-momentum-components-polar}
\end{figure}

The other relevant observables which can be calculated for simple wobbler within the HA are the two quadrupole moments $Q_{20,22}$ and the intraband + interband $B(E2)$ transition probabilities. The quadrupole components are expressed in terms of the intrinsic quadrupole moment $Q_0$ and the triaxiality parameter as \cite{shoji2006microscopic}:
\begin{align}
    Q_{20}=Q_0\cos\gamma\ ,\ Q_{22}=\frac{1}{\sqrt{2}}Q_0\sin\gamma\ .
\end{align}

These components can be furthermore used to determine the intraband $B(E2)$ transition probabilities \cite{wen2015wobbling}:
\begin{align}
    B(E2;(n,I)\to(n,I-2))=\frac{5}{16\pi}Q_{22}^2\ ,
\end{align}
and also the interband transitions:
\begin{align}
    B(E2;(n,I)\to(n-1,I-1))&=\frac{5}{16\pi}\frac{n}{I}\left(\sqrt{3}Q_{20}w_1+\sqrt{2}Q_{22}w_2\right)^2\ ,\\
    B(E2;(n,I)\to(n+1,I-1))&=\frac{5}{16\pi}\frac{n+1}{I}\left(\sqrt{3}Q_{20}w_2+\sqrt{2}Q_{22}w_1\right)^2\ .
\end{align}

Notice that for the intraband transitions, going from the state $I$ to $I-2$ will only depend quadratically on the quadrupole component $Q_{22}$, making thus the transitions spin-independent.

\subsubsection*{Triaxial rotor energy vs. wobbling energy}

An important discussion should be made regarding the nomenclature for energies when referring to wobbling motion. As shown in Eq. \ref{eq-wobbling-energy-evenA}, the energy spectrum for a simple wobbler can be determined for every phonon number and spin sequences. However, that is the `full' spectrum  of the wobbler, which is composed of the \emph{yrast} states with $n_w=0$ and the \emph{excited states} having $n_w=1,\dots$ and so on. On the other hand, the so-called \emph{wobbling energies} are defined in terms of these `absolute values' (i.e., $E_{I,n}$) with the following rules \cite{wen2015wobbling}:
\begin{align}
    E_\text{wob}(I_\text{even})&=E_{I,n}-E_{I,0}\ ,\\
    % E_\text{wob}(I_\text{even})&=E_{I,n}-E_{I,0}\ ,\nonumber \\
    E_\text{wob}(I_\text{odd})&=E_{I,n}-\frac{1}{2}\left(E_{I-1,0}+E_{I+1,0}\right)\ ,
    \label{eq-wobbling-energy-definition-evenA}
\end{align}
where the former wobbling energy corresponds to the even values of $I$ and the latter is applied for odd values of $I$. Very often within literature the energies calculated via Eq. \ref{eq-wobbling-energy-evenA} are referred to also as wobbling energies, which is not the same as Eq. \ref{eq-wobbling-energy-definition-evenA}, so a distinction should be made clear.

\subsection{Testing the Harmonic Approximation}

It is worth going further and apply the HA formalism for even-even nuclei for an existing spectrum. As such, one can take $^{130}$Ba as a testing example. As it will be discussed in a follow-up section, it turns out that experimental observations for wobbling structures in even-mass isotopes have been very scarce. Nevertheless,  very recently Petrache et al. identified a large collection of band structures in $^{130}$Ba \cite{petrache2019diversity}. Two of them are reported to be of wobbling nature \cite{chen2019transverse}. Having these two collective bands, one can check if the energy formula given in Eq. \ref{eq-wobbling-energy-evenA} for the simple wobbler can be applied for this isotope.

The method described in here will be based on a \emph{fitting procedure}, namely a set of parameters will be extracted from the expression of $E_{I,n}$ and they will adjusted such that the experimental data is best reproduced by the theoretical model. This kind of approach works really well for `well-behaved' model functions and if the input data is large enough to reach a good fit precision. More often than not, if the model function contains parameters which have a clear physical meaning then fitting becomes a suitable approach. In fact, in the following chapters, the developed formalism will verify the experimental data through similar fitting procedures (although their `core'-implementation will be more complex).

Looking at the energy formula from Eq. \ref{eq-wobbling-energy-evenA}, at a first glance, two fitting parameters would appear, namely the largest moment of inertia $\mathcal{I}_3$ and the wobbling frequency $\hbar\omega$. However, the wobbling frequency is furthermore dependent on the other two moments of inertia (as per Eq. \ref{wobbling-frequency-even-A}), meaning that one can use the set $\mathcal{P}_\text{fit}=\left[\mathcal{I}_1,\mathcal{I}_2,\mathcal{I}_3\right]$ as appropriate fitting parameters. The wobbling phonon number $n_w$ is attributed as follows: $n_w=0$ for the yrast band (denoted throughout calculations with B1) and $n_w=1$ for the first excited band (denoted with B2). The band B1 has signature $\alpha=0$ so it is the even-spin sequence, while B2 has odd spins. The experimental data regarding spins and energies for the two bands correspond to the measurements done in Ref. \cite{petrache2019diversity}.

Indeed, by following the procedure described above, the set $\mathcal{P}_\text{fit}$ is obtained. The parameters, i.e., the three moments of inertia are shown in Table \ref{table-params-ba130}. Remarking that fact that the largest MOI which was obtained via the fitting procedure is the one corresponding to the $3$-axis. With these parameters, the theoretical energies are determined numerically and the two bands are compared to the measured data in Fig. \ref{plot-ba130-excitation-energies}. Concerning the fitted energies, in the present calculations, instead of working with the `absolute energies' (i.e., the exact values of the energies that correspond to the measured spectrum), the \emph{excitation energies} were used instead \cite{raduta2017semiclassical,raduta2018wobbling,raduta2020towards}. These are determined by subtracting the band-head energy of B1 (the $10^+$ level) from each excited state of B1 and B2. Doing this improves the accuracy of the results. The excitation energy for a spin state $I$ is given as:
\begin{align}
    E(I)=E_\text{abs}(I)-E_\text{abs}(I_\text{band-head})\ ,
\end{align}
where `abs.' signifies the absolute value for $E$ at that particular spin state.

% \begin{table}
%     \centering
%     \resizebox{0.4\textwidth}{!}{%
%     \begin{tabular}{|c|c|c|l|}
%     \hline
%     $\mathcal{I}_1$ & $\mathcal{I}_3$ & $\mathcal{I}_2$ & Unit \\ \hline
%     27 &22 &43 &$\hbar ^ 2\text{MeV} ^ {-1}$ \\ \hline
%     \end{tabular}%
%     }
%     \caption{The parameter set $\mathcal{P}_\text{fit}$ obtained from the fitting procedure of the excitation energies of the two wobbling bands (B1 and B2) for $^{130}$Ba. The model function corresponds to the energy of a simple wobbler (see Eq. \ref{eq-wobbling-energy-evenA}).}
%     \label{table-params-ba130}
% \end{table}

\begin{table}
    \centering
    \resizebox{0.35\textwidth}{!}{%
    \begin{tabular}{|cccc|}
    \hline
    \multicolumn{4}{|c|}{$\mathcal{P}_\text{fit}$}                                                                                                                        \\ \hline
    \multicolumn{1}{|c|}{$\mathcal{I}_1$} & \multicolumn{1}{c|}{$\mathcal{I}_3$} & \multicolumn{1}{c|}{$\mathcal{I}_2$} & \multicolumn{1}{c|}{Unit}                     \\ \hline
    \multicolumn{1}{|c|}{27}                & \multicolumn{1}{c|}{22}                & \multicolumn{1}{c|}{43}                & \multicolumn{1}{c|}{$\hbar^2\text{MeV}^{-1}$} \\ \hline
    \end{tabular}%
    }
    \caption{The parameter set $\mathcal{P}_\text{fit}$ obtained from the fitting procedure of the excitation energies of the two wobbling bands (B1 and B2) for $^{130}$Ba. The model function corresponds to the energy of a simple wobbler (see Eq. \ref{eq-wobbling-energy-evenA}).}
    \label{table-params-ba130}
\end{table}

\begin{figure}
    \centering
    \includegraphics[width=0.46\textwidth]{Chapters/Figures/ba130-band1.pdf}
    \includegraphics[width=0.46\textwidth]{Chapters/Figures/ba130-band2.pdf}
    \caption{Comparison between the experimental and theoretical excitation energies for the two wobbling bands (B1 and B2) of $^{130}$Ba. Experimental data is taken from Ref. \cite{petrache2019diversity}. The theoretical data was obtained by fitting Eq. \ref{eq-wobbling-energy-evenA} with the parameters defined in table.}
    \label{plot-ba130-excitation-energies}
\end{figure}

An important remark should be made here regards to the fitting model used here. Even though the spectrum of this even-even nucleus has been quantitatively reproduced quite well and the values for the tree obtained MOI indicate a triaxial nucleus with main rotation around the third axis, it should not be considered a `realistic' tool in describing this isotope. This is because in another work, Chen et al. \cite{chen2019transverse} found through microscopic calculations that the wobbling motion does not occur as per a pure triaxial rotator, but it emerges from the coupling of two quasi-particles $\pi(h_{11/2})^2$ with a triaxial core. In fact, their work shows that $^{130}$Ba is the first nucleus in which a configuration with two quasi-particles generates stable triaxial deformation through wobbling motion. Consequently, the numerical implementation performed here only shows that HA can be a suitable tool to show that $^{130}$Ba does behave as a wobbler, but this pure triaxial rotator model \emph{hides} contributions coming from single-particle configuration within the final Hamiltonian. This translates to the fact that Eq. \ref{eq-wobbling-energy-evenA} contains the effects of the two $h_{11/2}$ protons hidden within $\hbar\omega_w$.

Having the excitation energies for the two bands, one can evaluate the theoretical wobbling energies as defined in Eq. \ref{eq-wobbling-energy-definition-evenA} and compare them with the experimental values. The two quantities are graphically represented in Fig. \ref{wobbling-energies-130ba-expVSth}.

\begin{figure}
    \centering
    \includegraphics[width=0.49\textwidth]{Chapters/Figures/ba130-wobbling-energies.pdf}
    \caption{The wobbling energies for $^{130}$Ba calculated with Eq. \ref{eq-wobbling-energy-definition-evenA}. For the first state of B2, the energy was determined as $E_\text{wob}(11^+)=E_\text{B2}(11^+)-\frac{1}{2}E_\text{B1}(12^+)$ since the band-head state of B1 is zero (as per the definition of excitation energy).}
    \label{wobbling-energies-130ba-expVSth}
\end{figure}

Using HA to describe the wobbling bands in the even-even $^{130}$Ba nucleus has been done independently by the current team, and the obtained results are unique to this present research.

Concluding this section on wobbling motion of even-even nuclei, a final sketch is depicted in Fig. \ref{simple-wobbler-geometrical-schematic}, where the main axes of the triaxial ellipsoid are represented and denoted with $m$, $s$, and $l$-axis (medium, short and long, respectively). The precession motion of the total angular momentum (the a.m. of the core itself) will be around the $m$-axis, having the largest moment of inertia.

\begin{figure}
    \centering
    \includegraphics[scale=0.65]{Chapters/Figures/simple-wobbler.pdf}
    \caption{A schematic representation with a \emph{simple wobbler}, with the total angular momentum doing a precessional motion about the axis with largest MOI. In this particular sketch, the axis with largest MOI is denoted with $m$ for intermediate/medium. $l$- and $s$-axis represent the long and short axes, respectively. The small-amplitude oscillations of $\mathbf{R}$ are depicted with the (red) encircled sine wave. This figure was adapted from Ref. \cite{poenaru2021extensive} and inspired from Ref. \cite{sensharma2020longitudinal}.}
    \label{simple-wobbler-geometrical-schematic}
\end{figure} % Chapter 5
\chapter{A New Theoretical Formalism} % Chapter 6
\chapter{Wobbling Motion Study via a Boson Description}
\label{extra-chapter-new-boson}

The last part of this work will be focused on the same wobbling phenomenon, but with a different approach than the ones employed in Chapters \ref{chapter-4-aw1-formalism} and \ref{chapter-5-novel}, as this method does not share the same foundational concepts as the $\mathbf{W_1}$ and $\mathbf{W_2}$ techniques. The results shown here correspond to a recent publication made by the team in Ref. \cite{raduta2020new}, this entire chapter consisting of a summary with all the unique features.

The $\mathbf{W_0}$ formalism studied in Ref. \cite{raduta2017semiclassical} described the wobbling motion for even-even nuclei using a \emph{boson expansion} of the angular momentum components. The Dyson boson representation was chosen \cite{dyson1956general} therein.
 % Chapter 7
\chapter{Conclusions}
\label{chapter-8-conclusions}

This thesis represents the work of several publications focused on the topic of Nuclear Structure, but on the theoretical side. The study of the nucleonic matter that lacks any axial symmetry was the main objective of the team. Nuclear triaxiality became a hot topic over the last decade due to its challenges of measuring it experimentally. Moreover, the theoretical description of triaxially deformed nuclei requires certain methods or approximations, which can become quite cumbersome.

Starting with Chapter \ref{chapter-2}, the nuclear surface is introduced in Eq. \ref{nuclear-shape} and it was parametrized in terms of the collective coordinates and spherical harmonics. The relevant excitation mode for triaxiality is given by the quadrupole deformation, having $\lambda=2$. The quadrupole deformation introduces two parameters that give an insight with respect to the elongation and departure from axial symmetry of a nucleus, by means of the quadrupole deformation parameter $\beta_2$ and triaxiality parameter $\gamma$, which are provided in Eq. \ref{bohr-deformation-params}. The two parameters dictate the stretching of the nuclear axes, and this was shown in Fig. \ref{nuclear-radius-elongation}. From the representation of a general ellipsoid in terms of $\beta_2$ and $\gamma$, all the possible shapes that posses axial symmetry emerge at certain values of $\gamma$, while the unique triaxial region is found in the region $\gamma\in(0,60)$ (see Fig. \ref{beta-gamma-plane}).

%ending phrase
Five publications are summarized herein, namely two research papers that introduce the re-normalization in terms of Signature Partner Bands for the first two triaxial bands in $^{161,163,165,167}$Lu (i.e., Refs. \cite{raduta2020approach,raduta2020towards}), two more papers that extends this formalism with the Parity Partner Bands in $^{163}$Lu (that is Refs. \cite{poenaru2021parity,poenaru2021extensive1}), and lastly a paper devoted to the geometry of the wobbling mode in odd-mass nuclei (i.e., Ref. \cite{poenaru2021extensive2}). % Chapter 8

% appendices 

\appendix

\chapter{Shell Model}
\label{appendix:shell-model}

\section{Shell model}

The idea that an atomic nucleus can have a structure that behaves rather similarly as the atom itself has been enforced by the experimental observations that were done across time. The sharp and discrete discontinuities of nuclear properties, such as the nucleon separation energy, point out that nucleus can be explained through the existence of \emph{shells}. Some examples of observations indicating this are:
\begin{itemize}
    \item When adding a nucleon to a nucleus, there are certain places where the \emph{binding energy} of the next nucleon becomes considerably smaller than the previous one. 
    \item Separation energies for both the protons and neutrons suffer drastic changes, having strong deviations from the predictions of the semi-empirical mass formula \cite{weizsacker1935theorie}, the discontinuities being represented by major shell closures (complete filling) \cite{krane1991introductory}.
    \item The neutron absorption cross-section has a substantial decrease in value at the neutron magic numbers
    \item Great abundance of nuclides where $Z$ and $N$ are magic numbers.
\end{itemize}

The sudden discontinuities occur at specific values of the proton $Z$ and neutron $N$ numbers: these are called \emph{magic numbers}. Currently, these magic numbers correspond to $Z$ or $N=2,8,20,28,50,82,126$, and they represent the major shells. There are also two \emph{weakly magic numbers}: 40 and 64. One can examine the values for the first excited states $2^+$ that are shown in Figs. \ref{e2plus_proton}, \ref{e2plus_neutron}. Indeed, these values show some peaks, each peak corresponding to a particular magic number. This results are part of the work of Raman et al. \cite{raman2001transition}, where the transition probabilities from the ground state to the first-excited $2^+$ state in even-even nuclei were evaluated.
\begin{figure}
    \centering
    \includegraphics[width=0.99\textwidth]{Chapters/Figures/E2plus_proton.pdf}
    \caption{The first excited energy states $2^+$ of nuclei with even $Z$ and $N$ are graphically represented with respect to the proton number. Each connecting line represents a set of isotopes. Figure taken from Ref. \cite{matta2017exotic}.}
    \label{e2plus_proton}
\end{figure}
\begin{figure}
    \centering
    \includegraphics[width=0.99\textwidth]{Chapters/Figures/E2plus_neutron.pdf}
    \caption{The first excited energy states $2^+$ of nuclei with even $Z$ and $N$ are graphically represented with respect to the neutron number. Each connecting line represents a set of isotopes. Figure taken from Ref. \cite{matta2017exotic}.}
    \label{e2plus_neutron}
\end{figure}

The shell model starts from the basic assumption that the nucleus is a \emph{mean-field potential}, where the motion of a single nucleon is caused by all the other nucleons. In other words, the nucleon is moving inside an average potential generated by all the other constituents of the nucleus. Of course that all the nucleons under the influence of this mean field occupy the energy levels which correspond to a series of sub-shells verifying the \textit{Pauli exclusion principle}. Having a general expression for the potential that reproduces all the magic numbers and the observed nuclear properties is therefore crucial. Since the model starts from the concept of independent (non-interacting) particle motion within an average potential, finding each energy will be equivalent of solving the Schrödinger equation:
\begin{align}
    -\frac{\hbar^2}{2m}\nabla ^2\psi_i(r)+V(r)\psi_i(r)=e_i\psi_i(r)\, 
    \label{schrodinger-single-particle-eq}
\end{align}
where $e_i$ represents the energy (eigenvalue), $\psi_i$ represents the wave-function (eigenstates), and $V(r)$ is the nuclear potential whose expression must be evaluated. The choice of $V(r)$ will be dictated by the reproduction of various experimental data (such as nuclear saturation, scattering, nuclear reactions, and so on). For the motion of an independent particle, an obvious first attempt would be the \emph{simple harmonic oscillator} (SHO), which has the known expression:
\begin{align}
    V(r)=\frac{1}{2}m(\omega_i r)^2\ ,
    \label{harmonic-potential}
\end{align}
with $\omega_i$ as the frequency of the basic harmonic-like motion of the particle in the nucleus. With Eq. \ref{harmonic-potential}, the motion of the nucleon has a straightforward expression:
\begin{align}
    \frac{\hbar^2}{2m}\nabla^2\psi_i(r)+\frac{1}{2}m(\omega r)^2\psi_i(r)=e_i\psi_i(r)\ .
\end{align}

This Schrödinger equation has its energy eigenvalues under to form:
\begin{align}
    e_N=\left(N+\frac{3}{2}\right)\hbar\omega\ ,
\end{align}
where $N$ is the number of oscillator quanta which describes each major shell (also called the \emph{principal quantum number}). One should keep in mind that such an expression is typical for a three-dimensional and isotropic harmonic oscillator. The principal quantum number $N$ is furthermore defined as:
\begin{align}
    N=2(n-1)+l\ ,
\end{align}
with $n$ and $l$ being the \emph{radial} quantum number and \emph{orbital angular momentum} quantum number, respectively, taking values $n=1,2,3,\dots$ and $l=0,1,2,\dots,n-1$. In this first approximation, all the levels with the same principal quantum number $N$ are \emph{degenerate}, with a maximal degeneracy given by $2(2l+1)$. However, by using only the SHO term as the expression of $V(r)$, only the first three magic numbers are reproduced, meaning that some additional term(s) might be needed in order to consistently obtain the series of magic numbers.

Furthermore, the steepness of the SHO can be corrected with an \emph{attractive} term proportional to $l$-squared. This acts as a centrifugal term which provides an angular momentum barrier, lifting the degeneracy between the levels with the same principal quantum number $N$ and different values for the orbital angular momentum $l$. This SHO+$l^2$ adjustment is still not enough, such that a so-called \emph{spin-orbit} coupling term of the form $\vec{l}\cdot\vec{s}$ must be also added. This term comes from the consideration that the nucleon-nucleon interaction has a spin dependence, and the potential depends on the intrinsic spin $s$ ($\vec{s}$) and the orbital angular momentum $l$ ($\vec{l}$) of a nucleon. Since $\vec{j}=\vec{l}+\vec{s}$, two possible states emerge from a single value of $l$ (depending on wether $\vec{s}$ is parallel or anti-parallel to $\vec{l}$). The final will consist of the \emph{Modified Harmonic Oscillator} (MHO).
\begin{align}
    V(r)=\frac{1}{2}(\omega r)^2+B\ \vec{l}^2+A\ \vec{l}\cdot\vec{s}\ .
    \label{modified-harmonic-oscillator-eq}
\end{align}

For the sake of simplicity, the centrifugal term will be denoted within formulas without the vector symbol. Since the intrinsic spin of a nucleon is $s=1/2$, for a given value of $l$, there can be two values for the \emph{total angular momentum} (a.m.) $j=l\pm1/2$: one for each spin orientation with respect to the direction of the orbital a.m. Moreover, for every value of $l=0,1,2,3,4,\dots$, there is a similar notation $l=s,p,d,f,g,\dots$, respectively. Regarding the spectroscopic notation, usually, the value of $j$ is considered as a subscript; $nl_j$ (for example $1p_{1/2}$ and $1p_{3/2}$). For high enough shells, there can be splittings between $j+1/2$ and $j-1/2$ that are large enough to lower the $j+1/2$ state from one oscillator shell $n$ to one located below $n-1$. These types of levels are called \emph{intruder states}, and they have opposite parity $\pi=(-1)^l$ with respect to the shell that these levels will occupy.

Going back to the expression of the $\vec{l}\cdot\vec{s}$ term from Eq. \ref{modified-harmonic-oscillator-eq} and denoting it with $V_{ls}(r)$, its contribution to the total potential can be regarded as a surface effect and it can be expressed as a function that depends on the radial coordinate as such \cite{casten2000nuclear}:
\begin{align}
    V_{ls}(r)=-a_{ls}\frac{\partial V(r)}{\partial r}\vec{l}\cdot\vec{s}\ ,
\end{align}
where $V(r)$ is the expression for a central potential and $a_{ls}$ is a strength constant.

The nuclear potential is now able to reproduce all the magic numbers. It is also possible to formulate the total energy of a single-particle within the average potential. Thus, the Hamiltonian of this simple system (i.e., the MHO) can be formulated as such:
\begin{align}
    H&=-\frac{\hbar^2}{2m}\nabla^2+V_\text{SHO}+(l^2)_\text{term}+(\vec{l}\cdot\vec{s})_\text{term}=-\frac{\hbar^2}{2m}\nabla^2+V_\text{MHO} , \nonumber\\
    H&=-\frac{\hbar^2}{2m}\nabla^2+\frac{1}{2}m(\omega r)^2+Bl^2+A\vec{l}\cdot\vec{s}\ .
\end{align}

The evolution from a SHO, then SHO+$l^2$, and finally SHO+$l^2+\vec{l}\cdot\vec{s}$ or modified oscillator potential is illustrated in Fig. \ref{energy-levels-mho}, where it can be seen how each extra term removes a degeneracy, with the complete reproduction of the magic numbers in the third column. The \emph{intruder} levels can also be observed, where states with $j=l+1/2$ from a particular $n$ are so low, that they lie below an $n-1$ adjacent level.
\begin{figure}
    \centering
    \includegraphics[width=0.99\textwidth]{Chapters/Figures/SM_level_scheme.png}
    \caption{The energy levels obtained via calculation of the shell model potential using the simple oscillator (SHO), the SHO amended with a centrifugal term $l^2$, and finally the modified oscillator (MHO) that contains a spin-orbit term. The `correct' magic numbers are the ones in the right-most column. Figure is adapted from Refs. \cite{krane1991introductory,matta2017exotic}.}
    \label{energy-levels-mho}
\end{figure}

Another, more realistic potential that can be used in order to reproduce the specific shell model calculation is the so-called Woods-Saxon potential. Because of the short-range character of the strong nuclear force, it is safe to assume that this potential should behave in the same manner as the density distribution of the nucleons. Since for medium and heavy nuclei, the Fermi-like functions (distributions) are the ones that best fit the experimentally measured data, this potential should have the following form \cite{woods1954diffuse}:
\begin{align}
    V_\text{ws}(r)=-\frac{V_0}{1+e^{\frac{r-R_0}{a}}}\ ,
    \label{woods-saxon-potential}
\end{align}
where $V_0$ represents the depth of the potential ($\approx 50$ MeV, in order to reproduce the experimental separation energies for the nucleons), $a$ is the surface thickness (or \emph{diffuseness parameter}, giving information about how fast the potential drops to zero) with a value of approximately $0.5$ fm, and $R_0$ is the nuclear radius ($R_0=1.2A^{1/3}\ \text{fm}$). The nature of this potential is of \emph{central type} and Eq. \ref{woods-saxon-potential} is not enough the reproduce the higher magic numbers. As such, the addition of a spin-orbit term, similarly as in the case of MHO potential, is required \cite{martin2017particle}: 
\begin{align}
    V_\text{total}=V_\text{ws}^\text{central}+V_{ls}(r)\vec{l}\cdot\vec{s}\ .
    \label{woods-saxon-so-potential}
\end{align}
The only good quantum numbers in the case of the WS potential are the total a.m. $j$ and the parity $\pi=(-1)^l$.
The expectation value of the spin-orbit term $\vec{l}\cdot\vec{s}$ can be given as:
\begin{align}
    \langle ls \rangle=\hbar^2\begin{cases}
        \frac{l}{2} \quad &\text{for} j=l+\frac{1}{2}\\
        -\frac{l+1}{2} &\text{for} j=l-\frac{1}{2}\ \\
    \end{cases}\ .
\end{align}
and the spacing between two levels can be furthermore expressed as \cite{martin2017particle}:
\begin{align}
    \Delta E_{ls}=\frac{2l+1}{2}\hbar^2\langle V_{ls}\rangle\ .
\end{align}
The experimental evidence points to the fact that $V_{ls}(r)$ is negative, meaning that states with $j=l-1/2$ are shifted higher than $j=l+1/2$. Some characteristics of the WS potential are the following:
\begin{enumerate}
    \item It increases with the increase of $R$, meaning that it has an \emph{attractive nature}
    \item It flattens out for large enough $A$ in the center of the nucleus
    \item It rapidly goes to zero as $R$ increases (given by the diffuseness parameter), indicating its short-range nature
    \item When $R=R_0$ (that is for the nucleons near the surface), a large force towards the center of the nucleus is experienced by the these nucleons.
\end{enumerate}
\begin{figure}
    \centering
    \includegraphics[scale=0.2]{Chapters/Figures/ws_potential_plot.png}
    \caption{The shape of the Woods-Saxon potential, defined by Eq. \ref{woods-saxon-potential}. The parameters are arbitrarily chosen as: $V_0=50$ MeV, $R=5.57$ fm, and $a=0.5$ fm.}
    \label{woods-saxon-plot}
\end{figure}

The Hamiltonian that describes the motion of the nucleon within the mean-field potential is given by:
\begin{align}
    %H&=-\frac{\hbar^2}{2m}\nabla^2+V_\text{ws}^\text{central}+(\vec{l}\cdot\vec{s})_\text{term}\ ,\\
    H&=-\frac{\hbar^2}{2m}\nabla^2-\frac{V_0}{1+e^{\frac{r-R_0}{a}}}+A\vec{l}\cdot\vec{s}\ ,
\end{align}
and the shape of a typical Woods-Saxon potential is shown in Fig. \ref{woods-saxon-plot}. A comparison between the Woods-Saxon potential, a SHO, and the square-well-like potential is made in Fig \ref{shell-model-functional-potentials}. The difference between the pure form of the Woods-Saxon potential and the total potential amended with the spin-orbit contribution can be seen in Fig. \ref{woods-saxon-energy-levels}.
\begin{figure}
    \centering
    \includegraphics[scale=0.2]{Chapters/Figures/functional-potentials-shell-model.png}
    \caption{A schematic representation with the three kind of potentials used to describe the shell model: harmonic oscillator, Woods-Saxon, and for completeness, the square-well.}
    \label{shell-model-functional-potentials}
\end{figure}
\begin{figure}
    \centering
    \includegraphics[width=0.99\textwidth]{Chapters/Figures/energy_levels_WS.png}
    \caption{The energy levels calculated for the Woods-Saxon potential given by Eq. \ref{woods-saxon-potential} (left-side), and the single-particle energies with the spin-orbit correction added, as in Eq. \ref{woods-saxon-so-potential} (right-side). Figure adapted from Ref. \cite{lewis2019lifetime}.}
    \label{woods-saxon-energy-levels}
\end{figure}

So far, the general discussion concerning nuclear models was made for the case where each nucleon is treated as an \emph{independent} particle moving in an average (mean-field) potential. However, such an assumption is not accurate enough (especially for the nuclei that lie far away from the closed shells), and this problem should be treated within a \emph{many-body} approach: considering the mutual interaction between the nucleons. These interactions are also called \emph{residual interactions} \cite{casten2000nuclear,bertulani2007nuclear}. With these residual interactions, an accurate depiction of the nucleus might be achieved. The \emph{Deformed Shell Model} will be employed in the following sections, reaching to the famous Nilsson model of describing the nucleus.


\appendix

\chapter{Rotational Bands in Nuclei}
\label{appendix:ral-dal-signature-scheme}

Based on the behavior of $\mathbf{j}$ and its coupling with the collective angular momentum, two rotational rotational modes can occur, each generating rotational bands. The two situations are called \emph{Deformation aligned bands} and \emph{Rotation aligned bands} \cite{uwitonze2015assignment}, and the two modes will be described in this chapter. Graphical representation with both rotational schemes can be seen in Fig. \ref{ral-dal-coupling-bands}.
\begin{figure}
    \centering
    \includegraphics[width=0.49\textwidth]{Chapters/Figures/DAL_scheme.pdf}
    \includegraphics[width=0.49\textwidth]{Chapters/Figures/RAL_scheme.pdf}
    \caption{A sketch with the geometrical interpretation of the two ways a nucleus can exhibit rotational bands: Deformed Aligned Bands (\textbf{left}) and Rotation Aligned Bands (\textbf{right}). The projections of the single-particle's total a.m. on the deformation and rotation axes are denoted with $K$ and $j_1$, respectively.}
    \label{ral-dal-coupling-bands}
\end{figure}

\section{Deformation Aligned Bands}

This case is also referred to as \emph{strong-coupling} limit \cite{bohr1953collective} since the particle's a.m. is tightly coupled to the deformation axis (i.e., the symmetry axis). The general Hamiltonian will be:
\begin{align}
    H_\text{rot}=H_0+H_\text{coupl}\ ,
\end{align}
where $H_0$ is the already discussed operator containing the squared components $I_k$ of $\mathbf{I}$ and the extra \emph{coupling term} represents the Coriolis force \cite{bertulani2007nuclear}. The Coriolis effect is reflected by the coupling of the collective motion of the nucleus to the odd nucleon's motion. Despite that, it can be neglected at small rotations.

The projection of the particle's a.m. on the symmetry axis is a good quantum number and if $\mathbf{R}$ is pointing in a direction that is perpendicular to the deformation axis, then $\Omega=K$, meaning that the energy spectrum will be given by:
\begin{align}
    E_\text{rot}(I)=\frac{\hbar^2}{2\mathcal{I}_\perp}\left[I(I+1)-K(K+1)\right]\ ,
\end{align}
or more formally:
\begin{align}
    E_\text{rot}(I)=\frac{\hbar^2}{2\mathcal{I}_\perp}\left[I(I+1)-K^2\right]+E_0(K)\ .
\end{align}
The rotational band will be constructed on the ground-state $E_0(K)$, where the total spin $I$ will be composed of a sequence $I=K,K+1,K+2,\dots$ having $K\neq\frac{1}{2}$. Consequently, this situation will lead to rotational bands where the difference between two consecutive states is only une unit of angular momentum. It should be pointed out that odd-$A$ nuclei can have multiple rotational bands that are built on different values of $K$. For $K=\frac{1}{2}$, the band structure will follow:
\begin{align}
    E_\text{rot}(I)=\frac{\hbar^2}{2\mathcal{I}_\perp}\left[I(I+1)+a(-)^{I+\frac{1}{2}}(I+\frac{1}{2})\right]\ .
    \label{deformation-aligned-energy}
\end{align}

The nature of $(-1)^{I+1/2}$ will be explained in the next section, which depicts the \emph{Rotation Aligned Bands}. Moreover, the term $a$ is called the \emph{decoupling parameter} \cite{bertulani2007nuclear}, and it can be determined from the first two experimental energy levels. Experimental data exhibiting rotational bands with $K=1/2$ and $K\neq 1/2$ are shown for two odd-$A$ nuclei in Fig. \ref{rotational-bands-odd-a}.
\begin{figure}
    \centering
    \includegraphics[scale=0.7]{Chapters/Figures/Tm165-Rotational-Bands.pdf}
    \includegraphics[scale=0.7]{Chapters/Figures/Lu175-Rotational-Bands.pdf}
    \caption{Rotational bands in odd-$A$ nuclei with the $K$ quantum number equal to $K=1/2$ (\textbf{left}) and $K\neq 1/2$ (\textbf{right}).}
    \label{rotational-bands-odd-a}
\end{figure}

\section{Rotation Aligned Bands}
\label{section-ral-signature}
This situation is also called the \emph{decoupling limit} \cite{bohr1998nuclear} and it leads to the apparition of \emph{decoupled bands}. Here the total angular momentum is more aligned to the axis of rotation, and its maximum projection is along this axis. This is usually happening at high-spins, meaning strong rotational motion, which makes the angular momentum of the odd-particle to depart from the symmetry axis and align more and more with the direction of rotation (through the Coriolis effect). As such, the coupling term $H_\text{coupl}$ from $H_\text{rot}$ is not neglected here:
\begin{align}
    H_\text{rot}=\frac{\hbar^2}{2\mathcal{I}_\perp}\left(\mathbf{I}^2+\mathbf{j}^2-2\mathbf{I}\cdot\mathbf{j}\right)\ .
\end{align}

It is usually preferred to work with the \emph{raising} and \emph{lowering} operators which correspond to the angular momentum operator: $\mathbf{I}_\pm=\mathbf{I}_1\pm i \mathbf{I}_2$ (equivalently can be done for $\mathbf{j}$), bringing $H_\text{rot}$ to the following expression:
\begin{align}
    H_\text{rot}&=\frac{\hbar^2}{2\mathcal{I}_\perp}\hat{I}^2+\frac{\hbar^2}{2\mathcal{I}_\perp}\hat{j}^2-\frac{\hbar^2}{\mathcal{I}_\perp}K^2+H_\text{Coriolis}\ ,\\
    H_\text{Coriolis}&=-\frac{\hbar^2}{2\mathcal{I}_\perp}(\mathbf{I}_+\mathbf{j}_-+\mathbf{I}_-\mathbf{j}_+)\ .
\end{align}

It is worth pointing out that the effect of the Coriolis term is to couple bands which differ in the $K$ quantum number with one unit, effect which is negligible at high deformations and low spins (since the single-particle motion is tightly bound to the bulk nucleus) while at very high rotations (spins) it becomes significant. Consequently, the Coriolis effect most probably occurs in prolate nuclei for an `almost empty' $j$-shell and oblate nuclei for an `almost full' $j$-shell.

When the single-particle angular momentum is orienting itself to the direction of rotation, the projection of $\mathbf{j}$ can be denoted by $j_1$ (keeping a consistency with Fig. \ref{rotational-coupling-schematic}). The spectrum of the decoupled bands will be:
\begin{align}
    E_\text{rot}(I)=\text{const.}+\frac{\hbar^2}{2\mathcal{I}_\perp}(I-j_1)(I-j_1+1)\ ,
\end{align}
where the coupling terms have been embedded in the constant term. This leads to a spin sequence $I=j_1,j_1+2,j_1+4\cdots$, which differs from the previous case via the constant $2\hbar$ angular momentum difference of two consecutive levels.

In order to understand the terms $(-1)^{I+1/2}$ from Eq. \ref{deformation-aligned-energy}, it is required to describe the wave-function corresponding to the particle-core system. Indeed, using the specific quantum numbers $I,K,M$ with their meaning explained in Fig. \ref{rotational-coupling-schematic}, the wave-function will be written as a combination of rotational (the Wigner $\mathcal{D}_{MK}^I$ functions) and single-particle components \cite{wang2007exotic,davydov1958rotational}:
\begin{align}
    \Psi_{MK}^I=\ket{IMK}=N\left[\phi_K \mathcal{D}_{MK}^I+(-)^{I+K}\phi_{-K}\mathcal{D}_{M-K}^I\right]\ ,
    \label{RAL-bands-wave-function}
\end{align}
where $N$ is the normalization constant, usually having the value $N=\sqrt{\frac{2I+1}{16\pi^2}}$. This linear combination of states with $K$ and $-K$ induces a degeneracy and it is due to the invariance of such a system with respect to rotations by $\pi$ around the rotational axis \cite{frauendorf1997tilted,bohr1998nuclear}. The factor $(-)^{I+K}\equiv\alpha$ is called the \emph{signature} and it reflects wether a system is invariant or not to such a rotation. More precisely, the \emph{signature quantum number} is given as \cite{sun1994varied}:
\begin{align}
    \alpha_I=\frac{1}{2}(-)^{I-\frac{1}{2}}\ ,
    \label{signature-quantum-number}
\end{align}
for a state of spin $I$ in an odd-$A$ nucleus, resulting in the favored states having $\alpha_\textbf{favored}=\frac{1}{2}$ and the unfavored states having $\alpha_\textbf{unfavored}=-\frac{1}{2}$.

Depending on the signature, the nuclear states can be divided into two sets: one that follows $I=K,K+2,K+4,\dots$ and $I=K+1,K+3,K+5,\dots$. This is the reason why for the decoupled bands, one can regard them as an `initial' rotational band $I,I+1,\dots$ that is `broken' apart in two sequences: one that is favored and one that is unfavored (opposite signature).
An example is an odd-$A$ nucleus where the favored bands have spins $I_\text{favored}=\frac{1}{2},\frac{5}{2},\frac{1}{2},\dots$, while their unfavored \emph{partner} bands have spins $I_\text{unfavored}=\frac{3}{2},\frac{7}{2},\frac{11}{2},\dots$ and opposite signature (also known as \emph{signature partners}). In fact, taking a closer look at the rotational bands specific to odd-$A$ nuclei shown in Fig. \ref{rotational-bands-odd-a}, each consecutive level is a state with different signature, meaning that each `group' of colors classifies into a set of favored (blue) and unfavored (magenta) states. The concept of signature partners will be crucial for a developed formalism that aims at describing rotational motion of highly deformed nuclei. This will be treated in Chapter \ref{chapter-6-aw1-formalism}.

This divided set of partner bands also has some characteristics that can be observed throughout experimental measurements. Firstly, the splitting of the two branches implies that the favored states will generally have lower excitation energy than their unfavored partners. This is also proved by the expression of the rotor energy given in Eq. \ref{deformation-aligned-energy}, where the decoupling parameter will cause an upward (downward) shift in energy for states with $I=1/2,5/2,9/2,\dots$ ($I=3/2,7/2,11/2,\dots$) if $a$ is positive (negative). The experimental data shown in Fig. \ref{level-scheme-signature-splitting} shows how the favored partner lies lower with respect to its unfavored partner bands, each having the corresponding spin sequence $\Delta I=2$ for intraband states and $\Delta I=1$ for interband states. Such spectra are very often met in the decay schemes for odd-mass nuclei in which the rotational motion is governed by the core + particle coupling scheme.
\begin{figure}
    \centering
    % \includegraphics[scale=0.45]{Chapters/Figures/Lu_163_K12-band.png}
    % \includegraphics[scale=0.28]{Chapters/Figures/Lu_163_signatureSplitting.png}
    \includegraphics[width=0.49\textwidth]{Chapters/Figures/Lu_163_K12-band.png}
    \includegraphics[width=0.49\textwidth]{Chapters/Figures/Lu_163_signatureSplitting.png}
    \caption{Experimental level schemes for $^{163}$Lu showing pairs of signature partner bands. \textbf{Left pair}: the two bands are built on a proton with $j=1/2$ and positive parity. \textbf{Right pair}: sequences built on a proton with $j=5/2$ with the same parity. The Nilsson quantum numbers are defined for each band. Note the lower energies for the favored states. Interband transitions are marked with the blue arrows. The experimental data is from Ref. \cite{reich2010nuclear} and the level schemes were taken from Ref. \cite{bhat1992evaluated}.}
    \label{level-scheme-signature-splitting}
\end{figure}


% references
\bibliography{bibliography}

\end{document}
